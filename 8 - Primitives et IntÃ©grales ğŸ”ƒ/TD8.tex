\documentclass[10pt]{article}

\usepackage[T1]{fontenc}
\usepackage[left=2cm, right=2cm, top=2cm, bottom=2cm]{geometry}
\usepackage[skins]{tcolorbox}
\usepackage{hyperref, fancyhdr, lastpage, tocloft, ragged2e, multicol}
\usepackage{amsmath, amssymb, amsthm, stmaryrd}
\usepackage{tkz-tab}

\def\pagetitle{Primitives et intégrales}

\title{\bf{\pagetitle}\\\large{Corrigé}}
\date{Octobre 2023}
\author{DARVOUX Théo}

\hypersetup{
    colorlinks=true,
    citecolor=black,
    linktoc=all,
    linkcolor=blue
}

\pagestyle{fancy}
\cfoot{\thepage\ sur \pageref*{LastPage}}


\begin{document}
\renewcommand*\contentsname{Exercices.}
\renewcommand*{\cftsecleader}{\cftdotfill{\cftdotsep}}
\maketitle
\hrule
\tableofcontents
\vspace{0.5cm}
\hrule

\thispagestyle{fancy}
\fancyhead[L]{MP2I Paul Valéry}
\fancyhead[C]{\pagetitle}
\fancyhead[R]{2023-2024}
\allowdisplaybreaks

\pagebreak

\section*{Exercice 8.1 [$\blacklozenge\lozenge\lozenge$]}
\begin{tcolorbox}[enhanced, width=7in, center, size=fbox, fontupper=\large, drop shadow southwest]
    Donner les primitives des fonctions suivantes (on précisera l'intervalle que l'on considère).
    \begin{align*}
        &a:x\mapsto\cos{xe^{\sin{x}}}; \hspace{1cm} b:x\mapsto\frac{\cos x}{\sin x}; \hspace{1cm} c:x\mapsto\frac{\cos x}{\sqrt{\sin x}}; \hspace{1cm} d:x\mapsto\frac{1}{3x+1};\\
        &e:x\mapsto\frac{\ln x}{x}; \hspace{1cm} f:x\mapsto\frac{1}{x\ln x}; \hspace{1cm} g:x\mapsto\sqrt{3x+1}; \hspace{1cm} h:x\mapsto\frac{x+x^2}{1+x^2}.
    \end{align*}
    \begin{align*}
        &A:\begin{cases}\mathbb{R}\rightarrow\mathbb{R}\\x\mapsto e^{\sin x} + c\end{cases}; \hspace{0.3cm} B:\begin{cases}\mathbb{R}\setminus\{k\pi, k\in\mathbb{Z}\}\rightarrow\mathbb{R}\\x\mapsto\ln(\sin x) + c\end{cases}; \hspace{0.3cm} C:x\mapsto\begin{cases}]2k\pi, (2k+1)\pi[, k\in\mathbb{Z}\rightarrow\mathbb{R}\\x\mapsto2\sqrt{\sin x} + c\end{cases}\\
        &D:\begin{cases}\mathbb{R}\setminus\{-\frac{1}{3}\}\rightarrow\mathbb{R}\\x\mapsto\frac{1}{3}\ln(3x+1) + c\end{cases}; \hspace{0.4cm} E:\begin{cases}\mathbb{R_+^*}\rightarrow\mathbb{R}\\x\mapsto\frac{1}{2}\ln^2x + c\end{cases}; \hspace{0.4cm} F:\begin{cases}\mathbb{R_+^*}\rightarrow\mathbb{R}\\x\mapsto\ln(\ln x) + c\end{cases};\\
        &G:\begin{cases}[-\frac{1}{3}, +\infty]\rightarrow\mathbb{R}\\x\mapsto\frac{2}{9}(3x+1)^{\frac{3}{2}} + c\end{cases}; \hspace{0.5cm}H:\begin{cases}\mathbb{R}\rightarrow\mathbb{R}\\x\mapsto\frac{1}{2}\ln(1+x^2) + x - \arctan(x) + c\end{cases}.
    \end{align*}
    Avec $c$ les constantes d'intégration.
    \qed
\end{tcolorbox}

\addcontentsline{toc}{section}{\protect\numberline{}Exercice 8.1}

\section*{Exercice 8.2 [$\blacklozenge\lozenge\lozenge$] Issu du cahier de calcul}
\begin{tcolorbox}[enhanced, width=7in, center, size=fbox, fontupper=\large, drop shadow southwest]
    On rappelle que $\int_a^b{f(x)dx}$ est l'aire algébrique entre la courbe représentative de $f$ et l'axe des abscisses.\\
    1. Sans chercher à les calculer, donner le signe des intégrales suivantes.
    \begin{equation*}
        \int_{-2}^3{e^{-x^2}dx}; \hspace{1cm} \int_5^{-3}{|\sin x|dx}; \hspace{1cm} \int_1^a{\ln^7(x)dx} (a\in\mathbb{R_+^*}).
    \end{equation*}
    2. En vous ramenant à des aires, calculer de tête
    \begin{equation*}
        \int_1^3{7dx}; \hspace{1cm} \int_0^7{3xdx}; \hspace{1cm} \int_{-2}^1{|x|dx}.
    \end{equation*}
    1. La première est positive car $-2<3$ et la fonction est positive sur $[-2,3]$e.\\
    La seconde est négative car $5>-3$ et la fonction est positive sur $[-3,5]$.\\
    La dernière est positive lorsque $a\geq1$ et négative lorsque $a\leq1$ car $\ln^7$ est positive sur $[1,+\infty[$.\\
    2. La première vaut $2\times7=14$.\\
    La seconde vaut $\frac{7^2\times3}{2}=\frac{147}{2}$.\\
    La dernière vaut $\frac{1}{2}+\frac{2\times2}{2}=2.5$\\
    \qed
\end{tcolorbox}

\addcontentsline{toc}{section}{\protect\numberline{}Exercice 8.2}

\end{document}
 