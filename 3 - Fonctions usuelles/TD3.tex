\documentclass[10pt]{article}

\usepackage[T1]{fontenc}
\usepackage[left=2cm, right=2cm, top=2cm, bottom=2cm]{geometry}
\usepackage[skins]{tcolorbox}
\usepackage{hyperref, fancyhdr, lastpage, tocloft, ragged2e}
\usepackage{amsmath, amssymb, amsthm}

\def\pagetitle{Fonctions usuelles}

\title{\bf{\pagetitle}\\\large{Corrigé}}
\date{Septembre 2023}
\author{DARVOUX Théo}

\hypersetup{
    colorlinks=true,
    citecolor=black,
    linktoc=all,
    linkcolor=blue
}

\pagestyle{fancy}
\cfoot{\thepage\ sur \pageref*{LastPage}}


\begin{document}
\renewcommand*\contentsname{Exercices.}
\renewcommand*{\cftsecleader}{\cftdotfill{\cftdotsep}}
\maketitle
\hrule
\tableofcontents
\vspace{0.5cm}
\hrule

\thispagestyle{fancy}
\fancyhead[L]{MP2I Paul Valéry}
\fancyhead[C]{\pagetitle}
\fancyhead[R]{2023-2024}


\addcontentsline{toc}{section}{Exponentielle and friends.}

\section*{Exercice 3.1 [$\blacklozenge\lozenge\lozenge$]}
\begin{tcolorbox}[enhanced, width=6in, center, size=fbox, fontupper=\large, drop shadow southwest]
    Résoudre $2\ln\left(\frac{x+3}{2}\right)=\ln(x)+\ln(3)$, sur $\mathbb{R}^*_+$.\\
    Soit $x\in\mathbb{R^*_+}$.\\
    On a :
    \begin{align*}
        &2\ln\left(\frac{x+3}{2}\right)=\ln(x)+\ln(3)\\
        \iff&\ln\left(\left(\frac{x+3}{2}\right)^2\right)=\ln(3x)\\
        \iff&\frac{(x+3)^2}{4}=3x\\
        \iff&x^2-6x+9=0\\
        \iff&x=3
    \end{align*}
    Ainsi, $3$ est l'unique solution.
\end{tcolorbox}
\addcontentsline{toc}{section}{\protect\numberline{}Exercice 3.1}

\section*{Exercice 3.2 [$\blacklozenge\blacklozenge\lozenge$]}
\begin{tcolorbox}[enhanced, width=6in, center, size=fbox, fontupper=\large, drop shadow southwest]
    Résoudre l'équation $ch(x)=2$. Que dire des solutions ?\\
    Soit $x\in\mathbb{R}$.\\
    On a :
    \begin{align*}
        &\frac{e^x+e^{-x}}{2}=2\\
        \iff&e^x+e^{-x}=4\\
        \iff&e^{2x}-4e^x+1=0\\
        \iff&e^x=2\pm\sqrt{3}\\
        \iff&x=\ln(2\pm\sqrt{3})
    \end{align*}
    Ainsi, $\ln(2-\sqrt{3})$ et $\ln(2+\sqrt{3})$ sont les uniques solutions dans $\mathbb{R}$.\\
    On remarque que :
    \begin{align*}
        \ln(2+\sqrt{3})=-\ln\left(\frac{1}{2+\sqrt{3}}\right)=-\ln\left(2-\sqrt{3}\right)
    \end{align*}
    Les solutions sont opposées.
\end{tcolorbox}
\addcontentsline{toc}{section}{\protect\numberline{}Exercice 3.2}

\end{document}