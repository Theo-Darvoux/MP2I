\documentclass[10pt]{article}

\usepackage[T1]{fontenc}
\usepackage[left=2cm, right=2cm, top=2cm, bottom=2cm]{geometry}
\usepackage[skins]{tcolorbox}
\usepackage{hyperref, fancyhdr, lastpage, tocloft, ragged2e, multicol}
\usepackage{amsmath, amssymb, amsthm, stmaryrd}
\usepackage{tkz-tab}
\usepackage{systeme}

\def\pagetitle{Petits Systèmes Linéaires}

\title{\bf{\pagetitle}\\\large{Corrigé}}
\date{Novembre 2023}
\author{DARVOUX Théo}

\hypersetup{
    colorlinks=true,
    citecolor=black,
    linktoc=all,
    linkcolor=blue
}

\pagestyle{fancy}
\cfoot{\thepage\ sur \pageref*{LastPage}}


\begin{document}
\renewcommand*\contentsname{Exercices.}
\renewcommand*{\cftsecleader}{\cftdotfill{\cftdotsep}}
\maketitle
\begin{center}
    \large{MRC Ibrahim pour le 9.4}\\
    \large{C'est pas possible d'écrire les $L_1 \leftrightarrow L_3$ sous les $\iff$ (de manière esthétique) en \LaTeX, sadface}
\end{center}
\hrule
\tableofcontents
\vspace{0.5cm}
\hrule


\thispagestyle{fancy}
\fancyhead[L]{MP2I Paul Valéry}
\fancyhead[C]{\pagetitle}
\fancyhead[R]{2023-2024}
\allowdisplaybreaks

\pagebreak

\section*{Exercice 9.1 [$\blacklozenge\lozenge\lozenge$] [Un système de Cramer bête et méchant]}
\begin{tcolorbox}[enhanced, width=7in, center, size=fbox, fontupper=\large, drop shadow southwest]
    Résoudre le système suivant dans $\mathbb{R}^3$.
    \begin{equation*}
        \systeme{
            3x + y- 2z = 10,
            2x - y+ z = 3,
            x - y +2z = 2
        }
    \end{equation*}
    Soit $(x,y,z)\in\mathbb{R}^3$.
    \begin{align*}
        &(x,y,z) \text{ est solution } \\
        \iff&
        \systeme{
            3x + y -2z = 10,
            2x - y + z = 3,
            x - y + 2z = 2
        }\\ \iff&
        \systeme{
            x - y + 2z = 2,
            2x - y + z = 3,
            3x + y - 2z = 10
        }\\ \iff&
        \systeme{
            x-y+2z=2,
            y-3z=-1,
            4y-8z=4
        }\\ \iff&
        \systeme{
            x-y+2z=2,
            y-3z=-1,
            4z=8
        }\\ \iff&
        \begin{cases}
            x=3\\
            y=5\\
            z=2
        \end{cases}
    \end{align*}
    L'unique solution de système dans $\mathbb{R}^3$ est donc $(3,5,2)$.\\
    \qed
\end{tcolorbox}

\addcontentsline{toc}{section}{\protect\numberline{}Exercice 9.1}

\section*{Exercice 9.2 [$\blacklozenge\lozenge\lozenge$]}
\begin{tcolorbox}[enhanced, width=7in, center, size=fbox, fontupper=\large, drop shadow southwest]
    Résoudre le système suivant dans $\mathbb{R}^3$.
    \begin{equation*}
        \systeme{
            x+2y-z=2,
            x-2y+3z=-2,
            3x-2y+5z=-2
        }
    \end{equation*}
    Soit $(x,y,z)\in\mathbb{R}^3$.
    \begin{align*}
        &(x,y,z) \text{ est solution}\\
        \iff&
        \systeme{
            x+2y-z=2,
            x-2y+3z=-2,
            3x-2y+5z=-2
        }\\ \iff&
        \systeme{
            x+2y-z=2,
            -4y+4z=-4,
            -8y+8z=-8
        }\\ \iff&
        \systeme{
            x+2y-z=2,
            y - z = 1,
            z = y - 1
        }\\ \iff&
        \begin{cases}
            y=1-x\\
            z=-x
        \end{cases}\\
    \end{align*}
    L'ensemble $S$ des solutions est alors 
    \begin{equation*}
        S = \left\{(x, 1-x, -x) \hspace{0.2cm} | \hspace{0.2cm} x \in \mathbb{R}\right\} = \left\{(0, 1, 0) + x(1,-1,-1) \hspace{0.2cm} | \hspace{0.2cm} x\in\mathbb{R}\right\}
    \end{equation*}
    \qed
\end{tcolorbox}

\addcontentsline{toc}{section}{\protect\numberline{}Exercice 9.2}

\section*{Exercice 9.3 [$\blacklozenge\blacklozenge\lozenge$]}
\begin{tcolorbox}[enhanced, width=7in, center, size=fbox, fontupper=\large, drop shadow southwest]
    Soit $(a,b,c)\in\mathbb{R}^3$, $a \neq b$, $a \neq c$, $b \neq c$. Résoudre :
    \begin{equation*}
        \begin{cases}
            x+ay+a^2z=a^3\\
            x+by+b^2z=b^3\\
            x+cy+c^2z=c^3
        \end{cases}
    \end{equation*}
    Soit $(x,y,z)\in\mathbb{R}^3$.
    \begin{align*}
        &(x,y,z) \text{ est solution}\\ \iff&
        \begin{cases}
            x+ay+a^2z=a^3\\
            x+by+b^2z=b^3\\
            x+cy+c^2z=c^3
        \end{cases}
        \\\iff &
        \begin{cases}
            x+ay+a^2z=a^3\\
            (b-a)y+(b^2-a^2)z=b^3-a^3\\
            (c-a)y+(c^2-a^2)z=c^3-a^3
        \end{cases}
        \\\iff &
        \begin{cases}
            x+ay+a^2z=a^3\\
            (b-a)y+(b-a)(b+a)z=(b-a)(a^2+ab+b^2)\\
            (c-a)y+(c-a)(c+a)z=(c-a)(a^2+ac+c^2)
        \end{cases}
        \\\iff &
        \begin{cases}
            x+ay+a^2z=a^3\\
            y+(b+a)z=a^2+ab+b^2\\
            y+(c+a)z=a^2+ac+b^2
        \end{cases}
        \\\iff &
        \begin{cases}
            x+ay+a^2z=a^3\\
            y+(b+a)z=a^2+ab+b^2\\
            z =a + b + c
        \end{cases}
        \\\iff &
        \begin{cases}
            x+ay+a^2z=a^3\\
            y=-bc - ab - ac\\
            z = a + b + c
        \end{cases}
        \\\iff &
        \begin{cases}
            x=abc\\
            y=-(ab + bc + ca)\\
            z = a + b + c
        \end{cases}
    \end{align*}
    L'unique solution est donc $(abc, -(ab + bc + ca), a+b+c)$.\\
    \qed
\end{tcolorbox}

\addcontentsline{toc}{section}{\protect\numberline{}Exercice 9.3}

\section*{Exercice 9.4 [$\blacklozenge\blacklozenge\blacklozenge$]}
\begin{tcolorbox}[enhanced, width=7in, center, size=fbox, fontupper=\large, drop shadow southwest]
    Soit $\lambda$ un paramètre réel et le système :
    \begin{align*}
        \begin{cases}
            (2-\lambda)x + y + z = 0\\
            x + (2-\lambda)y + z = 0\\
            x + y + (2-\lambda)z = 0
        \end{cases}
    \end{align*}
    Le résoudre, en discutant selon les valeurs de $\lambda$.\\
    Soit $(x,y,z)\in\mathbb{R}^3$.
    \begin{align*}
        &(x,y,z) \text{ est solution}\\
        &\iff \begin{cases}
            x + y + (2-\lambda)z = 0\\
            x + (2-\lambda)y + z = 0\\
            (2 - \lambda)x + y + z = 0
        \end{cases}\\
        &\iff \begin{cases}
            x+y+(2-\lambda)z =0\\
            (1-\lambda)y + (\lambda-1)z=0\\
            (\lambda-1)y + (1-(2-\lambda)^2)z=0
        \end{cases}\\
        &\iff \begin{cases}
            x + y + (2-\lambda)z = 0\\
            (1-\lambda)y + (\lambda-1)z = 0\\
            (-\lambda^2 + 5\lambda - 4)z = 0
        \end{cases}
    \end{align*}
    Les racines du polynôme $-\lambda^2+5\lambda-4$ sont $1$ et $4$.\\
    $\circledcirc$ Premier cas : $\lambda\notin\{1,4\}$.
    \begin{equation*}
        (x,y,z) \text{ est solution}
        \iff\begin{cases}
            x=0\\
            y=0\\
            z=0
        \end{cases}
    \end{equation*}
    L'unique solution est le couple $(0,0,0)$.\\
    $\circledcirc$ Deuxième cas : $\lambda=1$.
    \begin{equation*}
        (x,y,z) \text{ est solution}
        \iff\begin{cases}
            x+y+z=0
        \end{cases}
    \end{equation*}
    L'ensemble $S$ des solutions est le plan vectoriel: 
    \begin{equation*}
        S=\{(-y-z,y,z)\,|\, (y,z)\in\mathbb{R}^2\}=\{y(-1,1,0) + z(-1,0,1)\,|\,(y,z)\in\mathbb{R}^2\}
    \end{equation*}
    $\circledcirc$ Dernier cas : $\lambda=4$
    \begin{equation*}
        (x,y,z) \text{ est solution} \iff \begin{cases}
            x+z-2z=0\\
            -3y+3z=0\\
            0=0
        \end{cases}
        \iff\begin{cases}
            x=z\\
            y=z\\
            0=0
        \end{cases}
    \end{equation*}
    L'ensemble S des solutions est la droite passant par l'origine:
    \begin{equation*}
        S = \{(z,z,z)\,|\,z\in\mathbb{R}\} = \{(0,0,0) + z(1,1,1)\ \,|\, z\in\mathbb{R}\}
    \end{equation*} 
    \qed
\end{tcolorbox}

\addcontentsline{toc}{section}{\protect\numberline{}Exercice 9.4}
\end{document}
 