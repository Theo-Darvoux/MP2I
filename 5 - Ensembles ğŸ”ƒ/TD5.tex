\documentclass[10pt]{article}

\usepackage[T1]{fontenc}
\usepackage[left=2cm, right=2cm, top=2cm, bottom=2cm]{geometry}
\usepackage[skins]{tcolorbox}
\usepackage{hyperref, fancyhdr, lastpage, tocloft, ragged2e, multicol}
\usepackage{amsmath, amssymb, amsthm, stmaryrd}
\usepackage{tkz-tab}

\def\pagetitle{Ensembles et applications}

\title{\bf{\pagetitle}\\\large{Corrigé}}
\date{Octobre 2023}
\author{DARVOUX Théo}

\hypersetup{
    colorlinks=true,
    citecolor=black,
    linktoc=all,
    linkcolor=blue
}

\pagestyle{fancy}
\cfoot{\thepage\ sur \pageref*{LastPage}}


\begin{document}
\renewcommand*\contentsname{Exercices.}
\renewcommand*{\cftsecleader}{\cftdotfill{\cftdotsep}}
\maketitle
\hrule
\tableofcontents
\vspace{0.5cm}
\hrule

\thispagestyle{fancy}
\fancyhead[L]{MP2I Paul Valéry}
\fancyhead[C]{\pagetitle}
\fancyhead[R]{2023-2024}
\allowdisplaybreaks

\pagebreak

\section*{Exercice 5.1 [$\blacklozenge\lozenge\lozenge$]}
\begin{tcolorbox}[enhanced, width=7in, center, size=fbox, fontupper=\large, drop shadow southwest]
    Soient $A,B$ deux parties d'un ensemble $E$. Établir que
    \begin{equation*}
        A \setminus (A \setminus B) = A \cap B \hspace{1cm} \text{ et } \hspace{1cm} A \setminus (A \cap B) = A \setminus B = (A \cup B) \setminus B.
    \end{equation*}
    On a :
    \begin{align*}
        A \setminus (A \setminus B) &= A \cap \overline{(A\ \cap \overline{B})}\\
        &=A \cap (\overline{A} \cup B)\\
        &= (A \cap \overline{A}) \cup(A \cap B)\\
        &= A \cap B
    \end{align*}
    D'autre part :
    \begin{align*}
        A \setminus (A \cap B) &= A \cap \overline{(A \cap B)}\\
        &= A \cap (\overline{A} \cup \overline{B})\\
        &= (A \cap \overline{A}) \cup (A \cap \overline{B})\\
        &= A \cap \overline{B}\\
        &= A \setminus B
    \end{align*}
    Et :
    \begin{align*}
        (A \cup B) \setminus B &= (A \cup B) \cap \overline{B}\\
        &= (A \cap \overline{B}) \cup (B \cap \overline{B})\\
        &= A \cap \overline{B}\\
        &= A \setminus B
    \end{align*}
    \qed
\end{tcolorbox}

\addcontentsline{toc}{section}{\protect\numberline{}Exercice 5.1}

\section*{Exercice 5.2 [$\blacklozenge\lozenge\lozenge$]}
\begin{tcolorbox}[enhanced, width=7in, center, size=fbox, fontupper=\large, drop shadow southwest]
    Soient $A,B,C,D$ quatre parties d'un ensemble $E$, telles que
    \begin{equation*}
        E = A \cup B \cup C, \hspace{1cm} A \cap D \subset B, \hspace{1cm} B \cap D \subset C, \hspace{1cm} C \cap D \subset A.
    \end{equation*}
    Montrer que $D \subset A \cap B \cap C$.\\
    Soit $x \in D$, on sait que $x \in E$. Alors $x \in A$ ou $x \in B$ ou $x \in C$.\\
    $\circledcirc$ Si $x \in A$, alors $x \in A \cap D$, donc $x \in B$.\\
    $\circledcirc$ Si $x \in B$, alors $x \in B \cap D$, donc $x \in C$.\\
    $\circledcirc$ Si $x \in C$, alors $x \in C \cap D$, donc $x \in A$.\\
    On en déduit que $x \in A \cap B \cap C$.\\
    Ainsi, $D \subset A \cap B \cap C$.\\
    \qed
\end{tcolorbox}

\addcontentsline{toc}{section}{\protect\numberline{}Exercice 5.2}

\section*{Exercice 5.3 [$\blacklozenge\blacklozenge\lozenge$]}
\begin{tcolorbox}[enhanced, width=7in, center, size=fbox, fontupper=\large, drop shadow southwest]
    Démontrer que
    \begin{equation*}
        \mathbb{R} = \left\lbrace{x \in \mathbb{R} \hspace{0.1cm} | \hspace{0.1cm} \exists a \in \mathbb{R}^*_+ \hspace{0.1cm} \exists b \in \mathbb{R}^*_- : x = a + b}\right\rbrace.
    \end{equation*}
    On note $A$ = $\left\lbrace{x \in \mathbb{R} \hspace{0.1cm} | \hspace{0.1cm} \exists a \in \mathbb{R}^*_+ \hspace{0.1cm} \exists b \in \mathbb{R}^*_- : x = a + b}\right\rbrace$\\
    $\circledcirc$ Montrons que $\mathbb{R} \subset A$.\\
    Soit $x \in \mathbb{R}$.\\
    $\circ$ Si $x \leq 0$, On pose $a=1$ et $b=x-1$, ainsi $x = a + b$ donc $x \in A$.\\
    $\circ$ Si $x > 0$, On pose $a=x+1$ et $b=-1$, ainsi $x = a + b$ donc $x \in A$.\\
    Dans tous les cas $x \in A$, on en conclut que $\mathbb{R} \subset A$.\\
    $\circledcirc$ Montrons que $A \subset \mathbb{R}$.\\
    Soit $x \in A$, alors il existe $a \in \mathbb{R}^*_+$ et $b \in \mathbb{R}^*_-$ tels que $x = a + b$.\\
    Or $a + b \in \mathbb{R}$, donc $x \in \mathbb{R}$. On en conclut que $A \subset \mathbb{R}$.\\
    \qed 
\end{tcolorbox}

\addcontentsline{toc}{section}{\protect\numberline{}Exercice 5.3}

\section*{Exercice 5.4 [$\blacklozenge\blacklozenge\lozenge$]}
\begin{tcolorbox}[enhanced, width=7in, center, size=fbox, fontupper=\large, drop shadow southwest]
    Soit $n \in \mathbb{N}^*$ et $A_1, A_2, \dots, A_n$ $n$ parties de $E$ telles que
    \begin{equation*}
        A_n = E \hspace{1cm} \text{et} \hspace{1cm} A_1 \subset A_2 \subset \dots \subset A_n.
    \end{equation*}
    On pose $B_1=A_1$ et pour $k \in \llbracket{2, n}\rrbracket$, on pose $B_k = A_k \setminus A_{k-1}$.\\
    Prouver que $(B_k)_{1 \leq k \leq n}$ est un recouvrement disjoint de $E$.\\[0.25cm]
    Soit $x \in E$. Alors $x \in A_n$. Il existe alors $k$ le plus petit entier tel que $x \in A_k$. Ainsi, $x \in B_k$ puisque $x \in A_k \wedge x \notin A_{k-1}$ par définition de $k$.\\
    On en déduit que tout élément de $E$ appartient à au moins un $(B_k)$.\\[0.25cm]
    Montrons maintenant que tout élément de $E$ appartient aussi au plus à un $B_k$.\\
    Soit $x \in E$. Supposons qu'il existe $i,j \in \llbracket1,n\rrbracket$ tels que $i < j$ et $x \in B_i$ et $x \in B_j$.\\
    Or, puisque $x \in B_j$ et $i<j$, $x \notin A_i$. De plus, puisque $x \in B_i$, $x \in A_i$ ce qui est absurde.\\
    Ainsi, tout élément de $E$ appartient au plus à un $(B_k)$.\\[0.25cm]
    $(B_k)_{1 \leq k \leq n}$ est donc un recouvrement disjoint de E.\\
    \qed
\end{tcolorbox}

\addcontentsline{toc}{section}{\protect\numberline{}Exercice 5.4}

\section*{Exercice 5.5 [$\blacklozenge\blacklozenge\lozenge$]}
\begin{tcolorbox}[enhanced, width=7in, center, size=fbox, fontupper=\large, drop shadow southwest]
    Soit $E$ un ensemble et $A,B$ deux parties de $E$. Démontrer que
    \begin{equation*}
        B \subset A \iff (\forall X \in \mathcal{P}(E) \hspace{0.5cm} (A \cap X) \cup B = A \cap (X \cup B)).
    \end{equation*}
    Supposons $B \subset A$.\\
    Soit $X \in \mathcal{P}(E)$.\\
    On a :
    \begin{align*}
        (A \cap X) \cup B &= (A \cup B) \cap (X \cup B) = A \cap (X \cup B)
    \end{align*}
    Supposons $(\forall X \in \mathcal{P}(E) \hspace{0.5cm} (A \cap X) \cup B = A \cap (X \cup B))$.\\
    On a $B \in \mathcal{P}(E)$, donc :
    \begin{align*}
        (A \cap B) \cup B = A \cap (B \cup B) &\iff (A \cup B) \cap B = A \cap B\\
        &\iff (A \cup B) = A\\
        &\iff B \subset A
    \end{align*}
    \qed
\end{tcolorbox}

\addcontentsline{toc}{section}{\protect\numberline{}Exercice 5.5}

\section*{Exercice 5.6 [$\blacklozenge\blacklozenge\blacklozenge$]}
\begin{tcolorbox}[enhanced, width=7in, center, size=fbox, fontupper=\large, drop shadow southwest]
    Expliciter les ensembles
    \begin{equation*}
        A = \bigcap_{n\in\mathbb{N}^*}{\left\lbrack{\frac{1}{n+1},\frac{1}{n}}\right\rbrack} \hspace{0.5cm} \text{et} \hspace{0.5cm} B =\bigcup_{n\in\mathbb{N}^*}{\left\lbrack{\frac{1}{n+1}, \frac{1}{n}}\right\rbrack}.
    \end{equation*}
    A est l'ensemble vide, puisque l'intersection est commutative, on peut prendre $n=1$ et $n=10$, par exemple, et remarquer que leur intersection est nulle, ce qui se propage à toutes les intersections.\\[0.25cm]
    Montrons que B est l'ensemble $\rbrack0,1\rbrack$ par double inclusion.\\
    $\circledcirc$ Montrons que $B \subset \rbrack0,1\rbrack$.\\
    Soit $x \in B$. Il existe $n\in\mathbb{N}^*$ tel que $\frac{1}{n+1}\leq x \leq \frac{1}{n}$. Ainsi, $0 < x \leq 1$. Donc $x\in\rbrack0,1\rbrack$.\\
    $\circledcirc$ Montrons que $\rbrack0,1\rbrack \subset B$.\\
    Soit $x \in \rbrack 0,1 \rbrack$. Il existe $n\in\mathbb{N}^*$ tel que $n+1 \geq \frac{1}{x} \geq n$. Donc que $\frac{1}{n+1} \leq x \leq \frac{1}{n}$.\\
    Ainsi $x \in \left\lbrack\frac{1}{n+1},\frac{1}{n}\right\rbrack$ et donc $x \in B$.\\
    On en conclut que $B=\rbrack0,1\rbrack$. \qed
\end{tcolorbox}

\addcontentsline{toc}{section}{\protect\numberline{}Exercice 5.6}

\end{document}
