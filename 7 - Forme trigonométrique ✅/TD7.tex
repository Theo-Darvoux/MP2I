\documentclass[10pt]{article}

\usepackage[T1]{fontenc}
\usepackage[left=2cm, right=2cm, top=2cm, bottom=2cm]{geometry}
\usepackage[skins]{tcolorbox}
\usepackage{hyperref, fancyhdr, lastpage, tocloft, ragged2e, multicol}
\usepackage{amsmath, amssymb, amsthm, stmaryrd}
\usepackage{tkz-tab}

\def\pagetitle{Forme Trigonométrique}

\title{\bf{\pagetitle}\\\large{Corrigé}}
\date{Octobre 2023}
\author{DARVOUX Théo}

\hypersetup{
    colorlinks=true,
    citecolor=black,
    linktoc=all,
    linkcolor=blue
}

\pagestyle{fancy}
\cfoot{\thepage\ sur \pageref*{LastPage}}


\begin{document}
\renewcommand*\contentsname{Exercices.}
\renewcommand*{\cftsecleader}{\cftdotfill{\cftdotsep}}
\maketitle
\hrule
\tableofcontents
\vspace{0.5cm}
\hrule

\thispagestyle{fancy}
\fancyhead[L]{MP2I Paul Valéry}
\fancyhead[C]{\pagetitle}
\fancyhead[R]{2023-2024}
\allowdisplaybreaks

\begin{center}
    Pour $z\in\mathbb{C}$,\\
    $\Re(z)$ est la partie réelle de $z$.\\
    $\Im(z)$ est la partie imaginaire $z$.
\end{center}
\pagebreak

\section*{Exercice 7.1 [$\blacklozenge\lozenge\lozenge$]}
\begin{tcolorbox}[enhanced, width=7in, center, size=fbox, fontupper=\large, drop shadow southwest]
    Calculer $(1 + i)^2023$.\\
    On a :
    \begin{equation*}
        (1+i)^{2023}=(\sqrt{2}e^{i\frac{\pi}{4}})^{2023}
        = \sqrt{2}^{2023} e^{i\frac{2023\pi}{4}}
        = \sqrt{2}^{2023} e^{-i\frac{\pi}{4}}
    \end{equation*}
    \qed
\end{tcolorbox}

\addcontentsline{toc}{section}{\protect\numberline{}Exercice 7.1}

\section*{Exercice 7.2 [$\blacklozenge\blacklozenge\lozenge$]}
\begin{tcolorbox}[enhanced, width=7in, center, size=fbox, fontupper=\large, drop shadow southwest]
    Soient trois réels $x,y,z$ tels que $e^{ix} + e^{iy} + e^{iz} = 0$. Montrer que $e^{2ix} + e^{2iy} + e^{2iz}=0$.\\
    On a :
    \begin{align*}
        &e^{ix}+e^{iy}+e^{iz}=0\\
        \iff&e^{-ix}+e^{-iy}+e^{-iz}=0
    \end{align*}
    Et :
    \begin{align*}
        &(e^{ix}+e^{iy}+e^{iz})^2 = e^{2ix} + e^{2iy} + e^{2iz} + 2(e^{ixy} + e^{ixz} + e^{iyz})\\
        \iff&e^{2ix}+e^{2iy}+e^{2iz}=-2(e^{ixy}+e^{ixz}+e^{iyz})
    \end{align*}
    Or :
    \begin{align*}
        2(e^{ixy}+e^{ixz}+e^{iyz}) = 2e^{i(x+y+z)}(e^{-ix}+e^{-iy}+e^{-iz})=0
    \end{align*}
    Ainsi,
    \begin{equation*}
        e^{2ix}+e^{2iy}+e^{2iz}=0
    \end{equation*}
    \qed
\end{tcolorbox}

\addcontentsline{toc}{section}{\protect\numberline{}Exercice 7.2}

\section*{Exercice 7.3 [$\blacklozenge\lozenge\lozenge$]}
\begin{tcolorbox}[enhanced, width=7in, center, size=fbox, fontupper=\large, drop shadow southwest]
    1. Déterminer les formes algébriques et trigonométriques du nombre
    \begin{equation*}
        \frac{1+i\sqrt{3}}{2-2i}
    \end{equation*}
    2. En déduire l'expression de $\cos(\frac{7\pi}{12})$ et de $\sin(\frac{7\pi}{12})$ à l'aide de radicaux.\\[0.25cm]
    1. On a :
    \begin{align*}
        \frac{1+i\sqrt{3}}{2-2i}=\frac{1-\sqrt{3}}{4}+i\frac{1+\sqrt{3}}{4}=\frac{1}{\sqrt{2}}\left(\frac{\sqrt{2}(1-\sqrt{3})}{4}+i\frac{\sqrt{2}(1+\sqrt{3})}{4}\right)
    \end{align*}
    2. On a :
    \begin{equation*}
        \begin{cases}
            \cos(\frac{7\pi}{12})=\cos(\frac{\pi}{4}+\frac{\pi}{6})=\frac{\sqrt{6}}{4}-\frac{\sqrt{2}}{4}=\frac{\sqrt{2}(1-\sqrt{3})}{4}\\
            \sin(\frac{7\pi}{12})=\sin(\frac{\pi}{4}+\frac{\pi}{6})=\frac{\sqrt{2}}{4}+\frac{\sqrt{3}}{4}=\frac{\sqrt{2}(1+\sqrt{3})}{4}
        \end{cases}
        \text{Donc : }\frac{1+i\sqrt{3}}{2-2i}=\frac{1}{\sqrt{2}}e^{i\frac{7\pi}{12}}
    \end{equation*}
    \qed
\end{tcolorbox}

\addcontentsline{toc}{section}{\protect\numberline{}Exercice 7.3}

\section*{Exercice 7.4 [$\blacklozenge\lozenge\lozenge$]}
\begin{tcolorbox}[enhanced, width=7in, center, size=fbox, fontupper=\large, drop shadow southwest]
    Soit un réel $\theta$. Linéariser $(\cos\theta)^5$ et $(\sin\theta)^6$.\\
    On a :
    \begin{align*}
        (\cos\theta)^5&=\left(\frac{e^{i\theta}+e^{-i\theta}}{2}\right)^5\\
        &=\frac{1}{32}\left(e^{5i\theta}+5e^{3i\theta}+10e^{i\theta}+10e^{-i\theta}+5e^{-3i\theta}+e^{-5i\theta}\right)\\
        &=\frac{1}{32}\left(2\cos(5\theta)+10\cos(3\theta)+20\cos(\theta)\right)\\
        &=\frac{1}{16}\left(\cos(5\theta)+5\cos(3\theta)+10\cos(\theta)\right)
    \end{align*}
    Et :
    \begin{align*}
        (\sin\theta)^6&=\left(\frac{e^{i\theta}-e^{-i\theta}}{2i}\right)^6\\
        &=-\frac{1}{64}\left(e^{6i\theta}-6e^{4i\theta}+15e^{2i\theta}-20+15e^{-2i\theta}-6e^{-4i\theta}+e^{-6i\theta}\right)\\
        &=-\frac{1}{64}\left(2\cos(6\theta)-12\cos(4\theta)+30\cos(2\theta)-20\right)\\
        &=\frac{1}{32}\left(10 + 6\cos(4\theta) - 15\cos(2\theta) - \cos(6\theta)\right)
    \end{align*}
    \qed
\end{tcolorbox}

\addcontentsline{toc}{section}{\protect\numberline{}Exercice 7.4}

\section*{Exercice 7.5 [$\blacklozenge\blacklozenge\lozenge$]}
\begin{tcolorbox}[enhanced, width=7in, center, size=fbox, fontupper=\large, drop shadow southwest]
    1. Soit $x$ un réel. Exprimer $\cos(5x)$ comme un polynome en $\cos(x)$.\\
    2. Montrer que $\cos^2\left(\frac{\pi}{10}\right)$ est racine du trinôme $x \mapsto 16x^2 - 20x + 5$.\\
    3. En déduire l'égalité $\cos\left(\frac{\pi}{5}\right)=\frac{1+\sqrt{5}}{4}$.\\
    1. On a :
    \begin{align*}
        \cos(5x) &= \text{Re}\left((\cos(x)+i\sin(x))^5\right) \\
        &= \cos^5(x) - 10\cos^3(x)\sin^2(x) + 5\cos(x)\sin^4(x)\\
        &= \cos^5(x) - 10\cos^3(x)(1-\cos^2(x)) + 5\cos(x)(1 - 2\cos^2(x) + \cos^4(x))\\
        &= 16\cos^5(x) - 20\cos^3(x) + 5\cos(x)
    \end{align*}
    2. Posons $x = \cos^2\left( \frac{\pi}{10} \right)$ On a :
    \begin{align*}
        &\cos\left(5\cdot\frac{\pi}{10}\right)=16\cos^5\left(\frac{\pi}{10}\right)-20\cos^3\left( \frac{\pi}{10} \right) + 5\cos\left( \frac{\pi}{10} \right)\\
        \iff&16\cos^4\left( \frac{\pi}{10} \right) - 20\cos^2 \left( \frac{\pi}{10} \right) + 5 = 0\\
        \iff& 16x^2 - 20x + 5 = 0
    \end{align*}
    Ainsi $\cos^2\left( \frac{\pi}{10} \right)$ est racine de ce trinôme.
\end{tcolorbox}

\begin{tcolorbox}[enhanced, width=7in, center, size=fbox, fontupper=\large, drop shadow southwest]
    3. On a :
    \begin{align*}
        \cos^2 \left( \frac{\pi}{10} \right) = \frac{1+\cos(\frac{\pi}{5})}{2}
    \end{align*}
    Soit $x\in\mathbb{R}$, on a :
    \begin{align*}
        &16x^2-20x+5=0\\
        \iff& x = \frac{5 \pm \sqrt{5}}{8}
    \end{align*}
    Ainsi, $\frac{1+\cos\left(\frac{\pi}{5}\right)}{2}=\frac{5+\sqrt{5}}{8}$ car $\cos(\pi/5) > \cos(\pi/3) = 0.5$. On en déduit que $\cos\left( \frac{\pi}{5} \right)=\frac{1+\sqrt{5}}{4}$.
\end{tcolorbox}

\addcontentsline{toc}{section}{\protect\numberline{}Exercice 7.5}

\section*{Exercice 7.6 [$\blacklozenge\blacklozenge\lozenge$]}
\begin{tcolorbox}[enhanced, width=7in, center, size=fbox, fontupper=\large, drop shadow southwest]
    Soit $x\in\mathbb{R}$ et $n\in\mathbb{N}$. Calculer $S=\sum\limits_{k=0}^{n}{\binom{n}{k}\cos(kx)}$\\
    Notons : $S'=\sum\limits_{k=0}^{n}{\binom{n}{k}\sin(kx)}$
    On a :
    \begin{align*}
        S + iS' &= \sum_{k=0}^{n}{\binom{n}{k}\left(e^{ix}\right)^k}\\
        &= (1 + e^{ix})^n\\
        &= \left( e^{\frac{ix}{2}} \right)^n\left( e^{-\frac{ix}{2}} + e^{\frac{ix}{2}}\right)^n\\
        &= e^{\frac{inx}{2}}2^n\cos^n\left(\frac{x}{2}\right)
    \end{align*}
    Donc $S = \Re\left( 2^n\cos^n\left( \frac{x}{2} \right)e^{\frac{inx}{2}} \right) = 2^n\cos^n\left( \frac{x}{2} \right)\Re\left( e^{\frac{inx}{2}} \right)$.\\
    Or, on a :
    \begin{equation*}
        \Re\left( e^{\frac{inx}{2}} \right) = \Re\left( \cos\left(\frac{nx}{2}\right) + i\sin\left(\frac{nx}{2}\right) \right) = \cos\left( \frac{nx}{2} \right)
    \end{equation*}
    En conclusion :
    \begin{equation*}
        S = 2^n \cos\left( \frac{x}{2} \right)\cos\left(\frac{nx}{2}\right)
    \end{equation*}
    \qed
\end{tcolorbox}

\addcontentsline{toc}{section}{\protect\numberline{}Exercice 7.6}

\section*{Exercice 7.7 [$\blacklozenge\blacklozenge\lozenge$] Noyaux de Dirichlet et de Féjer.}
\begin{tcolorbox}[enhanced, width=7in, center, size=fbox, fontupper=\large, drop shadow southwest]
    Soit $n\in\mathbb{N}^*$ et $x\in\mathbb{R} \setminus \left\{ 2k\pi, k\in\mathbb{Z} \right\}$. On note
    \begin{equation*}
        D_n(x) = \sum_{k=-n}^{n}{e^{ikx}} \hspace{1.25cm} \text{et} \hspace{1.25cm} F_n(x) = \frac{1}{n}\sum_{k=0}^{n-1}{D_k(x)}.
    \end{equation*}
    1. Montrer que $D_n(x)=\frac{\sin\left( (n+\frac{1}{2})x \right)}{\sin\frac{x}{2}}$.\\
    2. Montrer que $F_n(x)=\frac{1}{n}\left( \frac{\sin\left( \frac{nx}{2} \right)}{\sin\frac{x}{2}} \right)^2$.\\
    1. 
    \begin{align*}
        \sum_{k=-n}^{n}{e^{ikx}}&=\sum_{k=-n}^{n}{\left(e^{ix}\right)^k}\\
        &=e^{-nx}\frac{1-e^{ix(2n+1)}}{1-e^{ix}}\\
        &=\frac{e^{-inx}-e^{ix(n+1)}}{1-e^{ix}}\\
        &=\frac{e^{ix/2}(e^{-ix(n+1/2)}-e^{ix(n+1/2)})}{e^{ix/2}(e^{-ix/2}-e^{ix/2})}\\
        &=\frac{-2i\sin(x(n+\frac{1}{2}))}{-2i\sin(\frac{x}{2})}\\
        &=\frac{\sin((n+\frac{1}{2})x)}{\sin\frac{x}{2}}\\
    \end{align*}
    2.
    \begin{align*}
        \frac{1}{n}\sum_{k=0}^{n-1}{D_k(x)} &= \frac{1}{n}\sum_{k=0}^{n-1}{\frac{\sin((k+\frac{1}{2})x)}{\sin\frac{x}{2}}}=\frac{1}{n\sin\frac{x}{2}}\sum_{k=0}^{n-1}{\sin((k+\frac{1}{2})x)}
    \end{align*}
    Calculons la somme des $\sin((k+1/2)x)$.
    \begin{align*}
        \sum^{n-1}_{k=0}{\sin((k+\frac{1}{2})x)}&=\Im\left( \sum_{k=0}^{n-1}{e^{ix(k+\frac{1}{2})}} \right) = \Im\left( e^{ix\frac{1}{2}}\sum_{k=0}^{n-1}{e^{ixk}} \right) = \Im\left( e^{ix\frac{1}{2}}\frac{1-e^{inx}}{1-e^{ix}} \right)\\
        &= \Im\left( e^{i\frac{x}{2}}\frac{e^{i\frac{nx}{2}}(\sin(\frac{nx}{2}))}{e^{i\frac{x}{2}}(\sin(\frac{x}{2}))} \right) = \Im\left(e^{i\frac{nx}{2}}\frac{\sin(\frac{nx}{2})}{\sin(\frac{x}{2})}\right)=\frac{\sin^2(\frac{nx}{2})}{\sin(\frac{x}{2})}
    \end{align*}
    Donc :
    \begin{align*}
        F_n(x)=\frac{1}{n\sin\frac{x}{2}}\cdot\frac{\sin^2(\frac{nx}{2})}{\sin\frac{x}{2}}=\frac{\sin^2\frac{nx}{2}}{n\sin^2\frac{x}{2}}=\frac{1}{n}\left(\frac{\sin\frac{nx}{2}}{\sin\frac{x}{2}}\right)^2
    \end{align*}
    \qed
\end{tcolorbox}

\addcontentsline{toc}{section}{\protect\numberline{}Exercice 7.7}

\section*{Exercice 7.8 [$\blacklozenge\blacklozenge\blacklozenge$]}
\begin{tcolorbox}[enhanced, width=7in, center, size=fbox, fontupper=\large, drop shadow southwest]
    Soit un quadrilatère $ABCD$ du plan. On construit les points $E, F, G, H$ à l'extérieur du quadrilatère tels que les triangles $EBA$, $FCB$, $GDC$ et $HAD$ soient des triangles directs, isocèles et rectangles en $E, F, G, H$.\\
    Démontrer que
    \begin{equation*}
        \overrightarrow{EG} \perp \overrightarrow{FH} \hspace{1cm} \text{et} \hspace{1cm} EG = FH.
    \end{equation*}
    \begin{center}
        \includegraphics*[scale=0.17]{./goat.png}
    \end{center}
\end{tcolorbox}

\addcontentsline{toc}{section}{\protect\numberline{}Exercice 7.8}

\section*{Exercice 7.9 [$\blacklozenge\blacklozenge\blacklozenge$]}
\begin{tcolorbox}[enhanced, width=7in, center, size=fbox, fontupper=\large, drop shadow southwest]
    Trouver les nombres complexes d'affixe $z\in\mathbb{C}$ tels que $1,z^2$ et $z^4$ sont alignés.\\
    C'est évident lorsque $z\in\{0,1\}$. Supposons $z\notin\{0,1\}$.\\
    Soit $r\in\mathbb{R_+^*}$ et $\theta\in\mathbb{R}$ tels que $z=re^{i\theta}$.\\ 
    On a :
    \begin{align*}
        &1, z^2, z^4 \text{ alignés } \\
        \iff&\frac{z^4-1}{z^2-1} \in \mathbb{R}\\
        \iff&\frac{(z^2-1)(z^2+1)}{z^2-1}\in\mathbb{R}\\
        \iff&z^2+1 \in \mathbb{R}\\
        \iff&z^2 \in \mathbb{R}\\
        \iff&r^2e^{2i\theta} = r^2e^{-2i\theta}\\
        \iff&e^{2i\theta} = e^{-2i\theta}\\
        \iff&e^{4i\theta}=1\\
        \iff&\theta=\frac{n\pi}{2}, n\in\mathbb{N}\\
    \end{align*}

    Donc $z\in\mathbb{R}\cup i\mathbb{R}$.\\
    \qed
\end{tcolorbox}

\addcontentsline{toc}{section}{\protect\numberline{}Exercice 7.9}

\end{document}
 