\documentclass[10pt]{article}

\usepackage[T1]{fontenc}
\usepackage[left=2cm, right=2cm, top=2cm, bottom=2cm]{geometry}
\usepackage[skins]{tcolorbox}
\usepackage{hyperref, fancyhdr, lastpage, tocloft, ragged2e}
\usepackage{amsmath, amssymb, amsthm}

\def\pagetitle{Logique}

\title{\bf{\pagetitle}\\\large{Corrigé}}
\date{Septembre 2023}
\author{DARVOUX Théo}

\hypersetup{
    colorlinks=true,
    citecolor=black,
    linktoc=all,
    linkcolor=blue
}

\pagestyle{fancy}
\cfoot{\thepage\ sur \pageref*{LastPage}}


\begin{document}
\renewcommand*\contentsname{Exercices.}
\renewcommand*{\cftsecleader}{\cftdotfill{\cftdotsep}}
\maketitle
\hrule
\tableofcontents
\vspace{0.5cm}
\hrule

\thispagestyle{fancy}
\fancyhead[L]{MP2I Paul Valéry}
\fancyhead[C]{\pagetitle}
\fancyhead[R]{2023-2024}



\section*{Exercice 0.1 [$\blacklozenge\lozenge\lozenge$]}
\begin{tcolorbox}[enhanced, width=7in, center, size=fbox, fontupper=\large, drop shadow southwest]
    Que répond un mathématicien à la question <<Vous êtes gaucher ou droitier ?>>.\\
    - << OUI >> \qed
\end{tcolorbox}
\addcontentsline{toc}{section}{\protect\numberline{}Exercice 0.1}

\section*{Exercice 0.2 [$\blacklozenge\lozenge\lozenge$]}
\begin{tcolorbox}[enhanced, width=7in, center, size=fbox, fontupper=\large, drop shadow southwest]
    C'est l'histoire de cinq mathématiciens qui vont au restaurant. Ils ont pris un menu avec thé ou (bien) café compris. À la fin du repas, le serveur demande : \textbf{tout le monde} prendra du café ? Le premier mathématicien répond <<Je ne sais pas>>. Idem pour le second, le troisième, et le quatrième. Le cinquième répond <<Quatre cafés et un thé SVP>>. Expliquer.\\[0.5cm]
    Le premier, deuxième, troisième et quatrième veulent un café car s'ils voulaient un thé, ils auraient répondu <<Non>>. Le cinquième sait alors qu'ils veulent tous les quatres un café et choisit un thé pour lui-même. 
    \qed
\end{tcolorbox}
\addcontentsline{toc}{section}{\protect\numberline{}Exercice 0.2}

\section*{Exercice 0.3 [$\blacklozenge\lozenge\lozenge$]}
\begin{tcolorbox}[enhanced, width=7in, center, size=fbox, fontupper=\large, drop shadow southwest]
    Soit $(u_n)_{n\in\mathbb{N}}$ une suite de réels.\\
    Plus tard dans l'année, nous définirons la phrase <<$u_n\underset{n\rightarrow+\infty}{\longrightarrow}+\infty$>> par l'assertion :
    \begin{equation*}
        \forall M\in\mathbb{R}\hspace{0.5cm}\exists p\in\mathbb{N}\hspace{0.5cm} \forall{n}\in\mathbb{N}\hspace{0.5cm}n\geq p\Longrightarrow u_n\geq M.
    \end{equation*}
    Écrire la négation de cette assertion.
    \begin{equation*}
        \exists{M\in\mathbb{R}}\hspace{0.5cm}\forall{p\in\mathbb{N}}\hspace{0.5cm}\exists{n\in\mathbb{N}}\hspace{0.5cm}\left(n\geq{p}\hspace{0.25cm}\text{et}\hspace{0.25cm}u_n<M\right)
    \end{equation*}
    \qed
\end{tcolorbox}
\addcontentsline{toc}{section}{\protect\numberline{}Exercice 0.3}

\section*{Exercice 0.4 [$\blacklozenge\lozenge\lozenge$]}
\begin{tcolorbox}[enhanced, width=7in, center, size=fbox, fontupper=\large, drop shadow southwest]
    1. Nier l'assertion :
    \begin{equation}
        \exists{x\in\mathbb{R}}\hspace{0.5cm}\forall{y\in\mathbb{R}}\hspace{0.5cm}x+y\geq0
    \end{equation}
    Sa négation est :
    \begin{equation*}
        \forall{x\in\mathbb{R}}\hspace{0.5cm}\exists{y\in\mathbb{R}}\hspace{0.5cm}x+y<0
    \end{equation*}
    Prouver que l'assertion (1) est fausse.\\
    Soit $x\in\mathbb{R}$. Posons $y=-(x+1)$.\\
    On a $x+y=-1$ c'est pourquoi $x+y<0$. Ainsi, la négation de (1) est toujours vraie.\\
    On en déduit que (1) est toujours fausse.\\
    \qed
\end{tcolorbox}
\addcontentsline{toc}{section}{\protect\numberline{}Exercice 0.4}

\section*{Exercice 0.5 [$\blacklozenge\lozenge\lozenge$]}
\begin{tcolorbox}[enhanced, width=7in, center, size=fbox, fontupper=\large, drop shadow southwest]
    <<S'il fait beau, je ne prends pas mon parapluie.>>\\
    Écrire la contraposée de la réciproque.\\[0.5cm]
    Sa réciproque est :\\
    <<Si je ne prends pas mon parapluie, il fait beau.>>\\[0.5cm]
    La contraposée de cette réciproque est :\\
    <<S'il ne fait pas beau, je prends mon parapluie.>> \qed
\end{tcolorbox}
\addcontentsline{toc}{section}{\protect\numberline{}Exercice 0.5}

\section*{Exercice 0.6 [$\blacklozenge\blacklozenge\lozenge$]}
\begin{tcolorbox}[enhanced, width=7in, center, size=fbox, fontupper=\large, drop shadow southwest]
    Soit $(u_n)$ une suite réelle. Démontrer l'équivalence des deux assertions :
    \begin{itemize}
        \item[1.] $P:\forall{n\in\mathbb{N}}$ \hspace{0.25cm} $u_{n+1} \geq u_n$. 
        \item[2.] $Q:\forall{(n,p)\in\mathbb{N}^2}$ \hspace{0.25cm} $p \geq n \Longrightarrow u_p \geq u_n$. 
    \end{itemize}
    C'est l'exercice 13.1\\
    \qed
    \end{tcolorbox}
\addcontentsline{toc}{section}{\protect\numberline{}Exercice 0.6}

\section*{Exercice 0.7 [$\blacklozenge\blacklozenge\lozenge$] [Récurrence Standard]}
\begin{tcolorbox}[enhanced, width=7in, center, size=fbox, fontupper=\large, drop shadow southwest]
    Déterminer les entiers naturels $n$ tels que $2^n\geq n^2$.\\
    Soit $\mathcal{P}_n$ la proposition : <<$2^n\geq n^2$>>.\\
    Montrons que $\mathcal{P}_n$ est vraie pour tout entier naturel $n$ supérieur à 4.\\[0.5cm]
    \emph{Initialisation.}\\
    Posons $n=4$.\\
    $2^4 \geq 4^2$. Ainsi, $\mathcal{P}_4$ est vérifiée.\\[0.5cm]
    \emph{Hérédité.}\\
    Soit $n$ un entier naturel supérieur à 4 fixé tel que $\mathcal{P}_n$ soit vraie. Montrons que $\mathcal{P}_{n+1}$ est vraie.\\
    On a :
    \begin{equation*}
        2^n \geq n^2 \iff 2^{n+1} \geq 2n^2
    \end{equation*}
    Or, pour tout n, on a : $n^2 \geq 2n + 1$ (polynome du second degré)\\
    Ainsi,
    \begin{equation*}
        2^n \geq n^2 \iff 2^{n+1} \geq n^2 + 2n + 1 \iff 2^{n+1} \geq (n+1)^2
    \end{equation*}
    C'est exactement $\mathcal{P}_{n+1}$.\\[0.5cm]
    \emph{Conclusion.}\\
    Par le principe du raisonnement par récurrence, $\mathcal{P}_n$ est vraie pour tout $n\in\mathbb{N} : n\geq4$.\\
    De plus, on peut vérifier facilement que $\mathcal{P}_n$ est vraie pour $n\in\{0,1,2\}$.\\
    Ainsi, $\mathcal{P}_n$ est vraie pour tout entier naturel $n$ différent de 3.\\
    \qed
\end{tcolorbox}
\addcontentsline{toc}{section}{\protect\numberline{}Exercice 0.7}

\section*{Exercice 0.8 [$\blacklozenge\blacklozenge\lozenge$] [Récurrence Double]}
\begin{tcolorbox}[enhanced, width=7in, center, size=fbox, fontupper=\large, drop shadow southwest]
    Soit $x$ un réel non nul.
    \begin{itemize}
        \item[1.] Pour $n$ un entier naturel, calculer $(x^n+\frac{1}{x^n})\cdot(x+\frac{1}{x})$
        \item[2.] Supposons que $x+\frac{1}{x}\in\mathbb{Z}$. Démontrer que $\forall n\in\mathbb{N}$ $x^n + \frac{1}{x}\in\mathbb{Z}$ 
    \end{itemize}
    1.
    \begin{equation*}
        (x^n + \frac{1}{x^n})\cdot(x+\frac{1}{x}) = x^{n+1}+x^{n-1}+x^{1-n}+x^{-n-1}
    \end{equation*}
    2.\\
    Soit $\mathcal{P}_n$ la proposition : <<$x^n+\frac{1}{x^n}\in\mathbb{Z}$>>.\\
    Montrons que $\mathcal{P}_n$ est vraie pour tout $n\in\mathbb{N}$.\\
    \emph{Initialisation.}\\
    Pour $n=0$.\\
    $x^0 + \frac{1}{x^0} = 1 + \frac{1}{1} = 2$. Or $2\in\mathbb{Z}$. Mézalor, $\mathcal{P}_0$ est vérifiée.\\
    Pour $n=1$.\\
    $x^1 + \frac{1}{x^1} = x + \frac{1}{x}$. Or on a supposé que ceci appartenait à $\mathbb{Z}$. Conséquemment, $\mathcal{P}_1$ est vérifiée.\\[0.5cm]
    \emph{Hérédité.}\\
    Soit $n\in\mathbb{N}$ tel que $\mathcal{P}_n$ et $\mathcal{P}_{n+1}$ soient vérifiées. Montrons $\mathcal{P}_{n+2}$.\\
    On a:
    \begin{equation*}
        (x^{n+1}+\frac{1}{x^{n+1}})\cdot(x+\frac{1}{x})=x^{n+2}+\frac{1}{x^{n+2}}+x^n+\frac{1}{x^{n}}
    \end{equation*}
    D'où :
    \begin{equation*}
        (x^{n+1}+\frac{1}{x^{n+1}})\cdot(x+\frac{1}{x})-(x^n+\frac{1}{x^n}) = x^{n+2}+\frac{1}{x^{n+2}}
    \end{equation*}
    Or, par hypothèse de récurrence, $x^{n+1}+\frac{1}{x^{n+1}}\in\mathbb{Z}$, ainsi que $x^n+\frac{1}{x^n}\in\mathbb{Z}$.\\
    Enfin, par stabilité de $\mathbb{Z}$ en somme et en produit, on obtient que :
    \begin{equation*}
        (x^{n+1}+\frac{1}{x^{n+1}})\cdot(x+\frac{1}{x})-(x^n+\frac{1}{x^n})\in\mathbb{Z}
    \end{equation*}
    Alors :
    \begin{align*}
        x^{n+2} + \frac{1}{x^{n+2}} \in \mathbb{Z}
    \end{align*}
    C'est exactement $\mathcal{P}_{n+2}$.\\
    \emph{Conclusion.}\\
    Par le principe de récurrence double, $\mathcal{P}_n$ est vraie pour tout $n\in\mathbb{N}$.\\
    \qed
\end{tcolorbox}
\addcontentsline{toc}{section}{\protect\numberline{}Exercice 0.8}

\section*{Exercice 0.9 [$\blacklozenge\lozenge\lozenge$] [Récurrence Forte]}
\begin{tcolorbox}[enhanced, width=7in, center, size=fbox, fontupper=\large, drop shadow southwest]
    Soit $(u_n)$, définie par récurrence par
    \begin{equation*}
        \begin{cases}
            u_1 = 3\\
            \forall{n\geq1} \hspace{0.25cm} u_{n+1}= \frac{2}{n}\sum\limits^{n}_{k=1}{u_k}
        \end{cases}
    \end{equation*}
    Démontrer par récurrence forte que $\forall{n\geq1}$ $u_n=3n$.\\
    Soit $\mathcal{P}_n$ la proposition : <<$\forall{n\geq1}$, $u_n=3n$>>.\\
    \emph{Initialisation.}\\
    Initialisation triviale pour $n=1$.\\
    \emph{Hérédité.}\\
    Soit $n\geq1$ tel que $\forall k\in[1,n], \hspace{0.25cm} \mathcal{P}_k$ soit vraie. Montrons que $\mathcal{P}_{n+1}$ est vraie.\\
    On a :
    \begin{align*}
        u_{n+1} 
        &= \frac{2}{n}\sum^{n}_{k=1}{u_k}\\
        &\stackrel{HR}{=} \frac{2}{n}\sum^{n}_{k=1}{3k}\\
        &= \frac{6}{n}\sum^{n}_{k=1}{k}\\
        &= \frac{3n(n+1)}{n}\\
        &= 3(n+1)
    \end{align*}
    C'est exactement $\mathcal{P}_{n+1}$.\\
    \emph{Conclusion.}\\
    Par le principe de récurrence forte, $\mathcal{P}_n$ est vraie pour tout $n\geq1$.\\
    \qed
\end{tcolorbox}
\addcontentsline{toc}{section}{\protect\numberline{}Exercice 0.9}

\section*{Exercice 0.10 [$\blacklozenge\blacklozenge\lozenge$]}
\begin{tcolorbox}[enhanced, width=7in, center, size=fbox, fontupper=\large, drop shadow southwest]
    Comme on l'a fait plus haut pour le principe de récurrence, écrire un <<principe de récurrence forte>> à l'aide d'une suite de quantificateurs.\\
    Soit une assertion dont le sens dépend d'un entier $n$, et que l'on note $\mathcal{P}_n$.
    \begin{equation*}
        [\forall{n\in\mathbb{N} \hspace{0.25cm}}(\forall{k\in[0, n] \hspace{0.25cm} \mathcal{P}_k}) \Rightarrow \mathcal{P}_{n+1}] \Longrightarrow (\forall{n\in\mathbb{N}} \hspace{0.25cm} \mathcal{P}_n)
    \end{equation*}
    \qed
\end{tcolorbox}
\addcontentsline{toc}{section}{\protect\numberline{}Exercice 0.10}

\section*{Exercice 0.11 [$\blacklozenge\blacklozenge\lozenge$]}
\begin{tcolorbox}[enhanced, width=7in, center, size=fbox, fontupper=\large, drop shadow southwest]
    Déterminer les fonctions $f: \mathbb{R} \rightarrow \mathbb{R}$ telles que 
    \begin{equation*}
        \forall{x\in\mathbb{R},\forall{y\in\mathbb{R}}, \hspace{0.5cm} f(x)f(y) - f(xy) = x + y}
    \end{equation*}
    \emph{Analyse.}\\
    Supposons qu'il existe $f$ une telle fonction. Soit $(x,y)\in\mathbb{R}^2$.
    \begin{itemize}
        \item[1.] Lorsque $x=y=0$, $f^2(0)-f(0)=0$. Donc $f(0) = 0$ ou $f(0) = 1$.
        \item[2.] Lorsque $y=0$, $f(x)f(0)-f(0)=x$. Ainsi, la possibilité $f(0)=0$ est éliminée.
    \end{itemize}
    Ainsi, $f(x)f(0)-f(0)=x$. Donc $f(x)=x+1$.\\
    Alors $f$ est solution si elle existe.\\[0.25cm]
    \emph{Synthèse.}\\
    Soient $(x,y)\in\mathbb{R}^2$.\\
    Posons $g: \begin{cases}\mathbb{R}\rightarrow\mathbb{R}\\x\mapsto{x+1}\end{cases}$.\\
    On a :
    \begin{align*}
        g(x)g(y)-g(xy)
        &=(x+1)(y+1)-(xy+1)\\
        &=xy+x+y+1-xy-1\\
        &=x+y
    \end{align*}
    \emph{Conclusion:}
    La fonction $x\mapsto x+1$ est alors solution unique.\\
    \qed
\end{tcolorbox}
\addcontentsline{toc}{section}{\protect\numberline{}Exercice 0.11}

\section*{Exercice 0.12 [$\blacklozenge\blacklozenge\lozenge$]}
\begin{tcolorbox}[enhanced, width=7in, center, size=fbox, fontupper=\large, drop shadow southwest]
    Déterminer les fonctions $f: \mathbb{R} \rightarrow \mathbb{R}$ telles que 
    \begin{equation*}
        \forall{x\in\mathbb{R}},\forall{y\in\mathbb{R}}, \hspace{0.25cm} f(x-f(y)) = 1-x-y
    \end{equation*}
    \emph{Analyse.}\\
    Supposons qu'il existe $f$ une telle fonction. Soit $(x,y)\in\mathbb{R}^2$.
    \begin{itemize}
        \item[1.] Lorsque $x=f(0), y=0$, $f(0)=1-f(0)$. Ainsi, $2f(0)=1$, donc $f(0)=\frac{1}{2}$.
        \item[2.] Lorsque $y=0$, $f(x-\frac{1}{2})=1-x$.
    \end{itemize}
    Posons $z=x-\frac{1}{2}$.\\
    On a $f(z)=1-z-\frac{1}{2} = \frac{1}{2}-z$.\\
    Donc $f(x)=\frac{1}{2}-x$ \emph{(car $z\in\mathbb{R}$)}.
    Alors $f$ est solution si elle existe.\\[0.25cm]
    \emph{Synthèse.}\\
    Soient $(x,y)\in\mathbb{R}^2$.\\
    Posons $g:\begin{cases}\mathbb{R}\rightarrow\mathbb{R}\\x\mapsto{\frac{1}{2}-x}\end{cases}$.\\
    On a :
    \begin{equation*}
        g(x-g(y))=\frac{1}{2}-(x - (\frac{1}{2}-y)) = 1 - x - y
    \end{equation*}
    \emph{Conclusion:} 
    La fonction $x\mapsto\frac{1}{2}-x$ est alors solution unique. \qed
\end{tcolorbox}
\addcontentsline{toc}{section}{\protect\numberline{}Exercice 0.12}

\section*{Exercice 0.13 [$\blacklozenge\lozenge\lozenge$]}
\begin{tcolorbox}[enhanced, width=7in, center, size=fbox, fontupper=\large, drop shadow southwest]
    Déterminer les fonctions $f: \mathbb{R}_+^* \rightarrow \mathbb{R}_+^*$ telles que 
    \begin{equation*}
        \forall{x,y\in\mathbb{R}^*_+}, \hspace{0.25cm} xf(xy) = f(y).
    \end{equation*}
    \emph{Analyse.}\\
    Supposons qu'il existe $f$ une telle fonction. Soient $x,y\in\mathbb{R}^*_+$
    \begin{itemize}
        \item[1.] Lorsque $y=1$, $f(1)=xf(x)$.
        \item[2.] Lorsque $x=\frac{1}{y}$, $f(1)=yf(y)$
    \end{itemize}
    Alors $xf(x)=yf(y)$, donc $f(x)=\frac{yf(y)}{x}=\frac{f(1)}{x}$.\\
    En posant $a=f(1)$, $f(x)=\frac{a}{x}$.
    Donc $f$ est solution si elle existe.\\[0.2cm]
    \emph{Synthèse.}\\
    Soient $x,y\in\mathbb{R}^*_+$.\\
    Soit $a\in\mathbb{R^*_+}$, posons $g:\begin{cases}\mathbb{R}^*_+\rightarrow\mathbb{R}^*_+\\x\mapsto\frac{a}{x}\end{cases}$.\\
    On a :
    \begin{equation*}
        xg(xy)=x\frac{a}{xy}=\frac{a}{y}=g(y).
    \end{equation*}
    \emph{Conclusion:}
    Les solutions sont donc les fonctions de la forme $x\mapsto\frac{a}{x}$ avec $a\in\mathbb{R^*_+}$.\\
    \qed
\end{tcolorbox}
\addcontentsline{toc}{section}{\protect\numberline{}Exercice 0.13}

\section*{Exercice 0.14 [$\blacklozenge\blacklozenge\blacklozenge$]}
\begin{tcolorbox}[enhanced, width=7in, center, size=fbox, fontupper=\large, drop shadow southwest]
    Déterminer les fonctions $f: \mathbb{R} \rightarrow \mathbb{R}$ telles que 
    \begin{equation*}
        \forall{(x,y)\in\mathbb{R}}, \hspace{0.25cm} f(f(x)+y) = f(x^2-y)+4x^2y.
    \end{equation*}
    \emph{Analyse.}
    Supposons qu'il existe $f$ une telle fonction. Soit $(x,y)\in\mathbb{R}^2$. 
    \begin{itemize}
        \item[(1)] Lorsque $y=x^2$, $f(f(x)+x^2)=f(0)+4x^4$
        \item[(2)] Lorsque $y=-f(x)$, $f(0)=f(x^2+f(x))-4x^2f(x)$ 
        \item[(1)] + (2) : $4x^2(x^2 - f(x)) = 0$.
    \end{itemize}
    Ainsi, lorsque $x\neq0$, on a: $x^2-f(x) = 0$, c'est-à-dire $f(x)=x^2$.\\
    Donc $f$ est solution si elle existe.\\[0.2cm]
    \emph{Synthèse.}\\
    Soient $x,y\in\mathbb{R}$.\\
    Posons $g:\begin{cases}\mathbb{R}\rightarrow\mathbb{R}\\x\mapsto x^2\end{cases}$\\
    On a :
    \begin{equation*}
        (x^2+y)^2 = x^4 + 2x^2y + y^2 = (x^4-2x^2y+y^2)+4x^2y=(x^2-y)^2+4x^2y
    \end{equation*}
    \emph{Conclusion:}
    La fonction $x\mapsto x^2$ est donc solution unique.\\
    \qed
\end{tcolorbox}
\addcontentsline{toc}{section}{\protect\numberline{}Exercice 0.14}

\section*{Exercice 0.15 [$\blacklozenge\blacklozenge\lozenge$]}
\begin{tcolorbox}[enhanced, width=7in, center, size=fbox, fontupper=\large, drop shadow southwest]
    Résoudre l'équation
    \begin{equation*}
        x^2 + x\sqrt{1-x^2}-1=0
    \end{equation*}
    \emph{Analyse.}
    Supposons qu'il existe $x\in\mathbb{R}$ vérifiant cette équation.
    \begin{align*}
        x^2-1=-x\sqrt{1-x^2} 
        &\Rightarrow (x^2-1)^2=(-x\sqrt{1-x^2})^2\\
        &\Rightarrow x^4 - 2x^2 + 1 = -x^4+x^2\\
        &\Rightarrow 2x^4-3x^2+1=0
    \end{align*}
    Posons $\omega=x^2$. On a alors $2\omega^2-3\omega+1=0$.\\
    Les racines de ce polynome sont : $\omega_1=\frac{1}{2}$ et $\omega_1=1$.\\
    Ainsi, si $x$ existe alors $x\in\{-\frac{\sqrt{2}}{2},-1,1,\frac{\sqrt{2}}{2}\}$.\\[0.25cm]
    \emph{Synthèse.}\\
    On remarque que l'équation est vérifiée lorsque $x\in\{-1,1,\frac{\sqrt{2}}{2}\}$ uniquement.\\[0.25cm]
    \emph{Conclusion.}\\
    L'équation admet donc pour ensemble de solutions : $\{-1,1,\frac{\sqrt{2}}{2}\}$.\\
    \qed
\end{tcolorbox}
\addcontentsline{toc}{section}{\protect\numberline{}Exercice 0.15}

\section*{Exercice 0.16 [$\blacklozenge\blacklozenge\blacklozenge$]}
\begin{tcolorbox}[enhanced, width=7in, center, size=fbox, fontupper=\large, drop shadow southwest]
    Soit $E$ l'ensemble des fonctions dérivables sur $\mathbb{R}$.\\
    On considère $F$ l'ensemble des fonctions affines, et $G$ l'ensemble des fonctions $g$ dérivables sur $\mathbb{R}$ telles que $g(0)=g'(0)=0$.\\
    Démontrer que toute fonction de $E$ s'écrit de manière unique comme la somme d'une fonction de $F$ et d'une fonction de $G$.\\[0.5cm]
    Soit $h$ une fonction dérivable sur $\mathbb{R}$.\\
    \emph{Analyse.}\\
    On suppose qu'il existe $f\in F$ et $g\in G$ telles que :
    \begin{equation*}
        \forall{x\in\mathbb{R}}, \hspace{0.25cm} h(x)=f(x)+g(x)
    \end{equation*}
    On a :
    \begin{itemize}
        \item $f\in F$ donc $\exists(a,b)\in\mathbb{R}^2$ | $\forall{x\in\mathbb{R}}$, $f(x)=ax+b$.
        \item $f$ est dérivable sur $\mathbb{R}$ et $\forall{x\in\mathbb{R}}$, $f'(x)=a$.
        \item $g\in G$ donc $g(0)=g'(0)=0$.  
        \item $\forall{x\in\mathbb{R}}$, $h(x)=f(x)+g(x)$ donc $h'(x)=f'(x)+g'(x)=a+g'(x)$.
        \item $h(0)=f(0)+g(0)=b$ et $h'(0)=f'(0)+g'(0)=a$
    \end{itemize}
    Ainsi, si $f$ existe, $\forall{x\in\mathbb{R}}, f(x) = h'(0)x+h(0)$.\\
    De plus, si $g$ existe, $\forall{x\in\mathbb{R}}, g(x) = h(x)-f(x)$.\\[0.5cm]
    \emph{Synthèse.}\\
    Soient $f$ et $g$ deux fonctions définies comme :
    \begin{equation*}
        f:\begin{cases}\mathbb{R}\rightarrow\mathbb{R}\\x\mapsto h'(0)x+h(0)\end{cases} g:\begin{cases}\mathbb{R}\rightarrow\mathbb{R}\\x\mapsto h(x)-h'(0)x-h(0)\end{cases}
    \end{equation*}
    \begin{itemize}
        \item $f$ est une fonction affine de coefficient $h'(0)$ et d'ordonnée à l'origine $h(0)$ donc $f\in F$.
        \item $g$ est dérivable sur $\mathbb{R}$, et $\forall{x\in\mathbb{R}}$, $g'(x)=h'(x)-h'(0)$.
        \item $g(0)=0$ et $g'(0)=0$. Ainsi, $g\in G$.
    \end{itemize}
    Soit $x\in\mathbb{R}$.
    \begin{align*}
        f(x) + g(x) 
        &= h'(0)x + h(0) + h(x) - h'(0)x - h(0)\\
        &= h(x)
    \end{align*}
    \emph{Conclusion.}\\
    Toute fonction dérivable dans $\mathbb{R}$ s'écrit de manière unique comme la somme d'une fonction affine $f:x\mapsto h'(0)x+h(0)$ et d'une fonction $g:x\mapsto h(x)-h'(0)x-h(0)$ dérivable sur $\mathbb{R}$ tel que $g(0)=g'(0)=0$.\\
    \qed
\end{tcolorbox}
\addcontentsline{toc}{section}{\protect\numberline{}Exercice 0.16}

\end{document}
