il\documentclass[10pt]{article}

\usepackage[T1]{fontenc}
\usepackage[left=2cm, right=2cm, top=2cm, bottom=2cm, paperheight=31cm]{geometry}
\usepackage[skins]{tcolorbox}
\usepackage{hyperref, fancyhdr, lastpage, tocloft, ragged2e, multicol, changepage}
\usepackage{amsmath, amssymb, amsthm, stmaryrd}
\usepackage{tkz-tab}
\usepackage{systeme}
\def\pagetitle{Applications}
\setlength{\headheight}{13pt}

\title{\bf{\pagetitle}\\\large{Corrigé}}
\date{Décembre 2023}
\author{DARVOUX Théo}

\DeclareMathOperator{\ch}{ch}

\hypersetup{
    colorlinks=true,
    citecolor=black,
    linktoc=all,
    linkcolor=blue
}

\pagestyle{fancy}
\cfoot{\thepage\ sur \pageref*{LastPage}}

\begin{document}
\renewcommand*\contentsname{Exercices.}
\renewcommand*{\cftsecleader}{\cftdotfill{\cftdotsep}}
\maketitle

\hrule
\tableofcontents
\vspace{0.5cm}
\hrule

\thispagestyle{fancy}
\fancyhead[L]{MP2I Paul Valéry}
\fancyhead[C]{\pagetitle}
\fancyhead[R]{2023-2024}
\allowdisplaybreaks

\pagebreak


\section*{Exercice 15.1 [$\blacklozenge\lozenge\lozenge$]}
\begin{tcolorbox}[enhanced, width=7.6in, center, size=fbox, fontupper=\large, drop shadow southwest]
    Soit $f:E\to F$ une application. Soient deux parties $A \subset E$ et $B \subset F$. Montrer l'égalité
    \begin{equation*}
        f(A) \cap B = f(A \cap f^{-1}(B)).
    \end{equation*}
    Procédons par double inclusion.\\
    $\circledcirc$ Soit $y\in f(A) \cap B$. Montrons que $y\in f(A \cap f^{-1}(B))$.\\
    On a $y\in f(A)$ et $y\in B$.\\
    $\exists x\in A ~ | ~ y = f(x)$ donc $x\in A$ et $x\in f^{-1}(B)$ car $y\in B$.\\
    Ainsi $x\in A\cap f^{-1}(B)$ et $f(x) = y \in f(A \cap f^{-1}(B))$\\[0.15cm]
    $\circledcirc$ Soit $y\in f(A \cap f^{-1}(B))$ Montrons que $y\in f(A) \cap B$.\\
    $\exists x \in A \cap f^{-1}(B) ~ | ~ y = f(x)$ donc $x\in A$ et $x \in f^{-1}(B)$.\\
    Ainsi, $f(x) = y \in f(A)$ et $f(x) = y \in B$ : $y\in f(A)\cap B$.\\
    \qed
\end{tcolorbox}
\addcontentsline{toc}{section}{Images directes, images réciproques.}
\addcontentsline{toc}{section}{\protect\numberline{}Exercice 15.1}

\section*{Exercice 15.2 [$\blacklozenge\blacklozenge\lozenge$]}
\begin{tcolorbox}[enhanced, width=7.6in, center, size=fbox, fontupper=\large, drop shadow southwest]
    Soit $f:E\to F$ une application. Soit $A$ une partie de $E$ et $B$ une partie de $F$.\\
    1. (a) Montrer que $A \subset f^{-1}(f(A))$.\\
    (b) Montrer que si $f$ est injective, la réciproque est vraie.\\
    2. (a) Montrer que $f(f^{-1}(B)) \subset B$.\\
    (b) Démontrer que si $f$ est surjective, la réciproque est vraie.\\
    3. Montrer que $f(f^{-1}(f(A))) = f(A)$.\\
    4. Montrer que $f^{-1}(f(f^{-1}(B)))=f^{-1}(B)$.\\[0.15cm]
    1.\\
    a) Soit $x\in A$. Montrons que $x\in f^{-1}(f(A))$.\\
    On a $x\in A$ alors $f(x) \in f(A)$ et $x\in f^{-1}(f(A))$.\\
    b) On suppose $f$ injective, soit $x \in f^{-1}(f(A))$.\\
    On applique $f$ : $f(x) \in f(A)$. Par injectivité de $f$, $x \in A$.\\[0.2cm]
    2.\\
    a) Soit $y \in f(f^{-1}(B))$.\\
    On a $\exists x \in f^{-1}(B) ~ | ~ y = f(x)$. Ainsi, $f(x)\in B$ : $y\in B$.\\
    b) Supposons $f$ surjective, soit $y\in B$.\\
    On a $\exists x \in f^{-1}(B) ~ | ~ y = f(x)$ et $f(x) = y \in f(f^{-1}(B))$.\\[0.2cm]
    3) Soit $y\in f(f^{-1}(f(A)))$. Montrons que $y\in f(A)$.\\
    On a $\exists x \in f^{-1}(f(A)) ~ | ~ y = f(x)$ et $f(x) \in f(A)$ donc $y \in f(A)$.\\
    Soit $y\in f(A)$. Montrons que $y\in f(f^{-1}(f(A)))$.\\
    On a $\exists x \in A ~ | ~ y = f(x)$ alors $f(x) \in f(A)$ et $x\in f^{-1}(f(A))$. Donc $f(x) = y \in f(f^{-1}(f(A)))$.\\[0.2cm]
    4) Soit $y \in f^{-1}(f(f^{-1}(B)))$. Montrons que $y \in f^{-1}(B)$.\\
    On a $f(y) \in f(f^{-1}(B))$ alors $y \in f^{-1}(B)$.\\
    Soit $y\in f^{-1}(B)$. Montrons que $y\in f^{-1}(f(f^{-1}(B)))$.\\
    On a $f(y) \in f(f^{-1}(B))$ donc $y \in f^{-1}(f(f^{-1}(B)))$.\\
    \qed
\end{tcolorbox}
\addcontentsline{toc}{section}{\protect\numberline{}Exercice 15.2}

\section*{Exercice 15.3 [$\blacklozenge\blacklozenge\blacklozenge$]}
\begin{tcolorbox}[enhanced, width=7.6in, center, size=fbox, fontupper=\large, drop shadow southwest]
    Soit $f:E\to F$ une application. Montrer que
    \begin{equation*}
        f \text{ est injective } \iff [\forall A,B \in \mathcal{P}(E) ~ f(A \cap B) = f(A) \cap f(B)]
    \end{equation*}
    $\circledcirc$ Supposons $f$ injective. Soient $A,B \in \mathcal{P}(E)$.\\
    On sait déjà que $f(A \cap B) \subset f(A) \cap f(B)$.\\
    Montrons alors que $f(A) \cap f(B) \subset f(A \cap B)$.\\
    Soit $y \in f(A) \cap f(B)$. On a que $y \in f(A) \wedge y \in f(B)$.\\
    Ainsi, $\exists x_A \in A ~ | ~ y = f(x_A)$ et $\exists x_B \in B ~ | ~ y = f(x_B)$.\\
    Or $f$ est injective : $x_A = x_B$, ainsi $x_A \in A \cap B$.\\
    On a enfin que $f(x_A) \in f(A \cap B)$, alors $y \in f(A \cap B)$.\\[0.2cm]
    $\circledcirc$ Supposons $[\forall A,B \in \mathcal{P}(E) ~ f(A \cap B) = f(A) \cap f(B)]$. Montrons que $f$ est injective.\\
    Soient $A,B \in \mathcal{P}(E)$.\\
    Soient $x,x' \in E$. On suppose que $f(x) = f(x')$. Montrons que $x = x'$.\\
    On a que $\{x\}$ et $\{x'\} \in \mathcal{P}(E)$.\\
    Ainsi : $f(\{x\} \cap \{x'\}) = f(\{x\}) \cap f(\{x'\})$.\\
    Supposons que $x \neq x'$. On a alors : $f(\varnothing) = f(\{x\}) \cap f(\{x'\})$ : $\varnothing = \{f(x)\} \cap \{f(x')\}$.\\
    Or $f(x) = f(x')$ donc $\{f(x)\} \cap \{f(x')\} \neq \varnothing$. C'est absurde : $x = x'$.\\
    On a bien montré que $f$ est injective.\\
    \qed 
\end{tcolorbox}
\addcontentsline{toc}{section}{\protect\numberline{}Exercice 15.3}

\section*{Exercice 15.4 [$\blacklozenge\lozenge\lozenge$]}
\begin{tcolorbox}[enhanced, width=7.6in, center, size=fbox, fontupper=\large, drop shadow southwest]
    Soient
    \begin{equation*}
        f : \begin{cases}\mathbb{N}^2 \to \mathbb{Z} \\ (n,p) \mapsto (-1)^np \end{cases} \hspace{1.25cm} \text{et} \hspace{1.25cm} g : \begin{cases}\mathbb{R} \to \mathbb{C} \\ x\mapsto \frac{1+ix}{1-ix} \end{cases}
    \end{equation*}
    Ces fonctions sont-elles injectives ? Surjectives ?\\[0.2cm]
    On a que $f$ n'est pas injective : $f(0,1) = f(2,1) = 1$.\\
    Montrons que $f$ est surjective.\\
    Soit $y\in\mathbb{Z}$. Montrons que $\exists (n,p)\in\mathbb{N}^2 ~ | ~ f(n,p) = y$.\\
    Si $y \geq 0$, on prend $n = 0$ et $p = |y|$.\\
    Si $y \leq 0$, on prend $n = 1$ et $p = |y|$.\\[0.2cm]
    On a que $g$ n'est pas surjective : $0$ n'a aucun antécédent par $g$.\\
    Montrons que $g$ est injective.\\
    Soient $x,x' \in \mathbb{R}$, supposons $g(x) = g(x')$. Montrons que $x=x'$.\\
    On a :
    \begin{align*}
        g(x) = g(x') &\iff \frac{1+ix}{1-ix} = \frac{1+ix'}{1-ix'}\\
        &\iff (1+ix)(1-ix') = (1+ix')(1-ix)\\
        &\iff 1 - ix' + ix + xx' = 1 - ix + ix' + xx'\\
        &\iff 2ix = 2ix'\\
        &\iff x = x' 
    \end{align*}
    On a bien que $g$ est injective.\\
    \qed
\end{tcolorbox}
\addcontentsline{toc}{section}{Applications injectives, surjectives.}
\addcontentsline{toc}{section}{\protect\numberline{}Exercice 15.4}

\section*{Exercice 15.5 [$\blacklozenge\lozenge\lozenge$]}
\begin{tcolorbox}[enhanced, width=7.6in, center, size=fbox, fontupper=\large, drop shadow southwest]
    Dans cet exercice, on admet que $\pi$ est irrationnel.\\
    Démontrer que $\cos_{|\mathbb{Q}}$ n'est pas injective et que $\sin_{|\mathbb{Q}}$ l'est.\\[0.2cm]
    On sait que $\cos$ est paire : $\cos_{|\mathbb{Q}}$ l'est aussi.\\
    Alors $\cos_{|\mathbb{Q}}(\frac{1}{2}) = \cos_{|\mathbb{Q}}(-\frac{1}{2})$. Or $\frac{1}{2} \neq -\frac{1}{2}$ : $\cos_{|\mathbb{Q}}$ n'est pas injective.\\[0.2cm]
    Soient $x,x'\in\mathbb{Q}^2$. Supposons que $\sin_{|\mathbb{Q}}(x) = \sin_{|\mathbb{Q}}(x')$. Montrons que $x=x'$.\\
    On a :
    \begin{align*}
        \sin_{|\mathbb{Q}}(x) = \sin_{|\mathbb{Q}}(x') &\iff x \equiv x' [2\pi] ~ (2\pi\text{-périodicité}) \\
        &\iff x = x' + 2k\pi ~ (k\in\mathbb{Z})\\
    \end{align*}
    Or, $\forall{k\in\mathbb{Z}^*}, ~ x' + 2k\pi \notin \mathbb{Q}$. On a alors que $k=0$ :
    \begin{equation*}
        \sin_{|\mathbb{Q}}(x) = \sin_{|\mathbb{Q}}(x') \iff x = x' + 2\cdot0\pi \iff x = x'
    \end{equation*}
    \qed
\end{tcolorbox}
\addcontentsline{toc}{section}{\protect\numberline{}Exercice 15.5}

\section*{Exercice 15.6 [$\blacklozenge\blacklozenge\lozenge$]}
\begin{tcolorbox}[enhanced, width=7.6in, center, size=fbox, fontupper=\large, drop shadow southwest]
    Soit l'application $f:\mathbb{R} \to \mathbb{R}$ définie par $f(x)=\begin{cases}x^2 ~ \text{si x $\geq$ 0} \\ 2x^2 ~ \text{si x < 0}\end{cases}$\\
    1. Montrer que $f$ n'est pas injective.\\
    2. Montrer que $f_{|\mathbb{Q}}$ est injective.\\[0.2cm]
    1. On a $f(2) = 4$ et $f(-\sqrt{2}) = 4$ : $f$ n'est pas injective.\\
    2. Soient $x,x'\in\mathbb{Q}$ tels que $f_{|\mathbb{Q}}(x)=f_{|\mathbb{Q}}(\widetilde{x})$. Montrons que $x=\widetilde{x}$.\\
    \underline{Cas n°1 : $x$ et $\widetilde{x}$ positifs} :
    \begin{align*}
        f_{|\mathbb{Q}}(x) = f_{|\mathbb{Q}}(\widetilde{x}) &\iff x^2 = \widetilde{x}^2 \iff x = \widetilde{x}
    \end{align*}
    \underline{Cas n°2 : $x$ et $\widetilde{x}$ strictement négatifs} :
    \begin{align*}
        f_{|\mathbb{Q}}(x) = f_{|\mathbb{Q}}(\widetilde{x}) &\iff 2x^2 = 2\widetilde{x}^2 \iff x^2 = \widetilde{x}^2 \iff x = \widetilde{x} ~ \text{ car $x,\widetilde{x}\in\mathbb{R}^*_-$}
    \end{align*}
    \underline{Cas n°3 : $x\geq0$ et $\widetilde{x}<0$} :
    \begin{align*}
        f_{|\mathbb{Q}}(x) = f_{|\mathbb{Q}}(\widetilde{x}) &\iff x^2 = 2\widetilde{x}^2 \iff x = -\sqrt{2}\widetilde{x} \iff -\frac{x}{\widetilde{x}}=\sqrt{2}
    \end{align*}
    Cela est impossible par stabilité de $\mathbb{Q}$ par la division. Donc $f_{|\mathbb{Q}}(x) \neq f_{|\mathbb{Q}}(\widetilde{x})$.\\
    Le cas où $x < 0$ et $\widetilde{x}\geq0$ est symétrique.\\
    On a prouvé que $f_{|\mathbb{Q}}$ est injective.\\
    \qed
\end{tcolorbox}
\addcontentsline{toc}{section}{\protect\numberline{}Exercice 15.6}

\section*{Exercice 15.7 [$\blacklozenge\lozenge\lozenge$]}
\begin{tcolorbox}[enhanced, width=7.6in, center, size=fbox, fontupper=\large, drop shadow southwest]
    Soit $f:E\to E$. Montrer que\\
    1. $f$ est injective si et seulement si $f \circ f$ est injective.\\
    2. $f$ est surjective si et seulement si $f \circ f$ est surjective.\\
    1. Supposons $f$ injective. D'après la proposition 18, $f \circ f$ est injective.\\
    Supposons $f \circ f$ injective. D'après la proposition 19, $f$ est injective.\\
    2. Supposons $f$ surjective. D'après la proposition 23, $f \circ f$ est surjective.\\
    Supposons $f \circ f$ surjective. D'après la proposition 24, $f$ est surjective.
    \qed
\end{tcolorbox}
\addcontentsline{toc}{section}{\protect\numberline{}Exercice 15.7}

\section*{Exercice 15.8 [$\blacklozenge\blacklozenge\lozenge$]}
\begin{tcolorbox}[enhanced, width=7.6in, center, size=fbox, fontupper=\large, drop shadow southwest]
    Soit $E$ un ensemble et $f:E\to E$ une application.\\
    On suppose que $f\circ f = f$ et que $f$ est injective ou surjective. Montrer que $f = \text{id}_E$.\\[0.2cm]
    $\circledcirc$ Supposons $f$ injective. Soit $x \in E$.\\
    On a $f \circ f (x) = f(x)$. Par injectivité de $f$, $f(x) = x$ donc $f = \text{id}_E$.\\[0.15cm]
    $\circledcirc$ Supposons $f$ surjective. Soit $y \in E$.\\
    On a $f \circ f (y) = f(y)$ et $\exists x \in E ~ | ~ f(x) = y$ par surjectivité de $f$.\\
    Donc $f \circ f \circ f (x) = f \circ f (x)$. Alors $f \circ f (x) = f(x)$ et $f(y) = y$ : $f = \text{id}_E$.\\
    \qed
\end{tcolorbox}
\addcontentsline{toc}{section}{\protect\numberline{}Exercice 15.8}

\section*{Exercice 15.9 [$\blacklozenge\blacklozenge\lozenge$]}
\begin{tcolorbox}[enhanced, width=7.6in, center, size=fbox, fontupper=\large, drop shadow southwest]
    Soit $E$ un ensemble non vide et $f:E\to E$ une application telle que $f \circ f \circ f = f$.\\
    Montrer que
    \begin{equation*}
        f \text{ est surjective} \iff f \text{ est injective}
    \end{equation*}
    $\circledcirc$ Supposons $f$ injective, montrons que $f$ est surjective.\\
    Soit $y \in E$. Par définition de $f$ : $f \circ f \circ f(y) = f(y)$.\\
    Par injectivité de $f$ : $f\circ f(y) = y$.\\
    Donc $f(y)$ est antécédent de $y$ : $f$ est surjective.\\[0.1cm]
    $\circledcirc$ Supposons $f$ surjective, montrons $f$ injective.\\
    Soient $y, y' \in E$ tels que $f(y) = f(y')$. Montrons que $y=y'$.\\
    Par surjectivité de $f$, $\exists x,x' \in E ~ | ~ f(x) = y ~ \wedge ~ f(x') = y'$.\\
    Ainsi, $f\circ f(x) = f \circ f (x')$.\\
    Appliquons $f$ : $f \circ f \circ f(x) = f \circ f \circ f(x')$.\\
    Alors : $f(x) = f(x')$ et donc $y = y'$.\\
    On a bien prouvé l'injectivité de $f$.\\
    \qed
\end{tcolorbox}
\addcontentsline{toc}{section}{\protect\numberline{}Exercice 15.9}

\section*{Exercice 15.10 [$\blacklozenge\lozenge\lozenge$]}
\begin{tcolorbox}[enhanced, width=7.6in, center, size=fbox, fontupper=\large, drop shadow southwest]
    Soit $f:\begin{cases}\mathbb{N} \to \mathbb{N} \\ n \mapsto n + (-1)^n\end{cases}$.\\
    Démontrer que $f$ est une bijection de $\mathbb{N}$ dans lui-même et donner sa réciproque.\\[0.2cm]
    Montrons que $f$ est un inverse à gauche et à droite d'elle-même.\\
    Soit $n\in\mathbb{N}$. On a :
    \begin{align*}
        f \circ f (n) = f(n + (-1)^n) &= n + (-1)^n + (-1)^{n + (-1)^n} \\
        &= n + (-1)^n(1+(-1)^{(-1)^n})
    \end{align*}
    Or $(-1)^n$ est toujours impair : $(-1)^{(-1)^n} = -1$. Ainsi :
    \begin{equation*}
        f\circ f(n) = n + (-1)^n(1-1) = n
    \end{equation*}
    On a bien que $f$ est un inverse à gauche et à droite d'elle même : $f$ est bijective et est sa propre réciproque.\\
    \qed
\end{tcolorbox}
\addcontentsline{toc}{section}{\protect\numberline{}Exercice 15.10}

\section*{Exercice 15.11 [$\blacklozenge\blacklozenge\blacklozenge$]}
\begin{tcolorbox}[enhanced, width=7.6in, center, size=fbox, fontupper=\large, drop shadow southwest]
    Soient $E$ un ensemble et $(A,B)\in\mathcal{P}(E)^2$. On définit
    \begin{equation*}
        \varPhi : \begin{cases}\mathcal{P}(E) \to \mathcal{P}(A) \times \mathcal{P}(B)\\X \mapsto (X \cap A, X \cap B)\end{cases}
    \end{equation*}
    1. Calculer $\varPhi(\varnothing)$ et $\varPhi(E \setminus (A \cup B))$. Que dire de $A$ et $B$ si $(A, \varnothing)$ admet un antécédent par $\varPhi$ ?\\
    2. Montrer que $\varPhi$ injective $\iff A \cup B = E$.\\
    3. Montrer que $\varPhi$ surjective $\iff A \cap B = \varnothing$.\\[0.2cm]
    1. On a $\varPhi(\varnothing) = (\varnothing, \varnothing)$ et $\varPhi(E \setminus (A \cup B)) = ((\overline{A} \cap \overline{B}) \cap A, (\overline{A} \cap \overline{B}) \cap B) = (\varnothing, \varnothing)$.\\
    Si $(A, \varnothing)$ admet un antécéddent par $\varPhi$ alors $A$ et $B$ sont disjoints : $A \cap B = \varnothing$.\\[0.15cm]
    2.\\
    $\circledcirc$ Supposons $\varPhi$ injective. Montrons $A \cup B = E$.\\
    On a que $\varPhi(E) = (A,B)$ et $\varPhi(A \cup B)=(A,B)$. Par injectivité de $\varPhi$, $E = A \cup B$.\\[0.1cm]
    $\circledcirc$ Supposons $A \cup B = E$. Montrons que $\varPhi$ est injective.\\
    Soient $X,Y \in \mathcal{P}(E)$ telles que $\varPhi(X) = \varPhi(Y)$. Montrons que $X = Y$.\\
    On a
    \begin{align*}
        &(X \cap A, X \cap B) = (Y \cap A, Y \cap B)\\
        \Longrightarrow&X\cap A = Y \cap A ~ \wedge ~ X \cap B = Y \cap B\\
        \Longrightarrow& (X \cap A) \cup (X \cap B) = (Y \cap A) \cup (Y \cap B)\\
        \Longrightarrow& X \cap (A \cup B) = Y \cap (A \cup B)\\
        \Longrightarrow& X = Y ~ \text{car } A \cup B = E 
    \end{align*}
    3.\\
    $\circledcirc$ Supposons $\varPhi$ surjective. Montrons $A \cap B = \varnothing$.\\
    On a que $\exists X \in \mathcal{P}(E) ~ | ~ \varPhi(X) = (A, \varnothing)$ puisque $(A, \varnothing) \in \mathcal{P}(a) \times \mathcal{P}(B)$ et que $\varPhi$ est surjective.\\
    Or, puisque $X$ existe, on a que $A$ et $B$ sont disjoints:  $A \cap B = \varnothing$.\\[0.1cm]
    $\circledcirc$ Supposons $A \cap B = \varnothing$. Montrons que $\varPhi$ est surjective.\\
    Soit $Y \in \mathcal{P}(A)$ et $Z \in \mathcal{P}(B)$. Montrons que $\exists X \in \mathcal{P}(E) ~ | ~ \varPhi(X) = (Y, Z)$.\\
    On choisit $X = Y \cup Z$. On a $\varPhi(X)=((Y \cup Z) \cap A, (Y \cup Z) \cap B)$.\\
    Or $A \cap B = \varnothing$. En particulier, $Y \cap B = \varnothing$ et $Z\cap A = \varnothing$ car $Y\in\mathcal{P}(A)$ et $Z \in \mathcal{P}(B)$.\\
    Alors, $\varPhi(X)=(Y \cap A, Z \cap B)=(Y,Z)$.\\
    On a bien que $X$ est un antécédent de $(Y,Z)$.\\
    \qed
\end{tcolorbox}
\addcontentsline{toc}{section}{\protect\numberline{}Exercice 15.11}

\section*{Exercice 15.12 [$\blacklozenge\blacklozenge\blacklozenge$]}
\begin{tcolorbox}[enhanced, width=7.6in, center, size=fbox, fontupper=\large, drop shadow southwest]
    Soit $f \in \mathcal{F}(E,F)$.\\
    1. Démontrer que $f$ est injective si et seulement si elle est inversible à gauche.\\
    Plus précisément, prouver l'assertion
    \begin{equation*}
        f \text{ est injective} \iff \exists g \in \mathcal{F}(F,E) ~ g \circ f = \text{id}_E
    \end{equation*}
    2. Démontrer que $f$ est surjective si et seulement si elle est inversible à \underline{droite}.\\
    Plus précisément, prouver l'assertion
    \begin{equation*}
        f \text{ est \underline{surjective}} \iff \exists g \in \mathcal{F}(F,E) ~ f \circ g = \text{id}_F 
    \end{equation*}
    1.\\
    $\circledcirc$ Supposons $f$ injective et soit $g:F\to E$.\\
    Soit $y \in F$.\\
    Si $y\in f(E)$, on a $\exists!x\in E ~ | ~ f(x) = y$, alors on pose $g(y)=x$.\\
    Si $y\notin f(E)$, on prend un élément $x\in F$ quelconque et on pose $g(y)=x$.\\
    On a que $g$ est bien définie sur $F$ et $\forall x \in E, ~ g(f(x))=x$ par définition.\\[0.15cm]
    $\circledcirc$ Supposons que $\exists g \in \mathcal{F}(F, E) ~ g \circ f = \text{id}_E$. Montrons que $f$ est injective.\\
    Soient $x,x'\in E$ tels que $f(x)=f(x')$.\\
    On a $f(x)=f(x') \iff g(f(x))=g(f(x')) \iff \text{id}_E(x) = \text{id}_E(x') \iff x = x'$.\\[0.2cm]
    2.\\
    $\circledcirc$ Supposons $f$ surjective et soit $g:F\to E$.\\
    Soit $y \in F$ : $\exists x \in E ~ | ~ y = f(x)$.\\
    Or il peut exister plusieurs $x$ différents dont $y$ est l'image, on fait le choix de n'en garder qu'un particulier.\\
    Alors on pose $g(y) = x$.\\
    Ainsi, on a $f(g(y)) = f(x)$, c'est-à-dire $f(g(y))=y$ : $f\circ g = \text{id}_F$.\\[0.15cm]
    $\circledcirc$ Supposons que $\exists g \in \mathcal{F}(F, E) ~ f \circ g = \text{id}_F$. Montrons que $f$ est surjective.\\
    Soit $y \in F$. On a que $f\circ g(y) = y$ car $f \circ g = \text{id}_F$.\\
    Ainsi, $y$ est l'image de $f\circ g(y)$ : $f$ est surjective.\\
    \qed
\end{tcolorbox}
\addcontentsline{toc}{section}{\protect\numberline{}Exercice 15.12}

\section*{Exercice 15.13 [$\blacklozenge\blacklozenge\blacklozenge$] Théorème de Cantor}
\begin{tcolorbox}[enhanced, width=7.6in, center, size=fbox, fontupper=\large, drop shadow southwest]
    Soit $f\in\mathcal{F}(E, \mathcal{P}(E))$. Montrer que $f$ n'est pas surjective.\\
    Indication : on pourra considérer $A = \{x \in E ~ | ~ x \notin f(x)\}$.\\[0.2cm]
    Montrons que $A$ n'a pas d'antécédent par $f$.\\
    Supposons qu'il en ait un.\\
    Alors $\exists \alpha \in E ~ | ~ A = f(\alpha)$.\\
    $\circledcirc$ Supposons que $\alpha \in A$. Alors $\alpha \in \{x \in E ~ | ~ x \notin f(x)\}$.\\
    Donc $\alpha \notin f(\alpha)$ donc $\alpha \notin A$. Absurde.\\[0.15cm]
    $\circledcirc$ Supposons que $\alpha \notin A$. Alors $\alpha \notin \{x \in E ~ | ~ x \notin f(x)\}$.\\
    Donc $\alpha \in A$. Absurde.\\[0.15cm]
    $\alpha$ n'existe pas : $A$ n'a pas d'antécédent par $f$ et $f$ n'est pas surjective.\\
    \qed
\end{tcolorbox}
\addcontentsline{toc}{section}{\protect\numberline{}Exercice 15.13}

\end{document}
 
