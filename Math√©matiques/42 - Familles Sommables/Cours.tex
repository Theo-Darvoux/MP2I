\documentclass[11pt]{article}

\def\chapitre{42}
\def\pagetitle{Familles sommables.}

\input{Latex/setup.tex}

\begin{document}

\input{Latex/title.tex}

\section*{Introduction.}

\begin{ex}{Pour poser le problème.}{}
    Soit la famille $(u_{n,p})_{(n,p)\in\N^2}$ définie pour tout $(n,p)\in\N^2$ par $u_{n,p}=\begin{cases}
        1 & \nt{si } n=p+1\\
        -1 & \nt{si } p=n+1\\
        0 & \nt{sinon}
    \end{cases}$
    \begin{equation*}
        \nt{Calculer:} \quad \sum_{n=0}^{+\infty}\sum_{p=0}^{+\infty}u_{n,p}, \quad \sum_{p=0}^{+\infty}\sum_{n=0}^{+\infty}u_{n,p}, \quad \sum_{N=0}^{+\infty}\sum_{n+p=N}u_{n,p}. \quad \nt{Commenter.}
    \end{equation*}
    \tcblower
    La première somme vaut $-1$, la deuxième vaut $1$ et la dernière n'est pas sommable.
\end{ex}

\section{Sommer des réels positifs.}
\setcounter{subsection}{-1}
\subsection{Travailler dans \texorpdfstring{$[0,+\infty]$}{Lg}}

On note \fbox{$[0,+\infty]=\R_+\cup\{+\infty\}$}.

\begin{defi}{}{}
    On appelle \bf{borne supérieure} d'une partie $A$ de $[0,+\infty]$ le plus petit des majorants de $A$ dans $[0,+\infty]$. Cet élément de $[0,+\infty]$ est noté $\sup(A)$.
\end{defi}

\begin{meth}{Passage au sup : l'argument clé du cours.}{}
    Soient $M\in[0,+\infty]$ un réel et $A$ une partie de $[0,+\infty]$. Pour démontrer l'inégalité $\sup(A)\leq M$, il suffira de montrer que $M$ est un majorant de $A$, autrement dit:
    \begin{equation*}
        (\forall x \in A\quad x\leq M) \ra \sup(A) \leq M.
    \end{equation*}
\end{meth}

\subsection{Somme d'une famille de réels positifs.}

\begin{defi}{}{}
    Soit $(u_i)_{i\in I}$ une famille de réels \bf{positifs}. On appelle \bf{somme} de cette famille, notée $\sum_{i\in I}u_i$ le nombre
    \begin{equation*}
        \sum_{i\in I}u_i=\sup\left\{\sum_{i\in F}u_i, ~ F\subset I, ~ F \nt{ finie}\right\} \quad (\in[0,+\infty]).
    \end{equation*}
\end{defi}

\begin{prop}{}{4}
    Soit $(u_i)_{i\in I}$ une famille de réels positifs et $I'\subset I$. On a \large$\sum\limits_{i\in I'}u_i\leq \sum\limits_{i\in I}u_i$.
    \tcblower
    Soit $F\subset I'$ finie. Alors $F\subset I$. On a: 
    \begin{equation*}
        \sum_{i\in F}u_i\leq \sum_{i\in I}u_i\quad\nt{alors}\quad\sum_{i\in I'}u_i \leq \sum_{i\in I}u_i
    \end{equation*}
    Par passage au sup sur $F$.
\end{prop}

\begin{prop}{}{5}
    Soit $(u_i)_{i\in I}$ et $(v_i)_{i\in I}$ deux familles de réels positifs telles que $\forall i \in I,~u_i\leq v_i$. On a \large$\sum_{i\in I}u_i\leq\sum_{i\in I}v_i$
    \tcblower
    Soit $F\subset I$ finie. Alors on a:
    \begin{equation*}
        \sum_{i\in F}u_i\leq\sum_{i\in F}v_i\leq\sum_{i\in I}v_i\quad\nt{alors}\quad\sum_{i\in I}u_i\leq\sum_{i\in I}v_i
    \end{equation*}
    Par passage au sup sur $F$.
\end{prop}

\begin{prop}{Lien avec les sommes finies, les sommes de séries.}{6}
    Soit $(u_i)_{i\in I}$ une famille de réels positifs.
    \begin{enumerate}[itemsep=-0.9 ex]
        \item Si $I$ est finie, le nombre $\sum\limits_{i\in I}u_i$ est à la fois la somme des nombres de la famille, et la somme de la famille.
        \item Si $I=\N$, alors $\sum\limits_{i\in I}u_i=\sum\limits_{k=0}^{+\infty}u_k$, le nombre à droite étant la somme de la série $\sum u_n$.
    \end{enumerate}
    \tcblower
    \boxed{1.} Notons $s$ la somme finie habituelle et $\s$ la nouvelle.\\
    $\circledcirc$ On a $I\subset I$ finie, donc $s\leq \s$.\\
    $\circledcirc$ Soit $F\subset I$ finie, on a
    \begin{equation*}
        \sum_{i\in F}u_i=\sum_{i\in I} u_i - \sum_{i\in I\setminus F}u_i\quad\nt{alors}\quad\sum_{i\in F}u_i\leq s\quad\nt{alors}\quad\s\leq s.
    \end{equation*}
    Par antisymétrie, $\s=s$.\n
    \boxed{2.} 
    $\circledcirc$ Soit $N\in\N$, $\lb0,N\rb\subset\N$ donc \large$\sum\limits_{i=0}^Nu_i\leq\sum\limits_{i\in I}u_i$. \normalsize Par passage à la limite, \large$\sum\limits_{i=0}^{+\infty}u_i\leq\sum\limits_{i\in I}u_i$.\normalsize\\
    $\circledcirc$ Soit $F\subset\N$ finie. On a
    \begin{equation*}
        \sum_{i\in F}u_i \leq \sum_{i=0}^Nu_i\quad\nt{avec }N=\max(F).
    \end{equation*}
    Alors, par passage à la limite, puis au sup:
    \begin{equation*}
        \sum_{i\in I}u_i\leq\sum_{i=0}^{+\infty}u_i.
    \end{equation*}
\end{prop}

\begin{prop}{Invariance de la somme par permutation, cas positif.}{}
    Soit $(u_i)_{i\in I}$ une famille de nombres réels positifs et $\s$ une bijection de $I$ dans $I$. On a
    \begin{equation*}
        \sum_{i\in I}u_{\s(i)}=\sum_{i\in I}u_i.
    \end{equation*}
    \tcblower
    $\circledcirc$ Soit $F\subset I$ finie. On a
    \begin{equation*}
        \sum_{i\in F}u_{\s(i)}=\sum_{i\in\s(F)}u_i\leq\sum_{i\in I}u_i\quad\nt{alors}\quad\sum_{i\in I}u_{\s(i)}\leq\sum_{i\in I}u_i.
    \end{equation*}
    $\circledcirc$ Soit $F\subset I$ finie. On a
    \begin{equation*}
        \sum_{i\in F}u_i = \sum_{i\in\s^{-1}(F)}u_{\s(i)}\leq\sum_{i\in I}u_{\s(i)}\quad\nt{alors}\quad\sum_{i\in I}u_i\leq \sum_{i\in I}u_{\s(i)}.
    \end{equation*}
\end{prop}

\subsection{Familles sommables de réels positifs.}

\begin{defi}{}{}
    Une famille de réels \bf{positifs} $(u_i)_{i\in I}$ est dite $\bf{sommable}$ si sa somme est finie, ce qui se note
    \begin{equation*}
        \sum_{i\in I}u_i < +\infty.
    \end{equation*}
\end{defi}

\begin{prop}{}{}
    Soit $(u_i)_{i\in I}$ et $(v_i)_{i\in I}$ deux familles de réels positifs (indexées par le même ensemble) et $\l\in\R_+$.
    \begin{itemize}[topsep=0pt,itemsep=-0.9 ex]
        \item La famille $(u_i+v_i)_{i\in I}$ est sommable ssi $(u_i)_{i\in I}$ et $(v_i)_{i\in I}$ le sont.
        \item Si $(u_i)_{i\in I}$ est sommable, alors $(\l u_i)_{i\in I}$ l'est aussi.
    \end{itemize}
    \tcblower
    Trivial.
\end{prop}

\subsection{Sommation par paquets.}

\begin{thm}{de sommation par paquets, cas positifs.}{}
    Soit $(u_i)_{i\in I}$ une famille de réels \bf{positifs}.\\
    On suppose que $I$ s'écrit comme une réunion \bf{disjointe} $I=\bigcup_{j\in J}I_j$. Alors
    \begin{equation*}
        \sum_{j\in J}\left( \sum_{i\in I_j}u_i \right) = \sum_{i\in I}u_i
    \end{equation*}
    \tcblower
    Preuve hors-programme. 
\end{thm}

\begin{corr}{si cette somme est finie, alors c'est sommable.}{}
    Une famille $(u_i)_{i\in I}$ de nombres positifs est sommable si et seulement si
    \begin{enumerate}[topsep=0pt,itemsep=-0.9 ex]
        \item pour tout $j\in J$, $(u_i)_{i\in I_j}$ est sommable,
        \item la famille $(\sum\limits_{i\in I_j}u_i)_{j\in J}$ est sommable.
    \end{enumerate}
\end{corr}

\begin{thm}{de Fubini positif.}{}
    Soit $(u_{i,j})_{(i,j)\in I\times J}$ une famille de réels positifs indexée par un produit cartésien $I\times J$, on a:
    \begin{equation*}
        \sum_{(i,j)\in I\times J}u_{i,j}=\sum_{i\in I}\sum_{j\in J}u_{i,j}=\sum_{j\in J}\sum_{i\in I}u_{i,j}.
    \end{equation*}
    \tcblower
    On a $I\times J=\bigcup_{i\in I}(\{i\}\times J)$ où les $\{i\}\times J$ sont un recouvrement disjoint de $I\times J$. Alors:
    \begin{equation*}
        \sum_{(i,j)\in I\times J}u_{i,j}=\sum_{i\in I}\sum_{(i,j)\in\{i\}\times J}u_{i,j}=\sum_{i\in I}\sum_{j\in J}u_{i,j}.
    \end{equation*}
\end{thm}

\begin{ex}{Sommes triangulaires, cas positif.}{}
    Soit $(u_{n,p})_{(n,p)\in\N^2}$ une famille de nombres réels positifs indexée par $\N^2$. On a
    \begin{equation*}
        \sum_{(n,p)\in\N^2}u_{n,p}=\sum_{n=0}^{+\infty}\sum_{k=0}^nu_{k,n-k}
    \end{equation*}
    \tcblower
    On a $\N^2=\bigcup_{n\in\N}I_n$ où $I_n=\{(k,n-k)\mid0\leq k \leq n\}$ recouvrement disjoint.
\end{ex}

\begin{ex}{}{}
    Calculer la somme de la famille \Large$\left( \frac{1}{p^2q^2} \right)_{(p,q)\in(\N^*)^2}$\normalsize.\\
    Montrer que la famille \Large$\left( \frac{1}{(p+q)^2} \right)_{(p,q)\in(\N^*)^2}$\normalsize n'est pas sommable.
    \tcblower
    On a:
    \begin{align*}
        \sum_{(p,q)\in(\N^*)^2}\frac{1}{p^2q^2}&=\sum_{p\in\N^*}\sum_{q\in\N^*}\frac{1}{p^2q^2}=\sum_{p\in\N^*}\frac{1}{p^2}\left( \sum_{q\in\N^*}\frac{1}{q^2} \right) =\frac{\pi^2}{6}\sum_{p\in\N^*}\frac{1}{p^2}\\
        &=\frac{\pi^4}{36}.
    \end{align*}
    On pose pour $n\in\N^*,~ I_n=\{(p,q)\in(\N^*)^2\mid p+q=n\}=\{(k,n-k)\mid k\in\lb1,n\rb\}$ (recouvrement disjoint).
    \begin{align*}
        \sum_{(p,q)\in(\N^*)^2}\frac{1}{(p+q)^2}&=\sum_{n\in\N^*}\sum_{(p,q)\in I_n}\frac{1}{(p+q)^2}\\
        &=\sum_{n\in\N^*}\frac{1}{n^2}|I_n|=\sum_{n\in\N^*}\frac{1}{n}\\
        &=+\infty
    \end{align*}
    Donc la famille n'est pas sommable.
\end{ex}

\begin{ex}{}{}
    Démontrer l'identité \Large$\sum\limits_{p=2}^{+\infty}(\zeta(p)-1)=1$.
    \tcblower
    On a:
    \begin{align*}
        \sum_{p=2}^{+\infty}(\zeta(p)-1) &= \sum_{p=2}^{+\infty}\sum_{n=2}^{+\infty}\frac{1}{n^p}=\sum_{n=2}^{+\infty}\sum_{p=2}^{+\infty}\frac{1}{n^p}=\sum_{n=2}^{+\infty}\frac{1}{n^2}\cdot\frac{n}{n-1}\\
        &=\sum_{n=2}^{+\infty}\frac{1}{n(n-1)}=\sum_{n=2}^{+\infty}\left(\frac{1}{n-1}-\frac{1}{n}\right)\\
        &=1
    \end{align*}
\end{ex}

\begin{ex}{}{}
    Soit $a\in[0,1[$. En considérant la famille $(a^{pq})_{(p,q)\in(\N^*)^2}$, démontrer l'identité:
    \begin{equation*}
        \sum_{n=1}^{+\infty}\frac{a^n}{1-a^n}=\sum_{n=1}^{+\infty}d(n)a^n,
    \end{equation*}
    où pour tout entier $n\in\N^*$, $d(n)$ est le nombre de diviseurs positifs de $n$.
    \tcblower
    On va poser pour $n\in\N^*, ~ I_n=\{(p,q)\in(\N^*)^2\mid pq=n\}$ (recouvrement disjoint):
    \begin{align*}
        \sum_{n=1}^{+\infty}\frac{a^n}{1-a^n}&=\sum_{n=1}^{+\infty}\sum_{p=1}^{+\infty}(a^n)^p=\sum_{(p,q)\in(\N^*)^2}a^{pq}\\
        &=\sum_{n\in\N^*}\sum_{(p,q)\in I_n}a^{pq}=\sum_{n\in\N^*}a^n|I_n|\\
        &=\sum_{n\in\N^*}a^{n}d(n)
    \end{align*}
    En effet, $I_n=\{(p,\frac{n}{p})\mid p \nt{ divise } n\}$. Donc $|I_n|=d(n)$.
\end{ex}

\section{Sommer des nombres complexes.}

\subsection{Familles sommables de nombres complexes: l'espace \texorpdfstring{$\ell^1$}{Lg}.}

\begin{defi}{}{}
    Une famille $(u_i)_{i\in I}$ de $\K^I$ est dite \bf{sommable} si $\sum\limits_{i\in I}|u_i|<+\infty$.\\
    L'ensemble des familles sommables de $\K^I$ est noté $\ell^1(I)$
\end{defi}

\begin{prop}{}{}
    Si $(u_i)_{i\in I}$ est une famille sommable, alors toute sous-famille de $(u_i)_{i\in I'}$ ($I\subset I'$) l'est aussi.
    \tcblower
    Supposons $(u_i)_{i\in I}$ sommable. D'après \ref{prop:4}
    \begin{equation*}
        \sum{i\in I'}|u_i|\leq\sum_{i\in I}|u_i|<+\infty.
    \end{equation*}
\end{prop}

\begin{prop}{}{}
    Soit $(u_i)_{i\in I}\in\K^I$ et $(v_i)_{i\in I}$ une famille de réels positifs telles que
    \begin{equation*}
        \forall i \in I, ~ |u_i|\leq v_i.
    \end{equation*}
    Si la famille $(v_i)_{i\in I}$ est sommable, alors $(u_i)_{i\in I}\in\ell^1(I)$.
    \tcblower
    On a $(|u_i|)_{i\in I}$ et $(v_i)_{i\in I}$ sont des familles positives. D'après \ref{prop:5}:
    \begin{equation*}
        \sum_{i\in I}|u_i|\leq\sum_{i\in I}v_i < +\infty
    \end{equation*}
\end{prop}

\begin{prop}{}{}
    L'ensemble $\ell^1(I)$ des familles sommables de $\K^I$ est un sous-espace vectoriel de $\K^I$.
    \tcblower
    $\circledcirc$ Posons $(\theta_i)_{i\in I}=0_{\K^I}$. Alors $\sum\theta_i=0$, donc $(\theta_i)_{i\in I}\in\ell^1(I)$.\n
    $\circledcirc$ Soient $(u_i)_{i\in I}$ et $(v_i)_{i\in I}$ deux familles sommables et $\l,\mu\in\K$. On a:
    \begin{align*}
        \sum_{i\in I}\left| \l u_i + \mu v_i \right| &\leq \sum_{i\in I}|\l u_i| + \sum_{i\in I}|\mu v_i|\\
        &= |\l|\sum_{i\in I}|u_i| + |\mu|\sum_{i\in I}|v_i|\\
        &< +\infty
    \end{align*}
\end{prop}

\subsection{Somme d'une famille sommable de nombres complexes.}

\begin{defi}{Somme d'une famille sommable réelle.}{}
    Soit $(u_i)_{i\in I}$ une famille \bf{sommable} de nombre réels. On pose
    \begin{equation*}
        I_+=\{i\in I\mid u_i\geq0\}\quad\nt{et}\quad I_-=\{i\in I\mid u_i<0\}.
    \end{equation*}
    Les famille s$(u_i)_{i\in I_+}$ et $-(u_i)_{i\in I_-}$ sont des familles sommables de réels positifs.\n
    On appelle \bf{somme} de la famille $(u_i)_{i\in I}$ et on note $\sum\limits_{i\in I}u_i$ le nombre réel défini par
    \begin{equation*}
        \sum_{i\in I}u_i=\sum_{i\in I_+}u_i - \sum_{i\in I_-}(-u_i)
    \end{equation*}
\end{defi}

\begin{defi}{Somme d'une famille sommable complexe.}{}
    Soit $(u_i)_{i\in I}$ une famille \bf{sommable} de nombres complexes.\\
    Les familles $(\Re(u_i))_{i\in I}$ et $(\Im(u_i))_{i\in I}$ sont des familles réelles sommables.\n
    On appelle \bf{somme} de la famille $(u_i)_{i\in I}$ et on note $\sum\limits_{i\in I}u_i$ le nombre complexe défini par
    \begin{equation*}
        \sum_{i\in I}u_i = \sum_{i\in I}\Re(u_i)+i\sum_{i\in I}\Im(u_i).
    \end{equation*}
\end{defi}

\begin{prop}{Lien avec les séries.}{}
    Une famille $(u_n)_{n\in\N}$ est sommable ssi la série $\sum u_n$ est absolument convergente.\\
    Si c'est le cas,
    \begin{equation*}
        \sum_{n\in\N}u_n=\sum_{n=0}^{+\infty}u_n.
    \end{equation*} 
    Le membre de gauche étant la somme de la famille, et celui de droite la somme de la série.
    \tcblower
    Proposition \ref{prop:6}.
    \begin{equation*}
        (u_n)_{n\in\N}\in\ell^1(I)\iff\sum_{n\in\N}|u_n|<+\infty\iff\sum_{n=0}^{+\infty}|u_n|<+\infty\iff\sum u_n\nt{ CVA.}.
    \end{equation*}
\end{prop}

\begin{prop}{Approcher la somme par une somme finie.}{}
    Soit $(u_i)_{i\in I}$ une famille sommable. Pour tout $\e>0$, il existe une partie $F$ finie de $I$ telle que
    \begin{equation*}
        \left|\sum_{i\in I}u_i - \sum_{i\in F}u_i\right|\leq \e
    \end{equation*}
    \tcblower
    Idée de preuve (cas réel). Soit $F$ partie finie de $I$. On a
    \begin{align*}
        \left|\sum_{i\in I}u_i - \sum_{i\in F}u_i\right|&=\left|\sum_{i\in I_+}u_i - \sum_{i\in I_-}(-u_i)-\sum_{i\in F\cap I_+}u_i + \sum_{i\in F\cap I_-}(-u_i)\right|\\
        &\leq \left|\sum_{i\in I_+} u_i - \sum_{i\in F\cap I_+}u_i\right| + \left| \sum_{i\in I_-}(-u_i) - \sum_{i\in F\cap I_-}(-u_i)\right|\\
        &\leq \left|\sum_{i\in I_+\setminus F}u_i\right| + \left|\sum_{i\in I_-\setminus F}(-u_i)\right|\\
        &\leq \frac{\e}{2} + \frac{\e}{2} \quad \nt{caractérisation de la borne sup}.\\
        &\leq \e
    \end{align*}
\end{prop}

\begin{thm}{Linéarité de la somme.}{}
    \begin{center}
        $(u_i)_{i\in I}\mapsto \sum\limits_{i\in I}u_i$ est une forme linéaire sur $\ell^1(I)$.
    \end{center}
    \tcblower
    Admis.
\end{thm}

\begin{prop}{Croissance de la somme.}{}
    Si $(u_i)_{i\in I}$ et $(v_i)_{i\in I}$ sont deux familles sommables telles que $\forall i\in I,~u_i\leq v_i$, alors:
    \begin{equation*}
        \sum_{i\in I}u_i\leq\sum_{i\in I}v_i.
    \end{equation*}
    \tcblower
    Admis.
\end{prop}

\begin{corr}{Inégalité triangulaire.}{}
    \begin{equation*}
        \forall (u_i)_{i\in I}\in\ell^1(I),~\left|\sum_{i\in I}u_i\right|\leq\sum_{i\in I}|u_i|.
    \end{equation*}
    \tcblower
    Admis.
\end{corr}

\begin{thm}{Permutation des termes de la somme d'une famille sommable.}{}
    Soit $(u_i)_{i\in I}$ une famille sommable de nombre complexes et $\s$ une bijection de $I$ dans $I$. On a
    \begin{equation*}
        \sum_{i\in I}u_{\s(i)}=\sum_{i\in I}u_i.
    \end{equation*}
    En particulier, on peut permuter les termes de la somme d'une série absolument convergente.
    \tcblower
    Admis.
\end{thm}

\subsection{Sommation par paquets.}

\begin{thm}{de sommation par paquets.}{}
    Soit $(u_i)_{i\in I}$ une famille de nombres complexes.\\
    On suppose que $I$ s'écrit comme une réunion disjointe $I=\bigcup\limits_{j\in J}I_j$.\\
    Si $u$ est sommable, alors
    \begin{equation*}
        \sum_{j\in J}\left( \sum_{i\in I_j}u_i \right) = \sum_{i\in I}u_i.
    \end{equation*}
\end{thm}

\begin{thm}{de Fubini.}{}
    Soit $(u_{i,j})_{(i,j)\in I\times J}$ une famille de nombres complexes indexée par un produit cartésien $I\times J$.\\
    Si $u$ est \bf{sommable},
    \begin{equation*}
        \sum_{(i,j)\in I\times J} u_{i,j}=\sum_{i\in I}\sum_{j\in J}u_{i,j}=\sum_{j\in J}\sum_{i\in I}u_{i,j}.
    \end{equation*}
\end{thm}

\begin{ex}{Sommes triangulaires.}{}
    Soit $(u_{n,p})_{(n,p)\in\N^2}$ une famille sommable de nombres complexes. On a
    \begin{equation*}
        \sum_{(n,p)\in\N^2}u_{n,p}=\sum_{n=0}^{+\infty}\sum_{k=0}^nu_{k,n-k}.
    \end{equation*}
\end{ex}

\begin{meth}{Le calcul d'abord, la justification ensuite.}{}
    En partique, on pourra écrire une sommation par paquets ou un échange de somme "sous réserve de sommabilité", le temps de voir si on aboutit ainsi à un résultat intéressant.\n
    Le cas échéant, il est encore temps de prouver la sommabilité en sommant les modules.\\
    On insiste sur le fait que les calculs de somme sur les modules sont justifiés par la seule positivité!
\end{meth}

\begin{ex}{Retour sur un exemple.}{}
    Soit $z\in\C$ tel que $|z|<1$. Démontrer que la famille $(z^{pq})_{(p,q)\in(\N^*)^2}$ est sommable.\\
    Montrer que
    \begin{equation*}
        \sum_{n=1}^{+\infty}\frac{z^n}{1-z^n}=\sum_{n=1}^{+\infty}d(n)z^n,
    \end{equation*}
    où pour tout entier $n\in\N^*$, $d(n)$ est le nombre de diviseurs positifs de $n$.
\end{ex}

\begin{ex}{}{}
    Soit $z\in\C$ tel que $|z|<1$. Montrer que:
    \begin{equation*}
        \sum_{n=1}^{+\infty}\frac{z^{2n-1}}{1-z^{2n-1}}=\sum_{n=1}^{+\infty}\frac{z^n}{1-z^{2n}},\quad\nt{et}\quad\sum_{n=0}^{+\infty}\frac{z^{2^n}}{1-z^{2^{n+1}}}=\frac{z}{1-z}.
    \end{equation*}
    \tcblower
    On a:
    \begin{equation*}
        \sum_{n=1}^{+\infty}\frac{2^{2n-1}}{1-z^{2n-1}}=\sum_{n=1}^{+\infty}\sum_{p=1}^{+\infty}(z^{2n-1})^p=\sum_{p=1}^{+\infty}\sum_{n=1}^{+\infty}z^{(2n-1)p}=\sum_{p=1}^{+\infty}z^{-p}\sum_{n=1}^{+\infty}(z^{2p})^n=\sum_{p=1}^{+\infty}\frac{z^p}{1-z^{2p}}.
    \end{equation*}
    Justifions que ces opérations sont possible: que $(z^{(2n-1)p})_{(n,p)\in(\N^*)^2}$ est sommable.\\
    Calculons la somme des modules:
    \begin{equation*}
        \sum_{(n,p)\in(\N^*)^2}|z^{(2n-1p)}|=\sum_{n=1}^{+\infty}\sum_{p=1}^{+\infty}|z|^{(2n-1)p}=\sum_{p=1}^{+\infty}\frac{|z|^p}{1-|z|^{2p}}
    \end{equation*}
    Convergente car $\frac{|z|^p}{1-|z|^{2p}}\sim|z|^p$ avec $|z|<1$.\n
    On a:
    \begin{equation*}
        \sum_{n=0}^{+\infty}\frac{z^{2^n}}{1-z^{2^{n+1}}}=\sum_{n=0}^{+\infty}z^{2^n}\sum_{p=0}^{+\infty}(z^{2^{n+1}})^p=\sum_{n=0}^{+\infty}\sum_{p=0}^{+\infty}z^{2^n(2p+1)}
    \end{equation*}
    Pour $q\in\N^*$, $I_q=\{(n,p)\in\N^2\mid 2^n(2p+1)=q\}$, on a $\N^2=\bigcup_{q\in\N^*}I_q$ et les $I_q$ sont disjoints deux-à-deux.\\
    Soit $(n,p)\in\N^2$, alors $(n,p)\in I_{2^n(2p+1)}$.\\
    Par sommation par paquets:
    \begin{equation*}
        \sum_{n=0}^{+\infty}\frac{z^{2^n}}{1-z^{2^{n+1}}}=\sum_{(n,p)\in\N^2}z^{2^n(2p+1)}=\sum_{q\in\N^*}\sum_{(n,p)\in I_q}z^{2^n(2p+1)}=\sum_{q\in\N^*}z^q|I_q|=\sum_{q\in\N^*}z^q=\frac{z}{1-z}.
    \end{equation*}
    En effet, pour $q\in\N^*$, d'après le TFAr, $q$ s'écrit de manière unique sous la forme $q=2^np_1^{\a_1}...p_r^{\a_r}$. Or les $p_i$ sont premiers impairs donc leur produit est impair, noté $(2p+1)$ $(p\in\N)$. On a donc $q=2^n(2p+1)$, donc $|I_q|=1$.\\
    Il reste à prouver que $(z^{2^n(2p+1)})_{(n,p)\in\N^2}$ est sommable.
    \begin{equation*}
        \sum_{(n,p)\in\N^2}|z^{2^n(2p+1)}|=\sum_{(n,p)\in\N^2}|z|^{2^n(2p+1)}=\frac{|z|}{1-|z|}<+\infty.
    \end{equation*}
\end{ex}


\subsection{Produits.}

\begin{prop}{}{}
    Si $(a_i)_{i\in I}$ et $(b_j)_{j\in J}$ sont deux familles sommables, alors la famille $(a_ib_j)_{(i,j)\in I\times J}$ est sommable et
    \begin{equation*}
        \sum_{(i,j)\in I\times J}a_ib_j=\left( \sum_{i\in I}a_i \right) \left( \sum_{j\in J} b_j \right)
    \end{equation*}
    Ce résultat s'étend à un produit à un produit fini de familles sommables.
    \tcblower
    On somme les modules.
    \begin{equation*}
        \sum_{(i,j)\in I \times J}|a_ib_j| = \sum_{i\in I}\sum_{i\in J}|a_i||b_j| = \left(\sum_{i\in I}|a_i|\right)\left(\sum_{j\in J}|b_j|\right)<+\infty.
    \end{equation*}
    La famille est donc sommable. Alors:
    \begin{equation*}
        \sum_{(i,j)\in I \times J}a_ib_j = \sum_{i\in I}\sum_{j\in J}a_ib_j = \left( \sum_{i\in I} a_i \right)\left( \sum_{j\in J} b_j \right).
    \end{equation*}
\end{prop}

\pagebreak

\begin{thm}{Produit de Cauchy de deux séries absolument convergentes.}{}
    Soit $\sum a_n$ et $\sum b_n$ deux séries de nombres complexes toutes deux absolument convergents.\n
    La série de terme général
    \begin{equation*}
        c_n:=\sum_{p+q=n}a_pb_q=\sum_{k=0}^na_kb_{n-k},
    \end{equation*}
    est absolument convergente. On l'appelle \bf{produit de Cauchy} de $\sum a_n$ et $\sum b_n$.\\
    On a
    \begin{equation*}
        \left( \sum_{n=0}^{+\infty} a_n \right) \left( \sum_{n=0}^{+\infty} b_n  \right) = \sum_{n=0}^{+\infty} \left( \sum_{p+q=n} a_pb_q \right)
    \end{equation*}
    \tcblower
    $\circledcirc$ Montrons que $\sum c_n$ CVA. Sommons les modules:
    \begin{equation*}
        \sum_{n=0}^{+\infty}|c_n|=\sum_{n=0}^{+\infty}\left|\sum_{k=0}^na_kb_{n-k}\right|\leq\sum_{n=0}^{+\infty}\sum_{k=0}^n|a_k||b_{n-k}|=\sum_{i=0}^{+\infty}\sum_{j=0}^{+\infty}|a_i||b_j|=\left( \sum_{i=0}^{+\infty}a_i \right)\left( \sum_{j=0}^{+\infty} b_j \right)<+\infty.
    \end{equation*}
    $\circledcirc$ En refaisant le calcul sans les modules, on obtient bien l'égalité.
\end{thm}

\begin{ex}{}{}
    Démontrer que pour tout $z\in\C$ tel que $|z|<1$,
    \begin{equation*}
        \frac{1}{(1-z)^2}=\sum_{n=0}^{+\infty}(n+1)z^n.
    \end{equation*}
    \tcblower
    On a:
    \begin{equation*}
        \frac{1}{(1-z^2)}=\frac{1}{1-z}\times\frac{1}{1-z}=\left(\sum_{n=0}^{+\infty}z^n\right)\left(\sum_{n=0}^{+\infty}z^n\right)=\sum_{n=0}^{+\infty}\sum_{k=0}^nz^kz^{n-k}=\sum_{n=0}^{+\infty}(n+1)z^n
    \end{equation*}
    On a utilisé le théorème du produit de Cauchy, car les séries sont absolument convergentes.
\end{ex}

\begin{ex}{Propriété de morphisme de l'exponentielle, et retour sur le début de l'année.}{}
    Pour tout $z\in\C$ on définit le nombre $\exp(z)$ par
    \begin{equation*}
        \exp(z):=\sum_{n=0}^{+\infty}\frac{z^n}{n!},
    \end{equation*}
    (on avait prouvé qu'il s'agit bien d'une série convergente avec d'Alembert).\n
    Démontrer que
    \begin{equation*}
        \forall (z,\tilde{z})\in\C^2\quad\exp(z+\tilde{z})=\exp(z)\exp(\tilde{z}).
    \end{equation*}
    Démontrer que $\exp:x\mapsto\exp(x)$ est dérivable sur $\R$ et qu'elle est sa propre dérivée.
    \tcblower
    Soient $z,\tilde{z}\in\C^2$. On a $\exp(z)$ et $\exp(\tilde{z})$ les sommes de deux séries absolument convergentes.
    On écrit le produit de Cauchy:
    \begin{align*}
        \exp(z)\exp(\tilde{z})&=\left(\sum_{n=0}^{+\infty}\frac{z^n}{n!}\right)\left( \sum_{n=0}^{+\infty}\frac{\tilde{z}^n}{n!} \right)=\sum_{n=0}^{+\infty}\sum_{k=0}^n\frac{z^k}{k!}\times\frac{\tilde{z}^{n-k}}{(n-k)!}\\
        &=\sum_{n=0}^{+\infty}\frac{1}{n!}\sum_{k=0}^{n}\binom{n}{k}z^{k}\tilde{z}^{n-k}=\sum_{n=0}^{+\infty}\frac{1}{n!}(z+\tilde{z})^n\\
        &=\exp(z+\tilde{z})
    \end{align*}
    Soit $a\in \R$ et $h\neq0$. On a
    \begin{align*}
        \left|\frac{\exp(a+h)-\exp(a)}{h}-\exp(a)\right|&=\exp(a)\left| \frac{\exp(h)-1}{h} - 1 \right|\\
        &=\exp(a)\left| \frac{\exp(h)-1-h}{h} \right|\\
        &=\exp(a)\left| \frac{1}{h}\sum\limits_{n=2}^{+\infty}\frac{h^n}{n!} \right|\\
        &=\exp(a)\left| h\sum_{n=2}^{+\infty}\frac{h^{n-2}}{n!} \right|\\
        &\leq |h|\sum_{n=2}^{+\infty}\left|\frac{h^n}{n!}\right|\\
        &\leq |h|\sum_{n=2}^{+\infty}\frac{1}{n!} \xrightarrow[h\to0]{} 0
    \end{align*}
    Donc $\frac{\exp(a+h)-\exp(a)}{h}\xrightarrow[h\to0]{}\exp(a)$.\\
    Donc $\exp$ est dérivable en $a$ et $\exp'(a)=\exp(a)$. 
\end{ex}

\begin{thm}{Théorème de réarrangement de Riemann.}{}
    Soit $\sum u_n$ une série de nombres réels convergente, mais pas absolument. Alors pour tout $l\in\overline{\R}$, il existe une bijection $\s:\N\to\N$ telle que \begin{equation*}\sum_{n=0}^Nu_{\s(n)}\xrightarrow[N\to+\infty]{}l.\end{equation*}
    \tcblower
    Hors-programme.
\end{thm}

\pagebreak

\section{Exercices.}

\begin{exercice}{$\blacklozenge\blacklozenge\lozenge$}{}
    Calculer la somme de \Large$\left( \frac{1}{(i+j^2)(i+j^2+1)} \right)_{i\geq 0,j\geq1}$ \normalsize (on suppose $\zeta(2)$ connu). En déduire
    \begin{equation*}
        \sum_{n=1}^{+\infty}\frac{\lf \sqrt{n} \rf}{n(n+1)}.
    \end{equation*}
    \tcblower
    On a:
    \begin{align*}
        \sum_{(i,j)\in\N\times\N^*}\frac{1}{(i+j^2)(i+j^2+1)}&=\sum_{j=1}^{+\infty}\sum_{i=0}^{+\infty}\frac{1}{(j^2+i)(j^2+i+1)}\\
        &=\sum_{j=1}^{+\infty}\sum_{i=0}^{+\infty}\left( \frac{1}{j^2+i}-\frac{1}{j^2+i+1} \right)\\
        &=\sum_{j=1}^{+\infty}\frac{1}{j^2}=\frac{\pi^2}{6}
    \end{align*}
    Pour $n\in\N^*$, on définit $I_n=\{(i,j)\in\N\times\N^*\mid n=i+j^2\}$ (recouvrement disjoint).
    \begin{align*}
        \sum_{(i,j)\in\N\times\N^*}\frac{1}{(i+j^2)(i+j^2+1)}&=\sum_{n\in\N^*}\sum_{(i,j)\in I_n}\frac{1}{(i+j^2)(i+j^2+1)}\\
        &=\sum_{n\in\N^*}\frac{1}{n(n+1)}|I_n|\\
        &=\sum_{n\in\N^*}\frac{\lf \sqrt{n} \rf}{n(n+1)}
    \end{align*}
    En effet, $I_n=\{(n-j^2,j^2)\mid j^2\in\lb1,n\rb\}=\{(n-j^2,j^2)\mid j\in\lb1,\sqrt{n}\rb\}$. Donc $|I_n|=\lf\sqrt{n}\rf$.\n
    En conclusion, cette somme vaut $\zeta(2)$.
\end{exercice}

\begin{exercice}{$\blacklozenge\blacklozenge\lozenge$}{}
    Montrer que \Large $\left( \frac{1}{(|p| + |q|)^s} \right)_{(p, q)\in\Z^{2} \setminus \{(0, 0)\}}$ \normalsize est une famille sommable si et seulement si $s > 2$.\n
    Exprimer alors sa somme à l'aide de la fonction $\zeta$.
    \tcblower
    Posons $I_{n} = \{(p, q) ~|~ (p, q) \in \Z^{2} \setminus \{(0, 0)\} ~|~ |p| + |q| = n \}$ (recouvrement disjoint).\n
    Ainsi $I_{n} = \{((-1)^{u}p, (-1)^{v}q) ~|~ (p, q) \in \N^{2} \setminus \{(0, 0)\} ~|~ (u, v) \in \{0, 1 \}^{2} ~|~ p + q = n\}$.\n
    Donc $I_{n} = \{((-1)^{u}p, (-1)^{v}(n-p)) ~|~ 1 \leq p \leq n ~|~ (u, v) \in \{0, 1 \}^{2}\}$.\n
    \begin{align*}
        \sum_{(p, q)\in\Z^{2} \setminus \{(0, 0)\}} \frac{1}{(|p| + |q|)^s} &= \sum_{n=1}^{+\infty} \sum_{(p, q) \in I_{n}} \frac{1}{(|p| + |q|)^s} ~~~~~~(\text{paquets positifs})\n
        &= \sum_{n=1}^{+\infty} \frac{1}{n^s} \times |I_{n}| ~~~~~~~~~~~~~~~~~~((p, q) \in I_{n}) \n
        &= \sum_{n=1}^{+\infty} \frac{4}{n^{s-1}} ~~~~~~~~~~~~~~~~~~~~~~~~(|I_{n}| = 4n)\n
    \end{align*}
    Ainsi par comparaison ( Serie de Riemman ) : \n
        \Large $\left( \frac{1}{(|p| + |q|)^s} \right)_{(p, q)\in\Z^{2} \setminus \{(0, 0)\}}$ \normalsize est une famille sommable si et seulement si s > 2. \n
        Ainsi 
        \begin{equation*}
        \forall s > 2~,~ \sum_{(p, q)\in\Z^{2} \setminus \{(0, 0)\}} \frac{1}{(|p| + |q|)^s} = 4~\zeta(s-1)
        \end{equation*}
\end{exercice}

\pagebreak

\begin{exercice}{$\blacklozenge\blacklozenge\lozenge$ Mines MP 2019}{}
    \begin{equation*}
       \text{Calculer } \sum_{n=2}^{+\infty} (n-1)(2^{n}(\zeta(n)-1) -1)
    \end{equation*}
    \tcblower
    \begin{align*}
        \sum_{n=2}^{+\infty} (n-1)(2^{n}(\zeta(n)-1) -1) &= \sum_{n=2}^{+\infty} (n-1)(2^{n}(\sum_{k=2}^{+\infty} \frac{1}{k^{n}}) -1) \n
        &= \sum_{n=2}^{+\infty} (n-1)(\sum_{k=2}^{+\infty} (\frac{2}{k})^{n} -1) \n
        &= \sum_{n=2}^{+\infty} \sum_{k=3}^{+\infty} (n-1)(\frac{2}{k})^{n} \n
        &= \sum_{k=3}^{+\infty} \sum_{n=2}^{+\infty} (n-1)(\frac{2}{k})^{n} ~~~~~~~~~~ (\text{Fubini positif}) \n
        &= \sum_{k=3}^{+\infty} (\frac{2}{k})^2 \sum_{n=2}^{+\infty} (n-1)(\frac{2}{k})^{n-2} \n
        &= \sum_{k=3}^{+\infty} (\frac{2}{k})^2 (\sum_{n=0}^{+\infty} (n+1)(\frac{2}{k})^{n}) \n
        &= \sum_{k=3}^{+\infty} (\frac{2}{k})^2 \frac{1}{(1-\frac{2}{k})^2} ~~~~~~~~~~ (\text{Exemple 36: } \forall k>3~,~ |\frac{2}{k}| < 1)\n
        &= \sum_{k=3}^{+\infty} (\frac{\frac{2}{k}}{1-\frac{2}{k}})^2 \n
        &= \sum_{k=3}^{+\infty} (\frac{2}{k-2})^2 \n
        &= 4\sum_{k=1}^{+\infty} (\frac{1}{k})^2 \n
        &= 4\zeta(2) \n
    \end{align*}
\end{exercice}

\begin{exercice}{$\blacklozenge\blacklozenge\lozenge$ Calcul de $\zeta(4)$}{}
    Pour tout $p > 1$, on note $\zeta(p) = \sum_{n=1}^{+\infty}\frac{1}{n^p}$ et on rappelle que $\zeta(2) = \frac{\pi^2}{6}$.\n
    Pour tout $(n, m) \in {\N^{*}}^{2}$, on note
    \begin{equation*}
        f(n, m) = \frac{2}{n^{3}m} + \frac{1}{n^{2}m^{2}} + \frac{2}{nm^{3}}
    \end{equation*}
    et
    \begin{equation*}
        u_{n, m} = f(n, m) - f(n, m + n) - f(m, m + n)
    \end{equation*}
    
    \begin{enumerate}
        \item Monter que, pour tout $(n, m) \in {\N^{*}}^{2}$, 
        \begin{equation*}
            u_{n, m} = \frac{2}{n^{2}m^{2}}
        \end{equation*}
        \item En deduire que la famille $\left( u_{n, m} \right)_{(p, q)\in {\N^{*}}^{2}}$ est sommable et calculer 
        \begin{align*}
            \sum_{n=1}^{+\infty} \sum_{m=1}^{+\infty} u_{n, m}
        \end{align*}
        \item Soit $N \in \N^{*}$. Exprimer $\sum_{n=1}^{+\infty} \sum_{m=1}^{+\infty} u_{n, m}$ en fonction de $\zeta(4)$ puis montrer que 
        \begin{equation*}
            \zeta(4) = \frac{\pi^{4}}{90}
        \end{equation*}
    \end{enumerate}

    \tcblower
    \boxed{1} 
    \begin{align*}
        u_{n, m} &= f(n, m) - f(n, m + n) - f(m, m + n) \n
    \end{align*}
    \Huge\centering TODO 
\end{exercice}

\end{document}