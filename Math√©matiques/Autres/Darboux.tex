\documentclass[10pt]{article}

\usepackage[T1]{fontenc}
\usepackage[left=2cm, right=2cm, top=2cm, bottom=2cm, paperheight=29cm]{geometry}
\usepackage[skins]{tcolorbox}
\usepackage{hyperref, fancyhdr, lastpage, tocloft, ragged2e, multicol, hyperref}
\usepackage{amsmath, amssymb, amsthm, stmaryrd}
\usepackage{tkz-tab}
\usepackage{systeme}


\renewcommand*{\epsilon}{\varepsilon}
\def\pagetitle{Bibmaths - Théorème de Darboux}
\setlength{\headheight}{13pt}

\title{\bf{\pagetitle}}
\author{DARVOUX Théo}

\hypersetup{
    colorlinks=true,
    citecolor=black,
    linktoc=all,
    linkcolor=blue
}

\pagestyle{fancy}
\cfoot{\thepage\ sur \pageref*{LastPage}}


\begin{document}
\maketitle
\Large{Un seul exercice parce que le nom est drôle : \href{https://www.bibmath.net/ressources/justeunexo.php?id=342}{\color{blue} Énoncé.}}

\thispagestyle{fancy}
\fancyhead[L]{MP2I Paul Valéry}
\fancyhead[C]{\pagetitle}
\fancyhead[R]{2023-2024}
\allowdisplaybreaks

\section*{ [$\bigstar\bigstar\bigstar\bigstar$]}
\begin{tcolorbox}[enhanced, width=7in, center, size=fbox, fontupper=\large, drop shadow southwest]
Soit $I$ un intervalle ouvert de $\mathbb{R}$, et $f$ une fonction dérivable sur $I$.\\
Montrons que $f'$ vérifie le TVI.\\
1. Ce n'est pas trivial car la dérivée n'est pas toujours continue.\\
2. Soient $a,b\in I^2 ~ | ~ f'(a) < f'(b)$ et $z\in]f'(a), f'(b)[$\\
Soit $\varepsilon>0$.
\begin{align*}
    &\frac{f(a+h)-f(a)}{h}\xrightarrow[h\to0]{} f'(a) \Rightarrow \exists \alpha_1>0,\forall h\in\,]0, \alpha_1], \frac{f(a+h)-f(a)}{h}\in [f'(a) - \epsilon, f'(a) + \epsilon]\\
    &\frac{f(b+h)-f(b)}{h}\xrightarrow[h\to0]{} f'(b) \Rightarrow \exists \alpha_2>0, \forall h \in ]0, \alpha_1], \frac{f(b+h)-f(b)}{h}\in [f'(b)-\epsilon, f'(b)+\epsilon]
\end{align*}
On pose donc $\alpha = \min(\alpha_1, \alpha_2)$ et $\epsilon = \min(z - f'(a), f'(b)-z)$ car $z\in]f'(a), f'(b)[$.
\begin{align*}
    &\frac{f(a+h)-f(a)}{h}\leq f'(a) + \epsilon \leq z\\
    &\frac{f(b+h)-f(b)}{h}\geq f'(b) - \epsilon \geq z
\end{align*}
Alors :
\begin{equation*}
    \frac{f(a+h)-f(a)}{h}\leq z \leq \frac{f(b+h)-f(b)}{h}
\end{equation*}
3. Soit $h\in[0,\alpha]$ la fonction $T_a : x\mapsto \frac{f(x+h)-f(x)}{h}$ est continue sur $I$.\\
On a $T_a(a) \leq z \leq T_a(b)$ donc par TVI, $\exists y\in [a,b] ~ | ~ T_a(y) = z$ et $y+h\in I$.\\[0.2cm]
4. $f$ est continue sur $[y,y+h]$ et dérivable sur $]y,y+h[$.\\
D'après le TAF, $\exists x\in ]y,y+h[ ~ | ~ \frac{f(y+h) - f(y)}{h} = z = f'(x)$.\\[0.2cm]
5. $\forall a,b\in I, \forall z \in ]f'(a), f'(b)|, \exists x \in I ~ | ~ z = f'(x) \iff z \in f(I)$.\\[0.2cm]
6. Pour $x\in]0,1], f'(x)=2x\sin(\frac{1}{x^2})-\frac{2}{x}\cos(\frac{1}{x})$.
\begin{equation*}
    \left|\frac{f(x)-f(0)}{x-0}\right|=|x\sin\left(\frac{1}{x}\right)|\leq|x|\xrightarrow[x\to0]{}0
\end{equation*}
Alors $f$ est dérivable en 0 et $f'(0)=0$.\\
On a $|f'(x)|\leq|2x-\frac{2}{x}|\xrightarrow[x\to0]{}+\infty$ donc $f'$ n'est pas continue en 0.\\
\end{tcolorbox}
\end{document}
 