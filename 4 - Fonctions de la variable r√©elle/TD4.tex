\documentclass[10pt]{article}

\usepackage[T1]{fontenc}
\usepackage[left=2cm, right=2cm, top=2cm, bottom=2cm]{geometry}
\usepackage[skins]{tcolorbox}
\usepackage{hyperref, fancyhdr, lastpage, tocloft, ragged2e, multicol}
\usepackage{amsmath, amssymb, amsthm}
\usepackage{tkz-tab}

\def\pagetitle{Fonctions usuelles}

\title{\bf{\pagetitle}\\\large{Corrigé}}
\date{Septembre 2023}
\author{DARVOUX Théo}

\DeclareMathOperator{\ch}{ch}
\DeclareMathOperator{\sh}{sh}
\DeclareMathOperator{\tah}{th}

\hypersetup{
    colorlinks=true,
    citecolor=black,
    linktoc=all,
    linkcolor=blue
}

\pagestyle{fancy}
\cfoot{\thepage\ sur \pageref*{LastPage}}


\begin{document}
\renewcommand*\contentsname{Exercices.}
\renewcommand*{\cftsecleader}{\cftdotfill{\cftdotsep}}
\maketitle
\hrule
\tableofcontents
\vspace{0.5cm}
\hrule

\thispagestyle{fancy}
\fancyhead[L]{MP2I Paul Valéry}
\fancyhead[C]{\pagetitle}
\fancyhead[R]{2023-2024}
\allowdisplaybreaks

\addcontentsline{toc}{section}{Vocabulaire sur les fonctions.}

\section*{Exercice 4.1 [$\blacklozenge\lozenge\lozenge$]}
\begin{tcolorbox}[enhanced, width=6in, center, size=fbox, fontupper=\large, drop shadow southwest]
    Soit $f:\mathbb{R}\rightarrow\mathbb{R}$ une fonction $2$-périodique et $3$-périodique. Montrer que $f$ est $1$-périodique.\\
    On a :
    \begin{align*}
        \forall{x\in\mathbb{R}}\begin{cases}x-2\in\mathbb{R}\\f(x-2)=f(x)\end{cases} \text{ et } \begin{cases}x+3\in\mathbb{R}\\f(x+3)=f(x)\end{cases}
    \end{align*}
    Alors :
    \begin{align*}
        \forall{x\in\mathbb{R}}\begin{cases}x-2+3\in\mathbb{R}\\f(x-2+3)=f(x-2)=f(x)\end{cases}
    \end{align*}
    \qed
\end{tcolorbox}
\addcontentsline{toc}{section}{\protect\numberline{}Exercice 4.1}

\end{document}