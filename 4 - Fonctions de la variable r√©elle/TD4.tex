\documentclass[10pt]{article}

\usepackage[T1]{fontenc}
\usepackage[left=2cm, right=2cm, top=2cm, bottom=2cm]{geometry}
\usepackage[skins]{tcolorbox}
\usepackage{hyperref, fancyhdr, lastpage, tocloft, ragged2e, multicol}
\usepackage{amsmath, amssymb, amsthm}
\usepackage{tkz-tab}

\def\pagetitle{Fonctions usuelles}

\title{\bf{\pagetitle}\\\large{Corrigé}}
\date{Septembre 2023}
\author{DARVOUX Théo}

\DeclareMathOperator{\ch}{ch}
\DeclareMathOperator{\sh}{sh}
\DeclareMathOperator{\tah}{th}

\hypersetup{
    colorlinks=true,
    citecolor=black,
    linktoc=all,
    linkcolor=blue
}

\pagestyle{fancy}
\cfoot{\thepage\ sur \pageref*{LastPage}}


\begin{document}
\renewcommand*\contentsname{Exercices.}
\renewcommand*{\cftsecleader}{\cftdotfill{\cftdotsep}}
\maketitle
\hrule
\tableofcontents
\vspace{0.5cm}
\hrule

\thispagestyle{fancy}
\fancyhead[L]{MP2I Paul Valéry}
\fancyhead[C]{\pagetitle}
\fancyhead[R]{2023-2024}
\allowdisplaybreaks

\addcontentsline{toc}{section}{Vocabulaire sur les fonctions.}

\section*{Exercice 4.1 [$\blacklozenge\lozenge\lozenge$]}
\begin{tcolorbox}[enhanced, width=6in, center, size=fbox, fontupper=\large, drop shadow southwest]
    Soit $f:\mathbb{R}\rightarrow\mathbb{R}$ une fonction $2$-périodique et $3$-périodique. Montrer que $f$ est $1$-périodique.\\
    On a :
    \begin{align*}
        \forall{x\in\mathbb{R}}\begin{cases}x-2\in\mathbb{R}\\f(x-2)=f(x)\end{cases} \text{ et } \begin{cases}x+3\in\mathbb{R}\\f(x+3)=f(x)\end{cases}
    \end{align*}
    Alors :
    \begin{align*}
        \forall{x\in\mathbb{R}}\begin{cases}x-2+3\in\mathbb{R}\\f(x-2+3)=f(x-2)=f(x)\end{cases}
    \end{align*}
    \qed
\end{tcolorbox}
\addcontentsline{toc}{section}{\protect\numberline{}Exercice 4.1}

\section*{Exercice 4.2 [$\blacklozenge\blacklozenge\blacklozenge$]}
\begin{tcolorbox}[enhanced, width=6in, center, size=fbox, fontupper=\large, drop shadow southwest]
    Déterminer toutes les fonctions croissantes $f:\mathbb{R}\rightarrow\mathbb{R}$ telles que
    \begin{equation*}
        \forall{x\in\mathbb{R}} \hspace{0.25cm} f(f(x))=x.
    \end{equation*}
    Soit $x\in\mathbb{R}$ et $f$ une solution du problème.\\
    On remarque que $f:x\mapsto x$ est solution du problème.\\
    Supposons $f(x)>x$, on a : $f(f(x))>f(x)$ par croissance de $f$. Or $f(f(x))=x$ donc $x>f(x)$, ce qui est absurde.\\
    Supposons $f(x)<x$, on a : $f(f(x))<f(x)$ par croissance de $f$. Or $f(f(x))=x$ donc $x<f(x)$, ce qui est absurde.\\
    Ainsi, la seule fonction de $\mathbb{R}$ vers $\mathbb{R}$ solution est $f:x\mapsto x$.
\end{tcolorbox}
\addcontentsline{toc}{section}{\protect\numberline{}Exercice 4.2}

\end{document}