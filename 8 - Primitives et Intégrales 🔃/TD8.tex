\documentclass[10pt]{article}

\usepackage[T1]{fontenc}
\usepackage[left=2cm, right=2cm, top=2cm, bottom=2cm]{geometry}
\usepackage[skins]{tcolorbox}
\usepackage{hyperref, fancyhdr, lastpage, tocloft, ragged2e, multicol}
\usepackage{amsmath, amssymb, amsthm, stmaryrd}
\usepackage{tkz-tab}

\def\pagetitle{Primitives et intégrales}

\title{\bf{\pagetitle}\\\large{Corrigé}}
\date{Octobre 2023}
\author{DARVOUX Théo}

\hypersetup{
    colorlinks=true,
    citecolor=black,
    linktoc=all,
    linkcolor=blue
}

\DeclareMathOperator{\ch}{ch}
\DeclareMathOperator{\sh}{sh}
\DeclareMathOperator{\tah}{th}

\pagestyle{fancy}
\cfoot{\thepage\ sur \pageref*{LastPage}}


\begin{document}
\renewcommand*\contentsname{Exercices.}
\renewcommand*{\cftsecleader}{\cftdotfill{\cftdotsep}}
\maketitle
\hrule
\tableofcontents
\vspace{0.5cm}
\hrule

\thispagestyle{fancy}
\fancyhead[L]{MP2I Paul Valéry}
\fancyhead[C]{\pagetitle}
\fancyhead[R]{2023-2024}
\allowdisplaybreaks

\pagebreak

\section*{Exercice 8.1 [$\blacklozenge\lozenge\lozenge$]}
\begin{tcolorbox}[enhanced, width=7in, center, size=fbox, fontupper=\large, drop shadow southwest]
    Donner les primitives des fonctions suivantes (on précisera l'intervalle que l'on considère).
    \begin{align*}
        &a:x\mapsto\cos{xe^{\sin{x}}}; \hspace{1cm} b:x\mapsto\frac{\cos x}{\sin x}; \hspace{1cm} c:x\mapsto\frac{\cos x}{\sqrt{\sin x}}; \hspace{1cm} d:x\mapsto\frac{1}{3x+1};\\
        &e:x\mapsto\frac{\ln x}{x}; \hspace{1cm} f:x\mapsto\frac{1}{x\ln x}; \hspace{1cm} g:x\mapsto\sqrt{3x+1}; \hspace{1cm} h:x\mapsto\frac{x+x^2}{1+x^2}.
    \end{align*}
    \begin{align*}
        &A:\begin{cases}\mathbb{R}\rightarrow\mathbb{R}\\x\mapsto e^{\sin x} + c\end{cases}; \hspace{0.5cm} B:\begin{cases}\mathbb{R}\setminus\{k\pi, k\in\mathbb{Z}\}\rightarrow\mathbb{R}\\x\mapsto\ln(\sin x) + c\end{cases}; \hspace{0.5cm}\\
        &C:\begin{cases}]2k\pi, (2k+1)\pi[, k\in\mathbb{Z}\rightarrow\mathbb{R}\\x\mapsto2\sqrt{\sin x} + c\end{cases}; \hspace{0.5cm} D:\begin{cases}\mathbb{R}\setminus\{-\frac{1}{3}\}\rightarrow\mathbb{R}\\x\mapsto\frac{1}{3}\ln(3x+1) + c\end{cases};\\
        &E:\begin{cases}\mathbb{R_+^*}\rightarrow\mathbb{R}\\x\mapsto\frac{1}{2}\ln^2x + c\end{cases}; \hspace{0.5cm} F:\begin{cases}\mathbb{R_+^*}\rightarrow\mathbb{R}\\x\mapsto\ln(\ln x) + c\end{cases};\\
        &G:\begin{cases}[-\frac{1}{3}, +\infty]\rightarrow\mathbb{R}\\x\mapsto\frac{2}{9}(3x+1)^{\frac{3}{2}} + c\end{cases}; \hspace{0.5cm}H:\begin{cases}\mathbb{R}\rightarrow\mathbb{R}\\x\mapsto\frac{1}{2}\ln(1+x^2) + x - \arctan(x) + c\end{cases}.
    \end{align*}
    Avec $c$ les constantes d'intégration.\\
    \qed
\end{tcolorbox}

\addcontentsline{toc}{section}{\protect\numberline{}Exercice 8.1}

\section*{Exercice 8.2 [$\blacklozenge\lozenge\lozenge$] Issu du cahier de calcul}
\begin{tcolorbox}[enhanced, width=7in, center, size=fbox, fontupper=\large, drop shadow southwest]
    On rappelle que $\int_a^b{f(x)dx}$ est l'aire algébrique entre la courbe représentative de $f$ et l'axe des abscisses.\\
    1. Sans chercher à les calculer, donner le signe des intégrales suivantes.
    \begin{equation*}
        \int_{-2}^3{e^{-x^2}dx}; \hspace{1cm} \int_5^{-3}{|\sin x|dx}; \hspace{1cm} \int_1^a{\ln^7(x)dx} (a\in\mathbb{R_+^*}).
    \end{equation*}
    2. En vous ramenant à des aires, calculer de tête
    \begin{equation*}
        \int_1^3{7dx}; \hspace{1cm} \int_0^7{3xdx}; \hspace{1cm} \int_{-2}^1{|x|dx}.
    \end{equation*}
    1.\\
    La première est positive car $-2<3$ et la fonction est positive sur $[-2,3]$e.\\
    La seconde est négative car $5>-3$ et la fonction est positive sur $[-3,5]$.\\
    La dernière est positive lorsque $a\geq1$ et négative lorsque $a\leq1$ car $\ln^7$ est positive sur $[1,+\infty[$.\\
    2.\\
    La première vaut $2\times7=14$.\\
    La seconde vaut $\frac{7^2\times3}{2}=\frac{147}{2}$.\\
    La dernière vaut $\frac{1}{2}+\frac{2\times2}{2}=2.5$\\
    \qed
\end{tcolorbox}

\addcontentsline{toc}{section}{\protect\numberline{}Exercice 8.2}

\section*{Exercice 8.3 [$\blacklozenge\lozenge\lozenge$]}
\begin{tcolorbox}[enhanced, width=7in, center, size=fbox, fontupper=\large, drop shadow southwest]
    Calculer les intégrales ci-dessous :
    \begin{align*}
        &I_1 = \int_0^1{x\sqrt{x}dx}, \hspace{0.5cm} I_2 = \int_{-1}^1{2^xdx}, \hspace{0.5cm} I_3=\int_1^e{\frac{\ln^3(t)}{t}dt}, \hspace{0.5cm} I_4=\int_0^1{\frac{x}{2x^2+3}dx},\\
        &I_5=\int_0^1{\frac{1}{2x^2+3}dx}, \hspace{0.5cm} I_6=\int_0^{\frac{\pi}{2}}{\cos^2xdx}, \hspace{0.5cm} I_7=\int_0^\pi{|\cos x|dx}, \hspace{0.5cm} I_8 = \int_0^{\frac{\pi}{2}}{\cos^3xdx}\\
        &I_9=\int_0^{\frac{\pi}{4}}{\tan^3xdx}.
    \end{align*}
    \begin{align*}
        &I_1 = \left[\frac{2}{5}x^{\frac{5}{2}}\right]_0^1=\frac{2}{5}, \hspace{0.5cm} I_2=\left[\frac{1}{\ln2}e^{x\ln2}\right]_{-1}^1=\frac{3}{\ln4}, \hspace{0.5cm} I_3=\left[\frac{\ln^4t}{4}\right]_1^e=\frac{1}{4},\\
        &I_4=\left[\frac{1}{4}\ln(2x^2+3)\right]_0^1=\frac{1}{4}\left(\ln\left(\frac{5}{3}\right)\right), \hspace{0.5cm} I_5 = \left[\frac{1}{\sqrt{6}}\arctan\left(\sqrt{\frac{2}{3}}x\right)\right]_0^1=\frac{1}{\sqrt{6}}\arctan\left(\sqrt{\frac{2}{3}}\right),\\
        &I_6=\frac{1}{2}\int_0^{\frac{\pi}{2}}{\cos2xdx}+\frac{\pi}{4}=\frac{1}{2}\left[-2\sin(2x)\right]_0^\frac{\pi}{2}+\frac{\pi}{4}=\frac{\pi}{4}, I_7=\left[2\sin x\right]_0^\pi=2,\\
        &I_8=\int_0^{\frac{\pi}{2}}{\cos x-\cos x\sin^2(x)dx}=\left[\sin x\right]_0^{\frac{\pi}{2}}-\left[\frac{1}{3}\sin^3x\right]_0^\frac{\pi}{2}=\frac{2}{3},\\
        &I_9=\int_0^\frac{\pi}{4}{\tan^3x+\tan x-\tan xdx}=\int_0^\frac{\pi}{4}{\tan x(\tan^2 x + 1)dx} - \frac{\ln2}{2}=\left[\frac{1}{2}\tan^2(x)\right]_0^\frac{\pi}{4}-\frac{\ln2}{2}\\
        &=\frac{1-\ln2}{2} 
    \end{align*}
    \qed
\end{tcolorbox}

\addcontentsline{toc}{section}{\protect\numberline{}Exercice 8.3}

\section*{Exercice 8.4 [$\blacklozenge\lozenge\lozenge$]}
\begin{tcolorbox}[enhanced, width=7in, center, size=fbox, fontupper=\large, drop shadow southwest]
    Calculer le nombre $\int_1^2{\frac{\ln x}{\sqrt{x}}dx}$.\\
    1. À l'aide d'une IPP.\\
    2. À l'aide du changement de variable $x=t^2$.\\
    1.
    \begin{align*}
        \int_1^2{\ln x\cdot x^{-\frac{1}{2}} dx}&=\left[\ln x \cdot 2\sqrt{x}\right]_1^2 - 2\int_1^2{x^{-\frac{1}{2}}dx}=2\sqrt{2}\ln2-2\left[2\sqrt{x}\right]_1^2=2\sqrt{2}(\ln2-2)+4
    \end{align*}
    2.
    \begin{align*}
        \int_1^2{\frac{\ln x}{\sqrt{x}}dx}=\int_1^{\sqrt{2}}{\frac{\ln t^2}{t}2tdt}=4\int_1^{\sqrt{2}}{\ln(t)dt}=4\left[t\ln t-t\right]_1^{\sqrt{2}}=4+2\sqrt{2}(\ln2-2)
    \end{align*}
    \qed
\end{tcolorbox}

\addcontentsline{toc}{section}{\protect\numberline{}Exercice 8.4}

\section*{Exercice 8.5 [$\blacklozenge\lozenge\lozenge$]}
\begin{tcolorbox}[enhanced, width=7in, center, size=fbox, fontupper=\large, drop shadow southwest]
    Calculer
    \begin{align*}
        \int_0^1{\frac{1}{(t+1)\sqrt{t}}dt} \hspace{1cm} \text{en posant }t=u^2.
    \end{align*}
    On a :
    \begin{align*}
       \int_0^1{\frac{1}{(t+1)\sqrt{t}}dt} = \int_0^1{\frac{1}{(u^2+1)u}2udu} = 2\int_0^1{\frac{1}{u^2+1}du}=2\left[\arctan(u)\right]_0^1 = \frac{\pi}{2}
    \end{align*}
\end{tcolorbox}

\addcontentsline{toc}{section}{\protect\numberline{}Exercice 8.5}

\section*{Exercice 8.6 [$\blacklozenge\lozenge\lozenge$]}
\begin{tcolorbox}[enhanced, width=7in, center, size=fbox, fontupper=\large, drop shadow southwest]
    Calculer
    \begin{align*}
        \int_0^1{\frac{t^9}{t^5+1}dt} \hspace{1cm} \text{en posant } u=t^5.
    \end{align*}
    On a :
    \begin{align*}
        \int_0^1{\frac{t^9}{t^5+1}dt}=\int_0^1{\frac{\frac{1}{5}t^5}{t^5+1}5t^4dt}=\frac{1}{5}\int^1_0{\frac{u}{u+1}du}=\frac{1}{5}\int^1_0{1-\frac{1}{u+1}du}=\frac{1}{5}\left(1-\ln2\right)
    \end{align*}
    \qed
\end{tcolorbox}

\addcontentsline{toc}{section}{\protect\numberline{}Exercice 8.6}

\section*{Exercice 8.7 [$\blacklozenge\blacklozenge\lozenge$]}
\begin{tcolorbox}[enhanced, width=7in, center, size=fbox, fontupper=\large, drop shadow southwest]
    En posant le changement de variable $u=\tan(x)$, calculer l'intégrale
    \begin{align*}
        \int_0^{\frac{\pi}{4}}{\frac{1}{1+\cos^2(x)}dx}&=\int_0^1{\frac{1}{1+\cos^2(\arctan(u))}\cdot\frac{1}{1+u^2}du}\\
        &=\int_0^1{\frac{1+u^2}{(2+u^2)(1+u^2)}du}\\
        &=\int_0^1{\frac{1}{2+u^2}du}\\
        &=\left[\frac{1}{\sqrt{2}}\arctan\left(\frac{u}{\sqrt{2}}\right)\right]_0^1\\
        &=\frac{1}{\sqrt{2}}\arctan\left(\frac{1}{\sqrt{2}}\right)\\
    \end{align*}
    \qed
\end{tcolorbox}

\addcontentsline{toc}{section}{\protect\numberline{}Exercice 8.7}

\section*{Exercice 8.8 [$\blacklozenge\lozenge\lozenge$]}
\begin{tcolorbox}[enhanced, width=7in, center, size=fbox, fontupper=\large, drop shadow southwest]
    On pose
    \begin{equation*}
        C=\int_0^{\frac{\pi}{2}}{\frac{\cos x}{\sin x + \cos x}dx} \hspace{0.5cm} \text{et} \hspace{0.5cm} S=\int_0^{\frac{\pi}{2}}{\frac{\sin x}{\sin x + \cos x}dx}
    \end{equation*}
    1. À l'aide du changement de variable $u=\frac{\pi}{2}-x$, prouver que $C=S$.\\
    2. Calculer $C + S$, en déduire la valeur commune de ces deux intégrales.
\end{tcolorbox}

\addcontentsline{toc}{section}{\protect\numberline{}Exercice 8.8 (W.I.P)}

\section*{Exercice 8.9 [$\blacklozenge\blacklozenge\blacklozenge$]}
\begin{tcolorbox}[enhanced, width=7in, center, size=fbox, fontupper=\large, drop shadow southwest]
    On considère les deux intégrales suivantes
    \begin{equation*}
        I=\int_0^{\frac{\pi}{2}}{\frac{\cos(t)}{\sqrt{1+\sin(2t)}}dt} \hspace{1cm} J=\int_0^{\frac{\pi}{2}}{\frac{\sin(t)}{\sqrt{1+\sin(2t)}}dt}
    \end{equation*}
    1. À l'aide du changement de variable $u=\frac{\pi}{4}-t$ calculer $I+J$.\\
    2. À l'aide du changement de variable $u=\frac{\pi}{2}-t$ montrer que $I=J$.\\
    3. En déduire $I$ et $J$.\\[0.1cm]
    1. On a :
    \begin{align*}
        I + J &= \int_0^{\frac{\pi}{2}}{\frac{\cos(t) + \sin(t)}{\sqrt{1+\sin(2t)}}dt}=\int_{-\frac{\pi}{4}}^{\frac{\pi}{4}}{\frac{\cos(\frac{\pi}{4}-u)+\sin(\frac{\pi}{4}-u)}{\sqrt{1+\cos(2u)}}du}\\
        &=\int_{-\frac{\pi}{4}}^{\frac{\pi}{4}}{\frac{\sqrt{2}\cos(u)}{\sqrt{2\cos^2(u)}}du}=\int_{-\frac{\pi}{4}}^{\frac{\pi}{4}}{\frac{\sqrt{2}\cos(u)}{\sqrt{2}|\cos(u)|}du}=\frac{\pi}{2}.
    \end{align*}
    2. On a :
    \begin{align*}
        I &= \int_0^{\frac{\pi}{2}}{\frac{\cos(t)}{\sqrt{1+\sin(2t)}}dt}=\int_0^{\frac{\pi}{2}}{\frac{\sin(u)}{\sqrt{1+\sin(\pi-u)}}du}=\int_0^{\frac{\pi}{2}}{\frac{\sin(u)}{\sqrt{1+\sin(u)}}du}=J
    \end{align*}
    3. On a $2I = 2J = I+J = \frac{\pi}{2}$. Donc $I = J = \frac{\pi}{4}$.\\
    \qed
\end{tcolorbox}

\addcontentsline{toc}{section}{\protect\numberline{}Exercice 8.9}

\section*{Exercice 8.10 [$\blacklozenge\lozenge\lozenge$]}
\begin{tcolorbox}[enhanced, width=7in, center, size=fbox, fontupper=\large, drop shadow southwest]
    Que vaut
    \begin{equation*}
        \int_{-666}^{666}{\ln\left(\frac{1+e^{\arctan(x)}}{1+e^{-\arctan(x)}}\right)dx} \text{ ?}
    \end{equation*}
\end{tcolorbox}

\addcontentsline{toc}{section}{\protect\numberline{}Exercice 8.10 (W.I.P)}

\section*{Exercice 8.11 [$\blacklozenge\blacklozenge\lozenge$]}
\begin{tcolorbox}[enhanced, width=7in, center, size=fbox, fontupper=\large, drop shadow southwest]
    Le but de cet exercice est de calculer les intégrales
    \begin{equation*}
        I = \int_0^1{\sqrt{1+x^2}dx} \hspace{1cm} \text{et} \hspace{1cm} J=\int_0^1{\frac{1}{\sqrt{1+x^2}}dx}.
    \end{equation*}
    1. Justifier que l'équation $\sh(x)=1$ possède une unique solution réelle que l'on notera dans la suite $\alpha$.\\
    Exprimer $\alpha$ à l'aide de la fonction $\ln$.\\
    2. Calculer $J$ en posant $x=\sh(t)$. On exprimera le résultat en fonction de $\alpha$.\\
    3. À l'aide d'une intégration par parties, obtenir une équation reliant $I$ et $J$.\\
    4. En déduire une expression de $I$ en fonction de $\alpha$.
\end{tcolorbox}

\addcontentsline{toc}{section}{\protect\numberline{}Exercice 8.11 (W.I.P)}

\section*{Exercice 8.12 [$\blacklozenge\blacklozenge\blacklozenge$]}
\begin{tcolorbox}[enhanced, width=7in, center, size=fbox, fontupper=\large, drop shadow southwest]
    Calculer $\int_0^1{\arctan(x^{1/3})dx}$ en posant d'abord $x=t^3$.\\
    On a :
    \begin{align*}
        \int_0^1{\arctan(x^{\frac{1}{3}})dx}&=\int_0^1{\arctan(t)\cdot3t^2dt}\\
        &=\left[\arctan(t)\cdot t^3\right]_0^1 - \int_0^1{\frac{t^3}{1+t^2} dt}\\
        &=\frac{\pi}{4} - \int_0^1{tdt}+\int_0^1{\frac{t}{1+t^2}dt}\\
        &=\frac{\pi}{4} - \frac{1}{2} + \left[\frac{1}{2}\ln(1+t^2)\right]_0^1\\
        &=\frac{\pi}{4} - \frac{1}{2} + \frac{1}{2}\ln(2)\\
        &=\frac{1}{4}\left(\pi - 2 + \ln(4)\right)
    \end{align*}
    \qed
\end{tcolorbox}

\addcontentsline{toc}{section}{\protect\numberline{}Exercice 8.12}
\end{document}
 