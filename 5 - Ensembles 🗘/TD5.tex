\documentclass[10pt]{article}

\usepackage[T1]{fontenc}
\usepackage[left=2cm, right=2cm, top=2cm, bottom=2cm]{geometry}
\usepackage[skins]{tcolorbox}
\usepackage{hyperref, fancyhdr, lastpage, tocloft, ragged2e, multicol}
\usepackage{amsmath, amssymb, amsthm}
\usepackage{tkz-tab}

\def\pagetitle{Ensembles et applications}

\title{\bf{\pagetitle}\\\large{Corrigé}}
\date{Octobre 2023}
\author{DARVOUX Théo}

\hypersetup{
    colorlinks=true,
    citecolor=black,
    linktoc=all,
    linkcolor=blue
}

\pagestyle{fancy}
\cfoot{\thepage\ sur \pageref*{LastPage}}


\begin{document}
\renewcommand*\contentsname{Exercices.}
\renewcommand*{\cftsecleader}{\cftdotfill{\cftdotsep}}
\maketitle
\hrule
\tableofcontents
\vspace{0.5cm}
\hrule

\thispagestyle{fancy}
\fancyhead[L]{MP2I Paul Valéry}
\fancyhead[C]{\pagetitle}
\fancyhead[R]{2023-2024}
\allowdisplaybreaks

\pagebreak

\section*{Exercice 5.1 [$\blacklozenge\lozenge\lozenge$]}
\begin{tcolorbox}[enhanced, width=7in, center, size=fbox, fontupper=\large, drop shadow southwest]
    Soient $A,B$ deux parties d'un ensemble $E$. Établir que
    \begin{equation*}
        A \setminus (A \setminus B) = A \cap B \hspace{1cm} \text{ et } \hspace{1cm} A \setminus (A \cap B) = A \setminus B = (A \cup B) \setminus B.
    \end{equation*}
    On a :
    \begin{align*}
        A \setminus (A \setminus B) &= A \cap \overline{(A\ \cap \overline{B})}\\
        &=A \cap (\overline{A} \cup B)\\
        &= (A \cap \overline{A}) \cup(A \cap B)\\
        &= A \cap B
    \end{align*}
    D'autre part :
    \begin{align*}
        A \setminus (A \cap B) &= A \cap \overline{(A \cap B)}\\
        &= A \cap (\overline{A} \cup \overline{B})\\
        &= (A \cap \overline{A}) \cup (A \cap \overline{B})\\
        &= A \cap \overline{B}\\
        &= A \setminus B
    \end{align*}
    Et :
    \begin{align*}
        (A \cup B) \setminus B &= (A \cup B) \cap \overline{B}\\
        &= (A \cap \overline{B}) \cup (B \cap \overline{B})\\
        &= A \cap \overline{B}\\
        &= A \setminus B
    \end{align*}
    \qed
\end{tcolorbox}

\addcontentsline{toc}{section}{\protect\numberline{}Exercice 5.1}

\section*{Exercice 5.2 [$\blacklozenge\lozenge\lozenge$]}
\begin{tcolorbox}[enhanced, width=7in, center, size=fbox, fontupper=\large, drop shadow southwest]
    Soient $A,B,C,D$ quatre parties d'un ensemble $E$, telles que
    \begin{equation*}
        E = A \cup B \cup C, \hspace{1cm} A \cap D \subset B, \hspace{1cm} B \cap D \subset C, \hspace{1cm} C \cap D \subset A.
    \end{equation*}
    Montrer que $D \subset A \cap B \cap C$.\\
    Soit $x \in D$, on sait que $x \in E$. Alors $x \in A$ ou $x \in B$ ou $x \in C$.\\
    $\circledcirc$ Si $x \in A$, alors $x \in A \cap D$, donc $x \in B$.\\
    $\circledcirc$ Si $x \in B$, alors $x \in B \cap D$, donc $x \in C$.\\
    $\circledcirc$ Si $x \in C$, alors $x \in C \cap D$, donc $x \in A$.\\
    On en déduit que $x \in A \cap B \cap C$.\\
    Ainsi, $D \subset A \cap B \cap C$.\\
    \qed
\end{tcolorbox}

\addcontentsline{toc}{section}{\protect\numberline{}Exercice 5.2}



\end{document}
