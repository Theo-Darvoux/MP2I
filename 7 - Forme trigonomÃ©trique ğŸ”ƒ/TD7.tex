\documentclass[10pt]{article}

\usepackage[T1]{fontenc}
\usepackage[left=2cm, right=2cm, top=2cm, bottom=2cm]{geometry}
\usepackage[skins]{tcolorbox}
\usepackage{hyperref, fancyhdr, lastpage, tocloft, ragged2e, multicol}
\usepackage{amsmath, amssymb, amsthm, stmaryrd}
\usepackage{tkz-tab}

\def\pagetitle{Forme Trigonométrique}

\title{\bf{\pagetitle}\\\large{Corrigé}}
\date{Octobre 2023}
\author{DARVOUX Théo}

\hypersetup{
    colorlinks=true,
    citecolor=black,
    linktoc=all,
    linkcolor=blue
}

\pagestyle{fancy}
\cfoot{\thepage\ sur \pageref*{LastPage}}


\begin{document}
\renewcommand*\contentsname{Exercices.}
\renewcommand*{\cftsecleader}{\cftdotfill{\cftdotsep}}
\maketitle
\hrule
\tableofcontents
\vspace{0.5cm}
\hrule

\thispagestyle{fancy}
\fancyhead[L]{MP2I Paul Valéry}
\fancyhead[C]{\pagetitle}
\fancyhead[R]{2023-2024}
\allowdisplaybreaks

\pagebreak

\section*{Exercice 7.1 [$\blacklozenge\lozenge\lozenge$]}
\begin{tcolorbox}[enhanced, width=7in, center, size=fbox, fontupper=\large, drop shadow southwest]
    Calculer $(1 + i)^2023$.\\
    On a :
    \begin{equation*}
        (1+i)^{2023}=(\sqrt{2}e^{i\frac{\pi}{4}})^{2023}
        = \sqrt{2}^{2023} e^{i\frac{2023\pi}{4}}
        = \sqrt{2}^{2023} e^{-i\frac{\pi}{4}}
    \end{equation*}
    \qed
\end{tcolorbox}

\addcontentsline{toc}{section}{\protect\numberline{}Exercice 7.1}


\end{document}
 