\documentclass[11pt]{article}

\def\chapitre{40}
\def\pagetitle{Arithmétique dans $\Z$.}

\input{/home/theo/MP2I/setup.tex}

\newcommand*{\PPCM}{\nt{PPCM}}
\newcommand*{\PGCD}{\nt{PGCD}}
\renewcommand*{\D}{\mathcal{D}}

\begin{document}

\input{/home/theo/MP2I/title.tex}

\section{Divisibilité dans \texorpdfstring{$\Z$}{Lg}}

\subsection{Définition et propriétés élémentaires.}

\begin{defi}{}{}
    Soit $(a,b)\in\Z^2$. On dit que $b$ \bf{divise} $a$ ($b\mid a$) s'il existe $k\in\Z$ tel que $a=kb$.\\
    On dit aussi que $b$ est \bf{diviseur} de $a$, ou que $a$ est \bf{multiple} de $b$.\n
    Notations pour les ensembles de diviseurs et multiples de $a\in\Z$:
    \begin{equation*}
        \D(a)=\{b\in\Z:b\mid a\} \qquad \nt{et} \qquad a\Z=\{ak,k\in\Z\}.
    \end{equation*}
\end{defi}

\begin{prop}{Faits immédiats.}{}
    Tous les entiers divisent 0 et 1 divise tous les entiers. Ajoutons que pour $(a,b,c)\in\Z^3$,
    \begin{enumerate}[topsep=0pt,itemsep=-0.9 ex]
        \item Si $b$ est diviseur de $a$ et si $a\neq0$, alors $|b|\leq|a|$.
        \item $b\mid a \iff a\Z \subset b\Z$.
        \item Si $c\mid a$ et $c\mid b$, alors $c\mid au+bv$, pour tous $u$ et $v$ dans $\Z$.
    \end{enumerate}
    \tcblower
    \boxed{1.} Supposons que $b\mid a$ et $a\neq 0$, alors $\exists k \in \Z \mid a = bk$ et $|a|=|b||k|$.\\
    De plus, $k\neq0$ car $a\neq0$, donc $|k|\geq1$ et $|kb|\geq |b|$, on obtient bien $|a|\geq |b|$.\n
    \boxed{2.} Supposons que $b\mid a$, alors $\exists k \in \Z \mid a = bk$, soit $m\in a\Z$ : $\exists k'\in\Z\mid m=ak'$ donc $m=bkk'$ donc $m\in b\Z$.\\
    Supposons $a\Z\subset b\Z$, on a $a\in a\Z$ donc $a\in b\Z$ donc $b\mid a$.\n
    \boxed{3.} Supposons que $c\mid a$ et $c \mid b$ : $\exists k,k'\in\Z\mid a=kc,~b=k'c$. Soient $u,v\in\Z$.\\
    On a alors $au+bv=kuc+k'vc=(ku+k'v)c$ avec $ku+k'v\in\Z$, donc $c\mid au+bv$.
\end{prop}

\begin{prop}{Plus une relation d'ordre!}{}
    Sur $\Z$, la relation \emph{divise} est réflexive, transitive, mais pas antisymétrique. On a
    \begin{equation*}
        \forall (a,b)\in\Z^2\quad(a\mid b \nt{ et } b \mid a) \iff (a=b \nt{ ou } a=-b).
    \end{equation*}
    Dans le cas où $(a\mid b)$ et $(b\mid a)$, ont dit que $a$ et $b$ sont \bf{associés}.
    \tcblower
    \fbox{$\la$} Trivial.\\
    \fbox{$\ra$} Supposons que $a\mid b$ et $b\mid a$. Alors $\exists k,k'\in\Z\mid a=kb$ et $b=k'a$.\\
    On a alors $b=bkk'$. Si $b=0$, alors $a=0$ donc $a=b$. Sinon, $kk'=1$ donc $k=\pm1$ et $a=\pm b$.
\end{prop}

\subsection{Division euclidienne.}

\begin{thm}{}{}
    Soit $(a,b)\in\Z\times\N^*$. Il existe un unique couple $(q,r)\in\Z^2$ tel que
    \begin{equation*}
        a=bq+r\quad\nt{et}\quad0\leq r < b.
    \end{equation*}
    Les entiers $q$ et $r$ sont appelés \bf{quotient} et \bf{reste} dans la division euclidienne de $a$ par $b$.
    \tcblower
    \bf{Unicité:}\\
    Soit $(q,r)\in\Z^2$ et $(q',r')\in\Z^2$ avec $0\leq r,r' <b$ tels que $a=bq+r$ et $a=bq'+r'$.\\
    On a $bq'+r'=bq+r$ donc $b(q'-q)=r-r'$. De plus, $0\leq r,r'<b$ donc $-b<-r'\leq0$.\\
    Ainsi, $-b<r-r'<b$ donc $-b<b(q'-q)<b$ donc $-1<q'-q<1$ donc $q'=q$ car $q-q'\in\Z$.\\
    Donc $r-r'=b\cdot0=0$ donc $(q,r)=(q',r')$.\n
    \bf{Existence:}\\
    On pose $q=\lf\frac{a}{b}\rf$ et $r=a-bq$. On a bien $a=bq+r$.\\
    On a $\lf\frac{a}{b}\rf\leq\frac{a}{b}<\lf\frac{a}{b}\rf+1$ donc $q\leq\frac{a}{b}<q+1$ donc $qb\leq a<qb+b$ donc $0\leq a-bq<b$ donc $0\leq r < b$.
\end{thm}

\begin{prop}{}{}
    Soient $a$ et $b$ deux entiers relatifs.\\
    L'entier $b$ divise $a$ si et seulement si le reste de la division euclidienne de $a$ par $|b|$ est nul.
    \tcblower
    \fbox{$\la$} Trivial.\\
    \fbox{$\ra$} Par unicité du reste.
\end{prop}

\subsection{PPCM de deux entiers.}

\begin{defi}{}{}
    Soient $a,b$ deux entiers relatifs.
    \begin{enumerate}[topsep=0pt,itemsep=-0.9 ex]
        \item Si $a$ et $b$ sont non nuls, on appelle \bf{Plus Petit Commun Multiple} de $a$ et $b$, note $a\lor b$, ou encore $\PPCM(a,b)$, le plus petit élément strictement positif de $a\Z\cap b\Z$.
        \item Si $a$ ou $b$ vaut 0, on pose $a\lor b=0$.
    \end{enumerate}
\end{defi}

\begin{prop}{}{}
    Soit $(a,b)\in\Z^2$. Leur PPCM $a\lor b$ est l'unique entier positif $m$ tel que
    \begin{equation*}
        a\Z \cap b\Z = m\Z.
    \end{equation*}
    \tcblower
    \bf{Unicité:}\\
    Soient $m,m'\in\N$ tels que $a\Z\cap b\Z=m\Z$ et $a\Z\cap b\Z=m'\Z$.\\
    Alors $m\Z = m'\Z$ donc $m$ et $m'$ sont associés (et positifs) donc $m=m'$.\n
    \bf{Existence:}\\
    On a $a\Z$ sous-groupe de $(\Z,+)$, $b\Z$ aussi, par intersection de groupes, $a\Z\cap b\Z$ l'est aussi.\\
    Or les sous-groupes de $\Z$ sont de la forme $m\Z$ avec $m\in\N$. Donc il existe un unique $m\in\N$ tel que $a\Z\cap b\Z=m\Z$.\\
    Vérifions que $m=\PPCM(a,b)$. Clair: $m$ est multiple commun de $a$ et $b$.\\
    De plus, $a\Z\cap b\Z\cap\N=m\Z\cap\N=\{0,m,2m,...\}$.\\
    Donc si $m=0$, $\PPCM(a,b)=0$, sinon $\PPCM(a,b)=m$. 
\end{prop}

\begin{thm}{}{}
    Soient $a$ et $b$ deux entier relatifs. Leur PPCM $a\lor b$ est l'unique entier positif $m$ tel que
    \begin{enumerate}[topsep=0pt,itemsep=-0.9 ex]
        \item $a\mid m$ et $b\mid m,\quad$ \emph{le PPCM est un multiple commun}.
        \item $\forall \mu \in \Z,~(a\mid\mu \nt{ et } b \mid \mu) \ra m \mid \mu,\quad$ \emph{tout multiple commun est multiple du PPCM}.
    \end{enumerate}
    \tcblower
    \bf{Unicité:} Soient $m,m'$ satisfaisant 1. et 2.\\
    On a $m\mid m'$ et $m'\mid m$, par antisymétrie sur $\N$, $m=m'$.\n
    \bf{Existence:} Posons $m=\PPCM(a,b)$.\\
    Il satisfait 1. par définition. Soit $\mu\in\Z$ un multiple commun, alors $\mu\in a\Z\cap b\Z=m\Z$, donc $m\mid \mu$.
\end{thm}

\subsection{PGCD de deux entiers.}

\begin{defi}{}{}
    Soient $a,b$ deux entiers relatifs.
    \begin{enumerate}[topsep=0pt,itemsep=-0.9 ex]
        \item Si $a$ et $b$ sont non nuls, on appelle \bf{Plus Grand Commun Diviseur} de $a$ et $b$, note $a\land b$, ou encore $\PGCD(a,b)$, le plus grand élément positif de $\D(a)\cap\D(b)$.
        \item Si $a=b=0$, on pose $a\land b=0$.
    \end{enumerate}
\end{defi}

\begin{prop}{}{}
    \begin{equation*}
        \forall (a,b)\in\Z^2\quad a\land b = |a|\land|b|
    \end{equation*}
    \tcblower
    On a:
    \begin{equation*}
        \D(a)\cap\D(b)=\D(|a|)\cap\D(|b|).
    \end{equation*}
    On n'a plus qu'à passer au $\max$.
\end{prop}

\begin{prop}{Lemme d'Euclide.}{}
    Soient $a,b,c,d$ quatre entiers relatifs. Si $a=bc+d$, alors on a $a\land b=b\land d$.
    \tcblower
    Supposons que $a=bc+d$. Se convaincre que $\D(a,b)=\D(b,d)$ puis passer au max.
\end{prop}

\begin{meth}{}{}
    Ce lemme est l'idée principale de l'algorithme d'Euclide, vu dans le "petit" cours d'arithmétique.\\
    Si $a\in\Z$ et $b\in\Z^*$, on peut appliquer cet algorithme à $|a|$ et $|b|$ pour calculer $a\land b$.
\end{meth}

\begin{prop}{Le sous-groupe de $\Z$ sous-jacent.}{}
    Soit $(a,b)\in\Z^2$. Notons $a\Z+b\Z=\{au+bv,~(u,v)\in\Z^2\}$. C'est un sous-groupe de $\Z$.\n
    Le $\PGCD~a\land b$ est l'unique entier positif $d$ tel que
    \begin{equation*}
        a\Z+b\Z=d\Z.
    \end{equation*}
    En particulier, il existe un couple $(u,v)\in\Z^2$ tel que $d=au+bv$ \bf{(relation de Bézout)}.
    \tcblower
    On a $a\Z+b\Z=\{au+bv\mid(u,v)\in\Z^2\}$.\\
    C'est un sous-groupe de $\Z$ car $0=a\cdot0+b\cdot0\in a\Z+b\Z$ et,\\Pour $(m,m')\in(a\Z+b\Z)^2$, $\exists(u,v,u',v')\in\Z\mid m=au+bv \nt{ et } m'=au'+bv'$\\
    Donc $m-m'=a(u-u')+b(v-v')\in a\Z+b\Z$.\n
    D'après le cours sur les structures algébriques, il existe $d\in\N\mid a\Z+b\Z=d\Z$.\n
    \bf{Unicité:} Si $d,d'$ conviennent, $d\Z=d'\Z$, ils sont associés et positits donc $d=d'$.\n
    Montrons que $d=\PGCD(a,b)$. \\
    On a $d\mid a$ et $d\mid b$ car $a\Z\subset a\Z+b\Z\subset d\Z$, pareil pour $b\Z$.\\
    Soit $\d\in\N$ diviseur de $a$ et $b$, on a $\exists (u,v)\in\Z^2\mid d=uv+bv$.\\
    Puisque $\d$ divise $a$ et $b$, alors $\d$ divise $au+bv=d$.\\
    Si $d\neq0$, $\d\mid d\ra \d\leq d$, sinon, $d=0$ donc $a=b=0$ donc $d=\PGCD(a,b)=0$.
\end{prop}

\begin{meth}{Écriture effective d'une relation de Bézout.}{}
    En \emph{remontant} les divisions euclidiennes écrites lors de l'exécution de l'algorithme d'Euclide.
\end{meth}

\begin{prop}{}{}
    \begin{equation*}
        \forall(a,b)\in\Z^2,\quad\forall k\in\Z,\quad\PGCD(ka,kb)=|k|\cdot\PGCD(a,b).
    \end{equation*}
    \tcblower
    On a $a\Z+b\Z=(a\land b)\Z$ donc $ka\Z+kb\Z=|k|(a\land b)\Z$.\\
    On a aussi $ka\land kb=|k|(a\land b)$.
\end{prop}

\begin{thm}{Une caractérisation du PGCD}{}
    Soient $a$ et $b$ deux entiers relatifs. Leur $\PGCD~a\land b$ est l'unique entier positif $d$ tel que
    \begin{enumerate}[topsep=0pt,itemsep=-0.9 ex]
        \item $d\in\D(a)\cap\D(b),\quad$ \emph{(le PGCD est un diviseur commun)}.
        \item $\forall \d\in\D(a)\cap\D(b)\quad\d\mid d\quad$\emph{(tous les diviseurs communs divisent le PGCD)}.
    \end{enumerate}
    \tcblower
    Notons $d=\PGCD(a,b)$, montrons que $d$ satisfait $1.$ et $2.$.\\
    Il satisfait 1. par définition.\\
    Soit $\d\in\Z\mid\d\mid a$ et $\d\mid b$, $\exists (u,v)\in\Z^2\mid d=au+bv$.\\
    Il est clair que $\d\mid au+bv$ donc $\d\mid d$, $d$ satisfait donc 2.\n
    Soit $d\in\N$ un entier qui satisfait 1. et 2.\\
    Si $d=0$, alors $a=b=0$ donc $d=\PGCD(a,b)=0$.\\
    Si $d\neq0$, alors $d\mid a$ et $d\mid b$, le plus grand d'après 2.
\end{thm}

\begin{corr}{}{}
    \begin{equation*}
        \forall(a,b)\in\Z^2\quad\D(a)\cap\D(b)=\D(a\land b).
    \end{equation*}
    \tcblower
    \fbox{$\supset$} claire par transitivité.\\
    \fbox{$\subset$} Soit $\d\in\D(a)\cap\D(b)$, on a établi qu'un diviseur commun divise le PGCD, donc $\d\in\D(a\land b)$.
\end{corr}

\begin{prop}{}{}
    \begin{equation*}
        \forall(a,b)\in\Z^2,\quad\PGCD(a,b)\cdot\PPCM(a,b)=|ab|.
    \end{equation*}
    \tcblower
    On note $d=a\land b$ et $m=a\lor b$.\\
    Puisque $d\mid a$ et $d\mid b$, $\exists (a',b')\in\Z^2\mid a=da'$ et $b=db'$.\\
    On a $da'b'=ab'=a'b$ donc $da'b'$ est multiple de $a$ et $b$, donc $m\mid da'b'$.\\
    Donc $md\mid (da')(db')$ donc $md\mid ab$.\n
    On a $\exists (u,v)\in\Z^2\mid d=au+bv$ et $\exists (k,k')\mid m=ak=bk'$, donc $md=amu+bmv=ab(k'u+kv)$ donc $md\mid ab$.\n
    Alors $ab$ et $md$ sont associés, $ab=\pm md$ donc $md=|ab|$. 
\end{prop}

\section{Entiers premiers entre eux.}
\subsection{Couples d'entiers premiers entre eux.}

\begin{defi}{}{}
    On dit que deux entiers sont \bf{premiers entre eux} si leur PGCD vaut 1.
\end{defi}

\begin{prop}{}{}
    Deux entiers naturels non nuls $a$ et $b$ sont premiers entre eux si et seulement si $a\lor b=|ab|$.
\end{prop}

\begin{prop}{}{}
    Soit $(a,b)\in\Z^2\setminus\{(0,0)\}$ et $d=a\land b$.\\
    Si $a'$ et $b'$ sont les deux entiers relatifs tels que $a=da'$ et $b=db'$, alors $a'\land b'=1$.
    \tcblower
    On a $\PGCD(a,b)=d$ donc $\PGCD(da',db')=d$ donc $d\PGCD(a',b')=d$ or $d\neq0$ car $(a,b)\neq(0,0)$.\\
    On retrouve bien que $\PGCD(a',b')=1$.
\end{prop}

\begin{thm}{de Bézout.}{}
    \begin{equation*}
        \forall(a,b)\in\Z^2\quad a\land b = 1 \iff \exists(u,v)\in\Z^2\mid au+bv=1.
    \end{equation*}
    \tcblower
    \fbox{$\la$} Supposons qu'il existe $(u,v)\in\Z^2$ tels que $au+bv=1$.\\
    Notons $d:=a\land b$, il divise $a$ et $b$ donc $au+bv$. Donc $d\mid 1$, c'est 1.\n
    \fbox{$\ra$} Supposons que $a\land b = 1$, alors $a\Z+b\Z=1\Z$ donc $1\in a\Z+b\Z$ donc $\exists (u,v)\in\Z^2\mid au+bv=1$.
\end{thm}

\begin{corr}{}{}
    Soit $(a,b,c)\in\Z^3$.
    \begin{enumerate}[topsep=0pt,itemsep=-0.9 ex]
        \item Si $a\land b=1$ et $a\land c=1$, alors $a\land(bc)=1$.
        \item Plus généralement, si $a$ est premier avec chacun des $m$ entiers $b_1,...,b_m$ ($m\in\N^*$), alors il est premier avec leur produit $b_1...b_m$.
        \item Si $a\land b=1$, alors pour tout $(n,p)\in\N^2,~a^n\land b^p=1$.
    \end{enumerate}
    \tcblower
    \boxed{1.} Supposons $a\land b=1$ et $a\land c = 1$.\\
    D'après le théorème de Bézout, $\exists (u,v)\in\Z^2 \mid au+bv=1$ et $\exists (u',v')\in\Z^2 \mid au'+cv'=1$.\\
    Donc $(au+bv)(au'+cv')=1$ donc $a(auu'+ucv'+bu'v)+bc(vv')=1$ donc $a\land bc = 1$.\n
    \boxed{2.} Tout pareil.\n
    \boxed{3.} Supposons $a\land b=1$ alors $a\land b^p=1$ et $b^p\land a = 1$ donc $b^p\land a^n=1$ (d'après 2, par récurrence).\\
    Donc $a^n\land b^p=1$.
\end{corr}

\subsection{Produit de diviseurs, diviseurs d'un produit.}

\begin{prop}{Produit de diviseurs premiers entre eux.}{}
    \begin{equation*}
        \forall (a_1,a_2,b)\in\Z^3\quad\begin{cases}
            a_1\mid b ~ \nt{ et } ~ a_2\mid b\\ a_1\land a_2=1
        \end{cases} 
        \ra a_1a_2\mid b.
    \end{equation*}
    \tcblower  
    Supposons que $a_1\mid b$ et $a_2\mid b$ et $a_1\land a_2=1$.\\
    Alors $|a_1a_2|=a_1\lor a_2$, or le PPCM divise tous les multiples communs, en particulier, $a_1a_2\mid b$.
\end{prop}

\begin{thm}{Lemme de Gauss.}{gauss}
    \begin{equation*}
        \forall(a,b,c)\in\Z^3,\quad\begin{cases}
            a\mid bc\\
            a\land b = 1
        \end{cases}\ra a \mid c.
    \end{equation*}
    \tcblower
    Supposons que $a\mid bc$ et $a\land b=1$ donc $\exists k \in \Z \mid bc = ak$.\\
    D'après le théorème de Bézout, $\exists (u,v)\in\Z^2\mid au+bv=1$.\\
    On a $c=acu+bcv=a(cu+kv)$ donc $a\mid c$.
\end{thm}

\begin{ex}{}{}
    1. Soit $P=a_nX^n+...+a_0\in\Z[X]$.\\
    Montrer que si $\frac{p}{q}$ est racine de $P$ avec $p\land q=1$, alors $p\mid a_0$ et $q\mid a_n$.\n
    2. Factoriser $X^3+2X^2-4X-3$ dans $\R[X]$.
    \tcblower
    \boxed{1.} On a $P(\frac{p}{q})=0$ donc $a_n\left(\frac{p}{q}\right)^n+...+a_0=0$ donc $a_np^n+...+a_0q^n=0$.\\
    Ainsi, $p(a_n^{n-1}+...+a_1q^{n-1})=-a_0q^n$ donc $p\mid a_0q^n$ or $p\land q^n=1$ donc $p\mid a_0$.\\
    En factorisant par $q$, on obtient aussi que $q\mid a_n$.\n
    \boxed{2.} D'après 1, $p\mid3$ et $q\mid1$ donc les seuls candidats : $\frac{p}{q}\in\{-3,-1,1,3\}$.\\
    On a alos $P=(X+3)(X^2-X-1)=(X+3)(X-\phi)(X-\psi)$ où $\phi=\frac{1+\sqrt{5}}{2}$ et $\psi=\frac{1-\sqrt{5}}{2}$.
\end{ex}

\subsection{Le cas des diviseurs premiers.}

\begin{defi}{}{}
    On appelle \bf{nombre premier} tout entier $p$ supérieur à 2 dont les diviseurs sont $1,p,-1$ et $-p$.
\end{defi}

\begin{prop}{}{}
    Tout entier naturel supérieur ou égal à 2 possède un diviseur premier.
    \tcblower
    On l'avait fait par récurrence forte au premier semestre.
\end{prop}

\begin{prop}{}{}
    Deux entiers relatifs sont premiers entre eux si et seulement si ils n'admettent aucun nombre premier comme diviseur commun.
    \tcblower
    \fbox{$\ra$} Par contraposée, supposons qu'il existe $p$ premier tel que $p\mid a$ et $p\mid b$.\\
    Puisque $p$ divise les deux, il divise le PGCD, or $p\geq2$ donc le PGCD est différent de 1.\n
    \fbox{$\la$} Par contraposée, supposons que $a$ et $b$ ne sont pas premiers entre-eux, alors $a\land b\geq2$.\\
    D'après la proposition précédente, le PGCD a un diviseur premier $p$, donc $p\mid a$ et $p\mid b$.
\end{prop}

\begin{prop}{}{28}
    Si $a$ est un entier et $p$ un nombre premier, alors $p\mid a$ ou $p$ est premier avec $a$.
    \tcblower
    Notons $d=p\land a$, il divise $p$, alors $d=p$ ou $d=1$.\\
    Mais si $d=p$, alors $p\mid a$, sinon si $d=1$, $a\land p=1$.
\end{prop}

\begin{prop}{}{}
    Soit $(a,b)\in\Z^2$ et $p$ un nombre premier.
    \begin{enumerate}[topsep=0pt,itemsep=-0.9 ex]
        \item Si $p\mid ab$, alors $p\mid a$ ou $p\mid b$.
        \item Si $p$ divise un produit d'entiers, alors il divise l'un des facteurs.
    \end{enumerate}
    \tcblower
    \boxed{1.} Supposons que $p\mid ab$.\\
    Si $p\mid a$, on a fini. Sinon, $p\land a=1$ d'après \ref{prop:28}, donc $p\mid b$ d'après \ref{thm:gauss}.\n 
    \boxed{2.} Récurrence, trivial.
\end{prop}

\subsection{Extension à un nombre fini de vecteurs.}

\begin{defi}{}{}
    Soit $n\in\N^*$ et $(a_1,...,a_n)\in\Z^n\setminus\{(0,...,0)\}$.\\
    Le plus grand diviseur positif commun à $a_1,...,a_n$ est appelé leur \bf{PGCD} et noté:
    \begin{equation*}
        a_1\land ... \land a_n.
    \end{equation*}
    On convient que le PGCD de $n$ entiers nuls vaut 0.
\end{defi}

\begin{prop}{}{}
    Soit $n\in\N^*$, $(a_1,...,a_n)\in\Z^n$. Leur PGCD est l'unique entier positif $d$ tel que
    \begin{equation*}
        a_1\Z+...+a_n\Z=d\Z
    \end{equation*}
    En particulier,
    \begin{equation*}
        \exists(u_1,...,u_n)\in\Z^n\quad d=a_1u_1+...+a_nu_n.
    \end{equation*}
\end{prop}

\begin{prop}{}{}
    \begin{equation*}
        \forall (a_1,...,a_n)\in\Z^n, \quad \forall k \in \Z, \quad \PGCD(ka_1,...,ka_n)=|k|\cdot\PGCD(a_1,...,a_n).
    \end{equation*}
\end{prop}

\begin{prop}{}{}
    Soit $n\in\N^*$ et $a_1,...,a_{n+1}$ des entiers relatifs. Alors,
    \begin{equation*}
        a_1\land...\land a_n\land a_{n+1}=(a_1\land ... \land a_n) \land a_{n+1}
    \end{equation*}
    \tcblower
    Notons $d_n=a_1\land ... \land a_n$, $d_{n+1}=a_1\land...\land a_n \land a_{n+1}$ et $d'_{n+1}=d_n\land a_{n+1}$.\\
    D'une part, d'après la proposition précédente:
    \begin{equation*}
        a_1\Z+...+a_n\Z+a_{n+1}\Z=d_{n+1}\Z.
    \end{equation*}
    D'autre part,
    \begin{align*}
        a_1\Z+...+a_n\Z+a_{n+1}\Z&=(a_1\Z+...+a_n\Z)+a_{n+1}\Z\\
        &=d_n\Z+a_{n+1}\Z\\
        &=(d_n\land a_{n+1})\Z\\
        &= d'_{n+1}\Z.
    \end{align*}
    Ceci amène que $d_{n+1}$ et $d'_{n+1}$ sont associés et donc égaux par positivité.
\end{prop}

\begin{prop}{}{}
    Soit $n\in\N^*$, $(a_1,...,a_n)\in\Z^n$ et $d$ leur PGCD, on a
    \begin{equation*}
        \bigcap_{k=1}^n\D(a_k)=\D(d).
    \end{equation*}
\end{prop}

\begin{defi}{}{}
    Des entiers relatifs $a_1,...,a_n$ sont dits \bf{premiers entre eux dans leur ensemble} si leur PGCD est égal à 1, ou de manière équivalente si $1$ et $-1$ sont les seuls diviseurs communs.\n
    Ils sont \bf{deux à deux premiers entre eux} si
    \begin{equation*}
        \forall (i,j) \in \lb1,n\rb^2, ~ i \neq j \ra a_i \land a_j = 1.
    \end{equation*}
\end{defi}

\begin{ex}{}{}
    Justifier que si $n$ entiers $(n\geq2)$ sont premiers entre eux deux-à-deux, ils le sont dans leur ensemble.\n
    Les entiers $6$, $10$ et $15$ sont premiers entre-eux dans leur ensemble, mais pas deux-à-deux.
    \tcblower
    Soit $a_1,...,a_n\in\Z^n$ premiers entre-eux deux-à-deux.\\
    Soit $d=a_1\land ... \land a_n$, alors $d\mid a_1$ et $d\mid a_2$ : il divise $a_1\land a_2=1$ donc $d=1$.
\end{ex}

\begin{thm}{}{}
    Soit $n\in\N^*$ et $(a_1,...,a_n)\in\Z^n$.\n
    $a_1,...,a_n$ sont premiers entre eux dans leur ensemble $\iff$ $\exists(u_1,...,u_n)\in\Z^n\quad\sum_{i=1}^na_iu_i=1.$
\end{thm}

\begin{prop}{}{}
    Soit $n\in\N^*$ et $(a_1,...,a_n)\in\Z^n$ et $b\in\Z$.\n
    Si tous les $a_i$ divisent $b$, et si les $a_i$ sont deux-à-deux premiers entre eux, alors $a_1...a_n$ divise $b$.
    \tcblower
    Supposons que $a_1,...,a_n$ divisent $b$ et sont deux-à-deux premiers entre eux.\\
    Alors, $a_1 \mid b$, $a_2\mid b$, et $a_1 \land a_2=1$ donc $a_1a_2\mid b$.\\
    De plus, $a_1a_2 \mid b$ et $a_3 \mid b$ et $a_1a_2\land a_3=1$ donc $a_1a_2a_3\mid b$.\\
    En itérant, on obtient le résultat.
\end{prop}

\section{Théorème fondamental de l'arithmétique et applications.}
\subsection{Le TFAr.}

\begin{thm}{Théorème fondamental de l'arithmétique.}{}
    Soit $n$ un entier supérieur à 2. Il existe un entier naturel $r$ non nul et $r$ nombres premiers $p_1<...<p_r$, ainsi que des entiers naturels non nuls $\a_1,...,\a_r$ tels que
    \begin{equation*}
        n=p_1^{\a_1}p_2^{\a_2}...p_r^{\a_r}.
    \end{equation*}
    Cette décomposition de $n$ en facteurs premiers est unique.
    \tcblower
    \bf{Existence:}\\
    Si $n$ est premier c'est bon. Sinon, $\exists n_1,n_2\in\lb2,n\rb\mid n=n_1n_2$.\\
    Il faut raisonner sur $n_1$ et $n_2$ et les décomposer par récurrence forte.\n
    \bf{Unicité:} On considère deux décompositions $n=p_1^{\a_1}...p_r^{\a_r}=q_1^{\b_1}...q_s^{\b_s}$ où $r,s\in\N^*$ et les $p_i,q_i$ sont premiers.\\
    On suppose les $p_i$ et $q_i$ distincts deux-à-deux.\n
    Montrons que $\{p_1,...,p_r\}=\{q_1,...,q_s\}$. Pour $i\in\lb1,r\rb$, on a que $p_i$ divise $q_1^{\b_1}...q_s^{\b_s}$.\\
    D'après le lemme d'euclide, $\exists j\in\lb1,s\rb\mid p_i \mid q_j$ donc $p_j=q_j$ car ils sont tous les deux premiers.\\
    On a donc $\{p_1,...,p_r\}\subset\{q_1,...,q_s\}$. On a l'autre inclusion de la même manière.\\
    Finalement, $\{p_1,...,p_r\}=\{q_1,...,q_s\}$, donc $r=s$ par égalité de cardinaux.\n
    On a $n=p_1^{\a_1}...p_r^{\a_r}=p_1^{\b_1}...p_r^{\b_r}$. Montrons que pour $i\in\lb1,r\rb$, on a $\a_i=\b_i$.\\
    Supposons que $\a_i<\b_i$ SPDG. Alors:
    \begin{equation*}
        p_i^{\a_i}\prod_{j\neq i}p_j^{\a_j}=p_i^{\b_i}\prod_{j\neq i}p_j^{\b_j}.
    \end{equation*}
    Puisque $\Z$ est intègre et que $p_i^{\a_i}\neq0$, on a:
    \begin{equation*}
        \prod_{j\neq i}p_j^{\a_j}=p_i^{\b_i-\a_i}\prod_{j\neq i}p_j^{\b_j}.
    \end{equation*}
    Donc $p_i\mid\prod_{j\neq i}p_j^{\a_j}$, donc $p_i$ divise l'un des $p_j$ pour $j\neq i$, ce qui est absurde.\\
    On a donc $\a_i=\b_i$.
\end{thm}

\subsection{Valuations \texorpdfstring{p}{Lg}-adiques.}

\begin{defi}{}{}
    Soit $p$ un nombre premier et $n$ un entier naturel non nul.\\
    On appelle \bf{valuation $p$-adique} de $n$, notée $v_p(n)$ l'exposant de $p$ dans la décomposition de $n$ en facteurs premiers (cet exposant valant 0 si $p$ ne figure pas dans la décomposition).
\end{defi}

\pagebreak

\begin{prop}{}{}
    Soit $n\in\N^*$, $p$ premier et $k\in\N$.
    \begin{equation*}
        v_p(n) = k \quad \iff \quad \exists q \in \N \quad n=p^kq \quad \nt{et} \quad p \land q = 1
    \end{equation*}
    \tcblower
    On distingue un entier $p_0$.\\
    \fbox{$\ra$} Supposons $v_{p_0}(n)=k$, on écrit la décomposition de $n$: $\prod_{p\in\P}p^{v_p(n)}$
    Notons $q=\prod_{p\neq p_0}p^{v_p(n)}$, alors $n=p_0^{v_{p_0}(n)}q$.\\
    De plus, $p\land q=1$ par produit.\n
    \fbox{$\la$} Supposons $\exists q \in \N\quad n=p^kq\quad\nt{et}\quad p\land q=1$.\\
    On a $q=\prod_{p\in\P}p^{v_p(q)}$, alors $n=p_0^k\prod_{p\in\P}p^{v_p(q)}$.\\
    Or $p_0\land q = 1$ donc $v_{p_0}(q)=0$ (car sinon $p_0\mid q$).\\
    On peut donc écrire $n=p_0^k\prod_{p\neq p_0}p^{v_p(q)}$.\\
    Par unicité, $v_{p_0}(n)=k$.
\end{prop}

\begin{prop}{}{}
    Soit $p$ un nombre premier.
    \begin{enumerate}[topsep=0pt,itemsep=-0.9 ex]
        \item $\forall(m,n)\in(\N^*)^2,\quad v_p(mn)=v_p(m)+v_p(n)$.
        \item $\forall n \in \N^*~\forall k\in\N\quad v_p(n^k)=kv_p(n)$.
    \end{enumerate}
    \tcblower
    \boxed{1.} On écrit les décomposition de $m$ et $n$.
    \begin{equation*}
        m=\prod_{p\in\P}p^{v_p(m)} \quad \nt{et} \quad n=\prod_{p\in\P}p^{v_p(n)}.
    \end{equation*}
    Donc:
    \begin{equation*}
        mn=\left( \prod_{p\in\P}p^{v_p(m)} \right)\left( \prod_{p\in\P}p^{v_p(n)} \right) = \prod_{p\in\P}p^{v_p(m)+v_p(n)}.
    \end{equation*}
    Par unicité de la décomposition, pour $p_0\in\P$, $v_{p_0}(mn)=v_{p_0}(m)+v_{p_0}(n)$.\n
    \boxed{2.} Récurrence facile.
\end{prop}

\begin{ex}{}{}
    Soit $n$ entier supérieur à 2. Montrer que $\sqrt{n}\in\Q \iff \exists k \in \Z \mid n = k^2$.
    \tcblower
    \fbox{$\la$} Supposons $n=m^2$ où $m\in\Q$ alors $\sqrt{n}=\sqrt{m^2}=m\in\Q$.\n
    \fbox{$\ra$} Supposons $\sqrt{n}\in\Q$, alors $\exists a,b\in\N\times\N^*$ tels que $\sqrt{n}=\frac{a}{b}$, alors $b^2n=a^2$.\\
    Soit $p\in\P$, on a $v_p(b^2n)=v_p(a^2)$ donc $2v_p(b)+v_p(n)=2v_p(a)$ donc $v_p(n)$ est pair :
    \begin{equation*}
        n=\prod_{p\in\P}p^{v_p(n)}=\prod_{p\in\P}p^{2v_p(n)/2}=\left( \prod_{p\in\P}p^{\frac{v_p(n)}{2}} \right)^2
    \end{equation*}
    Donc $n$ est bien un carré d'entiers.
\end{ex}

\begin{thm}{Description des diviseurs d'un entier.}{}
    Soit $n$ un entier naturel supérieur ou égal à 2, s'écrivant
    \begin{equation*}
        n = p_1^{\a_1}p_2^{\a_2}...p_r^{\a_r}.
    \end{equation*}
    Ses diviseurs positifs sont exactement les entiers de la forme
    \begin{equation*}
        p_1^{\b_1}p_2^{\b_2}...p_r^{\b_r}, \quad \nt{avec} \quad \forall i \in \lb1,r\rb ~ 0 \leq \b_i \leq \a_i.
    \end{equation*}
    \tcblower
    Découle du corrolaire \ref{corr:45}.
\end{thm}

\pagebreak

\begin{corr}{}{45}
    Soient $m,n\in\N^*$.
    \begin{center}
        $m\mid n \quad \iff \quad \forall p\in\P\quad v_p(m)\leq v_p(n)$.
    \end{center}
    \tcblower
    \fbox{$\ra$} Supposons $m \mid n$, alors $\exists k\in\N^*~n=mk$.\\
    Alors pour $p\in\P,~v_p(n)=v_p(m)+v_p(k)$ donc $v_p(m)\leq v_p(n)$.\n
    \fbox{$\la$} Supposons $\forall p\in\P\quad v_p(m)\leq v_p(n)$.\\
    On a:
    \begin{equation*}
        n=\prod_{p\in\P}p^{v_p(n)}=\prod_{p\in\P}p^{v_p(m)+v_p(n)-v_p(m)}=\prod_{p\in\P}p^{v_p(m)}\prod_{p\in\P}p^{v_p(n)-v_p(m)}.
    \end{equation*}
    Donc $m\mid n$.
\end{corr}

\begin{ex}{Un cas particulier important.}{}
    Soit $p$ un nombre premier et $\a$ un entier naturel non nul. Quels sont les diviseurs de $p^\a$ ?
    \tcblower
    On a $\D(p^\a)=\{p^\b\mid 0\leq\b\leq\a\}$.
\end{ex}

\begin{ex}{}{}
    Combien de diviseurs possède le nombre trente-six-milliards ?
    \tcblower
    Notons $N=36\times10^9$, alors $N=(2\times3)^2(2\times5)^9=2^{11}3^25^9$.\\
    Ainsi, $\D(N)=\{2^\a3^\b5^\g\mid (\a,\b,\g)\in\lb0,11\rb\times\lb0,2\rb\times\lb0,9\rb\}$ et $|\D(N)|=12\times3\times10=360$.
\end{ex}

\begin{prop}{Décomposition primaire du PGCD, du PPCM.}{}
    Soient $a$ et $b$ deux entiers naturels non nuls, dont une décomposition sur une même famille de nombres premiers $p_1<...<p_r$ est
    \begin{equation*}
        a=p_1^{\a_1}...p_r^{\a_r} \quad \nt{et} \quad b=p_1^{\b_1}...p_r^{\b_r}.
    \end{equation*}
    où les $\a_i$ et $\b_i$ sont des entiers naturels éventuellement nuls, on a alors
    \begin{equation*}
        a \land b = \prod_{i=1}^rp_i^{\min(\a_i,\b_i)}, \quad \nt{et} \quad a\lor b = \prod_{i=1}^rp_i^{\max(\a_i,\b_i)}.
    \end{equation*}
    De manière équivalente, pour tout nombre premier $p$,
    \begin{equation*}
        v_p(a\land b) = \min(v_p(a),v_p(b)) \quad \nt{et} \quad v_p(a\lor b)=\max(v_p(a), v_p(b)).
    \end{equation*}
    \tcblower
    C'est clair.
\end{prop}

\section{Congruences.}
\subsection{Une relation d'équivalence compatible avec la somme et le produit.}

\begin{defi}{}{}
    Soit $n\in\Z$. On dit que deux entiers relatifs sont \bf{congrus modulo} $n$, ce que l'on note $a\equiv b[n]$ s'il existe un entier relatif $k$ tel que $a=b+kn$.
\end{defi}

\begin{prop}{}{}
    Soit $n\in\Z$.
    \begin{enumerate}[topsep=0pt,itemsep=-0.9 ex]
        \item La relation de congruence modulo $n$ est une relation d'équivalence sur $\Z$.
        \item $\forall a \in \Z \quad n \mid a \iff a \equiv 0[n]$. En particulier $n\equiv 0[n]$.
        \item Compatible avec la somme et le produit:
        \begin{equation*}
            \forall (a,b,a',b')\in\Z^4\quad\begin{cases}
                a\equiv b &[n]\\
                a'\equiv b &[n]
            \end{cases}\ra\begin{cases}
                a+a'\equiv b+b' &[n]\\
                aa'\equiv bb' &[n]
            \end{cases}
        \end{equation*}
    \end{enumerate}
    \tcblower
    \boxed{1.} et \boxed{2.} Trivial.\n
    \boxed{3.} Supposons $a\equiv b[n]$ et $a'\equiv b'[n]$, $\exists k,k'\in\Z\mid b-a=nk$ et $b'-a'=nk'$.\\
    Alors $(b+b')-(a+a')=n(k+k')$, il est clair que $n\mid (b+b')-(a+a')$ donc $a+a'\equiv b+b'[n]$.\\
    De même, $bb'=(a+nk)(a'+nk')=aa'+n(ak'+ak+nkk')$ donc $aa'=bb'[n]$.
\end{prop}

\begin{prop}{}{}
    Soit $n\in\N^*$.
    \begin{enumerate}[topsep=0pt,itemsep=-0.9 ex]
        \item $\forall a \in \Z, ~ \exists!r\in\lb0,n-1\rb\mid a\equiv r[n]$.
        \item Il y a exactement $n$ classes d'équivalence pour la relation de congruence module $n$.
    \end{enumerate}
    \tcblower
    \boxed{1.} Soit $a\in\Z$, $\exists!(q,r)\in\Z\times N \mid a=nq+r \quad \nt{et} \quad 0\leq r \leq n-1$.\\
    On a bien $n\mid a-r$ donc $a\equiv r[n]$.\n
    \boxed{2.} Tout entier appartient à une unique classe d'équivalence modulo $n$ : celle de son reste dans sa division euclidienne avec $n$.
\end{prop}

\begin{ex}{}{}
    Démontrer qu'un entier naturel est un multiple de 3 si et seulement si la somme de ses chiffres est un multiple de 3, idem avec 9.
    \tcblower
    Soit $n\in\N$, notons $N=\sum_{k=0}^pc_k10^k$ avec $p\in\N$ et $\forall k\in\lb0,p\rb$, $c_k\in\lb0,9\rb$ le $k^{\nt{ème}}$ chiffre de $n$.\\
    Remarque: $10=1[3]$ donc $\forall k \in \N ~ 10^k=1[3]$.\\
    Par somme et produit modulo 3, on obtient:
    \begin{equation*}
        \sum_{k=0}^pc_k10^k\equiv\sum_{k=0}^pc_k\cdot1[3]
    \end{equation*}
    \begin{equation*}
        3 \mid n \iff n \equiv 0[3] \iff \sum_{k=0}^pc_k\equiv0[3] \iff 3 \mid \sum_{k=0}^pc_k.
    \end{equation*}
\end{ex}

\subsection{Inversibilité modulo \texorpdfstring{n}{Lg}.}

\begin{prop}{}{}
    Soit $n\in\N^*$ et $a\in\Z$.\\
    Si $a\land n=1$, alors $\exists b\in\Z \mid ab\equiv1[n]$ (la réciproque est vraie).\n
    Dans le cas $a\land n = 1$, si $x$ et $y$ sont deux entiers, on a
    \begin{equation*}
        ax\equiv y[n] \iff x\equiv by[n]
    \end{equation*}
    \tcblower
    Supposons que $a$ et $n$ sont premiers entre eux. Le théorème de Bézout donne alors l'existence d'un couple $(u,v)$ d'entiers relatifs tels que $au+nv=1$. Posons $b=u$ et passons modulo $n$:
    \begin{equation*}
        ab+0v\equiv 1[n]
    \end{equation*}
    ce qui montre que $b$ est un inverse de $a$ modulo $n$. Pour $x$ et $y$ dans $\Z$, on a
    \begin{equation*}
        ax\equiv y[n] \iff abx\equiv by[n]
    \end{equation*}
    (on multiplie par $b$ dans le sens direct, par $a$ dans le sens indirect en utilisant $ab\equiv1[n]$). On a bien
    \begin{equation*}
        ax\equiv y[n] \iff x\equiv by[n].
    \end{equation*}
\end{prop}

\begin{ex}{}{}
    Résoudre l'équation $7x\equiv11[31]$.
    \tcblower
    On remonte l'algorithme d'Euclide:\\
    On a $31=7\time4+3$, $7=3\times2+1$.\\
    Alors $1=7-3\times2=7-(31-7\times4)\times2=7\times9 - 2\times31$.\n
    Soit $x\in\Z$, $7x\equiv11[31]\iff9\times7x\equiv99[31]\iff x\equiv 99[31]\iff x\equiv 6[31]$.\\
    L'ensemble des solutions : $\{31k+6\mid k\in\Z\}$.
\end{ex}

\begin{corr}{}{}
    Soit $n\in\N^*$, ainsi que deux entiers relatifs $a$ et $y$.\\
    L'équation $ax\equiv y[n]$ possède une solution dans $\Z$ si et suelement si $a\land n$ divise $y$.\\
    Dans le cas où une solution existe, elle est unique modulo $\frac{n}{a\land n}$.
    \tcblower
    $\circledcirc$ Supposons qu'il existe $x\in\Z$ tel que $ax\equiv y[n]$. Alors, il existe $k\in\Z$ tel que $y=ax+kn$.\\
    Puisque $a\land n$ divise $a$ et $n$, il divise $y$.\\
    $\circledcirc$ Supposons réciproquement que $a\land n$ divise $y$. Notons alors $d=a\land n$; il existe $a'$ et $n'$ premiers entre eux tels que $a=da'$, $n=dn'$ et $y=dy'$. Pour $x\in\Z$, on a $ax\equiv y[n]\iff da'x\equiv dy'[dn']\iff a'x\equiv y'[n]$ .\\
    On a pu simplifier par $d$ par intégrité de $\Z$. Alors $a'$ et $n'$ sont premiers entre eux: il existe $b'$ tel que $a'b'\equiv1[n']$ et l'équation $a'x\equiv y'[n']$ possède $by'$ comme unique solution modulo $n'$ c'est à dire modulo $\frac{n}{a\land n}$.
\end{corr}

\subsection{Petit théorème de Fermat.}

\begin{prop}{}{}
    Soit $p$ un nombre premier.
    \begin{enumerate}[topsep=0pt,itemsep=-0.9 ex]
        \item $\forall k\in\lb1,p-1\rb,~ p\mid\binom{p}{k}$.
        \item $\forall (a,b)\in\Z^2~(a+b)^p\equiv a^p+b^p[p]$.
    \end{enumerate}
    \tcblower
    \boxed{1.} Soit $k\in\lb1,p-1\rb$. On a:
    \begin{equation*}
        k\binom{p}{k}=p\binom{p-1}{k-1}
    \end{equation*}
    Alors $p\mid k\binom{p}{k}$ et $p\land k=1$ donc d'après le Lemme de Gauss, $p\mid\binom{p}{k}$.\n
    \boxed{2.} Soient $a,b\in\Z$, on a:
    \begin{equation*}
        (a+b)^p=\sum_{k=0}^p\binom{p}{k}a^kb^{p-k}=a^p+b^p+\underbrace{\sum_{k=1}^{p-1}\binom{p}{k}a^kb^{p-k}}_{\nt{multiple de }p}\equiv a^p+b^p[p].
    \end{equation*}
\end{prop}

\begin{thm}{Petit théorème de Fermat.}{}
    Soit $n$ un entier relatif et $p$ un nombre premier. On a \fbox{${n^p\equiv n[p]}$}.\n
    Si de surcroît $n$ n'est pas un multiple de $p$, on a $n^{p-1}\equiv1[p]$.
    \tcblower
    Fixons $p$ premier. Par récurrence sur $n$:\\
    \bf{Initialisation:} $0^p=0[p]$.\\
    \bf{Hérédité:} Supposons pour $n\in\N$ que $n^p\equiv n[p]$. Alors:
    \begin{equation*}
        (n+1)^p\equiv n^p + 1^p [p] \equiv n+1[p]
    \end{equation*}
    On conclut par récurrence.\n
    Si $n$ n'est pas multiple de $p$, alors $n\land p=1$, alors $n$ est inversible modulo $p$ et on a bien que $n^{p-1}\equiv 1[p]$.
\end{thm}

\begin{ex}{Plus petit nombre de Carmichael.}{}
    1. Vérifier que
    \begin{equation*}
        561\equiv 1[2], \qquad 561\equiv1[10], \qquad 561\equiv 1[16].
    \end{equation*}
    2. Démontrer que pour tout entier $n$,
    \begin{equation*}
        n^{561}\equiv n[3], \qquad n^{561}\equiv n[11], \qquad n^{561}\equiv n[17].
    \end{equation*}
    3. Démontrer que pour tout $n\in\Z$, on a $n^{561}\equiv n[561]$.\\
    4. \emph{Le petit théorème de Fermat n'est pas un critère de primalité.} Expliquer.
    \tcblower
    \boxed{1.} Clair pour les deux premiers, et $561=35\times16+1$ donc $561\equiv1[16]$.\n
    \boxed{2.} On a $n^{561}\equiv n^{2\times280+1}[3]\equiv n(n^2)^{280}[3]$.\\
    Si $n$ n'est pas multiple de 3, alors $n^{2}\equiv 1[3]$ d'après le petit théorème de Fermat. Alors $n^{561}\equiv n[3]$.\\
    Si $n$ est multiple de 3, alors $n^{561}\equiv0[3]\equiv n[3]$.\\
    De même, $n^{561}\equiv n[11]$ et $n^{561}\equiv n[17]$.\n
    \boxed{3.} Soit $n\in\Z$, on a $561=17\times11\times3$, alors $3\mid n^{561}-n$ et $11\mid n^{561}-n$ et $17\mid n^{561}-n$.\\
    Ainsi, $3\times11\times17\mid n^{561}$ car $3$, $11$ et $17$ sont premiers entre-eux.\n
    \boxed{4.} Le petit théorème de Fermat ne caractérise pas les nombre premiers. En effet, 561 n'est pas premier mais vérifie le petit théorème de Fermat.
\end{ex}

\section{Exercices.}

\begin{exercice}{}{}
    Soit $n\in\N$, que vaut le PGCD de $3n+1$ et de $2n$ ?
    \tcblower
    On applique l'algorithme d'Euclide:
    \begin{align*}
        3n+1 &= 2n\times1+(n+1)\\
        2n &= (n+1)\times1+(n-1)\\
        n+1 &= (n-1)\times1+2
    \end{align*}
    Par transitivité, $\PGCD(3n+1,2n)=\PGCD(n-1,2)$.\\
    Le PGCD est donc 1 si $n$ est impair, et 2 si $n$ est pair.
\end{exercice}

\begin{exercice}{}{}
    Soient deux entiers relatifs $a$ et $b$ tels que $a^2\mid b^2$. Montrer que $a\mid b$.
    \tcblower
    Soit $d=a\land b$ ,alors $\exists (a',b')\in\Z^2\mid a=da',~b=db',~a'\land b'=1$.\\
    Par hypothèse, $\exists k \in \Z \mid b^2=a^2k$ donc $d^2{b'}^2=d^2{a'}^2k$.\\
    $\circledcirc$ Si $d=0$, alors $a=b=0$ et $a\mid b$.\\
    $\circledcirc$ Si $d\neq 0$, alors $(b')^2=(a')^2k$ donc $a'\mid(b')^2$ et $a'\land b'=1$.\\
    D'après le Lemme de Gauss, $a'\mid b'$ donc $da'\mid db'$ donc $a\mid b$.\n
    \bf{Solution :}\n
    Soit $p$ un nombre premier, $a^2\mid b^2$ donc $v_p(a^2)\leq v_p(b^2)$ donc $2v_p(a)\leq2v_p(b)$ donc $v_p(a)\leq v_p(b)$.\\
    Et ce pour tout $p$, donc $a\mid b$.
\end{exercice}



\end{document}