\documentclass[11pt]{article}

\def\chapitre{41}
\def\pagetitle{Fractions rationnelles.}

\input{/home/theo/MP2I/setup.tex}

\begin{document}

\input{/home/theo/MP2I/title.tex}

\section{Fractions rationnelles.}

\subsection{Le corps \texorpdfstring{$\K[X]$}{Lg}}


\begin{defi}{}{}
    On note $K(X)$ le corps des fractions de l'anneau $\K[X]$.\\
    Ses éléments, appelés \bf{fractions rationnelles} sont de la forme
    \begin{equation*}
        F=AB^{-1} \quad \nt{ou} \quad F=\frac{A}{B}.
    \end{equation*}
    où $A,B\in\K[X]$, avec $B\neq0$. Le couple $(A,B)$ est appelé un \bf{représentant} de $F$.
\end{defi}

\begin{prop}{}{}
    Soit $F\in\K(X)$. Il existe un unique $(A,B)\in\K[X]^2$ tel que
    \begin{equation*}
        B \nt{ est unitaire et non nul}, \quad A\land B = 1\quad\nt{et}\quad F=\frac{A}{B}.
    \end{equation*}
    C'est la \bf{forme irréductible} de la fraction $F$.
\end{prop}

\subsection*{Fonction rationnelle associée, composée, dérivation.}

\begin{defi}{Fonction rationnelle associée à une fraction rationnelle.}{}
    Soit $F\in\K(X)$ de forme irréductible $F=\frac{A}{B}$. On pose $D_F=\K\setminus\{\a\in\K\mid B(\a)=0\}$.\\
    On appelle \bf{fonction rationnelle} associée à $F$ l'application $\tilde{F}:\begin{cases}
        D_F &\to \quad \K\\
        x &\mapsto\quad \frac{A(x)}{B(x)}
    \end{cases}$
\end{defi}

\begin{ex}{}{}
    Soit $F=\frac{X^4+1}{X^4+X^2+1}$. Montrer que $F(-X)=F(X)$ et $F(\frac{1}{X})=F(X)$.
\end{ex}

\begin{ex}{}{}
    Soit $P$ un polynôme de $\K[X]$, non nul, de degré $d$. Justifier que $X^dP(\frac{1}{X})$ est un polynôme.\\
    Pourquoi l'appelle-t-on parfois polynôme symétrique de $P$ ?
    \tcblower
    Notons $P=\sum_{k=0}^da_kX^k$. Alors:
    \begin{equation*}
        X^dP\left(\frac{1}{X}\right)=X^d\sum_{k=0}^da_k\frac{1}{X^k}=\sum_{k=0}^da_kX^{d-k}=\sum_{j=0}^da_{d-j}X^j.
    \end{equation*}
    \bf{Remarque:} $f:P\mapsto X^dP(\frac{1}{X})$ est un endomorphisme de $R_d[X]$.
\end{ex}

\subsection*{Degré et partie entière d'une fraction rationnelle.}

\begin{defi}{}{}
    Soit $F\in\K(X)$ de forme irréductible $F=\frac{A}{B}$. Le \bf{degré} de $F$ est
    \begin{equation*}
        \deg(F) = \deg(A) - \deg(B).
    \end{equation*}
    On a donc $\deg(F)\in\Z\cup\{-\infty\}$.
\end{defi}

\begin{prop}{Degré et opérations.}{}
    Soient $F,G\in\K(X)$.
    \begin{enumerate}[topsep=0pt,itemsep=-0.9 ex]
        \item $\deg(F+G)\leq\max(\deg(F),\deg(G))$.
        \item $\forall \l\in\K, ~ \deg(\l F)\leq \deg(F)$.
        \item $\deg(FG)=\deg(F)+\deg(G)$.
        \item Si $G\neq0, ~ \deg(\frac{F}{G}) = \deg(F) - \deg(G)$
    \end{enumerate}
\end{prop}

\begin{ex}{}{}
    \begin{equation*}
        \deg\left( \frac{X^4}{X^2+1} \right) = 2; \quad \deg\left( \frac{X-1}{X^3+1} \right) = -2; \quad \deg\left( \frac{X^2+1}{X^2-1} \right) = 0.
    \end{equation*}
\end{ex}

\begin{prop}{}{}
    Soit $F=\frac{A}{B}\in\K(X)$. Alors
    \begin{equation*}
        \exists!(Q,R)\in(\K[X])^2\mid A = BQ+R \quad \nt{et} \quad \deg(R)<\deg(B).
    \end{equation*}
    On divise par $B$: $\frac{A}{B}=Q+\frac{R}{B}$.\\
    On appelle $Q$ la \bf{partie entière} de $F$, notée $E$, et $\frac{R}{B}$ est une fraction de degré strictement négatif.
\end{prop}

\begin{ex}{}{}
    Soit $n\in\N^*$. Quelle est la partie entière de $\frac{X^n}{X-1}$.
    \tcblower
    On a
    \begin{equation*}
        X^n=X^n-1+1=(X-1)\sum_{k=0}^{n-1}X^k+1
    \end{equation*}
    La partie entière est donc $\sum_{k=0}^{n-1}X^k$.
\end{ex}

\subsection{Zéros et pôles d'une fraction rationnelle.}

\begin{defi}{}{}
    Soit $F\in\K(X)$ de forme irréductible $F(X)=\frac{A(X)}{B(X)}$. Soient $\a\in\K$ et $m\in\N$.
    \begin{enumerate}[topsep=0pt,itemsep=-0.9 ex]
        \item $\a$ est un \bf{zéro} de $F$ si $A(\a)=0$.
        \item $\a$ est un \bf{pôle} de $F$ si $B(\a)=0$.
    \end{enumerate}
    On parlera de \bf{pôle simple} au sujet d'un pôle de multiplicité 1.
\end{defi}

\begin{ex}{}{}
    Déterminer les pôles de $\frac{1}{X^2+1}$ dans $\R$ et dans $\C$.
    \tcblower
    Pas de pôles réels, mais $i$ et $-i$ dans $\C$.
\end{ex}

\begin{ex}{}{}
    Justifier qu'une fraction rationnelle de $\C(X)$ sans pôle ne peut être qu'un polynôme.
    \tcblower
    Soit $F=\frac{A}{B}\in\K(X)$.\\
    On utilise le théorème de d'Alembert-Gauss sur $B$: les polynômes non constants ont des racines dans $\C$, ce n'est pas le cas de $B$ donc il est constant (non nul).\\
    Alors $\exists\l\in\K^*\mid B=\l$ et $F=\frac{1}{\l}A\in\K[X]$.
\end{ex}

\begin{prop}{}{}
    Soit $F\in\R(X)$ et $\a\in\C$.\\
    Alors $\a$ est pôle de $F$ de multiplicité $m$ ssi $\ov{a}$ l'est aussi.
\end{prop}

\section{Décomposition en éléments simples.}
\subsection{Les deux théorèmes.}

\begin{thm}{Décomposition en éléments simples de $\C(X)$.}{}
    Soit $F\in\C(X)$ sous forme irréductible, le dénominateur étant décomposé en facteurs irréductibles:
    \begin{equation*}
        F(X)=\frac{A(X)}{\prod\limits_{k=1}^r(X-\a_k)^{m_k}}
    \end{equation*}
    Les complexes $\a_k$ sont distincts deux-à-deux, les $m_k$ sont des entiers naturels non nuls. Alors il existe:\\
    --- Un unique polynôme $E\in\C[X]$ (partie entière de $F$).\\
    --- Une unique famille $(a_{k,j})$ de complexes.\\
    tels que
    \begin{equation*}
        F(X)=E(X)+\sum_{k=1}^r\left( \sum_{j=1}^{m_k}\frac{\a_{k,j}}{(X-\a_k)^j} \right)
    \end{equation*}
\end{thm}

\begin{thm}{Décomposition en éléments simples dans $\R(X)$.}{}
    Soit $F\in\R(X)$ sous forme irréductible:
    \begin{equation*}
        F(X)=\frac{A(X)}{\prod\limits_{k=1}^r(X-\a_k)^{m_k}\prod\limits_{l=1}^s(X^2+p_l+q_l)^{n_l}}
    \end{equation*}
    Les réels $\a_k$ sont distincts deux-à-deux; les irréductibles de degré 2: $X^2+p_l+q_l$ sont distrincts deux-à-deux, les $m_k$ et les $n_l$ sont dans $\N^*$.\\
    Alors il existe:\\
    --- Un unique polynôme $E\in\R[X]$ (partie entière de $F$).\\
    --- Une unique famille $(a_{k,j})$ de réels.\\
    --- Une unique famille $(b_{l,j}X+c_{l,j})$ de polynômes des $\R_1[X]$.\\
    tels que:
    \begin{equation*}
        F(X)=E(X)+\sum_{k=1}^r\left( \sum_{j=1}^{m_k}\frac{\a_{k,j}}{(X-\a_k)^j} \right)+\sum_{l=1}^s\left( \sum_{j=1}^{n_l}\frac{b_{l,j}X+c_{l,j}}{(X^2+p_lX+q_l)^j} \right).
    \end{equation*}
\end{thm}

\begin{ex}{Forme d'une décomposition en éléments simples.}{}
    1. Donner la forme de la décomposition en éléments simples dans $\C(X)$ de 
    \begin{equation*}
        F(X)=\frac{X}{(X+1)^3(X^2+X+1)}
    \end{equation*}
    2. Donner la forme de la décomposition en éléments simples dans $\R(X)$ de
    \begin{equation*}
        F(X)=\frac{X^2+2}{(X-1)^2(X+2)(X^2+X+1)^2}
    \end{equation*}
    \tcblower
    \boxed{1.} \begin{equation*}
        F = \frac{X}{(X+1)^3(X^2+X+1)}=\frac{X}{(X+1)^3(X-j)(X-\ov{j})}.
    \end{equation*}
    D'après le théorème: $\exists\a,\b,\g,\d,\e\in\C$ tels que
    \begin{equation*}
        F=\frac{\a}{X+1}+\frac{\b}{(X+1)^2}+\frac{\g}{(X+1)^3}+\frac{\d}{X-j}+\frac{\e}{X-\ov{j}}
    \end{equation*}
    \boxed{2.} \begin{equation*}
        F = \frac{X^2+2}{(X-1)^2(X+2)(X^2+X+1)^2}
    \end{equation*}
    Par théorème, $\exists \a_1,\a_2,\b,\g_1,\g_2,\d_1,\d_2\in\R$ tels que
    \begin{equation*}
        F(X)=\frac{\a_1}{X-1}+\frac{\a_2}{(X-1)^2} + \frac{\b}{(X+2)} + \frac{\g_1X + \d_1}{X^2+X+1} + \frac{\g_2X+\d_2}{(X^2+X+1)^2}.
    \end{equation*}
\end{ex}

\pagebreak
\subsection{Coefficient relatif à un pôle simple.}

\begin{prop}{Calcul du coefficient relatif à un pôle simple.}{}
    Soit $F\in\K(X)$ et $\a\in\K$ un \bf{pôle simple} de $F$.\\
    La décomposition en éléments simples de $F$ s'écrit donc
    \begin{equation*}
        F=\frac{c}{X-\a}+G\quad\nt{où $\a$ n'est pas un pôle de $G$}.
    \end{equation*}
    1. Formule du cache : $c=[(X-\a)F(X)](\a)$.\\
    2. Formule théorique: si $F=\frac{A}{B}$ et $B'\neq0$, alors $c=\frac{A(\a)}{B'(\a)}$.
    \tcblower
    \boxed{1.} On a $F(X)=\frac{c}{X-\a}+G$ donc $(X-\a)F(X)=c+(X-\a)G$, en évaluant en $\a$ on trouve $c=[(X-\a)F(X)](\a)$.\\
    \boxed{2.} Notons $B=(X-\a)Q$, avec $Q\in\K[X]$ tel que $Q(\a)\neq0$.\\
    On a $B'=(X-\a)Q'+Q$ donc $B'(\a)=Q(\a)$, or :
    \begin{equation*}
        \frac{A}{(X-\a)Q}=\frac{C}{X-\a}+G
    \end{equation*}
    Donc $\frac{A}{Q}=c+(X-\a)G$ et donc $\frac{A(\a)}{Q(\a)}=c$ donc $c=\frac{A(\a)}{Q(\a)}=\frac{A(\a)}{B'(\a)}$.
\end{prop}

\begin{ex}{}{}
    Décomposer en éléments simples dans $\R(X)$:
    \begin{equation*}
        F=\frac{2X-1}{X^3+3X^2+2X}
    \end{equation*}
    \begin{equation*}
        G=\frac{n!}{X(X-1)...(X-n)} \quad (n\in\N^*)
    \end{equation*}
    \tcblower
    \boxed{1.} On a:
    \begin{equation*}
        F=\frac{2X-1}{X(X+1)(X+2)}
    \end{equation*}
    Alors $\exists \a, \b, \g\in\R$ tels que
    \begin{equation*}
        F(X)=\frac{\a}{X}+\frac{\b}{X+1}+\frac{\g}{X+2}.
    \end{equation*}
    Alors $\a=-\frac{1}{2}$, $\b=3$, $\g=-\frac{5}{2}$.\\
    \boxed{2.} On a:
    \begin{equation*}
        G=\frac{n!}{X(X-1)...(X-n)}
    \end{equation*}
    Alors $\exists \a_0...\a_n\in\R$ tels que
    \begin{equation*}
        G(X)=\sum_{k=0}^n\frac{\a_k}{X-k}
    \end{equation*}
    Pour $k\in\lb0,n\rb$ fixé, $\a_k=\frac{n!}{k(k-1)...1\cdot(-1)...(k-n)}=(-1)^{n-k}\frac{n!}{k!(n-k)!}=(-1)^{n-k}\binom{n}{k}$.
\end{ex}

\begin{ex}{}{}
    Soit $P\in\K[X]$ de degré $n\geq1$ admettant $n$ racines distinctes non nulles $z_1,...,z_n$.\\
    1. Décomposer en éléments simples la fraction $P^{-1}$.\\
    2. En déduire la valeur de \begin{equation*}
        \sum_{k=1}^n\frac{1}{z_kP'(z_k)}.
    \end{equation*}
    \tcblower
    \boxed{1.} On a $P=\l\prod_{i=1}^n(X-z_k)$ où $\l$ est le coefficient dominant de $P$.\\
    Alors $\exists \a_1,...,\a_n\in\K$ tels que
    \begin{equation*}
        P^{-1}=\sum_{k=1}^n\frac{a_k}{X-z_k}.
    \end{equation*}
    Pour $k\in\lb1,n\rb$, la formule du cache donne $a_k=\frac{1}{\l\prod\limits_{i\neq k}^n(z_i-z_k)}$.\n
    La formule théorique donne $\forall k\in\lb1,n\rb~a_k=\frac{1}{P'(z_k)}$, alors
    \begin{equation*}
        P^{-1}=\sum_{k=1}^n\frac{1}{P'(z_k)(X-z_k)}
    \end{equation*}
    \boxed{2.} On évalue en 0:
    \begin{equation*}
        P^{-1}(0)=-\sum_{k=1}^n\frac{1}{z_kP'(z_k)}
    \end{equation*}
\end{ex}

\subsection{Une décomposition importante : celle de \texorpdfstring{$\frac{P'}{P}$}{Lg}.}

\begin{lemme}{Dérivée logarithmique d'un produit.}{}
    Soient $P_1,...,P_r\in\K[X]\setminus\{0\}$. Alors
    \begin{equation*}
        \frac{\left( \prod\limits_{k=1}^rP_k \right)'}{\prod\limits_{k=1}^rP_k}=\sum_{k=1}^r\frac{P_k'}{P_k}
    \end{equation*}
    \tcblower
    \begin{equation*}
        \left( \prod_{k=1}^r P_k \right)"=\sum_{k=1}^rP_k'\prod_{i\neq k}P_i.
    \end{equation*}
    On divise par $\prod_{i=1}^rP_i$:
    \begin{equation*}
        \frac{\left( \prod_{k=1}^rP_k' \right)}{\prod_{k=1}^rP_k}=\sum_{k=1}^r\frac{P_k'}{P_k}.
    \end{equation*}
\end{lemme}

\begin{thm}{Décomposition en éléments simples de $\frac{P'}{P}$}{}
    Soit $P$ un polynôme scindé sur $\K$ :
    \begin{equation*}
        P=\l\prod_{k=1}^r(X-\a_k)^{m_k}
    \end{equation*}
    Alors
    \begin{equation*}
        \frac{P'}{P}=\sum_{k=1}^r\frac{m_k}{X-\a_k}
    \end{equation*}
    \tcblower
    Posons $P_k=(X-\a_k)^{m_k}$. D'après le Lemme:
    \begin{equation*}
        \frac{P_k'}{P_k}=\frac{\left( \prod\limits_{k=1}^rP_k \right)'}{\prod\limits_{k=1}^rP_k}=\sum_{k=1}^r\frac{P_k'}{P_k}
    \end{equation*}
    Or:
    \begin{equation*} 
        \forall k \in \lb1,r\rb, ~ \frac{P_k'}{P_k}=\frac{m_k(X-a_k)^{m_k-1}}{(X-\a_k)^{m_k}}=\frac{m_k}{X-\a_k}
    \end{equation*}
    Donc $\frac{P'}{P}=\sum_{k=1}^r\frac{m_k}{X-\a_k}$.
\end{thm}

\begin{ex}{Utilisation d'une décomposition en éléments simples.}{}
    Soit $n\in\N$, $n\geq2$. Déterminer la forme irréductible de
    \begin{equation*}
        F=\sum_{k=1}^{n-1}\frac{1}{X-e^{\frac{2ik\pi}{n}}}
    \end{equation*}
    En déduire la valeur de
    \begin{equation*}
        \sum_{k=1}^{n-1}\frac{1}{1-e^{\frac{2ik\pi}{n}}}
    \end{equation*}
    \tcblower
    Posons $P=\prod\limits_{k=1}^{n-1}(X-e^{\frac{2ik\pi}{n}})$. On sait que $F=\frac{P'}{P}$ et on a:
    \begin{equation*}
        P=\frac{X^n-1}{X-1}=\sum_{k=0}^{n-1}X^k; \quad P'=\sum_{k=0}^{n-1}kX^{k-1}.
    \end{equation*}
    On a PGCD$(P',P)=1$ car $P$ n'a que des racines simples, donc $P$ et $P'$ n'ont pas de racines communes.\n
    On évalue en 1:
    \begin{equation*}
        \sum_{k=1}^n\frac{1}{1-e^{\frac{2ik\pi}{n}}}=\frac{P'(1)}{P(1)}=\frac{\frac{n(n-1)}{2}}{n}=\frac{n-1}{2}
    \end{equation*}
\end{ex}

\subsection{Pratique de la décomposition en éléments simples.}

\begin{meth}{comment aborder un calcul de décomposition en éléments simples.}{}
    Soit $F\in\K(X)$ dont on cherche la décomposition en éléments simples.
    \begin{enumerate}[topsep=0pt,itemsep=-0.9 ex]
        \item On écrit $F$ sous forme irréductible.\\On décompose le dénominateur en facteurs irréductibles.
        \item On cherche la partie entière de $F$ (si $\deg F < 0$, la partie entière est nulle).
        \item On écrit \emph{a priori} la décomposition en éléments simples de $F$.\\Elle fait intervenir des coefficients qu'il reste à calculer.
        \item On calcule les coefficients relatifs aux pôles simples (formule du cache).
        \item On calcule les autres coefficients comme on peut.\\On peut s'aider de la parité, de la conjugaison complexe, de l'évaluation en quelques valeurs, des limites en $+\infty$...
    \end{enumerate}
\end{meth}

\begin{ex}{}{}
    Décomposer en éléments simples dans $\C(X)$.
    \begin{equation*}
        F=\frac{X^2+2}{X^2(X^2+1)} \quad \nt{et} \quad G=\frac{1}{X^n-1} \quad (n\in\N^*)
    \end{equation*}
    \tcblower
    \boxed{1.} On a:
    \begin{equation*}
        F=\frac{X^2+2}{X^2(X-i)(X+i)}=\frac{\a}{X}+\frac{\b}{X^2}+\frac{\g}{X-i}+\frac{\d}{X+i}
    \end{equation*}
    Avec la formule du cache : $\g=\frac{i}{2}$ et $\d=-\frac{i}{2}$.\\
    On multiplie par $X^2$: $\frac{X^2+2}{X^2+1}=\a X + \b + \frac{\g X^2}{X-i} + \frac{\d X^2}{X+i}$.\\
    On évalue en 0: $\b=2$.\\
    On multiple par $X$: $\frac{X^2+2}{X(X^2+1)}=\a+\frac{\b}{X}+\frac{\g X}{X-i} + \frac{\d X}{X + i}$\\
    On passe à la limite: $0 = \a + 0 + \g + \d$ donc $\a=0$.\\
    Conclusion: $F=\frac{2}{X^2}+\frac{i/2}{X-i}-\frac{i/2}{X+i}$.\n
    \boxed{2.} On a:
    \begin{equation*}
        G=\frac{1}{X^n-1}=\frac{1}{\prod\limits_{k=0}^{n-1}\left(X-e^{\frac{2ik\pi}{n}}\right)}=\sum_{k=0}^{n-1}\frac{a_k}{X-\w_k}
    \end{equation*}
    Avec $\forall k \in \lb0,n-1\rb, ~ \w_k = e^{\frac{2ik\pi}{n}}$ et $a_0,...,a_{n-1}\in\C$.\\
    Avec la formule théorique, on sait que si $\a$ est un pôle simple de $\frac{A}{B}$, alors le coefficient vaut $\frac{A(\a)}{B(\a)}$.\\
    Ici, $A=1$, $B=X^n$ et $B'=nX^{n-1}$.\\
    On a:
    \begin{equation*}\forall k \in \lb0,n-1\rb~a_k=\frac{1}{n\w_k^{n-1}}=\frac{\w_k}{n}\end{equation*}
    Donc:
    \begin{equation*}
        \frac{1}{X^n-1}=\sum_{k=0}^{n-1}\frac{\w_k}{n(X-\w_k)}.
    \end{equation*}
\end{ex}

\begin{ex}{}{}
    Décomposer en éléments simples dans $\R(X)$.
    \begin{equation*}
        F=\frac{1}{X^3+1} \quad \nt{et} \quad G=\frac{4X^2}{X^4+1}
    \end{equation*}
    \tcblower
    \boxed{1.} On a:
    \begin{equation*}
        F=\frac{1}{X^3+1}=\frac{1}{(X+1)(X^2-X+1)}=\frac{\a}{X+1}+\frac{\b X + \g}{X^2-X+1}
    \end{equation*}
    Formule du cache : $\a=\frac{1}{3}$.\\
    On évalue en 0: $F(0)=1=\a+\g$ donc $\g=\frac{2}{3}$.\\
    On évalue en 1: $F(1)=\frac{1}{2}=\frac{1}{6}+\b+\frac{2}{3}$ donc $\b=-\frac{1}{3}$.\\
    Conclusion: $F=\frac{1}{3}\left( \frac{1}{X+1} - \frac{X-2}{X^2-X+1} \right)$.\n
    \boxed{2.} On a
    \begin{equation*}
        G=\frac{4X^2}{X^4+1}=\frac{4X^2}{(X^2-\sqrt{2}X+1)(X^2+\sqrt{2}X+1)}=\frac{\a X + \b}{X^2-\sqrt{2}X+1}+\frac{\g X + \d}{X^2+\sqrt{2}X+1}
    \end{equation*}
    On passe à la limite dans $XG(X)$ : $0=\a+\g$.\\
    On évalue en 0: $0=\b+\d$.\\
    Il en faut deux autres...
\end{ex}

\end{document}