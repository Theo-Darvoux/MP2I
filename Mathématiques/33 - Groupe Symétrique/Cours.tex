\documentclass[11pt]{article}

\def\chapitre{33}
\def\pagetitle{Groupe symétrique}

\input{/home/theo/MP2I/setup.tex}

\begin{document}

\input{/home/theo/MP2I/title.tex}

\section{Permutations}

\begin{defi}{}{}
    Une bijection de $\lb 1,n \rb$ dans lui-même est appelée une \textbf{permutation} de $\lb 1,n \rb$.\n
    L'ensemble des permutations de $\lb 1,n \rb$ sera noté $S_n$.
\end{defi}

\begin{ex}{}{}
    Soient
    \begin{equation*}
        \s=\begin{pmatrix}
            1&2&3&4&5\n
            2&5&4&3&1
        \end{pmatrix} \quad \text{ et } \quad
        \s'=\begin{pmatrix}
            1&2&3&4&5\n
            2&3&4&1&5
        \end{pmatrix}
    \end{equation*}
    Calculer $\s\s'$, $\s'\s$, $\s^2$ et $\s^{-1}$.
    \tcblower
    On a :
    \begin{align*}
        \s\s' &= \begin{pmatrix}
            1 & 2 & 3 & 4 & 5 \n
            5 & 4 & 3 & 2 & 1
        \end{pmatrix}
        &\s'\s = \begin{pmatrix}
            1 & 2 & 3 & 4 & 5 \n
            3 & 5 & 1 & 4 & 2
        \end{pmatrix}\n
        \s^2 &= \begin{pmatrix}
            1&2&3&4&5\n
            5&1&3&4&2
        \end{pmatrix}
        &\s^{-1} = \begin{pmatrix}
            1&2&3&4&5\n
            5&1&4&3&2
        \end{pmatrix}
    \end{align*}
\end{ex}

\begin{prop}{}{}
    \begin{enumerate}[topsep=0pt,itemsep=-0.9 ex]
        \item $(S_n, \circ)$ est une groupe, appelé \textbf{groupe symétrique}.
        \item $S_n$ est fini et son cardinal vaut $n!$.
        \item Ce groupe n'est pas abélien dès que $n\geq3$.
    \end{enumerate}
    \tcblower
    \boxed{1} Cours sur les structures algébriques.\\
    \boxed{2} On pose $\Phi:\begin{cases}S_n \to \mathcal{A}(\lb1,n\rb)\n\s\mapsto(\s(1),...,\s(n))\end{cases}$ bijective et $|\mathcal{A}(\lb1,n\rb)|=n!$.\\
    \boxed{3} $S_3$ n'est pas abélien car $\t:=...$ et $\t'=...$ ne commutent pas.\\
    Soient $\s,\s'\in S_n ~ | ~ \s_{|\{1,2,3\}}=\t$ et $\s'_{|\{1,2,3\}}=\t'$, fixes sur $\lb4,n\rb$, alors $\s\s'\neq\s'\s$.
\end{prop}

\begin{defi}{Vocabulaire}{}
    Soit $\s\in S_n$.
    \begin{enumerate}[topsep=0pt,itemsep=-0.9 ex]
        \item Si $x\in\lb1,n\rb$, l'ensemble $\{\s^k(x),k\in\Z\}$ est appelé \textbf{orbite} de $x$.
        \item On dit que $x$ est un \textbf{point fixe} de $\s$ si $\s(x)=x$. 
        \item On appelle \textbf{support} de $\s$ l'ensemble des éléments de $\lb1,n\rb$ qui ne sont pas des points fixes.
        \item Deux permutations $\s$ et $\s'$ sont dites \textbf{conjuguées} s'il existe $\alpha\in S_n$ tel que $\s'=\alpha\s\alpha^{-1}$.
    \end{enumerate}
\end{defi}

\begin{prop}{}{}
    Deux permutations dont les supports sont disjoints commutent.
    \tcblower
    Soient $\s,\s'\in S_n$. On note $S(\s)=\{x \in \lb1,n\rb ~ | ~ \s(x)\neq x\}$.\\
    Supposons $S(\s)\cap S(\s') = \0$.\\
    Soit $x\in\lb1,n\rb$.
    \begin{itemize}[topsep=0pt,itemsep=-0.9 ex]
        \item Si $x\in S(\s)$ : $x\notin S(\s')$ donc $\s\s'(x)=\s(x)\in S(\s)$ par bijectivité de $\s$.
        \item Si $x\notin S(\s)$ : Soit $x\in S(\s')$ et on se ramène au 1er cas, soit $x\notin S(\s')$ et $\s\s'(x)=x=\s'\s(x)$.
    \end{itemize}
    Dans tous les cas, $\s\s'(x)=\s'\s(x)$
\end{prop}

\section{Cycles.}

\begin{defi}{$p$-cycles}{}
    Soit $p$ un entier supérieur à 2.\\
    Une permutation $\g$ est appellée un $p$\textbf{-cycle} s'il existe $p$ éléments distincts $a_1,...,a_p$ de $\lb1,n\rb$ tels que
    \begin{equation*}
        a_1\overset{\g}{\mapsto}a_2\overset{\g}{\mapsto}...\overset{\g}{\mapsto}a_p\overset{\g}{\mapsto}a_1 \quad \text{et} \quad \forall b \in \lb1,n\rb\setminus\{a_1,...,a_p\} ~ \g(b)=b.
    \end{equation*}
    On note alors $\g=(a_1~a_2~...~a_p)$.
\end{defi}


\begin{ex}{Conjugué d'un cycle}{}
    Soit $\g=(a_1,...,a_p)$ un $p$-cycle et $\s\in S_n$. Montrer que
    \begin{equation*}
        \s\g\s^{-1}=(\s(a_1)~\s(a_2)~...~\s(a_p)).
    \end{equation*}
    \tcblower
    Soit $b\in\lb1,n\rb\setminus\{\s(a_1),...,\s(a_p)\}$.\\
    Alors $\s\g\s^{-1}(b)=\s\g(\s^{-1}(b))=\s\s^{-1}(b)=b$ car $b\notin\{\s(a_1),...,\s(a_p)\}$ donc $\s^{-1}(b)\notin\{a_1,..,a_p\}$ donc c'est un point fixe de $\g$.\\
    Soit $j\in\lb1,p\rb$, on a $\s\g\s^{-1}(\s(a_j))=\s\g(a_j)=\s(a_{j+1})$ avec $a_{p+1}:=a_1$.\\
    On a bien que $\s\g\s^{-1}$ et $(\s(a_1)...\s(a_p))$ sont égaux en tout point.\\
    \textbf{Remarque:} Ceci démontre que tous les $p$-cycles sont conjugués.\\
    Soient $\g=(a_1~...~a_p)$ et $\g'=(b_1~...~b_p)$ deux $p$-cycles.\\
    Posons $\s\in S_n$ telle que :
    \begin{itemize}[topsep=0pt,itemsep=-0.9 ex]
        \item $\forall j \in \lb 1, p \rb ~ \s(a_j) = b_j$.
        \item Notons $\lb 1, n \rb \setminus \{a_1, ..., a_p\} := \{a_1', ..., a_{n-p}'\}$ et $\lb 1, n \rb \setminus \{b_1, ... b_p\} := \{b_1', ... , b_{n-p}'\}$.
    \end{itemize}
    On pose alors $\forall i\in\lb1,n-p\rb ~ \s(a_i')=b_i'$.\\
    Alors $\s$ est bien une bijection de $\lb1,n\rb$ dans lui-même car injective et de même cardinal.\\
    On a donc $\g'=(b_1~...~b_p) = (\s(a_1)~...~\s(a_p))=\s\g\s^{-1}$ donc $\g$ et $\g'$ sont conjugués.
\end{ex}

\begin{ex}{Calculs sur un cycle}{}
    Soit $\g=(a_1~...~a_p)$. Déterminer $\g^{-1}$ et $\g^p$.
    \tcblower
    \textbf{La réciproque $\g^{-1}$ :}\n
    Si $\g(b)=b$ alors $\g^{-1}(b)=b$ car c'est un point fixe.\n
    Soit $j\in\lb1,p-1\rb$, $\g(a_j)=a_{j+1}$ donc $a_j=\g^{-1}(a_{j+1})$.\n
    Alors $\forall k \in \lb2,p\rb$, $\g^{-1}(a_k)=a_{k-1}$, et $\g^{-1}(a_1)=a_p$.\n
    Ainsi, $\g^{-1}=(a_p ~ a_{p-1} ~ ... ~ a_2 ~ a_1)$.\n
    \textbf{La puissance $\g^p$ :}\n
    On a $\g=(a, \g(a), ..., \g^{p-1}(a))$ pour un $a\in\lb1,n\rb$.
    \begin{itemize}
        \item $\g^p(a)=\g(\g^{p-1}(a))=a$.
        \item Soit $j\in\lb1,p-1\rb$, $\g^p(\g^j(a))=\g^j(\g^p(a))=\g^j(a)$.
        \item Soit $b\in\lb1,n\rb\setminus\{a,\g(a),...,\g^{p-1}(a)\}$, alors $\g^p(b)=b$ car point fixe.
    \end{itemize}
    Ainsi, $\forall x \in \lb1,n\rb, ~ \g^p(x)=x$ donc $\g^p=id$.\n
    \textbf{Remarque:} On pourrait aussi prouver que $p=\min\{j\in\N^*~|~\g^j=id\}$.
\end{ex}

\section{Transpositions}

\begin{defi}{}{}
    Une permutation $\t$ qui est un $2$-cycle est appelé une \textbf{transposition}.\n
    Une transposition est donc une permutation de la forme $(a,b)$ où $\{a,b\}$ est une paire de $\lb1,n\rb$.
\end{defi}

\begin{prop}{Involutivité}{}
    Si $\t$ est une transposition, alors 
    \begin{equation*}
        \t^2=id \quad \text{et} \quad \t^{-1}=\t
    \end{equation*}
    \tcblower
    C'est un $2$-cycle donc $\t^2=id$.\n
    On en déduit que $\t^{-1}=\t$.
\end{prop}

\begin{lemme}{Décomposition d'un cycle en produit de transpositions}{}
    Soit $\g=(a_1~...~a_p)$. Alors
    \begin{equation*}
        \g=(a_1~a_2)(a_2~a_3)...(a_{p-1}~a_p) \qquad \text{ou} \qquad \g=(a_1~a_p)(a_1~a_{p-1})...(a_1~a_2)
    \end{equation*}
    \tcblower
    Notons $\pi=(a_1~a_2)(a_2~a_3)...(a_{p-1}~a_p)$. Montrons que $\g=\pi$.
    \begin{itemize}
        \item Soit $b\in\lb1,n\rb\setminus\{a_1,...,a_p\}$ : $\g(b)=b$ et $\forall j \in \lb1,p-1\rb, ~ (a_j~a_{j+1})(b)=b$ car $b\notin\{a_j,a_{j+1}\}$.\n
        Alors $\gamma(b)=\pi(b)=b$.
        \item Soit $j\in\lb1,p-1\rb$. Alors $\pi(a_j)=\left[...(a_{j-1}~a_j)(a_j~a_{j+1})...\right](a_j)=\left[...(a_{j-1}~a_j)\right](a_{j+1})=a_{j+1}$.
        \item $\pi(a_p)=[(a_1~a_2)...(a_{p-1}~a_p)](a_p)=[(a_1~a_2)...(a_{p-2}~a_{p-1})](a_{p-1})=...=a_1$
    \end{itemize}
    Donc $\forall x \in \lb1,n\rb ~ \g(x)=\pi(x)$\n
    \textbf{Remarque:} On retrouve que $(1~2)(2~3)=(1~2~3)$ et $(2~3)(1~2) = (3~2)(2~1)=(3~2~1)=(1~3~2)$\n
    On a $(1~2)(2~3)\neq(2~3)(1~2)$.
\end{lemme}

\section{Théorème de décomposition.}

\begin{thm}{Décomposition en produit de cycles à supports disjoints}{}
    Soit $\s\in S_n$. Il existe $\g_1,...,\g_r$ $r$ cycles à supports disjoints tels que
    \begin{equation*}
        \s=\g_1\g_2...\g_r.
    \end{equation*}
    Les $\g_i$ commutent et cette décomposition est unique à l'ordre près.
    \tcblower
    Soit $\s\in S_n$.\\
    \textbf{Une relation d'équivalence sur $\lb1,n\rb$}.\\
    Pour $i,j\in\lb1,n\rb$, on note $i\sim j$ si $\exists k \in \Z ~ | ~ j = \s^k(i)$.
    \begin{itemize}[topsep=0pt,itemsep=-0.9 ex]
        \item Soit $i\in\lb1,n\rb$. $i=\s^0(i)$ donc $i \sim i$.
        \item Soient $i,j\in\lb1,n\rb ~ | ~ i \sim j$. Alors $\exists k \in \Z ~ | ~ j = \s^k(i) ~ : ~ i = \s^{-k}(j)$ et $j \sim i$.
        \item Soient $h,i,j\in\lb1,n\rb ~ | ~ h \sim i$ et $i \sim j : \exists k,l \in \Z ~ | ~ i = \s^k(h)$ et $j=\s^l(i)$ donc $j=\s^{l+k}(h)$ et $j \sim h$.
    \end{itemize}
    Il existe alors une partition de $\lb1,n\rb$ en classes d'équivalences.\\
    On fixe $x\in\lb1,n\rb$, prouvons qu'il existe $p\in\N^*$ tel que $[x]=\{x,\s(x),...,\s^{p-1}(x)\}$.\\
    On pose $p=\min\{k\in\N^* ~ | ~ \s^k(x)=x\}$. Cet ensemble est minoré et non-vide car :
    \begin{equation*}
        S : \begin{cases}
            \Z \to \lb1,n\rb\\
            k\mapsto \s^k(x)
        \end{cases}
        \text{n'est pas injective.}
    \end{equation*}
    Ainsi, $\exists k,k' \in \Z ~ | ~ k < k' ~ \text{et} ~ \s^k(x)=\s^{k'}(x)$ donc $\s^{k'-k}(x)=x$.\\
    Or $k'-k\in\N^*$, donc $\{k\in\N^* ~ | ~ \s^k(x)=x\}\neq\0$.\\
    Il faut montrer que $[x]=\{x,\s(x),...,\s^{p-1}(x)\}$.\\
    \fbox{$\supset$} est trivial.\\
    \fbox{$\subset$} Soit $y\in[x]$ : $\exists k \in \Z ~ | ~ y = \s^k(x)$.\\
    Par division euclidienne : $\exists!(q,r)\in\Z^2 ~ | ~ k = qp + r ~ \text{et} ~ 0 \leq r \leq p-1$.\\
    Donc $y = \s^k(x) = \s^{pq+r}(x) = \s^r(\s^{pq}(x))=\s^r(x)$ : $y\in\{x,\s(x),...,\s^{p-1}(x)\}$.\\
    Notons $A_1,...,A_r$ les classes d'équivalences non triviales de $\sim$. On a prouvé que :
    \begin{equation*}
        \forall j \in \lb1,r\rb ~ \exists x_j \in \lb 1, n \rb ~ \exists p_j \in \N^* ~ | ~ A_j = \{x_j, \s(x_j), ..., \s^{p_j-1}(x_j)\}.
    \end{equation*}
    On pose alors $\g_j=(x_j~\s(x_j)~...~\s^{p_j-1}(x_j))$, il est clair que $\s=\g_1\g_2...\g_r$.
\end{thm}

\begin{ex}{Une décomposition}{}
    Soit $\s=\begin{pmatrix}1&2&3&4&5&6&7&8\n5&4&1&7&8&6&2&3\end{pmatrix}$.
    \begin{enumerate}[topsep=0pt,itemsep=-0.9 ex]
        \item Décomposer $\s$ en produit de cycles à supports disjoints.
        \item Déterminer $\s^4$, $\s^{12}$ et $\s^{666}$.
    \end{enumerate}
    \tcblower
    \boxed{1} $\s=(1~5~8~3)(2~4~7)$\\
    \boxed{2} On a:
    \begin{itemize}[topsep=0pt,itemsep=-0.9 ex]
        \item $\s^4=(\g_1\g_2)^4\underset{\text{comm}}{=}\g_1^4\g_2^4=\g_2$ car $\g_1^4=id$ et $\g_2^4=\g_2^3\g_2=\g_2$.
        \item $\s^{12}=(\gamma_1^{4})^3(\g_2^3)^4=id$
        \item $\s^{666}=(1~8)(3~5)$ car $\s^{666}=\underset{id^{55}}{\cancel{\s^{12\times55}}}\s^6$.
    \end{itemize}
\end{ex}

\begin{corr}{}{}
    Toute permutation est un produit de transpositions.\\
    La décomposition n'est pas unique et les transpositions ne commutent pas nécéssairement.
    \tcblower
    Soit $\s \in S_{n}$.\\
    Le théorème 12 nous dit que : $\s$ s'écrit comme un produit de cycles. (à supports disjoints)\\
    Or tout cycle s'écrit comme un produit de transpositions.\\
    Donc, si $\g = (a_{1} a_{2}...a_{p})$, alors $\g = (a_{1} a_{2})...(a_{p-1} a_{p})$ et $\s$ s'écrit comme produit de produit de transpositions.
\end{corr}

\begin{ex}{}{}
    Décomposer en produit de transpositions la permutation :
    \begin{equation*}
        \s = \begin{pmatrix}
            1 & 2 & 3 & 4 & 5 & 6 & 7 \\
            7 & 5 & 1 & 2 & 4 & 6 & 3
        \end{pmatrix}
    \end{equation*}
    \tcblower
    $\s = (1 7 3) (2 5 4)$ (produit de cycles)\\
    $\s = (1 7)(7 3)(2 5)(5 4)$
\end{ex}

\section{Signature}

\begin{defi}{}{}
    Soit $\s \in S_{n}$
    \begin{enumerate}[topsep=0pt,itemsep=-0.9 ex]
        \item Une paire $\{ i, j \}$ de $\lb 1, n \rb$ est une \textbf{inversion} pour $\s$ si $i - j$ et $\s(i) - \s(j)$ sont de signe opposé.
        \item Le nombre d'inversion de $\s$ est noté $Inv(\s)$
        \item On appelle \textbf{signature} de $\s$ le nombre $\e(\s) = (-1)^{Inv(\s)}$
    \end{enumerate}
\end{defi}

\begin{ex}{}{}
    Après avoir calculé son nombre d'inversions, donner la signature de
    \begin{equation*}
        \s = \begin{pmatrix}
            1 & 2 & 3 & 4 & 5 \\
            4 & 1 & 2 & 5 & 3
        \end{pmatrix}
    \end{equation*}
    \tcblower
    On va calculer $\e(\s)$ en comptant le nombre d'inversions.\\
    Il y a $\binom{5}{2}$ paires dans $\lb 1, 5 \rb$.
    \begin{center}
        \begin{tabular}{c|c|c|c|c|c|c|c|c|c|c}
            paire & \{1, 2\} & \{1, 3\} & \{1, 4\} & \{1, 5\} & \{2, 3\} & \{2, 4\} & \{2, 5\} & \{3, 4\} & \{3, 5\} & \{4, 5\} \\
            \hline
            inversion & \cmark & \cmark & \xmark & \cmark & \xmark & \xmark & \xmark & \xmark & \xmark & \cmark
        \end{tabular}
    \end{center}
    Ainsi on a $Inv(\s) = 4$ donc $\e(\s) = (-1)^{4} = 1$
\end{ex}

\begin{prop}{}{}
    \begin{enumerate}[topsep=0pt,itemsep=-0.9 ex]
        \item L'identité a pour signature $1$.
        \item Les transpositions ont pour signature $-1$.
    \end{enumerate}
    \tcblower
    \boxed{1} Il est clair que $\text{Inv}(\id_{\lb 1, n \rb}) = 0$ donc $\e(\s)=1^{0} = 1$.\\
    \boxed{2} Soit $\{i,j\}$ une paire de $\lb1,n\rb$, $\tau\in S_n$ : $\exists(a,b)\in\lb1,n\rb~|~\tau=(a~b)$ où $a\leq b$.
    \begin{itemize}[topsep=0pt,itemsep=-0.9 ex]
        \item Cas $\{i,j\}\cap\{a,b\}=\0:~\tau(i)=i$ et $\tau(j)=j$ donc $i-j$ est de même signe.
        \item Cas $i=a$ et $j\neq b$ : $\tau(a)=b$ et $\tau(j)=j ~ : ~ |\lb a+1,b-1\rb$.
        \item Cas $i\neq a, j=b$ : $\tau(i)=i$ et $\tau(b)=a$ : $|\lb a+1,b-1 \rb$.
        \item Cas $\{i,j\}=\{a,b\}$ : $\tau(a)=b$ et $\tau(b)=a$, c'est une inversion.
    \end{itemize}
    Bilan : $\text{Inv}(\tau)=2|\lb a+1,b-1\rb|+1=2(b-a)-1$, impair.\\
    Donc $\e(\tau)=-1$.
\end{prop}

\begin{prop}{La signature comme un produit}{}
    \begin{equation*}
        \forall  \s \in S_{n} ~ \e(\s) = \prod_{\{i, j\}} \frac{\s(i) - \s(j) }{i - j}
    \end{equation*}
    \tcblower
    Fixons $\{i, j \} \in \mathcal{P}_{2}(\lb 1, n \rb)$ (ensembles des paires)\\
    On a 
    \begin{equation*}
        \frac{\s(i) - \s(j) }{i - j} = (-1)^{x_{\{i, j \}}} \Big|\frac{\s(i) - \s(j) }{i - j}\Big| \quad \text{où} \quad x_{\{i, j \}} = 
        \begin{cases}
            0 & \mbox{si {i, j} n'est pas une inversion.} \\
            1 & \mbox{sinon.}
        \end{cases}.
    \end{equation*}
    Alors :
    \begin{equation*}
        \prod_{\{i, j\}} \frac{\s(i) - \s(j) }{i - j} = \prod_{\{i, j\}} (-1)^{x_{\{i, j \}}} \Big|\frac{\s(i) - \s(j) }{i - j}\Big| = (-1)^{\sum_{\{i, j \}} x_{\{i, j \}}} \times \prod_{\{i, j\}} \Big|\frac{\s(i) - \s(j) }{i - j}\Big|
    \end{equation*}
    Or :
    \begin{equation*}
        \sum_{\{i, j \}} x_{\{i, j \}} = \text{Inv}(\s) \quad \text{donc} \quad (-1)^{\sum_{\{i, j \}}} = \e(\s)
    \end{equation*}
    Le produit vaut 1 car :
    $\phi : \begin{cases}
        \mathcal{P}_{2}(\lb 1, n \rb) \to \mathcal{P}_{2}(\lb 1, n \rb) \\
        \{ i, j \} \mapsto \{ \s(i), \s(j) \}
    \end{cases}$ est une bijection.\\
    On pose alors le changement d'indice $\{ u, v \} = \{ \s(i), \s(j) \}$ :
    \begin{equation*}
        \prod_{\{i, j\}} |\s(i) - \s(j)| = \prod_{\{u, v\}} |u - v| = \prod_{\{i, j\}} |i - j|
    \end{equation*}
    Donc le produit vaut bien 1.
\end{prop}

\begin{thm}{}{}
    La signature est l'unique application $\e : S_{n} \to \{ -1, 1 \}$ telle que
    \begin{enumerate}[topsep=0pt,itemsep=-0.9 ex]
        \item $\forall \s, \s^{'} \in S_{n} ~~ \e(\s \s^{'}) = \e(\s) \e(\s^{'})$
        \item Pour toute transposition $\tau \in S_{n}, ~ \e(\tau) = -1$
    \end{enumerate}
    \tcblower
    \textbf{Unicité} : Soit $\d:S_n\to\{-1,1\} ~ | 1. \text{ et } 2.$\\
    Soit $\s\in S_n, \exists r\in\N^* ~ \exists \t_1,...,\t_r$ transpositions : $\s=\t_1...\t_r$.\\
    Alors $\d(\s)=\d(\t_1)...\d(\t_r)=\prod_{i=1}^m(-1)=\e(\t_1)...\e(\t_r)=\e(\t_1,...,\t_r)=\e(\s)$.\n
    \boxed{1} Soient $\s,\s'\in S_n$.
    \begin{equation*}
        \e(\s\s')=\prod_{\{i,j\}}\frac{\s\s'(i)-\s\s'(j)}{i-j}=\prod_{\{i,j\}}\frac{\s\s'(i)-\s\s'(j)}{\s'(i)-\s'(j)}\cdot\prod_{\{i,j\}}\frac{\s'(i)-\s'(j)}{i-j}
    \end{equation*}
    Or, 
    \begin{equation*}
        \prod_{\{i,j\}}\frac{\s(\s'(i))=\s(\s'(j))}{\s'(i)-\s'(j)}=\prod_{\{i,j\}}\frac{\s(i)-\s(j)}{i-j}=\e(\s) \quad \text{car bijection } \{i,j\}\mapsto\{\s(i),\s(j)\}.
    \end{equation*}
    Donc $\e(\s\s')=\e(\s)\e(\s')$.\n
    \boxed{2} On le sait déjà (proposition 18).
\end{thm}

\begin{corr}{}{}
    La signature est l'unique morphisme de groupes non trivial de $(S_{n}, \circ)$ dans $(\C^{*}, \times)$
    \tcblower
    Montrons l'existence dans un premier point puis l'unicité.
    \begin{itemize}[topsep=0pt,itemsep=-0.9 ex]
        \item La fonction constante
        $\mathds{1} : \left\{ \begin{array}{ll}
             S_{n} \to \C^{*}  \n
             \s \mapsto 1
        \end{array}\right.$ est un \underline{morphisme de groupes} dit morphisme \underline{trivial}.
        \item La signature $\e$ est un morphisme de groupes de $S_{n}$ dans $\C^{*}$. Il est non trivial car $\e(\tau) = -1$.
        \item \textbf{Unicité} Soit $f : S_{n} \to \C^{*}$ un morphisme de groupes, soit $\t$ transpositions fixée. $\t^2 = \id$.\\
        Appliquons $f$ : $f(\t^2) = f(\id) = 1 \ra f(\tau)^{2} = -1$ ou $1$.
        \begin{enumerate}[topsep=0pt,itemsep=-0.7 ex]
            \item $f(\tau) = 1$. Soit $\tau'$, conjuguée à $\tau$ : $\exists \alpha \in S_{n}$, $\tau' = \alpha \tau \alpha^{-1}$
            $f(\tau') = f(\alpha \tau \alpha^{-1}) = f(\alpha) \cancel{f(\tau)} f(\alpha)^{-1} = 1$.\\
            Or toute permutation est produit de transpositions $\ra ~ \forall \s \in S_{n}$, $f(\s) = 1$.
            \item $f(\tau) = -1$. Par conjugaison, pour toute transposition $\t'$, $f(\t') = -1$ donc $f$ est un morphisme de groupe envoyant sur $-1$.
        \end{enumerate}
    \end{itemize}
\end{corr}

\end{document}
