\documentclass[11pt]{article}

\usepackage[paperheight=15in, left=2cm, right=2cm, top=2cm, bottom=2cm]{geometry}
\usepackage[most]{tcolorbox}
\usepackage{hyperref, fancyhdr, lastpage, tocloft, changepage}
\usepackage{enumitem}
\usepackage{amsmath, amssymb, amsthm, stmaryrd, cancel}

\def\pagetitle{Groupe Symétrique}
\setlength{\headheight}{14pt}
\newcommand*{\F}{\mathcal{F}}
\newcommand*{\R}{\mathbb{R}}
\newcommand*{\C}{\mathbb{C}}
\newcommand*{\Z}{\mathbb{Z}}
\newcommand*{\K}{\mathbb{K}}
\newcommand*{\N}{\mathbb{N}}
\newcommand*{\m}{\mathcal}
\newcommand*{\lb}{\llbracket}
\newcommand*{\rb}{\rrbracket}
\newcommand*{\id}{\text{id}}
\newcommand*{\n}{\\[0.2cm]}
\renewcommand*{\t}{\tau}
\newcommand{\0}{\varnothing}

\renewcommand*{\phi}{\varphi}
\newcommand*{\e}{\varepsilon}
\newcommand*{\g}{\gamma}
\newcommand*{\s}{\sigma}

\title{\bf{\pagetitle}\\\large{Corrigé}}

\newcommand{\providetcbcountername}[1]{%
  \@ifundefined{c@tcb@cnt@#1}{%
    --undefined--%
  }{%
    tcb@cnt@#1%
  }
}

\newcommand{\settcbcounter}[2]{%
  \@ifundefined{c@tcb@cnt@#1}{%
    \GenericError{Error}{counter name #1 is no tcb counter }{}{}%
  }{%
    \setcounter{tcb@cnt@#1}{#2}%
   }%
}%

\newcommand{\displaytcbcounter}[1]{% Wrapper for \the...
  \@ifundefined{thetcb@cnt@#1}{%
    \GenericError{Error}{counter name #1 is no tcb counter }{}{}%
  }{%
    \csname thetcb@cnt@#1\endcsname% 
  }%
}

\newtcbtheorem{thm}{Théorème}
{
    enhanced,frame empty,interior empty,
    colframe=red,
    after skip = 1cm,
    borderline west={1pt}{0pt}{green!25!red},
    borderline south={1pt}{0pt}{green!25!red},
    left=0.2cm,
    attach boxed title to top left={yshift=-2mm,xshift=-2mm},
    coltitle=black,
    fonttitle=\bfseries,
    colbacktitle=white,
    boxed title style={boxrule=.4pt,sharp corners},
    before lower = {\textbf{Preuve :}\n}
}{thm}

\newtcbtheorem[use counter from = thm]{defi}{Définition}
{
    enhanced,frame empty,interior empty,
    colframe=green,
    after skip = 1cm,
    borderline west={1pt}{0pt}{green},
    borderline south={1pt}{0pt}{green},
    left=0.2cm,
    attach boxed title to top left={yshift=-2mm,xshift=-2mm},
    coltitle=black,
    fonttitle=\bfseries,
    colbacktitle=white,
    boxed title style={boxrule=.4pt,sharp corners},
    before lower = {\textbf{Preuve :}\n}
}{defi}

\newtcbtheorem[use counter from = thm]{prop}{Proposition}
{
    enhanced,frame empty,interior empty,
    colframe=blue,
    after skip = 1cm,
    borderline west={1pt}{0pt}{green!25!blue},
    borderline south={1pt}{0pt}{green!25!blue},
    left=0.2cm,
    attach boxed title to top left={yshift=-2mm,xshift=-2mm},
    coltitle=black,
    fonttitle=\bfseries,
    colbacktitle=white,
    boxed title style={boxrule=.4pt,sharp corners},
    before lower = {\textbf{Preuve :}\n}
}{prop}

\newtcbtheorem[use counter from = thm]{ex}{Exemple}
{
    enhanced,frame empty,interior empty,
    colframe=orange,
    after skip = 1cm,
    borderline west={1pt}{0pt}{green!25!orange},
    borderline south={1pt}{0pt}{green!25!orange},
    left=0.2cm,
    attach boxed title to top left={yshift=-2mm,xshift=-2mm},
    coltitle=black,
    fonttitle=\bfseries,
    colbacktitle=white,
    boxed title style={boxrule=.4pt,sharp corners},
    before lower = {\textbf{Preuve :}\n}
}{ex}


\hypersetup{
    colorlinks=true,
    citecolor=black,
    linktoc=all,
    linkcolor=blue
}

\pagestyle{fancy}
\cfoot{\thepage\ sur \pageref*{LastPage}}

\begin{document}

\thispagestyle{fancy}
\fancyhead[L]{MP2I Paul Valéry}
\fancyhead[C]{\pagetitle}
\fancyhead[R]{2023-2024}

\hrule
\begin{center}
    \LARGE{\textbf{Chapitre 33}}\\
    \large{\pagetitle}\\
    \rule{0.8\textwidth}{0.5pt}
\end{center}


\vspace{0.5cm}

\section{Permutations}

\begin{defi}{}{}
    Une bijection de $\lb1,n\rb$ dans lui-même est appelée une \textbf{permutation} de $\lb1,n\rb$.\\
    L'ensemble des permutations de $\lb1,n\rb$ sera noté $S_n$.
\end{defi}

\begin{ex}{}{}
    Soient
    \begin{equation*}
        \s=\begin{pmatrix}
            1&2&3&4&5\\
            2&5&4&3&1
        \end{pmatrix} \quad \text{ et } \quad
        \s'=\begin{pmatrix}
            1&2&3&4&5\\
            2&3&4&1&5
        \end{pmatrix}
    \end{equation*}
    Calculer $\s\s'$, $\s'\s$, $\s^2$ et $\s^{-1}$.
    \tcblower
    On a :
    \begin{align*}
        \s\s' &= \begin{pmatrix}
            1&2&3&4&5\\
            5&4&3&2&1
        \end{pmatrix}
        &\s'\s = \begin{pmatrix}
            1&2&3&4&5\\
            3&5&1&4&2
        \end{pmatrix}\\
        \s^2 &= \begin{pmatrix}
            1&2&3&4&5\\
            5&1&3&4&2
        \end{pmatrix}
        &\s^{-1} = \begin{pmatrix}
            1&2&3&4&5\\
            5&1&4&3&2
        \end{pmatrix}
    \end{align*}
\end{ex}

\begin{prop}{}{}
    \begin{enumerate}[topsep=0pt,itemsep=-0.9 ex]
        \item $(S_n, \circ)$ est une groupe, appelé \textbf{groupe symétrique}.
        \item $S_n$ est fini et son cardinal vaut $n!$.
        \item Ce groupe n'est pas abélien dès que $n\geq3$.
    \end{enumerate}
    \tcblower
    \boxed{1.} Cours sur les structures algébriques.\\
    \boxed{2.} On pose $\Phi:\begin{cases}S_n\to\m{A}(\lb1,n\rb)\\\s\mapsto(\s(1),...,\s(n))\end{cases}$ bijective et $|\m{A}(\lb1,n\rb)|=n!$.\\
    \boxed{3.} $S_3$ n'est pas abélien car $\t:=...$ et $\t'=...$ ne commutent pas.\\
    Soient $\s,\s'\in S_n ~ | ~ \s_{|\{1,2,3\}}=\t$ et $\s'_{|\{1,2,3\}}=\t'$, fixes sur $\lb4,n\rb$, alors $\s\s'\neq\s'\s$.
\end{prop}

\begin{defi}{Vocabulaire}{}
    Soit $\s\in S_n$.
    \begin{enumerate}[topsep=0pt,itemsep=-0.9 ex]
        \item Si $x\in\lb1,n\rb$, l'ensemble $\{\s^k(x),k\in\Z\}$ est appelé \textbf{orbite} de $x$.
        \item On dit que $x$ est un \textbf{point fixe} de $\s$ si $\s(x)=x$. 
        \item On appelle \textbf{support} de $\s$ l'ensemble des éléments de $\lb1,n\rb$ qui ne sont pas des points fixes.
        \item Deux permutations $\s$ et $\s'$ sont dites \textbf{conjuguées} s'il existe $\alpha\in S_n$ tel que $\s'=\alpha\s\alpha^{-1}$.
    \end{enumerate}

\end{defi}

\begin{prop}{}{}
    Deux permutations dont les supports sont disjoints commutent.
    \tcblower
    Soient $\s,\s'\in S_n$. On note $S(\s)=\{x \in \lb1,n\rb ~ | ~ \s(x)\neq x\}$.\\
    Supposons $S(\s)\cap S(\s') = \0$.\\
    Soit $x\in\lb1,n\rb$.\\
    $\circledcirc$ Si $x\in S(\s)$ : $x\notin S(\s')$ donc $\s\s'(x)=\s(x)\in S(\s)$ par bijectivité de $\s$.\\
    $\circledcirc$ Si $x\notin S(\s)$ : Soit $x\in S(\s')$ et on se ramène au 1er cas, soit $x\notin S(\s')$ et $\s\s'(x)=x=\s'\s(x)$.\\
    Dans tous les cas, $\s\s'(x)=\s'\s(x)$
\end{prop}

\section{Cycles.}

\begin{defi}{}{}
    Soit $p$ un entier supérieur à 2.\\
    Une permutation $\g$ est appellée un $p$\textbf{-cycle} s'il existe $p$ éléments distincts $a_1,...,a_p$ de $\lb1,n\rb$ tels que
    \begin{align*}
        &a_1\overset{\g}{\mapsto}a_2\overset{\g}{\mapsto}...\overset{\g}{\mapsto}a_p\overset{\g}{\mapsto}a_1.\\
        \text{et} \quad &\forall b \in \lb1,n\rb\setminus\{a_1,...,a_p\} ~ \g(b)=b.
    \end{align*}
    On note alors $\g=(a_1~a_2~...~a_p)$.
\end{defi}


\begin{ex}{Conjugué d'un cycle}{}
    Soit $\g=(a_1,...,a_p)$ un $p$-cycle et $\s\in S_n$. Montrer que
    \begin{equation*}
        \s\g\s^{-1}=(\s(a_1)~\s(a_2)~...~\s(a_p)).
    \end{equation*}
    \tcblower
    Soit $b\in\lb1,n\rb\setminus\{\s(a_1),...,\s(a_p)\}$.\\
    Alors $\s\g\s^{-1}(b)=\s\g(\s^{-1}(b))=\s\s^{-1}(b)=b$ car $b\notin\{\s(a_1),...,\s(a_p)\}$ donc $\s^{-1}(b)\notin\{a_1,..,a_p\}$ donc c'est un point fixe de $\g$.\\[0.2cm]
    Soit $j\in\lb1,p\rb$.\\
    Alors $\s\g\s^{-1}(\s(a_j))=\s\g(a_j)=\s(a_{j+1})$ avec $a_{p+1}:=a_1$.\\
    On a bien que $\s\g\s^{-1}$ et $(\s(a_1)...\s(a_p))$ sont égaux en tout point.\\[0.2cm]
    \textbf{Remarque:} Ceci démontre que tous les $p$-cycles sont conjugués.\\
    Soient $\g=(a_1~...~a_p)$ et $\g'=(b_1~...~b_p)$ deux $p$-cycles.\\
    Posons $\s\in S_n$ telle que :
    \begin{itemize}
        \item $\forall j \in \lb 1, p \rb ~ \s(a_j) = b_j$.
        \item Notons $\lb 1, n \rb \setminus \{a_1, ..., a_p\} := \{a_1', ..., a_{n-p}'\}$ et $\lb 1, n \rb \setminus \{b_1, ... b_p\} := \{b_1', ... , b_{n-p}'\}$.
    \end{itemize}
    On pose alors $\forall i\in\lb1,n-p\rb ~ \s(a_i')=b_i'$.\\
    Alors $\s$ est bien une bijection de $\lb1,n\rb$ dans lui-même car injective et de même cardinal.\\
    On a donc $\g'=(b_1~...~b_p) = (\s(a_1)~...~\s(a_p))=\s\g\s^{-1}$ donc $\g$ et $\g'$ sont conjugués.
\end{ex}

\begin{ex}{Calculs sur un cycle}{}
    Soit $\g=(a_1~...~a_p)$. Déterminer $\g^{-1}$ et $\g^p$.
    \tcblower
    \textbf{La réciproque $\g^{-1}$ :}\\
    Si $\g(b)=b$ alors $\g^{-1}(b)=b$ car c'est un point fixe.\\
    Soit $j\in\lb1,p-1\rb$, $\g(a_j)=a_{j+1}$ donc $a_j=\g^{-1}(a_{j+1})$.\\
    Alors $\forall k \in \lb2,p\rb$, $\g^{-1}(a_k)=a_{k-1}$, et $\g^{-1}(a_1)=a_p$.\\
    Ainsi, $\g^{-1}=(a_p ~ a_{p-1} ~ ... ~ a_2 ~ a_1)$.\\[0.2cm]
    \textbf{La puissance $\g^p$ :}\\
    On a $\g=(a, \g(a), ..., \g^{p-1}(a))$ pour un $a\in\lb1,n\rb$.\\
    $\circledcirc$ $\g^p(a)=\g(\g^{p-1}(a))=a$.\\
    $\circledcirc$ Soit $j\in\lb1,p-1\rb$, $\g^p(\g^j(a))=\g^j(\g^p(a))=\g^j(a)$.\\
    $\circledcirc$ Soit $b\in\lb1,n\rb\setminus\{a,\g(a),...,\g^{p-1}(a)\}$, alors $\g^p(b)=b$ car point fixe.\\
    Ainsi, $\forall x \in \lb1,n\rb, ~ \g^p(x)=x$ donc $\g^p=\id$.\n
    \textbf{Remarque:} On pourrait aussi prouver que $p=\min\{j\in\N^*~|~\g^j=\id\}$.
\end{ex}

\section{Transpositions}

\begin{defi}{}{}
    Une permutation $\t$ qui est un $2$-cycle est appelé une \textbf{transposition}.\\
    Une transposition est donc une permutation de la forme $(a,b)$ où $\{a,b\}$ est une paire de $\lb1,n\rb$.
\end{defi}

\begin{prop}{Involutivité}{}
    Si $\t$ est une transposition, alors 
    \begin{equation*}
        \t^2=\id \quad \text{et} \quad \t^{-1}=\t
    \end{equation*}
    \tcblower
    C'est un $2$-cycle donc $\t^2=\id$.\\
    On en déduit que $\t^{-1}=\t$.
\end{prop}

\begin{prop}{Décomposition d'un cycle en produit de transpositions}{}
    Soit $\g=(a_1~...~a_p)$. Alors
    \begin{equation*}
        \g=(a_1~a_2)(a_2~a_3)...(a_{p-1}~a_p) \qquad \text{ou} \qquad \g=(a_1~a_p)(a_1~a_{p-1})...(a_1~a_2)
    \end{equation*}
    \tcblower
    Notons $\pi=(a_1~a_2)(a_2~a_3)...(a_{p-1}~a_p)$. Montrons que $\g=\pi$.\\
    $\circledcirc$ Soit $b\in\lb1,n\rb\setminus\{a_1,...,a_p\}$ : $\g(b)=b$ et $\forall j \in \lb1,p-1\rb, ~ (a_j~a_{j+1})(b)=b$ car $b\notin\{a_j,a_{j+1}\}$.\\
    Alors $\gamma(b)=\pi(b)=b$.\\
    $\circledcirc$ Soit $j\in\lb1,p-1\rb$. Alors $\pi(a_j)=\left[...(a_{j-1}~a_j)(a_j~a_{j+1})...\right](a_j)=\left[...(a_{j-1}~a_j)\right](a_{j+1})=a_{j+1}$.\\
    $\circledcirc$ $\pi(a_p)=[(a_1~a_2)...(a_{p-1}~a_p)](a_p)=[(a_1~a_2)...(a_{p-2}~a_{p-1})](a_{p-1})=...=a_1$\\
    Donc $\forall x \in \lb1,n\rb ~ \g(x)=\pi(x)$\n
    \textbf{Remarque:} On retrouve que $(1~2)(2~3)=(1~2~3)$ et $(2~3)(1~2) = (3~2)(2~1)=(3~2~1)=(1~3~2)$\\
    On a $(1~2)(2~3)\neq(2~3)(1~2)$.
\end{prop}

\section{Théorème de décomposition.}

\begin{thm}{Décomposition en produit de cycles à supports disjoints}{}
    Soit $\s\in S_n$. Il existe $\g_1,...,\g_r$ $r$ cycles à supports disjoints tels que
    \begin{equation*}
        \s=\g_1\g_2...\g_r.
    \end{equation*}
    Les $\g_i$ commutent et cette décomposition est unique à l'ordre près.
    \tcblower
    Soit $\s\in S_n$.\\
    \textbf{Une relation d'équivalence sur $\lb1,n\rb$}.\\
    Pour $i,j\in\lb1,n\rb$, on note $i\sim j$ si $\exists k \in \Z ~ | ~ j = \s^k(i)$.\\
    $\circledcirc$ Soit $i\in\lb1,n\rb$. $i=\s^0(i)$ donc $i \sim i$.\\
    $\circledcirc$ Soient $i,j\in\lb1,n\rb ~ | ~ i \sim j$. Alors $\exists k \in \Z ~ | ~ j = \s^k(i) ~ : ~ i = \s^{-k}(j)$ et $j \sim i$.\\
    $\circledcirc$ Soient $h,i,j\in\lb1,n\rb ~ | ~ h \sim i$ et $i \sim j : \exists k,l \in \Z ~ | ~ i = \s^k(h)$ et $j=\s^l(i)$ donc $j=\s^{l+k}(h)$ et $j \sim h$.
    Il existe alors une partition de $\lb1,n\rb$ en classes d'équivalences.\n
    On fixe $x\in\lb1,n\rb$.\\
    Prouvons qu'il existe $p\in\N^*$ tel que $[x]=\{x,\s(x),...,\s^{p-1}(x)\}$.\\
    On pose $p=\min\{k\in\N^* ~ | ~ \s^k(x)=x\}$. Cet ensemble est minoré. Il est non-vide car :
    \begin{equation*}
        S : \begin{cases}
            \Z \to \lb1,n\rb\\
            k\mapsto \s^k(x)
        \end{cases}
        \text{n'est pas injective.}
    \end{equation*}
    Ainsi, $\exists k,k' \in \Z ~ | ~ k < k' ~ \text{et} ~ \s^k(x)=\s^{k'}(x)$ donc $\s^{k'-k}(x)=x$.\\
    Or $k'-k\in\N^*$, donc $\{k\in\N^* ~ | ~ \s^k(x)=x\}\neq\0$.\\
    Il faut montrer que $[x]=\{x,\s(x),...,\s^{p-1}(x)\}$.\\
    \fbox{$\supset$} est trivial.\\
    \fbox{$\subset$} Soit $y\in[x]$ : $\exists k \in \Z ~ | ~ y = \s^k(x)$.\\
    Par division euclidienne : $\exists!(q,r)\in\Z^2 ~ | ~ k = qp + r ~ \text{et} ~ 0 \leq r \leq p-1$.\\
    Donc $y = \s^k(x) = \s^{pq+r}(x) = \s^r(\s^{pq}(x))=\s^r(x)$ : $y\in\{x,\s(x),...,\s^{p-1}(x)\}$.\n
    Notons $A_1,...,A_r$ les classes d'équivalences non triviales de $\sim$. On a prouvé que :
    \begin{equation*}
        \forall j \in \lb1,r\rb ~ \exists x_j \in \lb 1, n \rb ~ \exists p_j \in \N^* ~ | ~ A_j = \{x_j, \s(x_j), ..., \s^{p_j-1}(x_j)\}.
    \end{equation*}
    On pose alors $\g_j=(x_j~\s(x_j)~...~\s^{p_j-1}(x_j))$, il est clair que $\s=\g_1\g_2...\g_r$.
\end{thm}

\begin{ex}{Une décomposition}{}
    Soit $\s=\begin{pmatrix}1&2&3&4&5&6&7&8\\5&4&1&7&8&6&2&3\end{pmatrix}$.\\
    1. Décomposer $\s$ en produit de cycles à supports disjoints.\\
    2. Déterminer $\s^4$, $\s^{12}$ et $\s^{666}$.
    \tcblower
    \boxed{1.} $\s=(1~5~8~3)(2~4~7)$\\
    \boxed{2.}\\
    $\circledcirc ~ \s^4=(\g_1\g_2)^4\underset{\text{comm}}{=}\g_1^4\g_2^4=\g_2$ car $\g_1^4=\id$ et $\g_2^4=\g_2^3\g_2=\g_2$.\\
    $\circledcirc ~ \s^{12}=(\gamma_1^{4})^3(\g_2^3)^4=\id$\\
    $\circledcirc ~ \s^{666}=(1~8)(3~5)$ car $\s^{666}=\cancel{\s^{12\times55}}\s^6$.
\end{ex}

\end{document}
 