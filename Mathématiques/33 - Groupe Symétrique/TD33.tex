\documentclass[11pt]{article}

\usepackage[paperheight=45in, left=2cm, right=2cm, top=2cm, bottom=2cm]{geometry}
\usepackage[most]{tcolorbox}
\usepackage{hyperref, fancyhdr, lastpage, tocloft, changepage}
\usepackage{enumitem}
\usepackage{amsmath, amssymb, amsthm, stmaryrd}

\def\pagetitle{Groupe Symétrique}
\setlength{\headheight}{14pt}
\newcommand*{\F}{\mathcal{F}}
\newcommand*{\R}{\mathbb{R}}
\newcommand*{\C}{\mathbb{C}}
\newcommand*{\Z}{\mathbb{Z}}
\newcommand*{\K}{\mathbb{K}}
\newcommand*{\N}{\mathbb{N}}
\newcommand*{\m}{\mathcal}
\newcommand*{\lb}{\llbracket}
\newcommand*{\rb}{\rrbracket}
\newcommand*{\id}{\text{id}}
\newcommand*{\n}{\\[0.2cm]}
\renewcommand*{\t}{\tau}
\newcommand{\0}{\varnothing}

\renewcommand*{\phi}{\varphi}
\newcommand*{\e}{\varepsilon}
\newcommand*{\g}{\gamma}
\newcommand*{\s}{\sigma}

\title{\bf{\pagetitle}\\\large{Corrigé}}

\newtcbtheorem{exercise}{Exercice}
{
    enhanced,frame empty,interior empty,
    colframe=blue,
    after skip = 1cm,
    borderline west={1pt}{0pt}{green!25!blue},
    borderline south={1pt}{0pt}{green!25!blue},
    left=0.2cm,
    attach boxed title to top left={yshift=-2mm,xshift=-2mm},
    coltitle=black,
    fonttitle=\bfseries,
    colbacktitle=white,
    boxed title style={boxrule=.4pt,sharp corners},
    before lower = {\textbf{Solution :}}
}{exercise}

\hypersetup{
    colorlinks=true,
    citecolor=black,
    linktoc=all,
    linkcolor=blue
}

\pagestyle{fancy}
\cfoot{\thepage\ sur \pageref*{LastPage}}

\begin{document}

\thispagestyle{fancy}
\fancyhead[L]{MP2I Paul Valéry}
\fancyhead[C]{\pagetitle}
\fancyhead[R]{2023-2024}

\hrule
\begin{center}
    \LARGE{\textbf{Chapitre 33}}\\
    \large{\pagetitle}\\
    \rule{0.8\textwidth}{0.5pt}
\end{center}


\vspace{0.5cm}

\begin{exercise}{$\blacklozenge\lozenge\lozenge$}{}
    Écrire explicitement $s_{1}$, $s_{2}$ et $s_{3}$.
    \tcblower\\[0.2cm]
    $s_{1} = \{ id_{1} \}$, $s_{2} = \{ id_{\llbracket 1, 2 \rrbracket}; 
    \begin{pmatrix}
        1 & 2 \\
        2 & 1 
    \end{pmatrix}
    \}$ \\
    $s_{3} = \{ id_{\llbracket 1, 3 \rrbracket}; 
    \begin{pmatrix}
        1 & 2 & 3\\
        2 & 1 & 3
    \end{pmatrix};
    \begin{pmatrix}
        1 & 2 & 3\\
        1 & 3 & 2
    \end{pmatrix};
    \begin{pmatrix}
        1 & 2 & 3\\
        3 & 2 & 1
    \end{pmatrix};
    \begin{pmatrix}
        1 & 2 & 3\\
        2 & 3 & 1
    \end{pmatrix};
    \begin{pmatrix}
        1 & 2 & 3\\
        3 & 1 & 2
    \end{pmatrix}
    \}$ 
\end{exercise}

\begin{exercise}{$\blacklozenge\lozenge\lozenge$}{}
    Soit $n$ et $p$ deux entiers naturels supérieurs à 2 tels que $p \leq n$\\
    Combien $S_{n}$ contient-il de $p$-cycles ?
    \tcblower\\[0.2cm]
    Choisir un $p$-cycles, c'est choisir un $p$-uplet d'éléments distinct deux à deux, on a donc une bijection entre l'ensemble des $p$-cycles et $A_{p}(\llbracket 1, n \rrbracket)$\\
    Or $|A_{p}(\llbracket 1, n \rrbracket)|$ = $\frac{n!}{(n-p)!}$\\\\
    Ainsi on a exactement $\frac{n!}{(n-p)!}$, $p$-cycles distincts.
\end{exercise}

\begin{exercise}{$\blacklozenge\blacklozenge\blacklozenge$}{}
    Centre de $S_{n}$\\
    On note $Z(S_{n})$ le centre de $S_{n}$, c'est-à-dire l'ensemble des permutations qui commutent avec toutes les autres.\\
    \begin{enumerate}
        \item Que vaut $Z(S_{2})$ ?
        \item Montrer que $Z(S_{n})$ est trivial dès que $n \geq 3$.
    \end{enumerate}
    \tcblower\\[0.2cm]
    \boxed{1} $S_{2}$ est un groupe abélien donc on a $Z(S_{2}) = S_{2}$\\\\
    \boxed{2} Soit $n \in \N_{\geq 3}$, $id_{\llbracket 1, n \rrbracket} \in Z(S_{n})$ étant donné que $id_{\llbracket 1, n \rrbracket}$ est le neutre du groupe $S_{n}$\\\\
    Supposons qu'il en existe au moins un autre, on le notera $\gamma$\\
    $\gamma \neq id_{\llbracket 1, n \rrbracket}$ donc $\exists k \in \llbracket 1, n \rrbracket ~|~ \gamma(k) \neq k$\\
    Notons $z \in (\llbracket 1, n \rrbracket - \{\gamma^{2}(k)\})$ (possible $n \geq 3$) \\\\
    $\gamma = \begin{pmatrix}
        k & \gamma(k) & ... & ... \\
        \gamma(k) & \gamma^{2}(k) & ... & ...
    \end{pmatrix}$\\\\
    Posons $\beta = \begin{pmatrix}
        k & \gamma(k) & ... & ... \\
        \gamma(k) & z & ... & ...
    \end{pmatrix}$\\\\
    $\beta \circ \gamma(k) = z$ et $\gamma \circ \beta(k) = \gamma \circ \gamma(k) = \gamma^{2}(k)$\\
    Donc $\beta \circ \gamma(k) \neq \gamma \circ \beta(k)$ (voir ensemble de définition de $z$)\\\\
    Ainsi on a $\beta \circ \gamma \neq \gamma \circ \beta$\\\\
    On en deduis que $\forall n \in \N_{\geq 3}$, $Z(S_{n}) = \{ id_{\llbracket 1, n \rrbracket} \}$
\end{exercise}
\end{document}
