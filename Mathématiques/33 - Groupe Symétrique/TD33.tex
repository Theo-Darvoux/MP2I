\documentclass[11pt]{article}

\usepackage[paperheight=15in, left=2cm, right=2cm, top=2cm, bottom=2cm]{geometry}
\usepackage[most]{tcolorbox}
\usepackage{amsmath, amssymb, amsthm, enumitem, stmaryrd, cancel, pifont, dsfont, hyperref, fancyhdr, lastpage, tocloft, changepage}

\def\pagetitle{Groupe Symétrique}
\setlength{\headheight}{14pt}

\title{\bf{\pagetitle}\\\large{Corrigé}}

\hypersetup{
    colorlinks=true,
    citecolor=black,
    linktoc=all,
    linkcolor=blue
}

\pagestyle{fancy}
\cfoot{\thepage\ sur \pageref*{LastPage}}

\begin{document}

\newcommand{\providetcbcountername}[1]{%
  \@ifundefined{c@tcb@cnt@#1}{%
    --undefined--%
  }{%
    tcb@cnt@#1%
  }
}

\newcommand{\settcbcounter}[2]{%
  \@ifundefined{c@tcb@cnt@#1}{%
    \GenericError{Error}{counter name #1 is no tcb counter }{}{}%
  }{%
    \setcounter{tcb@cnt@#1}{#2}%
   }%
}%

\newcommand{\displaytcbcounter}[1]{% Wrapper for \the...
  \@ifundefined{thetcb@cnt@#1}{%
    \GenericError{Error}{counter name #1 is no tcb counter }{}{}%
  }{%
    \csname thetcb@cnt@#1\endcsname% 
  }%
}

% MATHS %
\newtcbtheorem{thm}{Théorème}
{
    enhanced,frame empty,interior empty,
    colframe=red,
    after skip = 1cm,
    borderline west={1pt}{0pt}{green!25!red},
    borderline south={1pt}{0pt}{green!25!red},
    left=0.2cm,
    attach boxed title to top left={yshift=-2mm,xshift=-2mm},
    coltitle=black,
    fonttitle=\bfseries,
    colbacktitle=white,
    boxed title style={boxrule=.4pt,sharp corners},
    before lower = {\textbf{Preuve :}\n}
}{thm}

\newtcbtheorem[use counter from = thm]{defi}{Définition}
{
    enhanced,frame empty,interior empty,
    colframe=green,
    after skip = 1cm,
    borderline west={1pt}{0pt}{green},
    borderline south={1pt}{0pt}{green},
    left=0.2cm,
    attach boxed title to top left={yshift=-2mm,xshift=-2mm},
    coltitle=black,
    fonttitle=\bfseries,
    colbacktitle=white,
    boxed title style={boxrule=.4pt,sharp corners},
    before lower = {\textbf{Preuve :}\n}
}{defi}

\newtcbtheorem[use counter from = thm]{prop}{Proposition}
{
    enhanced,frame empty,interior empty,
    colframe=blue,
    after skip = 1cm,
    borderline west={1pt}{0pt}{green!25!blue},
    borderline south={1pt}{0pt}{green!25!blue},
    left=0.2cm,
    attach boxed title to top left={yshift=-2mm,xshift=-2mm},
    coltitle=black,
    fonttitle=\bfseries,
    colbacktitle=white,
    boxed title style={boxrule=.4pt,sharp corners},
    before lower = {\textbf{Preuve :}\n}
}{prop}

\newtcbtheorem[use counter from = thm]{corr}{Corrolaire}
{
    enhanced,frame empty,interior empty,
    colframe=blue,
    after skip = 1cm,
    borderline west={1pt}{0pt}{green!25!blue},
    borderline south={1pt}{0pt}{green!25!blue},
    left=0.2cm,
    attach boxed title to top left={yshift=-2mm,xshift=-2mm},
    coltitle=black,
    fonttitle=\bfseries,
    colbacktitle=white,
    boxed title style={boxrule=.4pt,sharp corners},
    before lower = {\textbf{Preuve :}\n}
}{corr}

\newtcbtheorem[use counter from = thm]{lem}{Lemme}
{
    enhanced,frame empty,interior empty,
    colframe=blue,
    after skip = 1cm,
    borderline west={1pt}{0pt}{green!25!blue},
    borderline south={1pt}{0pt}{green!25!blue},
    left=0.2cm,
    attach boxed title to top left={yshift=-2mm,xshift=-2mm},
    coltitle=black,
    fonttitle=\bfseries,
    colbacktitle=white,
    boxed title style={boxrule=.4pt,sharp corners},
    before lower = {\textbf{Preuve :}\n}
}{lem}

\newtcbtheorem[use counter from = thm]{ex}{Exemple}
{
    enhanced,frame empty,interior empty,
    colframe=orange,
    after skip = 1cm,
    borderline west={1pt}{0pt}{green!25!orange},
    borderline south={1pt}{0pt}{green!25!orange},
    left=0.2cm,
    attach boxed title to top left={yshift=-2mm,xshift=-2mm},
    coltitle=black,
    fonttitle=\bfseries,
    colbacktitle=white,
    boxed title style={boxrule=.4pt,sharp corners},
    before lower = {\textbf{Preuve :}\n}
}{ex}

\newtcbtheorem[use counter from = thm]{meth}{Méthode}
{
    enhanced,frame empty,interior empty,
    colframe=purple,
    after skip = 1cm,
    borderline west={1pt}{0pt}{purple},
    borderline south={1pt}{0pt}{purple},
    left=0.2cm,
    attach boxed title to top left={yshift=-2mm,xshift=-2mm},
    coltitle=black,
    fonttitle=\bfseries,
    colbacktitle=white,
    boxed title style={boxrule=.4pt,sharp corners},
    before lower = {\textbf{Preuve :}\n}
}{meth}

\newtcbtheorem[use counter from = thm]{exercice}{Exercice}
{
    enhanced,frame empty,interior empty,
    colframe=blue,
    after skip = 1cm,
    borderline west={1pt}{0pt}{green!25!blue},
    borderline south={1pt}{0pt}{green!25!blue},
    left=0.2cm,
    attach boxed title to top left={yshift=-2mm,xshift=-2mm},
    coltitle=black,
    fonttitle=\bfseries,
    colbacktitle=white,
    boxed title style={boxrule=.4pt,sharp corners},
    before lower = {\textbf{Preuve :}\n}
}{exercice}

% PHYSIQUE %
\newtcbtheorem[use counter from = thm]{qc}{Question de Cours}
{
    enhanced,frame empty,interior empty,
    colframe=red,
    after skip = 1cm,
    borderline west={1pt}{0pt}{green!25!red},
    borderline south={1pt}{0pt}{green!25!red},
    left=0.2cm,
    attach boxed title to top left={yshift=-2mm,xshift=-2mm},
    coltitle=black,
    fonttitle=\bfseries,
    colbacktitle=white,
    boxed title style={boxrule=.4pt,sharp corners},
    before lower = {\textbf{Preuve :}\n}
}{qc}
\newtcbtheorem[use counter from = thm]{app}{Application}
{
    enhanced,frame empty,interior empty,
    colframe=blue,
    after skip = 1cm,
    borderline west={1pt}{0pt}{green!25!blue},
    borderline south={1pt}{0pt}{green!25!blue},
    left=0.2cm,
    attach boxed title to top left={yshift=-2mm,xshift=-2mm},
    coltitle=black,
    fonttitle=\bfseries,
    colbacktitle=white,
    boxed title style={boxrule=.4pt,sharp corners},
    before lower = {\textbf{Preuve :}\n}
}{app}
\newcommand*{\K}{\mathbb{K}}
\newcommand*{\C}{\mathbb{C}}
\newcommand*{\R}{\mathbb{R}}
\newcommand*{\Q}{\mathbb{Q}}
\newcommand*{\Z}{\mathbb{Z}}
\newcommand*{\N}{\mathbb{N}}
\newcommand*{\F}{\mathcal{F}}

\newcommand{\0}{\varnothing}
\newcommand*{\e}{\varepsilon}
\newcommand*{\g}{\gamma}
\newcommand*{\s}{\sigma}

\newcommand*{\ra}{\Rightarrow}
\newcommand*{\lb}{\llbracket}
\newcommand*{\rb}{\rrbracket}
\newcommand*{\n}{\\[0.2cm]}

\newcommand*{\cmark}{\ding{51}}
\newcommand*{\xmark}{\ding{55}}

\newcommand{\rg}[1]{\textrm{rg}(#1)}
\newcommand{\vect}[1]{\textrm{Vect}(#1)}
\newcommand{\tr}[1]{\textrm{Tr}(#1)}

\renewcommand{\dim}[1]{\textrm{dim}~#1}
\renewcommand*{\ker}[1]{\textrm{Ker}(#1)}
\renewcommand{\Im}[1]{\textrm{Im}(#1)}

\renewcommand*{\t}{\tau}
\renewcommand*{\phi}{\varphi}

\thispagestyle{fancy}
\fancyhead[L]{MP2I Paul Valéry}
\fancyhead[C]{\pagetitle}
\fancyhead[R]{2023-2024}

\hrule
\begin{center}
    \LARGE{\textbf{Chapitre 33}}\\
    \large{\pagetitle}\\
    \rule{0.8\textwidth}{0.5pt}
\end{center}


\vspace{0.5cm}

\begin{exercise}{$\blacklozenge\lozenge\lozenge$}{}
    Écrire explicitement $s_{1}$, $s_{2}$ et $s_{3}$.
    \tcblower\\[0.2cm]
    $s_{1} = \{ id_{1} \}$, $s_{2} = \{ id_{\llbracket 1, 2 \rrbracket}; 
    \begin{pmatrix}
        1 & 2 \\
        2 & 1 
    \end{pmatrix}
    \}$ \\
    $s_{3} = \{ id_{\llbracket 1, 3 \rrbracket}; 
    \begin{pmatrix}
        1 & 2 & 3\\
        2 & 1 & 3
    \end{pmatrix};
    \begin{pmatrix}
        1 & 2 & 3\\
        1 & 3 & 2
    \end{pmatrix};
    \begin{pmatrix}
        1 & 2 & 3\\
        3 & 2 & 1
    \end{pmatrix};
    \begin{pmatrix}
        1 & 2 & 3\\
        2 & 3 & 1
    \end{pmatrix};
    \begin{pmatrix}
        1 & 2 & 3\\
        3 & 1 & 2
    \end{pmatrix}
    \}$ 
\end{exercise}

\begin{exercise}{$\blacklozenge\lozenge\lozenge$}{}
    Soit $n$ et $p$ deux entiers naturels supérieurs à 2 tels que $p \leq n$\\
    Combien $S_{n}$ contient-il de $p$-cycles ?
    \tcblower\\[0.2cm]
    Choisir un $p$-cycles, c'est choisir un $p$-uplet d'éléments distinct deux à deux, on a donc une bijection entre l'ensemble des $p$-cycles et $A_{p}(\llbracket 1, n \rrbracket)$\\
    Or $|A_{p}(\llbracket 1, n \rrbracket)|$ = $\frac{n!}{(n-p)!}$\\\\
    Ainsi on a exactement $\frac{n!}{(n-p)!}$, $p$-cycles distincts.
\end{exercise}

\begin{exercise}{$\blacklozenge\blacklozenge\blacklozenge$}{}
    Centre de $S_{n}$\\
    On note $Z(S_{n})$ le centre de $S_{n}$, c'est-à-dire l'ensemble des permutations qui commutent avec toutes les autres.\\
    \begin{enumerate}
        \item Que vaut $Z(S_{2})$ ?
        \item Montrer que $Z(S_{n})$ est trivial dès que $n \geq 3$.
    \end{enumerate}
    \tcblower\\[0.2cm]
    \boxed{1} $S_{2}$ est un groupe abélien donc on a $Z(S_{2}) = S_{2}$\\\\
    \boxed{2} Soit $n \in \N_{\geq 3}$, $id_{\llbracket 1, n \rrbracket} \in Z(S_{n})$ étant donné que $id_{\llbracket 1, n \rrbracket}$ est le neutre du groupe $S_{n}$\\\\
    Supposons qu'il en existe au moins un autre, on le notera $\gamma$\\
    $\gamma \neq id_{\llbracket 1, n \rrbracket}$ donc $\exists k \in \llbracket 1, n \rrbracket ~|~ \gamma(k) \neq k$\\
    Notons $z \in (\llbracket 1, n \rrbracket - \{\gamma^{2}(k)\})$ (possible $n \geq 3$) \\\\
    $\gamma = \begin{pmatrix}
        k & \gamma(k) & ... & ... \\
        \gamma(k) & \gamma^{2}(k) & ... & ...
    \end{pmatrix}$\\\\
    Posons $\beta = \begin{pmatrix}
        k & \gamma(k) & ... & ... \\
        \gamma(k) & z & ... & ...
    \end{pmatrix}$\\\\
    $\beta \circ \gamma(k) = z$ et $\gamma \circ \beta(k) = \gamma \circ \gamma(k) = \gamma^{2}(k)$\\
    Donc $\beta \circ \gamma(k) \neq \gamma \circ \beta(k)$ (voir ensemble de définition de $z$)\\\\
    Ainsi on a $\beta \circ \gamma \neq \gamma \circ \beta$\\\\
    On en deduis que $\forall n \in \N_{\geq 3}$, $Z(S_{n}) = \{ id_{\llbracket 1, n \rrbracket} \}$
\end{exercise}
\end{document}
