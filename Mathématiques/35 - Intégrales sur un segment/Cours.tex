\documentclass[11pt]{article}
\usepackage[paperheight=15in, left=2cm, right=2cm, top=2cm, bottom=2cm]{geometry}
\usepackage[most]{tcolorbox}
\usepackage{amsmath, amssymb, amsthm, enumitem, stmaryrd, cancel, pifont, dsfont, hyperref, fancyhdr, lastpage, tocloft, changepage}

\newcommand*{\K}{\mathbb{K}}
\newcommand*{\C}{\mathbb{C}}
\newcommand*{\R}{\mathbb{R}}
\newcommand*{\Q}{\mathbb{Q}}
\newcommand*{\Z}{\mathbb{Z}}
\newcommand*{\N}{\mathbb{N}}
\newcommand*{\F}{\mathcal{F}}
\newcommand*{\CM}{\mathcal{CM}}

\newcommand{\0}{\varnothing}
\newcommand*{\e}{\varepsilon}
\newcommand*{\g}{\gamma}
\newcommand*{\s}{\sigma}
\renewcommand*{\l}{\lambda}
\renewcommand*{\t}{\tau}

\newcommand*{\ra}{\Longrightarrow}
\newcommand*{\la}{\Longleftarrow}
\newcommand*{\rla}{\Longleftrightarrow}
\newcommand*{\lb}{\llbracket}
\newcommand*{\rb}{\rrbracket}
\newcommand*{\n}{\\[0.2cm]}

\newcommand*{\cmark}{\ding{51}}
\newcommand*{\xmark}{\ding{55}}

\newcommand{\dx}{\textrm{d}x}
\newcommand{\dt}{\textrm{d}t}
\newcommand{\rg}[1]{\textrm{rg}(#1)}
\newcommand{\vect}[1]{\textrm{Vect}(#1)}
\newcommand{\tr}[1]{\textrm{Tr}(#1)}

\renewcommand*{\Re}{\textrm{Re}}
\renewcommand{\dim}[1]{\textrm{dim}~#1}
\renewcommand*{\ker}[1]{\textrm{Ker}(#1)}
\renewcommand{\Im}{\textrm{Im}}

\newcommand{\providetcbcountername}[1]{%
  \@ifundefined{c@tcb@cnt@#1}{%
    --undefined--%
  }{%
    tcb@cnt@#1%
  }
}

\newcommand{\settcbcounter}[2]{%
  \@ifundefined{c@tcb@cnt@#1}{%
    \GenericError{Error}{counter name #1 is no tcb counter }{}{}%
  }{%
    \setcounter{tcb@cnt@#1}{#2}%
   }%
}%

\newcommand{\displaytcbcounter}[1]{% Wrapper for \the...
  \@ifundefined{thetcb@cnt@#1}{%
    \GenericError{Error}{counter name #1 is no tcb counter }{}{}%
  }{%
    \csname thetcb@cnt@#1\endcsname% 
  }%
}

% MATHS %
\newtcbtheorem{thm}{Théorème}
{
    enhanced,frame empty,interior empty,
    colframe=red,
    after skip = 1cm,
    borderline west={1pt}{0pt}{green!25!red},
    borderline south={1pt}{0pt}{green!25!red},
    left=0.2cm,
    attach boxed title to top left={yshift=-2mm,xshift=-2mm},
    coltitle=black,
    fonttitle=\bfseries,
    colbacktitle=white,
    boxed title style={boxrule=.4pt,sharp corners},
    before lower = {\textbf{Preuve :}\n}
}{thm}

\newtcbtheorem[use counter from = thm]{defi}{Définition}
{
    enhanced,frame empty,interior empty,
    colframe=green,
    after skip = 1cm,
    borderline west={1pt}{0pt}{green},
    borderline south={1pt}{0pt}{green},
    left=0.2cm,
    attach boxed title to top left={yshift=-2mm,xshift=-2mm},
    coltitle=black,
    fonttitle=\bfseries,
    colbacktitle=white,
    boxed title style={boxrule=.4pt,sharp corners},
    before lower = {\textbf{Preuve :}\n}
}{defi}

\newtcbtheorem[use counter from = thm]{prop}{Proposition}
{
    enhanced,frame empty,interior empty,
    colframe=blue,
    after skip = 1cm,
    borderline west={1pt}{0pt}{green!25!blue},
    borderline south={1pt}{0pt}{green!25!blue},
    left=0.2cm,
    attach boxed title to top left={yshift=-2mm,xshift=-2mm},
    coltitle=black,
    fonttitle=\bfseries,
    colbacktitle=white,
    boxed title style={boxrule=.4pt,sharp corners},
    before lower = {\textbf{Preuve :}\n}
}{prop}

\newtcbtheorem[use counter from = thm]{corr}{Corrolaire}
{
    enhanced,frame empty,interior empty,
    colframe=blue,
    after skip = 1cm,
    borderline west={1pt}{0pt}{green!25!blue},
    borderline south={1pt}{0pt}{green!25!blue},
    left=0.2cm,
    attach boxed title to top left={yshift=-2mm,xshift=-2mm},
    coltitle=black,
    fonttitle=\bfseries,
    colbacktitle=white,
    boxed title style={boxrule=.4pt,sharp corners},
    before lower = {\textbf{Preuve :}\n}
}{corr}

\newtcbtheorem[use counter from = thm]{lem}{Lemme}
{
    enhanced,frame empty,interior empty,
    colframe=blue,
    after skip = 1cm,
    borderline west={1pt}{0pt}{green!25!blue},
    borderline south={1pt}{0pt}{green!25!blue},
    left=0.2cm,
    attach boxed title to top left={yshift=-2mm,xshift=-2mm},
    coltitle=black,
    fonttitle=\bfseries,
    colbacktitle=white,
    boxed title style={boxrule=.4pt,sharp corners},
    before lower = {\textbf{Preuve :}\n}
}{lem}

\newtcbtheorem[use counter from = thm]{ex}{Exemple}
{
    enhanced,frame empty,interior empty,
    colframe=orange,
    after skip = 1cm,
    borderline west={1pt}{0pt}{green!25!orange},
    borderline south={1pt}{0pt}{green!25!orange},
    left=0.2cm,
    attach boxed title to top left={yshift=-2mm,xshift=-2mm},
    coltitle=black,
    fonttitle=\bfseries,
    colbacktitle=white,
    boxed title style={boxrule=.4pt,sharp corners},
    before lower = {\textbf{Preuve :}\n}
}{ex}

\newtcbtheorem[use counter from = thm]{meth}{Méthode}
{
    enhanced,frame empty,interior empty,
    colframe=purple,
    after skip = 1cm,
    borderline west={1pt}{0pt}{purple},
    borderline south={1pt}{0pt}{purple},
    left=0.2cm,
    attach boxed title to top left={yshift=-2mm,xshift=-2mm},
    coltitle=black,
    fonttitle=\bfseries,
    colbacktitle=white,
    boxed title style={boxrule=.4pt,sharp corners},
    before lower = {\textbf{Preuve :}\n}
}{meth}

\newtcbtheorem[use counter from = thm]{exercice}{Exercice}
{
    enhanced,frame empty,interior empty,
    colframe=blue,
    after skip = 1cm,
    borderline west={1pt}{0pt}{green!25!blue},
    borderline south={1pt}{0pt}{green!25!blue},
    left=0.2cm,
    attach boxed title to top left={yshift=-2mm,xshift=-2mm},
    coltitle=black,
    fonttitle=\bfseries,
    colbacktitle=white,
    boxed title style={boxrule=.4pt,sharp corners},
    before lower = {\textbf{Preuve :}\n}
}{exercice}

% PHYSIQUE %
\newtcbtheorem[use counter from = thm]{qc}{Question de Cours}
{
    enhanced,frame empty,interior empty,
    colframe=red,
    after skip = 1cm,
    borderline west={1pt}{0pt}{green!25!red},
    borderline south={1pt}{0pt}{green!25!red},
    left=0.2cm,
    attach boxed title to top left={yshift=-2mm,xshift=-2mm},
    coltitle=black,
    fonttitle=\bfseries,
    colbacktitle=white,
    boxed title style={boxrule=.4pt,sharp corners},
    before lower = {\textbf{Preuve :}\n}
}{qc}
\newtcbtheorem[use counter from = thm]{app}{Application}
{
    enhanced,frame empty,interior empty,
    colframe=blue,
    after skip = 1cm,
    borderline west={1pt}{0pt}{green!25!blue},
    borderline south={1pt}{0pt}{green!25!blue},
    left=0.2cm,
    attach boxed title to top left={yshift=-2mm,xshift=-2mm},
    coltitle=black,
    fonttitle=\bfseries,
    colbacktitle=white,
    boxed title style={boxrule=.4pt,sharp corners},
    before lower = {\textbf{Preuve :}\n}
}{app}

\def\pagetitle{Intégrales sur un segment}
\setlength{\headheight}{14pt}

\title{\bf{\pagetitle}\n\large{Corrigé}}

\hypersetup{
    colorlinks=true,
    citecolor=black,
    linktoc=all,
    linkcolor=blue
}

\pagestyle{fancy}
\cfoot{\thepage\ sur \pageref*{LastPage}}

\begin{document}

\thispagestyle{fancy}
\fancyhead[L]{MP2I Paul Valéry}
\fancyhead[C]{\pagetitle}
\fancyhead[R]{2023-2024}

\hrule
\begin{center}
    \LARGE{\textbf{Chapitre 35}}\n
    \large{\pagetitle}\n
    \rule{0.8\textwidth}{0.5pt}
\end{center}


\vspace{0.5cm}

\section{Intégrale d'une fonction continue sur un segment}
\subsection{Ensemble $\CM(I,\K)$}

\begin{defi}{Fonction continue par morceaux sur un intervalle.}{}
    Soit $I$ un intervalle et $f:I\to\K$. On dit que $f$ est \textbf{continue par morceaux} sur $I$ si pour tout segment $[a,b]\subset I$, $f_{|[a,b]}$ est continue par morceaux sur $[a,b]$.\\
    On note $\CM(I,\K)$ l'ensemble des fonctions continues par morceaux sur $I$.
\end{defi}

\begin{ex}{$x\mapsto\lfloor\frac{1}{x}\rfloor$}{}
La fonction $x\mapsto\lfloor\frac{1}{x}\rfloor$ est continue par morceaux sur $\R^*_+$. Expliquer.
\tcblower
Soit $[a,b]\subset\R^*_+$. Notons $S=\{\frac{1}{n}\mid n\in\N^*\}\cap]a,b[$.\\
Cet ensemble est finito : pour $n\in\N^*$, $a<\frac{1}{n}<b\iff\frac{1}{b}<n<\frac{1}{a}\iff\lfloor\frac{1}{b}\rfloor+1 \leq n \leq \lfloor\frac{1}{a}\rfloor$.\\
$S$ contient donc au plus $\lfloor\frac{1}{a}\rfloor-\lfloor\frac{1}{b}\rfloor$ points.\\
Notons $n=|S|$ puis $S=\{a_1,...,a_n\}$, avec $a_1<a_2<...<a_n$.\\
Posons $\s=(a_0,a_1,...,a_n,a_{n+1})$ avec $a_0 := a$ et $a_{n+1} := b$.\\
Soit $i\in\lb0,\rb$, $f_{|]a_i,a_{i+1}[}$ est constante, elle y est donc continue et prolongeable par continuité aux bords.
Ainsi, $f\in\CM(\R^*_+,\R)$.\\
\textbf{Remarque:} En posant $f(0):=0$, ça ne marche plus car $f_{|[0,b]}$ n'est pas cpm sur $[0,b]$.
\end{ex}

\subsection{Intégrale d'une fonction continue par morceaux entre deux bornes}
\begin{defi}{}{}
    Soit $f\in\CM(I,\R)$ et $a,b\in I$. On note $\int_a^bf(x)\dx$, ou plus simplement $\int_a^bf$ le réel défini par :
    \begin{equation*}
        \int_a^bf(x)\dx:=\int_[a,b]f ~ \text{si} ~ a<b, \quad \int_a^af(x)\dx:=0,\quad \text{et}\quad\int_a^bf(x)\dx:=-\int_{[b,a]}f~\text{si } a>b.
    \end{equation*}
\end{defi}

\begin{prop}{}{}
    Soit $f\in\CM(I,\C)$.\\
    Les fonctions $x\mapsto\Re(f(x))$ et $x\mapsto\Im(f(x))$ sont continues par morceaux sur $I$.\\
    Pour $a,b\in I$, on pose :
    \begin{equation*}
        \int_a^bf(x)\dx:=\int_a^b\Re(f(x))\dx+i\int_a^b\Im(f(x))\dx.
    \end{equation*}
    Ainsi, la partie réelle de l'intégrale est l'intégrale de la partie réelle, idem pour la partie imaginaire.
    \tcblower
    Pour prouver la continuité par morceaux de $\Re(f)$ et $\Im(f)$ à partir de celle de $f$, on introduit une subdivision adaptée à $f$ $\s=(a_0,...,a_n)$ et on prouve qu'elle est adaptée à sa partie réelle et à sa partie imaginaire. On peut utiliser :
    \begin{equation*}
        \forall x\in I ~ \Re(f(x)) = \frac{1}{2}(f(x)+\overline{f(x)}) ~ \text{et} ~ \Im(f(x))=\frac{1}{2i}(f(x)-\overline{f(x)}).
    \end{equation*}
    En effet, ces relations donnent que pour $i\in\lb0,n-1\rb$, les restrictions de $\Re(f)$ et $\Im(f)$ à $]a_i,a_{i+1}[$ y sont continues, et prolongeables par continuité sur les bords.
\end{prop}

\pagebreak
\subsection{Relation de Chasles.}
\begin{prop}{Relation de Chasles}{}
    Soient $f\in\CM(I,\K)$ et $a,b,c\in I$.
    \begin{equation*}
        \int_a^bf=\int_a^cf+\int_c^bf.
    \end{equation*}
    \tcblower
    La relation a été établie dans le cours de construction pour une fonction à valeurs réelles dans le cas où $a<c<b$.\\
    $\bullet$ cas $a<b<c$ :
    \begin{equation*}
        \int_a^cf+\int_c^bf=\int_{[a,c]}f-\int_{[b,c]}f=\int_{[a,b]}f+\int_{[b,c]}f-\int_{[b,c]}f=\int_{[a,b]}f=\int_a^bf.
    \end{equation*}
    $\bullet$ cas $b=c<a$ :\\
    D'une part $\int_a^bf=-\int_{[b,a]}f$, d'autre part : $\int_a^cf+\int_c^bf=-\int_c^af=-\int_[b,a]f$.\n
    Les autres cas sont similaires.
\end{prop}

\subsection{Linéarité.}

\begin{prop}{Linéarité de l'intégrale.}{}
    Soient $f,g\in\CM(I,\K)$, et $a,b\in I$. Pour tous scalaires $\l,\mu\in\K$,
    \begin{equation*}
        \int_a^b(\l f+\mu g) = \l\int_{a}^bf+\mu\int_a^bg.
    \end{equation*}
    \tcblower
    On l'a prouvé pour $a<b$ et $f,g$ à valeurs réelles. Il faut le vérifier dans les autres cas.
\end{prop}

\subsection{Intégrales et inégalités.}
\begin{prop}{Positivité}{}
    Soit $f\in\CM([a,b],\R)$ où le segment $[a,b]$ est tel que \fbox{$a\leq b$}.\\
    Si $f$ est positive sur $[a,b]$, alors l'intégrale $\int_a^bf(x)\dx$ est un nombre positif.\\
    Si $f$ est négative sur $[a,b]$, alors cette intégrale est un nombre négatif.
    \tcblower
    On l'a déjà prouvé.
\end{prop}

\begin{prop}{Intégrale nulle d'une fonction positive et continue}{}
    Soit $f:[a,b]\to\R$, avec \fbox{$a<b$}, continue et positive sur $[a,b]$.\\
    Si $\int_a^bf(x)\dx=0$, alors $f$ est nulle sur $[a,b]$.\\
    Par contraposée, si $\exists c \in [a,b] ~ f(c)>0$, alors $\int_a^bf>0$.
    \tcblower
    Il y a aussi la preuve suivante dans \textbf{L'Exercice 79} de la banque CCINP :\\
    On suppose $f$ continue et positive sur $[a,b]$ et $\int_a^bf=0$.\\
    Posons $F:x\mapsto\int_a^xf(t)\dt$ définie sur $[a,b]$, $f$ étant continue sur $[a,b]$, $F$ est une primitive de $f$ sur $[a,b]$ d'après le TFA (prouvé plus loin).\\
    Donc $\forall x \in [a,b], ~ F'(x)=f(x)\geq0$, ainsi $F$ est croissante sur $[a,b]$.\\
    Or, $F(b)=\int_a^bf=0$, de plus, $F(a)=\int_a^af=0$.\\
    Par croissance, $\forall x \in [a,b], ~ F(a) \leq F(x)\leq F(b)$ donc $F(x)=0$.\\
    Donc $F$ est constante sur $[a,b]$, on a $a<b$ donc $\forall x\in[a,b], ~ F'(x)=f(x)=0$.
    \n\textbf{Remarque:} Pourquoi continue et pas continue par morceaux ?\\
    Soit $f:\begin{cases}[0,1]\to\R\\x\mapsto\begin{cases}0 \text{ si } x\neq\frac{1}{2}\\1 \text{ si } x=\frac{1}{2}\end{cases}\end{cases}$, son intégrale est nulle, mais $f$ ne l'est pas.
\end{prop}

\end{document}
