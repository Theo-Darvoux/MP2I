\documentclass[11pt]{article}

\def\chapitre{7}
\def\pagetitle{Forme trigonométrique.}

\input{/home/theo/MP2I/setup.tex}

\newcommand*{\U}{\mathbb{U}}

\begin{document}

\input{/home/theo/MP2I/title.tex}

\thispagestyle{fancy}

\section{Nombres complexes de module 1 et trigonométrie.}

\subsection{Paramétrisation du cercle trigonométrique par \texorpdfstring{$\theta\mapsto e^{i\theta}$}{Lg}}

\begin{defi}{}{}
    On note $\U$ l'ensemble des nombres complexes de module 1:
    \begin{equation*}
        \U=\{\w\in\C\mid|\w|=1\}.
    \end{equation*}
    Si on identifie $\C$ avec le plan muni d'un repère orthonormé, alors $\U$ est le cercle trigonométrique.
\end{defi}

\begin{defi}{}{}
    Soit $\theta\in\R$. On note $e^{i\theta}$ (<< exponentielle de $i\theta$ >>) le nombre complexe de module 1 suivant :
    \begin{center}
        \boxed{e^{i\theta}:=\cos\theta+i\sin\theta}.
    \end{center}
    Par définition même de $e^{i\theta}$, on a $\cos\theta=\Re(e^{i\theta})$ et $\sin\theta=\Im(e^{i\theta})$.
\end{defi}

\begin{prop}{Paramétrisation de $\U$}{}
    \begin{equation*}
        \forall z \in \C\quad z\in\U\iff \exists\theta\in\R~z=e^{i\theta}.
    \end{equation*}
    Par conséquent,
    \begin{equation*}
        \U = \left\{ e^{i\theta }\mid \theta\in\R \right\}.
    \end{equation*}
    \tcblower
    \boxed{\la} Supposons $\exists \theta\in\R\mid z=e^{i\theta}$. Alors $|z|^2=|e^{i\theta}|^2=\cos^2\theta+\sin^2\theta=1$ donc $|z|=1$ : $z\in\U$.\\
    \boxed{\ra} Supposons $|z|=1$, notons $M$ le point d'affixe $z$, $\exists \theta\in\R\mid M=(\cos \theta, \sin \theta)$, donc $z=\cos\theta+\sin\theta=e^{i\theta}$.
\end{prop}

\begin{ex}{}{}
    \begin{align*}
        -1=e^{i\pi},\qquad 1=e^{i0}=e^{2i\pi}, \qquad i=e^{i\frac{\pi}{2}}, \qquad -i=e^{-i\frac{\pi}{2}}\\
        \frac{\sqrt{2}}{2}+\frac{\sqrt{2}}{2}i=e^{i\frac{\pi}{4}}, \qquad \frac{1}{2}+\frac{\sqrt{3}}{2}i=e^{i\frac{\pi}{3}}, \qquad \frac{\sqrt{3}}{2}+\frac{1}{2}i=e^{i\frac{\pi}{6}}.
    \end{align*}
\end{ex}

\quad Le rapport entre les nombres $e^{i\theta}$ qui ont été définis ci-dessus et la fonction exponentielle définie sur $\R$ est à ce stade de l'année encore flou. On se contente pour l'instant de remarquer que ces deux exponentielles partagent la même propriété de morphisme.

\begin{prop}{Propriété de morphisme pour $e^i$.}{}
    \begin{equation*}
        \forall \theta,\theta'\in\R\quad e^{i(\theta+\theta')}=e^{i\theta}\cdot e^{i\theta'}.
    \end{equation*}
    Par conséquent, pour tout $\theta,\theta'$ réels,
    \begin{equation*}
        \left( e^{i\theta} \right)-1=e^{-i\theta}=\ov{e^{i\theta}},\quad e^{i(\theta-\theta')}=\frac{e^{i\theta}}{e^{i\theta'}}\quad\et\quad \forall k\in\Z\quad (e^{i\theta})^k=e^{ik\theta}.
    \end{equation*}
    \begin{equation*}
        e^{i\theta}=e^{i\theta'}\iff\theta\equiv\theta'[2\pi]
    \end{equation*}
    \tcblower
    Soient $\theta,\theta'\in\R$. On a
    \begin{equation*}
        e^{i\theta}e^{i\theta}=(\cos\theta+i\sin\theta)(\cos\theta'+i\sin\theta')=...=\cos(\theta+\theta')+i\sin(\theta+\theta')=e^{i(\theta+\theta')}
    \end{equation*}
\end{prop}

On retiendra notamment, que $\forall \w \in \U,~\ov{\w}=\w^{-1}$.

\subsection{Applications à la trigonométrie.}

\begin{prop}{Formule d'Euler.}{}
    \begin{equation*}
        \forall \theta \in \R, \quad \cos\theta=\frac{e^{i\theta}+e^{-i\theta}}{2}\quad\sin \theta=\frac{e^{i\theta}-e^{-i\theta}}{2i}.
    \end{equation*}
    \tcblower
    Soit $\theta\in\R$. On a\\
    $\bullet$  $2\cos\theta=2\Re(e^{i\theta})=e^{i\theta}+\ov{e^{ei\theta}}=e^{i\theta}+e^{-i\theta}$.\\
    $\bullet$ $2i\sin(\theta)=2i\Im(e^{i\theta})=e^{i\theta}-\ov{e^{i\theta}}=e^{i\theta}-e^{-i\theta}$.
\end{prop}

\begin{meth}{Linéarisation des puissances de cos et sin. $\star$}{}
    Soient $p$ et $q$ deux entiers naturels. Pour linéariser $(\cos \theta)^p(\sin\theta)^q$, on peut toujours :
    \begin{itemize}
        \item transformer $\cos\theta$ et $\sin\theta$ par les formules d'Euler;
        \item développer grâce à la formule du binôme de Newton;
        \item regrouper les exponentielles conjuguées de $e^{ik\theta}$ et $e^{-ik\theta}$;
        \item reconnaître des termes $\cos(k\theta)$ et $\sin(k\theta)$ ($k\in\N$) par les formules d'Euler.
    \end{itemize}
    On peut ainsi transformer $(\cos\theta)^p(\sin\theta)^q$ en une combinaison linéare de termes $\cos(k\theta)$ et $\sin(k\theta)$ où $k\in\N$.
\end{meth}

\begin{ex}{}{}
    Soit $\theta\in\R$. Linéariser $(\cos\theta)^4$, $(\sin\theta)^3$ et $(\cos\theta)^3\sin\theta$. Calculer $\int_0^\pi(\cos x)^4\dx$.
    \tcblower
    \boxed{1.} \begin{align*}
        \cos^4\theta&=\left( \frac{e^{i\theta}+e^{-i\theta}}{2} \right)^4=\frac{1}{2^4}\left( (e^{i\theta})^4+4(e^{i\theta})^3e^{-i\theta} + 6(e^{i\theta})^3(e^{-i\theta})^2 + 4e^{i\theta}(e^{-i\theta})^3+(e^{-i\theta})^4 \right)\\
        &=\frac{1}{2^4}\left( e^{4i\theta} + 4e^{2i\theta} + 6 + 4e^{-2i\theta} + e^{-4i\theta} \right)=\frac{1}{2^4}\left( 2\cos(4\theta) + 9\cos(2\theta) + 6 \right)\\
        &=\frac{1}{2^3}\left( \cos(4\theta) + 4\cos(2\theta) + 3\right)
    \end{align*}
    (...)\\
    \boxed{4.}
    \begin{align*}
        \int_0^\pi(\cos x)^4\dx&=\frac{1}{2^3}\int_0^\pi\cos(4x)\dx+\frac{4}{2^3}\int_{0}^\pi\cos(2x)\dx+\frac{3}{2^3}\int_0^\pi\dx\\
        &=\frac{1}{2^3}\left[ \frac{\sin(4x)}{4} \right]_0^\pi+\frac{4}{2^3}\left[ \frac{\sin(2x)}{2} \right]_0^\pi+\frac{3}{2^3}\left[ x \right]_0^\pi\\
        &=\frac{1}{2^3}\left( 0-0 \right)+\frac{4}{2^3}\left( 0-0 \right)+\frac{3}{2^3}\pi=\frac{3\pi}{8}
    \end{align*}
\end{ex}

\begin{meth}{Technique de l'angle moitié.}{}
    Cette factorisation permet de faire apparaître une formule d'Euler :
    \begin{equation*}
        1+e^{i\theta}=e^{i\frac{\theta}{2}}\left( e^{-i\theta/2}+e^{i\theta/2} \right)=e^{i\frac{\theta}{2}}\cdot 2\cos\left( \frac{\theta}{2} \right).
    \end{equation*}
    \begin{equation*}
        1-e^{i\theta}=e^{i\frac{\theta}{2}}\left( e^{-i\theta/2} - e^{i\theta/2} \right)=-2i\sin\left( \frac{\theta}{2} \right)e^{i\frac{\theta}{2}}
    \end{equation*}
    Pour factoriser $e^{ia}+e^{ib}$, on peut factoriser par $e^{i\frac{a+b}{2}}$: (angle moyen).
\end{meth}

\begin{ex}{$\star$}{}
    Pour $\theta\in\R\setminus\{2k\pi, k\in\Z\}$, on établit les formules
    \begin{equation*}
        \sum_{k=0}^n\cos(k\theta)=\frac{\sin\left( \frac{(n+1)\theta}{2} \right)\cos\left( \frac{n\theta}{2} \right)}{\sin\left( \frac{\theta}{2} \right)}\quad\et\quad\sum_{k=0}^n\sin(k\theta)=\frac{\sin\left( \frac{(n+1)\theta}{2} \right)\sin\left( \frac{n\theta}{2} \right)}{\sin\left( \frac{\theta}{2} \right)}
    \end{equation*}
    \tcblower
    Puisque $\theta\not\equiv0[2\pi]$, $e^{i\theta}\neq1$ et donc:
    \begin{align*}
        \sum_{k=0}^ne^{ik\theta}&=\frac{1-e^{i(n+1)\theta}}{1-e^{i\theta}}=\frac{e^{i\frac{n+1}{2}\theta}\left( e^{-i\frac{n+1}{2}\theta}-e^{i\frac{n+1}{2}\theta} \right)}{e^{i\frac{\theta}{2}}\left( e^{-i\frac{\theta}{2}-e^{i\frac{\theta}{2}}} \right)}\\
        &=e^{i(n/2)\theta}\cdot\frac{-2i\sin\left( \frac{(n+1)\theta}{2} \right)}{-2i\sin\left( \frac{\theta}{2} \right)}=e^{i(n/2)\theta}\cdot\frac{\sin\left( \frac{(n+1)\theta}{2} \right)}{\sin\left( \frac{\theta}{2} \right)}
    \end{align*}
    En passant à la partie réelle/imaginaire, on obtient bien les égalités.
\end{ex}

\begin{ex}{Somme de cos, somme de sin. $\star$}{}
    Soient $p,q\in\R$. On retrouve les égalités:
    \begin{align*}
        &\cos p+\cos q\quad=\quad2\cos\left( \frac{p-q}{2} \right)\cos\left( \frac{p+q}{2} \right) \qquad\qquad~ \sin p+\sin q\quad=\quad2\cos\left( \frac{p-q}{2} \right)\sin\left( \frac{p+q}{2} \right)\\
        &\cos p-\cos q\quad=\quad-2\sin\left( \frac{p-q}{2} \right)\sin\left( \frac{p+q}{2} \right)\qquad\qquad \sin(p)-\sin(q)\quad=\quad2\sin\left( \frac{p-q}{2} \right)\cos\left( \frac{p+q}{2} \right)
    \end{align*}
    \tcblower
    \boxed{1.}
    \begin{align*}
        \cos p + \cos q &= \Re(e^{ip})+\Re(e^{iq})=\Re(e^{ip}+e^{iq})=\Re\left(e^{i\frac{p+q}{2}}\left( e^{i\frac{p-q}{2}} + e^{-i\frac{p-q}{2}} \right)\right)\\
        &=2\cos\left( \frac{p+q}{2} \right)\cos\left( \frac{p-q}{2} \right)
    \end{align*}
    \boxed{2.}
    \begin{align*}
        \sin p + \sin q &= \Im(e^{ip})+\Im(e^{iq})=\Im(e^{ip}+e^{iq})=\Im\left(e^{i\frac{p+q}{2}}\left( e^{i\frac{p-q}{2}} + e^{-i\frac{p-q}{2}} \right)\right)\\
        &=2\cos\left( \frac{p-q}{2} \right)\sin\left( \frac{p+q}{2} \right)
    \end{align*}
\end{ex}

\begin{prop}{Formule de Moivre.}{}
    \begin{equation*}
        \forall \theta \in \R\quad \forall n \in \N\quad (\cos\theta+i\sin\theta)^n=\cos(n\theta)+i\sin(n\theta).
    \end{equation*}
\end{prop}

\begin{meth}{<< Délinéarisation >> : exprimer $\cos(n\theta)$ et $\sin(n\theta)$ en fonction de $\cos\theta$ et $\sin\theta$. $\star$}{}
    On peut toujours
    \begin{itemize}
        \item écrire la formule de Moivre:
        \begin{equation*}
            \cos(n\theta)+i\sin(n\theta)=(\cos\theta+i\sin\theta)^n
        \end{equation*}
        \item développer par la formule du binôme de Newton;
        \item identifier les parties réelles et imaginaires.
    \end{itemize}
    On exprime ainsi $\cos(n\theta)$ et $\sin(n\theta)$ en fonction de $\cos \theta$ et $\sin\theta$.\
    En utilisant la relation $\cos^2\theta+\sin^2\theta=1$, on poursuit les simplifications.\\
    On obtiendra toujours deux polynômes $T_n$ et $U_{n-1}$ tels que
    \begin{equation*}
        \cos(n\theta)=T_n(\cos\theta \quad \et \quad \sin(n\theta)=(\sin \theta)U_{n-1}(\cos\theta)
    \end{equation*}
\end{meth}

\begin{ex}{}{}
    Exprimer $\cos3\theta$ et $\sin5\theta$ en fonction de $\cos\theta$ et de $\sin\theta$.
    \tcblower
    \boxed{1.}
    \begin{align*}
        \cos3\theta&=\Re((\cos \theta + i\sin\theta)^3)=\Re(\cos^3\theta+3i\cos^2\theta\sin\theta-3\cos\theta\sin^2\theta-i\sin^3\theta)\\
        &=\cos^3\theta-3\cos\theta\sin^2\theta=\cos^3\theta-3\cos\theta(1-\cos^2\theta)=4\cos^3\theta-3\cos\theta.
    \end{align*}
    \boxed{2.}
    \begin{align*}
        \sin5\theta&=\Im(e^{5i\theta})=\Im((\cos\theta+i\sin\theta)^5)=5\cos^4\theta\sin\theta-10\cos^2\theta\sin^3\theta+\sin^5\theta\\
        &=\sin\theta\left[ 5\cos^4\theta-10\cos^2\theta(1-\cos^2\theta)+(1-\cos^2\theta)^2 \right]\\
        &=\sin\theta\left[ 16\cos^4\theta - 12\cos^2\theta + 1 \right]
    \end{align*}
\end{ex}

\section{Forme trigonométrique d'un nombre complexe non nul.}

\subsection{Exemples et applications.}

\begin{prop}{}{}
    Tout nombre complexe $z$ non nul peut s'écrire sous \bf{forme trigonométrique}:
    \begin{center}
        \boxed{z=re^{i\theta}}, où $r\in\R^*_+$ et $\theta\in\R$.
    \end{center}
    --- Le nombre $r$ est le module de $z$,\\
    --- On appelle $\theta$ un \bf{argument} de $z$.
    \tcblower
    Soit $z\in\C\setminus\{0\}$. On a $\frac{z}{|z|}\in\U$ donc il existe $\theta\in\R$ tel que $\frac{z}{|z|}=e^{i\theta}$, donc $z=|z|e^{i\theta}$.
\end{prop}

\begin{meth}{Passer de la forme algébrique à la forme trigonométrique.}{}
    Pour mettre un nombre complexe non nul sous forme trigonométrique, il suffit de mettre son module en facteur. On va peut-être reconnaître un argument classique $(\frac{\pi}{3},\frac{\pi}{4},...)$. Sinon, on peut exprimer l'argument à l'aide de la fonction arctan, comme dans l'exemple ci-dessous.
\end{meth}

\begin{ex}{De la forme algébrique à la forme trigonométrique, et réciproquement.}{}
    \begin{enumerate}
        \item Mettre les nombres suivants sous forme trigonométrique.
        \begin{equation*}
            -1, \quad 1-i, \quad \sin\theta+i\cos\theta~(\theta\in\R),\quad i^{35}
        \end{equation*}
        Donner la forme trigonométrique de $1+2i$ et expliciter son argument.
        \item \begin{enumerate}
            \item Donner la forme algébrique de $(\sqrt{3}+i)^{666}$.
            \item Soit $\theta\not\equiv0[2\pi]$. Donner la forme algébrique de $\frac{e^{i\theta}-1}{e^{i\theta}+1}$.
        \end{enumerate}
    \end{enumerate}
    \tcblower
    \boxed{1.} $-1=e^{i\pi}$; $1-i=\sqrt{2}e^{-i\frac{\pi}{4}}$; $\sin\theta+i\cos\theta=e^{i\theta}$ et $i^{35}=-i=e^{i\frac{\pi}{2}}$.\\
    Et $1+2i=\sqrt{5}\left( \frac{1}{\sqrt{5}}+i\frac{2}{\sqrt{5}} \right) = \sqrt{5}\left( \frac{\sqrt{5}}{5}+i\frac{2\sqrt{5}}{5} \right)$.
    \begin{equation*}
        \exists\theta\in\R\mid\frac{1}{\sqrt{5}}+i\frac{2}{\sqrt{5}}=e^{i\theta} \iff \begin{cases} \cos\theta=\frac{1}{\sqrt{5}}\\ \sin\theta=\frac{2}{\sqrt{5}} \end{cases} ~\nt{donc}~\tan\theta=\frac{2\sqrt{5}}{5}\frac{5}{\sqrt{5}}=2.
    \end{equation*}
    On a $\cos\theta>0$ donc $\theta\in]-\frac{\pi}{2},\frac{\pi}{2}[$, alors $\theta=\arctan(2)$ et $1+2i=\sqrt{5}e^{i\arctan(2)}$.\n
    \boxed{2.a)} $\sqrt{3}+i=2\left( \frac{\sqrt{3}}{2} + \frac{1}{2}i \right)=2e^{i\pi/6}$.\\
    Alors $(\sqrt{3}+i)^{666}=(2e^{i\frac{\pi}{6}})^{666}=2^{666}e^{666i\frac{\pi}{6}}=2^{666}e^{111i\pi}=-2^{666}$.\\
    \boxed{2.b)} $\frac{e^{i\theta}-1}{e^{i\theta}+1}=\frac{e^{i\theta/2}(e^{i\theta/2}-e^{-i\theta/2})}{e^{i\theta/2}(e^{i\theta/2}+e^{i\theta/2})}=\frac{2i\sin(\theta/2)}{2\cos(\theta/2)}=i\tan(\theta/2)$.
\end{ex}

\begin{ex}{}{}
    Transformation de $a\cos t+b\sin t$ en $A\cos(t-\phi)$.
    \tcblower
    Soients $a,b,t\in\R$, on note $z=a+ib$.
    \begin{equation*}
        a\cos(t)+b\sin(t)=a\frac{e^{it}+e^{-it}}{2}+b\frac{e^{it}-e^{-it}}{2i}=\frac{1}{2}(ae^{it}+ae^{-it}+ibe^{it}+ibe^{-it})=\frac{1}{2}(\ov{z}e^{it}+ze^{-it})
    \end{equation*}
    Il existe donc $r\in\R_+$ et $\phi\in\R$ tels que $z=re^{i\phi}$ où $r=\sqrt{a^2+b^2}$.
    \begin{equation*}
        a\cos t + b\sin t=\frac{1}{2}(re^{-i\phi}e^{it}+re^{i\phi}e^{-it})=\frac{1}{2}(re^{i(t-\phi)}+re^{i(t-\phi)})=r\cos(t-\phi).
    \end{equation*}
\end{ex}

\begin{prop}{Égalité de formes trigonométriques : presque-unicité de l'écriture.}{}
    \begin{equation*}
        \forall r,r'\in\R_+^*,\quad \forall\theta,\theta'\in\R\quad re^{i\theta}=r'e^{i\theta'}\iff\begin{cases}r=r'\\\theta\equiv\theta'[2\pi]\end{cases}
    \end{equation*}
\end{prop}

\begin{ex}{$\star$}{}
    Comme on le verra dans la troisième partie de ce sours, la forme trigonométrique est particulièrement adaptée à la résolution de problèmes << multiplicatifs >>.\n
    En guise d'exemple, on résout sur $\C$ l'équation $z^3=-4|z|$.
    \tcblower
    Soit $z\in\Z$. On sait que $0$ est solution, on suppose $z\neq0$ : $\exists(r,\theta)\in\R_+^*\times\R\mid z=re^{i\theta}$.
    \begin{align*}
        z^3=-4|z|&\iff r^3e^{3i\theta} = 4re^{i\pi}\\
        &\iff r^3=4r\quad\et\quad 3\theta\equiv \pi[2\pi]\\
        &\iff r=2\quad\et\quad \theta\equiv\frac{\pi}{3}\left[\frac{2\pi}{3}\right]\\
    \end{align*}
    L'équation possède 4 solutions : $0;~2e^{i\pi/3};~-2;~2e^{-i\pi/3}$.
\end{ex}

\begin{defi}{}{}
    Parmi l'infinité d'arguments d'un même nombre complexe non nul, un seul appartient à l'intervalle $]-\pi,\pi]$. On l'appelle \bf{argument principal} de $z$ et on le note $\arg(z)$.
\end{defi}

\begin{prop}{}{}
    Soient $z,z'\in\C^*$. On a
    \begin{equation*}
        \arg(zz')\equiv\arg(z)+\arg(z')[2\pi]\quad\et\quad\arg\left( \frac{z}{z'} \right)\equiv\arg(z)-\arg(z')[2\pi].
    \end{equation*}
\end{prop}

\subsection{Un peu de géométrie.}

\quad On travaille ici dans le plan muni d'un repère orthonormé direct. On rappelle que si $A$ et $B$ sont deux points du plan d'affixes respectives $a$ et $b$, on appelle \bf{affixe du vecteur} $\v{AB}$ le nombre complexe $b-a$. Il s'agit de l'affixe du point $M$ tel que $\v{OM}=\v{AB}$.

\begin{prop}{}{}
    Soient $A,B,C,D$ quatres points du plan distincts deux-à-deux, d'affixes respectives $a$, $b$, $c$ et $d$.
    \begin{equation*}
        \left|\frac{d-c}{b-a}\right|=\frac{\|\v{CD}\|}{\|\v{AB}\|}.
    \end{equation*}
    Le nombre $\arg\left( \frac{d-c}{b-a} \right)$ est une mesure de l'angle orienté $(\v{AB},\v{CD})$.
\end{prop}

\begin{corr}{}{}
    Soient $A,B,C,D$ quatre points du plan distincts deux-à-deux d'affixes $a,b,c,d$.
    \begin{itemize}
        \item $(AB)\parallel(CD)\iff\frac{d-c}{b-a}\in\R.$
        \item En particulier, $A,B,C$ sont alignés ssi $\frac{c-a}{b-a}\in\R$.
        \item $(AB)\bot(CD)\iff\frac{d-c}{b-a}\in i\R$.
    \end{itemize}
    \tcblower
    Preuves sur l'autre poly.
\end{corr}

\section{Exercices.}

\begin{exercice}{$\bww$}{}
    Calculer $(1 + i)^2023$.
    \tcblower
    On a :
    \begin{equation*}
        (1+i)^{2023}=(\sqrt{2}e^{i\frac{\pi}{4}})^{2023}
        = \sqrt{2}^{2023} e^{i\frac{2023\pi}{4}}
        = \sqrt{2}^{2023} e^{-i\frac{\pi}{4}}
    \end{equation*}
\end{exercice}

\begin{exercice}{$\bbw$}{}
    Soient trois réels $x,y,z$ tels que $e^{ix} + e^{iy} + e^{iz} = 0$. Montrer que $e^{2ix} + e^{2iy} + e^{2iz}=0$.
    \tcblower
    On a :
    \begin{align*}
        e^{ix}+e^{iy}+e^{iz}=0\iff e^{-ix}+e^{-iy}+e^{-iz}=0
    \end{align*}
    Et :
    \begin{align*}
        &(e^{ix}+e^{iy}+e^{iz})^2 = e^{2ix} + e^{2iy} + e^{2iz} + 2(e^{ixy} + e^{ixz} + e^{iyz})\\
        \iff&e^{2ix}+e^{2iy}+e^{2iz}=-2(e^{ixy}+e^{ixz}+e^{iyz})
    \end{align*}
    Or :
    \begin{align*}
        2(e^{ixy}+e^{ixz}+e^{iyz}) = 2e^{i(x+y+z)}(e^{-ix}+e^{-iy}+e^{-iz})=0
    \end{align*}
    Ainsi,
    \begin{equation*}
        e^{2ix}+e^{2iy}+e^{2iz}=0
    \end{equation*}
\end{exercice}

\begin{exercice}{$\bww$}{}
    1. Déterminer les formes algébriques et trigonométriques du nombre
    \begin{equation*}
        \frac{1+i\sqrt{3}}{2-2i}
    \end{equation*}
    2. En déduire l'expression de $\cos(\frac{7\pi}{12})$ et de $\sin(\frac{7\pi}{12})$ à l'aide de radicaux.
    \tcblower
    \boxed{1.} On a :
    \begin{align*}
        \frac{1+i\sqrt{3}}{2-2i}=\frac{1-\sqrt{3}}{4}+i\frac{1+\sqrt{3}}{4}=\frac{1}{\sqrt{2}}\left(\frac{\sqrt{2}(1-\sqrt{3})}{4}+i\frac{\sqrt{2}(1+\sqrt{3})}{4}\right)
    \end{align*}
    \boxed{2.} On a :
    \begin{equation*}
        \begin{cases}
            \cos(\frac{7\pi}{12})=\cos(\frac{\pi}{4}+\frac{\pi}{6})=\frac{\sqrt{6}}{4}-\frac{\sqrt{2}}{4}=\frac{\sqrt{2}(1-\sqrt{3})}{4}\\
            \sin(\frac{7\pi}{12})=\sin(\frac{\pi}{4}+\frac{\pi}{6})=\frac{\sqrt{2}}{4}+\frac{\sqrt{3}}{4}=\frac{\sqrt{2}(1+\sqrt{3})}{4}
        \end{cases}
        \text{Donc : }\frac{1+i\sqrt{3}}{2-2i}=\frac{1}{\sqrt{2}}e^{i\frac{7\pi}{12}}
    \end{equation*}
\end{exercice}

\begin{exercice}{$\bww$}{}
    Soit un réel $\theta$. Linéariser $(\cos\theta)^5$ et $(\sin\theta)^6$.
    \tcblower
    On a :
    \begin{align*}
        (\cos\theta)^5&=\left(\frac{e^{i\theta}+e^{-i\theta}}{2}\right)^5\\
        &=\frac{1}{32}\left(e^{5i\theta}+5e^{3i\theta}+10e^{i\theta}+10e^{-i\theta}+5e^{-3i\theta}+e^{-5i\theta}\right)\\
        &=\frac{1}{32}\left(2\cos(5\theta)+10\cos(3\theta)+20\cos(\theta)\right)\\
        &=\frac{1}{16}\left(\cos(5\theta)+5\cos(3\theta)+10\cos(\theta)\right)
    \end{align*}
    Et :
    \begin{align*}
        (\sin\theta)^6&=\left(\frac{e^{i\theta}-e^{-i\theta}}{2i}\right)^6\\
        &=-\frac{1}{64}\left(e^{6i\theta}-6e^{4i\theta}+15e^{2i\theta}-20+15e^{-2i\theta}-6e^{-4i\theta}+e^{-6i\theta}\right)\\
        &=-\frac{1}{64}\left(2\cos(6\theta)-12\cos(4\theta)+30\cos(2\theta)-20\right)\\
        &=\frac{1}{32}\left(10 + 6\cos(4\theta) - 15\cos(2\theta) - \cos(6\theta)\right)
    \end{align*}
\end{exercice}

\begin{exercice}{$\bbw$}{}
    1. Soit $x$ un réel. Exprimer $\cos(5x)$ comme un polynome en $\cos(x)$.\\
    2. Montrer que $\cos^2\left(\frac{\pi}{10}\right)$ est racine du trinôme $x \mapsto 16x^2 - 20x + 5$.\\
    3. En déduire l'égalité $\cos\left(\frac{\pi}{5}\right)=\frac{1+\sqrt{5}}{4}$.
    \tcblower
    \boxed{1.} On a :
    \begin{align*}
        \cos(5x) &= \text{Re}\left((\cos(x)+i\sin(x))^5\right) \\
        &= \cos^5(x) - 10\cos^3(x)\sin^2(x) + 5\cos(x)\sin^4(x)\\
        &= \cos^5(x) - 10\cos^3(x)(1-\cos^2(x)) + 5\cos(x)(1 - 2\cos^2(x) + \cos^4(x))\\
        &= 16\cos^5(x) - 20\cos^3(x) + 5\cos(x)
    \end{align*}
    \boxed{2.} Posons $x = \cos^2\left( \frac{\pi}{10} \right)$ On a :
    \begin{align*}
        &\cos\left(5\cdot\frac{\pi}{10}\right)=16\cos^5\left(\frac{\pi}{10}\right)-20\cos^3\left( \frac{\pi}{10} \right) + 5\cos\left( \frac{\pi}{10} \right)\\
        \iff&16\cos^4\left( \frac{\pi}{10} \right) - 20\cos^2 \left( \frac{\pi}{10} \right) + 5 = 0\\
        \iff& 16x^2 - 20x + 5 = 0
    \end{align*}
    Ainsi $\cos^2\left( \frac{\pi}{10} \right)$ est racine de ce trinôme.\n
    \boxed{3.} On a :
    \begin{align*}
        \cos^2 \left( \frac{\pi}{10} \right) = \frac{1+\cos(\frac{\pi}{5})}{2}
    \end{align*}
    Soit $x\in\mathbb{R}$, on a :
    \begin{align*}
        &16x^2-20x+5=0\\
        \iff& x = \frac{5 \pm \sqrt{5}}{8}
    \end{align*}
    Ainsi, $\frac{1+\cos\left(\frac{\pi}{5}\right)}{2}=\frac{5+\sqrt{5}}{8}$ car $\cos(\pi/5) > \cos(\pi/3) = 0.5$. On en déduit que $\cos\left( \frac{\pi}{5} \right)=\frac{1+\sqrt{5}}{4}$.
\end{exercice}

\begin{exercice}{$\bbw$}{}
    Soit $x\in\mathbb{R}$ et $n\in\mathbb{N}$. Calculer $S=\sum\limits_{k=0}^{n}{\binom{n}{k}\cos(kx)}$
    \tcblower
    Notons : $S'=\sum\limits_{k=0}^{n}{\binom{n}{k}\sin(kx)}$. On a :
    \begin{align*}
        S + iS' &= \sum_{k=0}^{n}{\binom{n}{k}\left(e^{ix}\right)^k} = (1 + e^{ix})^n = \left( e^{\frac{ix}{2}} \right)^n\left( e^{-\frac{ix}{2}} + e^{\frac{ix}{2}}\right)^n = e^{\frac{inx}{2}}2^n\cos^n\left(\frac{x}{2}\right)
    \end{align*}
    Donc $S = \Re\left( 2^n\cos^n\left( \frac{x}{2} \right)e^{\frac{inx}{2}} \right) = 2^n\cos^n\left( \frac{x}{2} \right)\Re\left( e^{\frac{inx}{2}} \right)$.\\
    Or, on a :
    \begin{equation*}
        \Re\left( e^{\frac{inx}{2}} \right) = \Re\left( \cos\left(\frac{nx}{2}\right) + i\sin\left(\frac{nx}{2}\right) \right) = \cos\left( \frac{nx}{2} \right)
    \end{equation*}
    En conclusion :
    \begin{equation*}
        S = 2^n \cos\left( \frac{x}{2} \right)\cos\left(\frac{nx}{2}\right)
    \end{equation*}
\end{exercice}

\begin{exercice}{$\bbw$}{}
    Soit $n\in\mathbb{N}^*$ et $x\in\mathbb{R} \setminus \left\{ 2k\pi, k\in\mathbb{Z} \right\}$. On note
    \begin{equation*}
        D_n(x) = \sum_{k=-n}^{n}{e^{ikx}} \hspace{1.25cm} \text{et} \hspace{1.25cm} F_n(x) = \frac{1}{n}\sum_{k=0}^{n-1}{D_k(x)}.
    \end{equation*}
    1. Montrer que $D_n(x)=\frac{\sin\left( (n+\frac{1}{2})x \right)}{\sin\frac{x}{2}}$.\\
    2. Montrer que $F_n(x)=\frac{1}{n}\left( \frac{\sin\left( \frac{nx}{2} \right)}{\sin\frac{x}{2}} \right)^2$.
    \tcblower
    \boxed{1.} 
    \begin{align*}
        \sum_{k=-n}^{n}{e^{ikx}}&=\sum_{k=-n}^{n}{\left(e^{ix}\right)^k}=e^{-nx}\frac{1-e^{ix(2n+1)}}{1-e^{ix}}=\frac{e^{-inx}-e^{ix(n+1)}}{1-e^{ix}}\\
        &=\frac{e^{ix/2}(e^{-ix(n+1/2)}-e^{ix(n+1/2)})}{e^{ix/2}(e^{-ix/2}-e^{ix/2})}=\frac{-2i\sin(x(n+\frac{1}{2}))}{-2i\sin(\frac{x}{2})}=\frac{\sin((n+\frac{1}{2})x)}{\sin\frac{x}{2}}
    \end{align*}
    \boxed{2.}
    \begin{align*}
        \frac{1}{n}\sum_{k=0}^{n-1}{D_k(x)} &= \frac{1}{n}\sum_{k=0}^{n-1}{\frac{\sin((k+\frac{1}{2})x)}{\sin\frac{x}{2}}}=\frac{1}{n\sin\frac{x}{2}}\sum_{k=0}^{n-1}{\sin((k+\frac{1}{2})x)}
    \end{align*}
    Calculons la somme des $\sin((k+1/2)x)$.
    \begin{align*}
        \sum^{n-1}_{k=0}{\sin((k+\frac{1}{2})x)}&=\Im\left( \sum_{k=0}^{n-1}{e^{ix(k+\frac{1}{2})}} \right) = \Im\left( e^{ix\frac{1}{2}}\sum_{k=0}^{n-1}{e^{ixk}} \right) = \Im\left( e^{ix\frac{1}{2}}\frac{1-e^{inx}}{1-e^{ix}} \right)\\
        &= \Im\left( e^{i\frac{x}{2}}\frac{e^{i\frac{nx}{2}}(\sin(\frac{nx}{2}))}{e^{i\frac{x}{2}}(\sin(\frac{x}{2}))} \right) = \Im\left(e^{i\frac{nx}{2}}\frac{\sin(\frac{nx}{2})}{\sin(\frac{x}{2})}\right)=\frac{\sin^2(\frac{nx}{2})}{\sin(\frac{x}{2})}
    \end{align*}
    Donc :
    \begin{align*}
        F_n(x)=\frac{1}{n\sin\frac{x}{2}}\cdot\frac{\sin^2(\frac{nx}{2})}{\sin\frac{x}{2}}=\frac{\sin^2\frac{nx}{2}}{n\sin^2\frac{x}{2}}=\frac{1}{n}\left(\frac{\sin\frac{nx}{2}}{\sin\frac{x}{2}}\right)^2
    \end{align*}
\end{exercice}

\begin{exercice}{$\bbb$}{}
    Soit un quadrilatère $ABCD$ du plan. On construit les points $E, F, G, H$ à l'extérieur du quadrilatère tels que les triangles $EBA$, $FCB$, $GDC$ et $HAD$ soient des triangles directs, isocèles et rectangles en $E, F, G, H$.\\
    Démontrer que
    \begin{equation*}
        \overrightarrow{EG} \perp \overrightarrow{FH} \hspace{1cm} \text{et} \hspace{1cm} EG = FH.
    \end{equation*}
    \tcblower
    Pas de solution.
\end{exercice}

\begin{exercice}{$\bbb$}{}
    Trouver les nombres complexes d'affixe $z\in\mathbb{C}$ tels que $1,z^2$ et $z^4$ sont alignés.
    \tcblower
    C'est évident lorsque $z\in\{0,1\}$. Supposons $z\notin\{0,1\}$.\\
    Soit $r\in\mathbb{R_+^*}$ et $\theta\in\mathbb{R}$ tels que $z=re^{i\theta}$.\\ 
    On a :
    \begin{align*}
        1, z^2, z^4 \text{ alignés } \iff&\frac{z^4-1}{z^2-1} \in \mathbb{R} \iff\frac{(z^2-1)(z^2+1)}{z^2-1}\in\mathbb{R}\\
        \iff&z^2+1 \in \mathbb{R} \iff z^2 \in \mathbb{R} \iff r^2e^{2i\theta} = r^2e^{-2i\theta}\\
        \iff&e^{2i\theta} = e^{-2i\theta} \iff e^{4i\theta}=1 \iff\theta=\frac{n\pi}{2}, n\in\mathbb{N}
    \end{align*}
    Donc $z\in\mathbb{R}\cup i\mathbb{R}$.
\end{exercice}

\end{document}