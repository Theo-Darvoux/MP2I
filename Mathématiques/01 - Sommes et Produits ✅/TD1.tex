\documentclass[10pt]{article}

\usepackage[T1]{fontenc}
\usepackage[left=2cm, right=2cm, top=2cm, bottom=2cm]{geometry}
\usepackage[skins]{tcolorbox}
\usepackage{hyperref, fancyhdr, lastpage, tocloft, ragged2e}
\usepackage{amsmath, amssymb, amsthm}

\def\pagetitle{Sommes et Produits}

\title{\bf{\pagetitle}\\\large{Corrigé}}
\date{Septembre 2023}
\author{DARVOUX Théo}

\hypersetup{
    colorlinks=true,
    citecolor=black,
    linktoc=all,
    linkcolor=blue
}

\pagestyle{fancy}
\cfoot{\thepage\ sur \pageref*{LastPage}}

\DeclareMathOperator{\ch}{ch}
\DeclareMathOperator{\sh}{sh}


\begin{document}
\renewcommand*\contentsname{Exercices.}
\renewcommand*{\cftsecleader}{\cftdotfill{\cftdotsep}}
\maketitle
\centering
\large{Crédits: Étienne pour avoir aidé sur le 1.14 et 1.15}\\[0.5cm]
\justifying
\hrule
\tableofcontents
\vspace{0.5cm}
\hrule

\fancyhead[L]{MP2I Paul Valéry}
\fancyhead[C]{\pagetitle}
\fancyhead[R]{2023-2024}

\thispagestyle{fancy}

\section*{Exercice 1.1 [$\blacklozenge\lozenge\lozenge$]}

\begin{tcolorbox}[enhanced, width=7in, center, size=fbox, fontupper=\large, drop shadow southwest]
    Démontrer :
    \begin{equation*}
        \forall{n}\in\mathbb{N}^*,\hspace{0.5cm} \sum\limits^{n}_{k=1}{(-1)^kk} = \frac{(-1)^n(2n+1)-1}{4}.
    \end{equation*}
    Soit $n \in \mathbb{N^*}$. Notons $\mathcal{P}_n$ cette proposition.\\
    Montrons que $\mathcal{P}_n$ est vraie pour tout $n$.\\
    \emph{Initialisation.}\\
    Pour $n=1$, on a :
    \begin{equation*}
        \sum\limits^{n}_{k=1}{(-1)^kk} = \frac{(-1)^n(2n+1)-1}{4} = -1
    \end{equation*}
    $\mathcal{P}_1$ est donc vérifiée.\\
    \emph{Hérédité.}\\
    Soit $n\in\mathbb{N}^*$ tel que $\mathcal{P}_n$ soit vraie. Montrons que $\mathcal{P}_{n+1}$ est vraie.\\
    Par hypothèse de récurrence, on a :
    \begin{equation*}
        \sum\limits^{n}_{k=1}{(-1)^kk} = \frac{(-1)^n(2n+1)-1}{4}
    \end{equation*}
    Ainsi,
    \begin{equation*}
        \sum\limits^{n}_{k=1}{(-1)^kk} + (-1)^{n+1}(n+1) = \frac{(-1)^n(2n+1)-1}{4} +(-1)^{n+1}(n+1)
    \end{equation*}
    Donc :
    \begin{equation*}
        \begin{aligned}
            \sum\limits^{n+1}_{k=1}(-1)^kk 
            &= \frac{(-1)^n(2n+1)-1+4(-1)^{n+1}(n+1)}{4}\\
            &=\frac{(-1)^n(-2n - 3)-1}{4}=\frac{(-1)^{n+1}(2n+3)-1}{4}\\
            &=\frac{(-1)^{n+1}(2(n+1)+1)-1}{4}
        \end{aligned}
    \end{equation*}
    Ce qui est exactement $\mathcal{P}_{n+1}$.\\
    \emph{Conclusion.}\\
    Par le principe de récurrence, $\mathcal{P}_n$ est vrai pour tout $n\in\mathbb{N}^*$.\\
    $\rightline{\qed}$
\end{tcolorbox}
\addcontentsline{toc}{section}{\protect\numberline{}Exercice 1.1}
\pagebreak


\section*{Exercice 1.2 [$\blacklozenge\lozenge\lozenge$]}

\begin{tcolorbox}[enhanced, width=7in, center, size=fbox, fontupper=\large, drop shadow southwest]
    Soit $n\in\mathbb{N}$. Calculer
    \begin{equation*}
        \sum\limits^{n}_{k=1}{k(k+1)}, \hspace{0.5cm} \sum\limits^{2n}_{k=n}{e^{-k}}, \hspace{0.5cm} \sum\limits^{2n}_{k=0}{|k-n|}.
    \end{equation*}
    \begin{itemize}
        \item $\sum\limits^n_{k=1}{k(k+1)} = \sum\limits^n_{k=1}{k^2} + \sum\limits^n_{k=1}{k} = \frac{n(2n+1)(n+1)}{6}+\frac{3n(n+1)}{6}=\frac{n(n+1)(n + 2)}{3}$
        \item $\sum\limits^{2n}_{k=n}{e^{-k}} = \sum\limits^{n}_{k=0}{e^{-k-n}}=e^{-n}\sum\limits^{n}_{k=0}{(\frac{1}{e})^k}=e^{-n}\cdot\frac{1-e^{-n-1}}{1-e^{-1}}=\frac{e^{-n}-e^{-2n-1}}{1-e^{-1}}$
        \item $\sum\limits^{2n}_{k=0}{|k-n|}=\sum\limits^{n}_{k=0}{(-k+n)}+\sum\limits^{n}_{k=0}{(k+n-n)}=-\frac{n(n+1)}{2}+n(n+1)+\frac{n(n+1)}{2}=n(n+1)$
    \end{itemize}
    \qed
\end{tcolorbox}

\addcontentsline{toc}{section}{\protect\numberline{}Exercice 1.2}

\section*{Exercice 1.3 [$\blacklozenge\blacklozenge\lozenge$]}

\begin{tcolorbox}[enhanced, width=7in, center, size=fbox, fontupper=\large, drop shadow southwest]
    Soit $n\in\mathbb{N}^*$. Calculer
    \begin{equation*}
        \sum\limits^{n}_{k=-n}{(k+2)}, \hspace{0.5cm} \sum\limits^{2n}_{k=1}{(-1)^kk^2}.
    \end{equation*}
    \begin{itemize}
        \item $\sum\limits^{n}_{k=-n}{(k+2)}=\sum\limits^{2n}_{k=0}{(k-n+2)}=2(2n+1)=4n+2$
        \item $\sum\limits^{2n}_{k=1}{(-1)^kk^2}=\sum\limits^{n}_{k=1}{(-1)^{2k}4k^2}+\sum\limits^{n}_{k=1}{(-1)^{2k-1}(4k^2-4k+1)}=n(2n+1)$
    \end{itemize}
    \qed
\end{tcolorbox}

\addcontentsline{toc}{section}{\protect\numberline{}Exercice 1.3}

\section*{Exercice 1.4 \emph{Téléscopages} [$\blacklozenge\blacklozenge\lozenge$]}

\begin{tcolorbox}[enhanced, width=7in, center, size=fbox, fontupper=\large, drop shadow southwest]
    Soit $n\in\mathbb{N}$. Calculer
    \begin{equation*}
        \sum\limits^{n}_{k=1}{\ln(1+\frac{1}{k})}, \hspace{0.5cm} \sum\limits^{n}_{k=2}{\frac{1}{k(k-1)}}, \hspace{0.5cm} \sum\limits_{k=0}^{n}{k\cdot k!}.
    \end{equation*}
    \begin{itemize}
        \item $\sum\limits^{n}_{k=1}{\ln(1+\frac{1}{k})}=\sum\limits^n_{k=1}{\ln(\frac{k+1}{k})}=\sum\limits^n_{k=1}{(\ln(k+1)-\ln(k))}=\ln(n+1)$
        \item $\sum\limits^{n}_{k=2}{\frac{1}{k(k-1)}}=\sum\limits^n_{k=2}{(\frac{1}{k-1}-\frac{1}{k})}=1-\frac{1}{n}=\frac{n-1}{n}$
        \item $\sum\limits_{k=0}^{n}{k\cdot k!}=\sum\limits^n_{k=0}{(k+1-1)\cdot k!}=\sum\limits^n_{k=0}{((k+1)! - k!)}=(n+1)!-1$
    \end{itemize}
    \qed
\end{tcolorbox}

\addcontentsline{toc}{section}{\protect\numberline{}Exercice 1.4}

\section*{Exercice 1.5 [$\blacklozenge\blacklozenge\lozenge$]}

\begin{tcolorbox}[enhanced, width=7in, center, size=fbox, fontupper=\large, drop shadow southwest]
    Soit $(u_n)_{n\in\mathbb{N}}$ une suite réelle et $n\in\mathbb{N}^*$. Simplifier.
    \begin{equation*}
        \sum\limits^{n-1}_{k=0}{(u_{2k+2}-u_{2k})}, \hspace{0.5cm} \sum\limits^{n}_{k=1}{(u_{2k+1}-u_{2k-1})}.
    \end{equation*}
    \begin{itemize}
        \item $\sum\limits^{n-1}_{k=0}{(u_{2k+2}-u_{2k})}=\sum\limits^{n-1}_{k=0}{(u_{2(k+1)}-u_{2(k)})}=u_{2n}-u_0$
        \item $\sum\limits^{n}_{k=1}{(u_{2k+1}-u_{2k-1})}=\sum\limits^{n-1}_{k=0}{(u_{2k+3}-u_{2k+1})}$\\On a donc :\\$\sum\limits^{n-1}_{k=0}{(u_{2k+3}-u_{2k+1})}-\sum\limits^{n-1}_{k=0}{(u_{2k+2}-u_{2k})}\\=\sum\limits^{n-1}_{k=0}{(u_{2k+3}-u_{2k+2})} + \sum\limits^{n-1}_{k=0}{(u_{2k}-u_{2k+1})}\\=u_{2n+1}-u_{2n}-u_{2n-1}+u_{0}$\\On en déduit que :\\$\sum\limits^{n}_{k=1}{(u_{2k+1}-u_{2k-1})}=u_{2n+1}-u_{2n-1}$
    \end{itemize}
    \qed
\end{tcolorbox}

\addcontentsline{toc}{section}{\protect\numberline{}Exercice 1.5}

\section*{Exercice 1.6 [$\blacklozenge\blacklozenge\lozenge$]}

\begin{tcolorbox}[enhanced, width=7in, center, size=fbox, fontupper=\large, drop shadow southwest]
    Soient $q\in\mathbb{R}$ et $n\in\mathbb{N}^*$. On cherche à calculer la somme $S_n=\sum\limits^{n}_{k=0}{kq^{k-1}}$.\\
    Que vaut-elle si $q=1$ ? Désormais, on supposera $q\neq 1$.\\
    Pour $q=1$, $S_n=\sum\limits^{n}_{k=0}{k}=\frac{n(n+1)}{2}$\\
    Soit la fonction $f_n:x\mapsto\sum\limits^{n}_{k=1}{kx^{k-1}}$. En la voyant comme la dérivée d'une autre que l'on calculera, calculer $S_n$.\\
    On remarque que $\sum\limits^{n}_{k=1}{kq^{k-1}}$ est la dérivée de $\sum\limits^{n}_{k=1}{q^k}$ à une constante près.\\
    Or :
    \begin{equation*}
        \sum\limits^{n}_{k=1}{q^k}=\frac{q-q^{n+1}}{1-q}
    \end{equation*}
    Et sa dérivée est :
    \begin{equation*}
        \frac{1-(n+1)q^n+nq^{n+1}}{(1-q)^2}
    \end{equation*}
    On en déduit que $S_n=\frac{nq^{n+1}-(n+1)q^n+1}{(1-q)^2}$.\\
    \qed
\end{tcolorbox}

\addcontentsline{toc}{section}{\protect\numberline{}Exercice 1.6}

\section*{Exercice 1.7 [$\blacklozenge\blacklozenge\lozenge$]}

\begin{tcolorbox}[enhanced, width=7in, center, size=fbox, fontupper=\large, drop shadow southwest]
    $0,999...=1$. Expliquer.\\
    Soit $n\in\mathbb{N}$. On a :
    \begin{equation*}
        0,999... = \sum\limits^{n}_{k=1}{\frac{9}{10^k}}=9\sum\limits^{n}_{k=1}{(\frac{1}{10})^k}=\frac{9-\frac{9}{10^n}}{9}.
    \end{equation*}
    Or :
    \begin{equation*}
        \lim_{n\rightarrow+\infty}{\frac{9-\frac{9}{10^n}}{9}}=\frac{9}{9}=1.
    \end{equation*}
    $\rightline{\qed}$\\
    \qed
\end{tcolorbox}

\addcontentsline{toc}{section}{\protect\numberline{}Exercice 1.7}

\section*{Exercice 1.8 [$\blacklozenge\blacklozenge\lozenge$]}

\begin{tcolorbox}[enhanced, width=7in, center, size=fbox, fontupper=\large, drop shadow southwest]
    Soit $n\in\mathbb{N}$ et $f_n:x\mapsto x^n$. On se donne un entier naturel $p$ et un réel $x$.\\
    Exprimer le nombre $f_{n}^{(p)}(x)$ à l'aide de factorielles.
    \begin{itemize}
        \item Lorsque $p\leq n$ : $\frac{n!}{(n-p)!}x^{n-p}$
        \item Lorsque $p>n$ : 0
    \end{itemize}
    \qed
\end{tcolorbox}

\addcontentsline{toc}{section}{\protect\numberline{}Exercice 1.8}

\section*{Exercice 1.9 [$\blacklozenge\blacklozenge\lozenge$]}

\begin{tcolorbox}[enhanced, width=7in, center, size=fbox, fontupper=\large, drop shadow southwest]
    Soit $n\in\mathbb{N}^*$ et $p\in\mathbb{N}$.\\
    1. À l'aide d'un téléscopage, démontrer l'identité :
    \begin{equation*}
        \sum\limits^{n}_{k=p}{\binom{k}{p}}=\binom{n+1}{p+1}
    \end{equation*}
    On a :
    \begin{equation*}
        \sum\limits^{n}_{k=p}{\binom{k}{p}}\stackrel{\text{Pascal}}{=}\sum\limits^{n}_{k=p}{(\binom{k+1}{p+1}-\binom{k}{p+1})}=\binom{n+1}{p+1}-\binom{p}{p+1}=\binom{n+1}{p+1}
    \end{equation*}
    \qed
\end{tcolorbox}

\begin{tcolorbox}[enhanced, width=7in, center, size=fbox, fontupper=\large, drop shadow southwest]
    2. Grâce au cas $p=1$, retrouver l'expression connue de $\sum\limits^{n}_{k=1}{k}$.\\
    On a :
    \begin{equation*}
        \sum\limits^{n}_{k=1}{\binom{k}{1}}=\sum\limits^{n}_{k=1}{k} \hspace{0.5cm} \mathrm{et} \hspace{0.5cm} \sum\limits^{n}_{k=1}{\binom{k}{1}}=\binom{n+1}{2}=\frac{(n+1)!}{2(n-1)!}=\frac{n(n+1)}{2}
    \end{equation*}
    On retrouve donc bien :
    \begin{equation*}
        \sum\limits^{n}_{k=1}{k}=\frac{n(n+1)}{2}
    \end{equation*}
    $\rightline{\qed}$
\end{tcolorbox}

\begin{tcolorbox}[enhanced, width=7in, center, size=fbox, fontupper=\large, drop shadow southwest]
    3. Grâce au cas $p=2$, retrouver l'expression connue de $\sum\limits^{n}_{k=1}{k^2}$.\\
    On a :
    \begin{equation*}
        \sum\limits^{n}_{k=2}{\binom{k}{2}}=\sum\limits^{n}_{k=2}{\frac{k!}{2(k-2)!}}=\sum\limits^{n}_{k=2}{\frac{k^2-k}{2}}=\frac{1}{2}\sum\limits^{n}_{k=2}{k^2}-\frac{1}{2}\sum\limits^{n}_{k=2}{k}
    \end{equation*}
    Et :
    \begin{equation*}
        \sum\limits^{n}_{k=2}{\binom{k}{2}}=\binom{n+1}{3}=\frac{(n+1)!}{6(n-2)!}=\frac{n(n+1)(n-1)}{6}
    \end{equation*}
    On en déduit donc que \emph{(on isole $\sum\limits^{n}_{k=2}{k^2}$ du premier résultat.)}:
    \begin{align*}
        \sum\limits^{n}_{k=2}{k^2}&=2\left(\frac{n(n+1)(n-1)}{6}+\frac{n(n+1)}{4}-\frac{1}{2}\right)\\
        &=\frac{2n(n+1)(n-1)+3n(n+1)-6}{6}\\
        &=\frac{n(n+1)(2n+1)-6}{6}
    \end{align*}
    On a donc :
    \begin{equation*}
        \sum\limits^{n}_{k=1}{k^2}=\sum\limits^{n}_{k=2}{k^2}+1=\frac{n(n+1)(2n+1)}{6}
    \end{equation*}
    $\rightline{\qed}$
\end{tcolorbox}

\addcontentsline{toc}{section}{\protect\numberline{}Exercice 1.9}

\section*{Exercice 1.10 [$\blacklozenge\blacklozenge\lozenge$]}

\begin{tcolorbox}[enhanced, width=7in, center, size=fbox, fontupper=\large, drop shadow southwest]
    Soit $x\in\mathbb{R}$ et $n\in\mathbb{N}$. Calculer $\sum\limits^{n}_{k=0}{\ch(kx)}$ et $\sum\limits^{n}_{k=0}{\binom{n}{k}\ch(kx)}$.\\
    On a :
    \begin{align*}
        \sum\limits^{n}_{k=0}{\ch(kx)}&=\sum\limits^{n}_{k=0}{\frac{e^{kx}+e^{-kx}}{2}}\\
        &=\frac{1}{2}\left(\sum\limits^{n}_{k=0}{(e^x)^k}+\sum\limits^{n}_{k=0}{\left(\frac{1}{e^x}\right)^{k}}\right)\\
        &=\frac{1-e^{(n+1)x}}{2-2e^x}+\frac{1-e^{-(n+1)x}}{2-2e^{-x}}
    \end{align*}
    La factorisation est laissée au lecteur $\heartsuit\heartsuit\heartsuit$.
\end{tcolorbox}

\begin{tcolorbox}[enhanced, width=7in, center, size=fbox, fontupper=\large, drop shadow southwest]
    Ensuite, on a :
    \begin{align*}
        \sum\limits^{n}_{k=0}{\binom{n}{k}\ch(kx)}&=\frac{1}{2}\sum\limits^{n}_{k=0}{\binom{n}{k}(e^x)^k+\frac{1}{2}\sum\limits^{n}_{k=0}{\binom{n}{k}{(e^{-x})^k}}}\\
        &\stackrel{Newton}{=}\frac{1}{2}\left((1+e^x)^n+(1+e^{-x})^n\right)\\
    \end{align*}
    \qed
\end{tcolorbox}

\addcontentsline{toc}{section}{\protect\numberline{}Exercice 1.10}

\section*{Exercice 1.11 [$\blacklozenge\blacklozenge\lozenge$]}
\begin{tcolorbox}[enhanced, width=7in, center, size=fbox, fontupper=\large, drop shadow southwest]
    Soit $n\in\mathbb{N}$. Calculer :
    \begin{equation*}
        \sum\limits_{0\leq{i}\leq{j}\leq{n}}{2^{-j}\binom{j}{i}}
    \end{equation*}
    On a :
    \begin{align*}
        \sum\limits_{0\leq{i}\leq{j}\leq{n}}{2^{-j}\binom{j}{i}}
        &=\sum\limits^{n}_{j=0}{2^{-j}\sum\limits^{j}_{i=0}{\binom{j}{i}}}\\
        &=\sum\limits^{n}_{j=0}{2^{-j}\cdot2^j}\\
        &=n+1
    \end{align*}
    \qed
\end{tcolorbox}
\addcontentsline{toc}{section}{\protect\numberline{}Exercice 1.11}

\section*{Exercice 1.12 [$\blacklozenge\blacklozenge\lozenge$]}
\begin{tcolorbox}[enhanced, width=7in, center, size=fbox, fontupper=\large, drop shadow southwest]
    Soit $n\in\mathbb{N}$. Calculer :
    \begin{equation*}
        S_n = \sum\limits_{1\leq i,j\leq n}{|i-j|}
    \end{equation*}
    On a :
    \begin{align*}
        \sum\limits_{1\leq i,j\leq n}{|i-j|}
        &=\sum\limits^{n}_{j=1}{\sum\limits^{n}_{i=1}{|i-j|}}\\
        &\stackrel{Chasles}{=}\sum\limits^{n}_{j=1}{\left(\sum\limits^{j}_{i=1}{|i-j|}+\sum\limits^{n}_{i=j+1}{|i-j|}\right)}\\
        &=\sum\limits^{n}_{j=1}{\left(\sum\limits^{n}_{i=j+1}{(i-j)}-\sum\limits^{j}_{i=1}{(i-j)}\right)}\\
        &=\sum\limits^{n}_{j=1}{\sum\limits^{n-j}_{i=1}{i}-\sum\limits^{n}_{j=1}\sum\limits^{j}_{i=1}{i}+\sum\limits^n_{j=1}j^2}\\
    \end{align*}
    Or,
    \begin{equation*}
        \sum^n_{j=1}{\sum^{n-j}_{i=1}{i}}=\frac{n(n^2-1)}{6}
    \end{equation*}
    Et :
    \begin{equation*}
        \sum^n_{j=1}\sum^j_{i=1}{i}=\sum^n_{j=1}{\frac{j(j+1)}{2}}=\frac{n(n+1)(n+2)}{6}
    \end{equation*}
    Donc :
    \begin{align*}
        S_n &= \frac{n(n+1)(n-1)}{6} - \frac{n(n+1)(n+2)}{6} + \frac{n(n+1)(2n+1)}{6} = \frac{n(n+1)(n-1)}{3}
    \end{align*}
    \qed
\end{tcolorbox}
\addcontentsline{toc}{section}{\protect\numberline{}Exercice 1.12}

\section*{Exercice 1.13 [$\blacklozenge\blacklozenge\lozenge$]}
\begin{tcolorbox}[enhanced, width=7in, center, size=fbox, fontupper=\large, drop shadow southwest]
    Soit $n\in\mathbb{N}^*$. Calculer :
    \begin{equation*}
        \sum^n_{k=0}\sum^n_{i=k}{\binom{n}{i}\binom{i}{k}}
    \end{equation*}
    On a :
    \begin{align*}
        \sum^n_{k=0}\sum^n_{i=k}{\binom{n}{i}\binom{i}{k}}&=\sum^n_{i=0}\binom{n}{i}\sum^i_{k=0}\binom{i}{k}=\sum^n_{i=0}\binom{n}{i}2^i=3^n
    \end{align*}
    \qed
\end{tcolorbox}
\addcontentsline{toc}{section}{\protect\numberline{}Exercice 1.13}

\section*{Exercice 1.14 [$\blacklozenge\blacklozenge\blacklozenge$]}
\begin{tcolorbox}[enhanced, width=7in, center, size=fbox, fontupper=\large, drop shadow southwest]
    Démontrer :
    \begin{equation*}
        \forall{n\in\mathbb{N}^*}\hspace{0.5cm}\sum^n_{k=1}{\binom{n}{k}\frac{(-1)^{k-1}}{k}}=\sum^n_{k=1}{\frac{1}{k}}
    \end{equation*}
    Soit $n\in\mathbb{N}^*$, on note $\mathcal{P}_n$ cette proposition. Montrons $\mathcal{P}_n$ pour tout $n$.\\
    \emph{Initialisation}.\\
    Pour $n=1$, on a :
    \begin{equation*}
        \sum^1_{k=1}{\binom{1}{k}\frac{(-1)^0}{k}}=\sum^1_{k=1}{\frac{1}{k}}=1
    \end{equation*}
    $\mathcal{P}_1$ est donc vérifiée.\\
    \emph{Hérédité}.\\
    Soit $n\in\mathbb{N}^*$ tel que $\mathcal{P}_n$ soit vraie. Montrons que $\mathcal{P}_{n+1}$ est vraie.\\
    Par hypothèse de récurrence, on a :
    \begin{equation*}
        \sum^n_{k=1}{\binom{n}{k}\frac{(-1)^{k-1}}{k}}=\sum^n_{k=1}{\frac{1}{k}}
    \end{equation*}
    Or :
    \begin{align*}
        \sum^{n+1}_{k=1}{\binom{n+1}{k}\frac{(-1)^{k-1}}{k}}
        &=\sum^{n+1}_{k=1}{\left(\binom{n}{k}+\binom{n}{k-1}\right)\frac{(-1)^{k-1}}{k}}\\
        &=\sum^{n+1}_{k=1}{\binom{n}{k}\frac{(-1)^{k-1}}{k}}+\sum^{n+1}_{k=1}{\binom{n}{k-1}\frac{(-1)^{k-1}}{k}}\\
        &\stackrel{HR}{=}\sum^n_{k=1}{\frac{1}{k}}+\sum^{n+1}_{k=1}{\binom{n}{k-1}\frac{(-1)^{k-1}}{k}}\\
        &=\sum^n_{k=1}{\frac{1}{k}}+\sum^{n}_{k=0}{\binom{n}{k}\frac{(-1)^k}{k+1}}\\
        &=\sum^n_{k=1}{\frac{1}{k}}+\frac{1}{n+1}\sum^n_{k=0}{(-1)^k\frac{(n+1)!}{(k+1)!(n-k)!}}\\
        &=\sum^n_{k=1}{\frac{1}{k}}-\frac{1}{n+1}\sum^n_{k=0}{\binom{n+1}{k+1}(-1)^{k+1}}\\
        &\stackrel{Newton}{=}\sum^n_{k=1}{\frac{1}{k}}+\frac{1}{n+1}\\
        &=\sum^{n+1}_{k=1}{\frac{1}{k}}
    \end{align*}
    Ce qui est exactement $\mathcal{P}_{n+1}$.\\
    \emph{Conclusion.}\\
    Par le principe de récurrence, $\mathcal{P}_n$ est vraie pour tout $n\in\mathbb{N}^*$\\
    $\rightline{\qed}$
\end{tcolorbox}
\addcontentsline{toc}{section}{\protect\numberline{}Exercice 1.14}

\section*{Exercice 1.15 [$\blacklozenge\blacklozenge\blacklozenge$]}
\begin{tcolorbox}[enhanced, width=7in, center, size=fbox, fontupper=\large, drop shadow southwest]
    Pour $k\in\mathbb{N}^*$, on pose $H_k=\sum\limits^k_{p=1}{\frac{1}{p}}$.\\
    Pour $n\in\mathbb{N}^*$, exprimer $\sum\limits^n_{k=1}H_k$ et $\sum\limits^n_{k=1}kH_k$ en fonction de $n$ et $H_n$.\\
    On a :
    \begin{align*}
        \sum^n_{k=1}{H_k}
        &=\sum^n_{k=1}{\sum^k_{p=1}{\frac{1}{p}}}\\
        &=\sum^n_{p=1}{\sum^n_{k=p}{\frac{1}{p}}} && \text{(Proposition 33 !)}\\
        &=\sum^n_{p=1}{\frac{1}{p}\sum^n_{k=p}{1}}\\
        &=\sum^n_{p=1}{\frac{n-p+1}{p}}\\
        &=n\sum^n_{p=1}{\frac{1}{p}}-\sum^n_{p=1}{\frac{p-1}{p}}\\
        &=nH_n + H_n - n\\
        &=(n+1)H_n - n
    \end{align*}
    $\rightline{$\mathbb{CQFD}$}$
\end{tcolorbox}

\begin{tcolorbox}[enhanced, width=7in, center, size=fbox, fontupper=\large, drop shadow southwest]
    On a :
    \begin{align*}
        \sum^n_{k=1}{kH_k}
        &=\sum^n_{k=1}{k\sum^k_{p=1}{\frac{1}{p}}}\\
        &=\sum^n_{k=1}{\sum^k_{p=1}{\frac{k}{p}}}\\
        &=\sum^n_{p=1}{\sum^n_{k=p}{\frac{k}{p}}} && \text{(Encore la proposition 33 \color{red}{$\heartsuit$}\color{black})}\\
        &=\sum^n_{p=1}{\frac{1}{p}\sum^n_{k=p}{k}}\\
        &=\sum^n_{p=1}{\frac{(n-p+1)(p+n)}{2p}}\\
        &=\frac{1}{2}\sum^n_{p=1}{\frac{n^2-p^2+p+n}{p}}\\
        &=\frac{1}{2}\left(\sum^n_{p=1}{\frac{n^2+n}{p}}+\sum^{n}_{p=1}{\frac{p(1-p)}{p}}\right)\\
        &=\frac{1}{2}\left(n(n+1)H_n+n\frac{n(n+1)}{2}\right)\\
        &=\frac{n(n+1)H_n+n}{2}-\frac{n(n+1)}{4}\\
        &=\frac{2n(n+1)H_n+2n-n(n+1)}{4}\\
        &=\frac{n(2(n+1)H_n-n+1)}{4}
    \end{align*}
    \rightline{$\mathbb{CQFD}$}
\end{tcolorbox}

\addcontentsline{toc}{section}{\protect\numberline{}Exercice 1.15}

    
\end{document}