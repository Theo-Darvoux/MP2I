\documentclass[10pt]{article}

\usepackage[T1]{fontenc}
\usepackage[left=2cm, right=2cm, top=2cm, bottom=2cm]{geometry}
\usepackage[skins]{tcolorbox}
\usepackage{hyperref, fancyhdr, lastpage, tocloft, ragged2e, multicol}
\usepackage{amsmath, amssymb, amsthm, stmaryrd}
\usepackage{tkz-tab}

\def\pagetitle{Primitives et intégrales}

\title{\bf{\pagetitle}\\\large{Corrigé}}
\date{Octobre 2023}
\author{DARVOUX Théo}

\hypersetup{
    colorlinks=true,
    citecolor=black,
    linktoc=all,
    linkcolor=blue
}

\DeclareMathOperator{\ch}{ch}
\DeclareMathOperator{\sh}{sh}
\DeclareMathOperator{\tah}{th}

\pagestyle{fancy}
\cfoot{\thepage\ sur \pageref*{LastPage}}


\begin{document}
\renewcommand*\contentsname{Exercices.}
\renewcommand*{\cftsecleader}{\cftdotfill{\cftdotsep}}
\maketitle
\begin{center}
    \large{\color{red}\textbf{MRC AMINE POUR LES EXOS 8.8, 8.10 ET 8.11}}
\end{center}

\hrule
\tableofcontents
\vspace{0.5cm}
\hrule


\thispagestyle{fancy}
\fancyhead[L]{MP2I Paul Valéry}
\fancyhead[C]{\pagetitle}
\fancyhead[R]{2023-2024}
\allowdisplaybreaks

\pagebreak

\section*{Exercice 8.1 [$\blacklozenge\lozenge\lozenge$]}
\begin{tcolorbox}[enhanced, width=7in, center, size=fbox, fontupper=\large, drop shadow southwest]
    Donner les primitives des fonctions suivantes (on précisera l'intervalle que l'on considère).
    \begin{align*}
        &a:x\mapsto\cos{xe^{\sin{x}}}; \hspace{1cm} b:x\mapsto\frac{\cos x}{\sin x}; \hspace{1cm} c:x\mapsto\frac{\cos x}{\sqrt{\sin x}}; \hspace{1cm} d:x\mapsto\frac{1}{3x+1};\\
        &e:x\mapsto\frac{\ln x}{x}; \hspace{1cm} f:x\mapsto\frac{1}{x\ln x}; \hspace{1cm} g:x\mapsto\sqrt{3x+1}; \hspace{1cm} h:x\mapsto\frac{x+x^2}{1+x^2}.
    \end{align*}
    \begin{align*}
        &A:\begin{cases}\mathbb{R}\rightarrow\mathbb{R}\\x\mapsto e^{\sin x} + c\end{cases}; \hspace{0.5cm} B:\begin{cases}\mathbb{R}\setminus\{k\pi, k\in\mathbb{Z}\}\rightarrow\mathbb{R}\\x\mapsto\ln(\sin x) + c\end{cases}; \hspace{0.5cm}\\
        &C:\begin{cases}]2k\pi, (2k+1)\pi[, k\in\mathbb{Z}\rightarrow\mathbb{R}\\x\mapsto2\sqrt{\sin x} + c\end{cases}; \hspace{0.5cm} D:\begin{cases}\mathbb{R}\setminus\{-\frac{1}{3}\}\rightarrow\mathbb{R}\\x\mapsto\frac{1}{3}\ln(3x+1) + c\end{cases};\\
        &E:\begin{cases}\mathbb{R_+^*}\rightarrow\mathbb{R}\\x\mapsto\frac{1}{2}\ln^2x + c\end{cases}; \hspace{0.5cm} F:\begin{cases}\mathbb{R_+^*}\rightarrow\mathbb{R}\\x\mapsto\ln(\ln x) + c\end{cases};\\
        &G:\begin{cases}[-\frac{1}{3}, +\infty]\rightarrow\mathbb{R}\\x\mapsto\frac{2}{9}(3x+1)^{\frac{3}{2}} + c\end{cases}; \hspace{0.5cm}H:\begin{cases}\mathbb{R}\rightarrow\mathbb{R}\\x\mapsto\frac{1}{2}\ln(1+x^2) + x - \arctan(x) + c\end{cases}.
    \end{align*}
    Avec $c$ les constantes d'intégration.\\
    \qed
\end{tcolorbox}

\addcontentsline{toc}{section}{\protect\numberline{}Exercice 8.1}

\section*{Exercice 8.2 [$\blacklozenge\lozenge\lozenge$] Issu du cahier de calcul}
\begin{tcolorbox}[enhanced, width=7in, center, size=fbox, fontupper=\large, drop shadow southwest]
    On rappelle que $\int_a^b{f(x)dx}$ est l'aire algébrique entre la courbe représentative de $f$ et l'axe des abscisses.\\
    1. Sans chercher à les calculer, donner le signe des intégrales suivantes.
    \begin{equation*}
        \int_{-2}^3{e^{-x^2}dx}; \hspace{1cm} \int_5^{-3}{|\sin x|dx}; \hspace{1cm} \int_1^a{\ln^7(x)dx} (a\in\mathbb{R_+^*}).
    \end{equation*}
    2. En vous ramenant à des aires, calculer de tête
    \begin{equation*}
        \int_1^3{7dx}; \hspace{1cm} \int_0^7{3xdx}; \hspace{1cm} \int_{-2}^1{|x|dx}.
    \end{equation*}
    1.\\
    La première est positive car $-2<3$ et la fonction est positive sur $[-2,3]$e.\\
    La seconde est négative car $5>-3$ et la fonction est positive sur $[-3,5]$.\\
    La dernière est positive lorsque $a\geq1$ et négative lorsque $a\leq1$ car $\ln^7$ est positive sur $[1,+\infty[$.\\
    2.\\
    La première vaut $2\times7=14$.\\
    La seconde vaut $\frac{7^2\times3}{2}=\frac{147}{2}$.\\
    La dernière vaut $\frac{1}{2}+\frac{2\times2}{2}=2.5$\\
    \qed
\end{tcolorbox}

\addcontentsline{toc}{section}{\protect\numberline{}Exercice 8.2}

\section*{Exercice 8.3 [$\blacklozenge\lozenge\lozenge$]}
\begin{tcolorbox}[enhanced, width=7in, center, size=fbox, fontupper=\large, drop shadow southwest]
    Calculer les intégrales ci-dessous :
    \begin{align*}
        &I_1 = \int_0^1{x\sqrt{x}dx}, \hspace{0.5cm} I_2 = \int_{-1}^1{2^xdx}, \hspace{0.5cm} I_3=\int_1^e{\frac{\ln^3(t)}{t}dt}, \hspace{0.5cm} I_4=\int_0^1{\frac{x}{2x^2+3}dx},\\
        &I_5=\int_0^1{\frac{1}{2x^2+3}dx}, \hspace{0.5cm} I_6=\int_0^{\frac{\pi}{2}}{\cos^2xdx}, \hspace{0.5cm} I_7=\int_0^\pi{|\cos x|dx}, \hspace{0.5cm} I_8 = \int_0^{\frac{\pi}{2}}{\cos^3xdx}\\
        &I_9=\int_0^{\frac{\pi}{4}}{\tan^3xdx}.
    \end{align*}
    \begin{align*}
        &I_1 = \left[\frac{2}{5}x^{\frac{5}{2}}\right]_0^1=\frac{2}{5}, \hspace{0.5cm} I_2=\left[\frac{1}{\ln2}e^{x\ln2}\right]_{-1}^1=\frac{3}{\ln4}, \hspace{0.5cm} I_3=\left[\frac{\ln^4t}{4}\right]_1^e=\frac{1}{4},\\
        &I_4=\left[\frac{1}{4}\ln(2x^2+3)\right]_0^1=\frac{1}{4}\left(\ln\left(\frac{5}{3}\right)\right), \hspace{0.5cm} I_5 = \left[\frac{1}{\sqrt{6}}\arctan\left(\sqrt{\frac{2}{3}}x\right)\right]_0^1=\frac{1}{\sqrt{6}}\arctan\left(\sqrt{\frac{2}{3}}\right),\\
        &I_6=\frac{1}{2}\int_0^{\frac{\pi}{2}}{\cos2xdx}+\frac{\pi}{4}=\frac{1}{2}\left[-2\sin(2x)\right]_0^\frac{\pi}{2}+\frac{\pi}{4}=\frac{\pi}{4}, I_7=\left[2\sin x\right]_0^\pi=2,\\
        &I_8=\int_0^{\frac{\pi}{2}}{\cos x-\cos x\sin^2(x)dx}=\left[\sin x\right]_0^{\frac{\pi}{2}}-\left[\frac{1}{3}\sin^3x\right]_0^\frac{\pi}{2}=\frac{2}{3},\\
        &I_9=\int_0^\frac{\pi}{4}{\tan^3x+\tan x-\tan xdx}=\int_0^\frac{\pi}{4}{\tan x(\tan^2 x + 1)dx} - \frac{\ln2}{2}=\left[\frac{1}{2}\tan^2(x)\right]_0^\frac{\pi}{4}-\frac{\ln2}{2}\\
        &=\frac{1-\ln2}{2} 
    \end{align*}
    \qed
\end{tcolorbox}

\addcontentsline{toc}{section}{\protect\numberline{}Exercice 8.3}

\section*{Exercice 8.4 [$\blacklozenge\lozenge\lozenge$]}
\begin{tcolorbox}[enhanced, width=7in, center, size=fbox, fontupper=\large, drop shadow southwest]
    Calculer le nombre $\int_1^2{\frac{\ln x}{\sqrt{x}}dx}$.\\
    1. À l'aide d'une IPP.\\
    2. À l'aide du changement de variable $x=t^2$.\\
    1.
    \begin{align*}
        \int_1^2{\ln x\cdot x^{-\frac{1}{2}} dx}&=\left[\ln x \cdot 2\sqrt{x}\right]_1^2 - 2\int_1^2{x^{-\frac{1}{2}}dx}=2\sqrt{2}\ln2-2\left[2\sqrt{x}\right]_1^2=2\sqrt{2}(\ln2-2)+4
    \end{align*}
    2.
    \begin{align*}
        \int_1^2{\frac{\ln x}{\sqrt{x}}dx}=\int_1^{\sqrt{2}}{\frac{\ln t^2}{t}2tdt}=4\int_1^{\sqrt{2}}{\ln(t)dt}=4\left[t\ln t-t\right]_1^{\sqrt{2}}=4+2\sqrt{2}(\ln2-2)
    \end{align*}
    \qed
\end{tcolorbox}

\addcontentsline{toc}{section}{\protect\numberline{}Exercice 8.4}

\section*{Exercice 8.5 [$\blacklozenge\lozenge\lozenge$]}
\begin{tcolorbox}[enhanced, width=7in, center, size=fbox, fontupper=\large, drop shadow southwest]
    Calculer
    \begin{align*}
        \int_0^1{\frac{1}{(t+1)\sqrt{t}}dt} \hspace{1cm} \text{en posant }t=u^2.
    \end{align*}
    On a :
    \begin{align*}
       \int_0^1{\frac{1}{(t+1)\sqrt{t}}dt} = \int_0^1{\frac{1}{(u^2+1)u}2udu} = 2\int_0^1{\frac{1}{u^2+1}du}=2\left[\arctan(u)\right]_0^1 = \frac{\pi}{2}
    \end{align*}
    \qed
\end{tcolorbox}

\addcontentsline{toc}{section}{\protect\numberline{}Exercice 8.5}

\section*{Exercice 8.6 [$\blacklozenge\lozenge\lozenge$]}
\begin{tcolorbox}[enhanced, width=7in, center, size=fbox, fontupper=\large, drop shadow southwest]
    Calculer
    \begin{align*}
        \int_0^1{\frac{t^9}{t^5+1}dt} \hspace{1cm} \text{en posant } u=t^5.
    \end{align*}
    On a :
    \begin{align*}
        \int_0^1{\frac{t^9}{t^5+1}dt}=\int_0^1{\frac{\frac{1}{5}t^5}{t^5+1}5t^4dt}=\frac{1}{5}\int^1_0{\frac{u}{u+1}du}=\frac{1}{5}\int^1_0{1-\frac{1}{u+1}du}=\frac{1}{5}\left(1-\ln2\right)
    \end{align*}
    \qed
\end{tcolorbox}

\addcontentsline{toc}{section}{\protect\numberline{}Exercice 8.6}

\section*{Exercice 8.7 [$\blacklozenge\blacklozenge\lozenge$]}
\begin{tcolorbox}[enhanced, width=7in, center, size=fbox, fontupper=\large, drop shadow southwest]
    En posant le changement de variable $u=\tan(x)$, calculer l'intégrale
    \begin{align*}
        \int_0^{\frac{\pi}{4}}{\frac{1}{1+\cos^2(x)}dx}&=\int_0^1{\frac{1}{1+\cos^2(\arctan(u))}\cdot\frac{1}{1+u^2}du}\\
        &=\int_0^1{\frac{1+u^2}{(2+u^2)(1+u^2)}du}\\
        &=\int_0^1{\frac{1}{2+u^2}du}\\
        &=\left[\frac{1}{\sqrt{2}}\arctan\left(\frac{u}{\sqrt{2}}\right)\right]_0^1\\
        &=\frac{1}{\sqrt{2}}\arctan\left(\frac{1}{\sqrt{2}}\right)\\
    \end{align*}
    \qed
\end{tcolorbox}

\addcontentsline{toc}{section}{\protect\numberline{}Exercice 8.7}

\section*{Exercice 8.8 [$\blacklozenge\lozenge\lozenge$]}
\begin{tcolorbox}[enhanced, width=7in, center, size=fbox, fontupper=\large, drop shadow southwest]
    On pose
    \begin{equation*}
        C=\int_0^{\frac{\pi}{2}}{\frac{\cos x}{\sin x + \cos x}dx} \hspace{0.5cm} \text{et} \hspace{0.5cm} S=\int_0^{\frac{\pi}{2}}{\frac{\sin x}{\sin x + \cos x}dx}
    \end{equation*}
    1. À l'aide du changement de variable $u=\frac{\pi}{2}-x$, prouver que $C=S$.\\
    2. Calculer $C + S$, en déduire la valeur commune de ces deux intégrales.\\[0.5cm]
    1. En posant le changement de variable $u = \frac{\pi}{2} - x$, on a :
    \begin{align*}
    S &= \int_{\frac{\pi}{2}}^{0} \frac{\sin\left(\frac{\pi}{2} - x\right)}{\sin\left(\frac{\pi}{2} - x\right) + \cos\left(\frac{\pi}{2} - x\right)} (-du) \\
    &= \int_{0}^{\frac{\pi}{2}} \frac{\cos(u)}{\cos(u) + \sin(u)} \, du\\
    \end{align*}
    Ainsi, $C = S$.\\[0.25cm]
    2. On a :
    \begin{align*}
        C + S &= \int_{0}^{\frac{\pi}{2}} \frac{\cos(x)}{\sin(x) + \cos(x)}dx + \int_{0}^{\frac{\pi}{2}} \frac{\sin(x)}{\cos(x) + \sin(x)} \, dx\\
        &= \int_{0}^{\frac{\pi}{2}} \frac{\cos(x) + \sin(x)}{\cos(x) + \sin(x)} \,dx\\
        &= \int_{0}^{\frac{\pi}{2}} 1 \, dx = \frac{\pi}{2}
    \end{align*}
    On en déduit que $C = S = \frac{\pi}{4}$.\\
    \qed
\end{tcolorbox}

\addcontentsline{toc}{section}{\protect\numberline{}Exercice 8.8}

\section*{Exercice 8.9 [$\blacklozenge\blacklozenge\blacklozenge$]}
\begin{tcolorbox}[enhanced, width=7in, center, size=fbox, fontupper=\large, drop shadow southwest]
    On considère les deux intégrales suivantes
    \begin{equation*}
        I=\int_0^{\frac{\pi}{2}}{\frac{\cos(t)}{\sqrt{1+\sin(2t)}}dt} \hspace{1cm} J=\int_0^{\frac{\pi}{2}}{\frac{\sin(t)}{\sqrt{1+\sin(2t)}}dt}
    \end{equation*}
    1. À l'aide du changement de variable $u=\frac{\pi}{4}-t$ calculer $I+J$.\\
    2. À l'aide du changement de variable $u=\frac{\pi}{2}-t$ montrer que $I=J$.\\
    3. En déduire $I$ et $J$.\\[0.1cm]
    1. On a :
    \begin{align*}
        I + J &= \int_0^{\frac{\pi}{2}}{\frac{\cos(t) + \sin(t)}{\sqrt{1+\sin(2t)}}dt}=\int_{-\frac{\pi}{4}}^{\frac{\pi}{4}}{\frac{\cos(\frac{\pi}{4}-u)+\sin(\frac{\pi}{4}-u)}{\sqrt{1+\cos(2u)}}du}\\
        &=\int_{-\frac{\pi}{4}}^{\frac{\pi}{4}}{\frac{\sqrt{2}\cos(u)}{\sqrt{2\cos^2(u)}}du}=\int_{-\frac{\pi}{4}}^{\frac{\pi}{4}}{\frac{\sqrt{2}\cos(u)}{\sqrt{2}|\cos(u)|}du}=\frac{\pi}{2}.
    \end{align*}
    2. On a :
    \begin{align*}
        I &= \int_0^{\frac{\pi}{2}}{\frac{\cos(t)}{\sqrt{1+\sin(2t)}}dt}=\int_0^{\frac{\pi}{2}}{\frac{\sin(u)}{\sqrt{1+\sin(\pi-u)}}du}=\int_0^{\frac{\pi}{2}}{\frac{\sin(u)}{\sqrt{1+\sin(u)}}du}=J
    \end{align*}
    3. On a $2I = 2J = I+J = \frac{\pi}{2}$. Donc $I = J = \frac{\pi}{4}$.\\
    \qed
\end{tcolorbox}

\addcontentsline{toc}{section}{\protect\numberline{}Exercice 8.9}

\section*{Exercice 8.10 [$\blacklozenge\lozenge\lozenge$]}
\begin{tcolorbox}[enhanced, width=7in, center, size=fbox, fontupper=\large, drop shadow southwest]
    Que vaut
    \begin{equation*}
        \int_{-666}^{666}{\ln\left(\frac{1+e^{\arctan(x)}}{1+e^{-\arctan(x)}}\right)dx} \text{ ?}
    \end{equation*}
    Soit $x\in[-666,666]$.\\
    Par imparité de $\arctan$, on a :
    \begin{align*}
        \ln\left(\frac{1+e^{\arctan(-x)}}{1+e^{-\arctan(-x)}}\right)=\ln\left(\frac{1+e^{-\arctan(x)}}{1+e^{\arctan(x)}}\right)=-\ln\left(\frac{1+e^{\arctan(x)}}{1+e^{-\arctan(x)}}\right)
    \end{align*}
    Ainsi, $\ln\left(\frac{1+e^{\arctan(x)}}{1+e^{-\arctan(x)}}\right)$ est impaire. Donc 
    \begin{equation*}
        \int_{-666}^{666}{\ln\left(\frac{1+e^{\arctan(x)}}{1+e^{-\arctan(x)}}\right)dx}=0.
    \end{equation*}
    \qed
\end{tcolorbox}

\addcontentsline{toc}{section}{\protect\numberline{}Exercice 8.10}

\section*{Exercice 8.11 [$\blacklozenge\blacklozenge\lozenge$]}
\begin{tcolorbox}[enhanced, width=7in, center, size=fbox, fontupper=\large, drop shadow southwest]
    Le but de cet exercice est de calculer les intégrales
    \begin{equation*}
        I = \int_0^1{\sqrt{1+x^2}dx} \hspace{1cm} \text{et} \hspace{1cm} J=\int_0^1{\frac{1}{\sqrt{1+x^2}}dx}.
    \end{equation*}
    1. Justifier que l'équation $\sh(x)=1$ possède une unique solution réelle que l'on notera dans la suite $\alpha$.\\
    Exprimer $\alpha$ à l'aide de la fonction $\ln$.\\
    2. Calculer $J$ en posant $x=\sh(t)$. On exprimera le résultat en fonction de $\alpha$.\\
    3. À l'aide d'une intégration par parties, obtenir une équation reliant $I$ et $J$.\\
    4. En déduire une expression de $I$ en fonction de $\alpha$.
    \\
    1. On a :
    \begin{equation*}
        \sh(\alpha) = 1 \iff \left( \frac{e^\alpha - e^{-\alpha}}{2} \right) = 1 \iff e^\alpha - 2e^{-\alpha} = 0 \iff e^{2\alpha} - 2e^\alpha - 1 = 0
    \end{equation*}
    Changement de variable : $X = e^\alpha$
    \begin{align*}
        &X^2 - 2X - 1 = 0 \\
        &\Delta = (-2)^2 - 4 \cdot (-1) = 8 \\
        &\Delta > 0 \text{, donc il y a 2 racines} \\
        &X_1 = \frac{4 + 2\sqrt{2}}{2} = 2 + \sqrt{2}, \quad X_2 = \frac{4 - 2\sqrt{2}}{2} = 2-\sqrt{2} \\
        &\alpha = \ln(2 + \sqrt{2}) \quad \text{(Impossible, car } \ln(2 - \sqrt{2}) < 0) \\
    \end{align*}
    Ainsi, $\alpha=\ln(2+\sqrt{2})$\\
    2. On a :
    \begin{equation*}
        J = \int_0^1 \frac{1}{\sqrt{1+x^2}} \, dx = \int_0^\alpha \frac{1}{\sqrt{1+\text{sh}^2(t)}} \cdot \text{ch}(t) \, dt = \int_0^\alpha \frac{\text{ch}(t)}{\sqrt{\text{ch}^2(t)}} \, dt = \int_0^\alpha 1 \, dt = \alpha
    \end{equation*}
    3. On a :
    \begin{align*}
        I &= \int_0^1{\sqrt{1+x^2}dx}=\left[x\sqrt{1+x^2}\right]_0^1 - \int_0^1{\frac{x^2}{\sqrt{1+x^2}}dx}\\
        &= \sqrt{2} - \int_0^1{\frac{1+x^2}{\sqrt{1+x^2}}-\frac{1}{\sqrt{1+x^2}}dx}\\
        &=\sqrt{2} - \int_0^1{\sqrt{1+x^2}dx} + \int_0^1{\frac{1}{\sqrt{1+x^2}}dx}\\
        &=\sqrt{2} - I + J
    \end{align*}
    Ainsi, $I = \frac{1}{2}\left(\sqrt{2} + J\right)$.\\
    4. Il vient immédiatement que $I=\frac{1}{2}\left(\sqrt{2}+\alpha\right)$\\
    \qed
\end{tcolorbox}

\addcontentsline{toc}{section}{\protect\numberline{}Exercice 8.11}

\section*{Exercice 8.12 [$\blacklozenge\blacklozenge\blacklozenge$]}
\begin{tcolorbox}[enhanced, width=7in, center, size=fbox, fontupper=\large, drop shadow southwest]
    Calculer $\int_0^1{\arctan(x^{1/3})dx}$ en posant d'abord $x=t^3$.\\
    On a :
    \begin{align*}
        \int_0^1{\arctan(x^{\frac{1}{3}})dx}&=\int_0^1{\arctan(t)\cdot3t^2dt}\\
        &=\left[\arctan(t)\cdot t^3\right]_0^1 - \int_0^1{\frac{t^3}{1+t^2} dt}\\
        &=\frac{\pi}{4} - \int_0^1{tdt}+\int_0^1{\frac{t}{1+t^2}dt}\\
        &=\frac{\pi}{4} - \frac{1}{2} + \left[\frac{1}{2}\ln(1+t^2)\right]_0^1\\
        &=\frac{\pi}{4} - \frac{1}{2} + \frac{1}{2}\ln(2)\\
        &=\frac{1}{4}\left(\pi - 2 + \ln(4)\right)
    \end{align*}
    \qed
\end{tcolorbox}

\addcontentsline{toc}{section}{\protect\numberline{}Exercice 8.12}

\section*{Exercice 8.13 [$\blacklozenge\blacklozenge\blacklozenge$]}
\begin{tcolorbox}[enhanced, width=7in, center, size=fbox, fontupper=\large, drop shadow southwest]
    Calculer l'intégrale $I = \int_0^{\frac{\pi}{4}}{\ln(1+\tan x)dx}$ en posant $x=\frac{\pi}{4}-u$.\\
    On a :
    \begin{align*}
        I &= \int_0^{\frac{\pi}{4}}{\ln(1+\tan x)dx} = \int_{0}^{\frac{\pi}{4}}{\ln\left(1+\tan\left(\frac{\pi}{4}-u\right)\right)du}\\
        &=\int^{\frac{\pi}{4}}_0{\ln\left(1+\frac{1-\tan u}{1+\tan u}\right)du}=\int_0^{\frac{\pi}{4}}{\ln\left(\frac{2}{1+\tan u}\right)du}\\
        &=\int_0^{\frac{\pi}{4}}{\ln2du}-\int_0^{\frac{\pi}{4}}{\ln(1+\tan u)du} = \frac{\pi}{4}\ln2 - I
    \end{align*}
    On en déduit que $2I = \frac{\pi}{4}\ln2$. Ainsi, $I=\frac{\pi}{8}\ln2$\\
    \qed
\end{tcolorbox}

\addcontentsline{toc}{section}{\protect\numberline{}Exercice 8.13}

\section*{Exercice 8.14 [$\blacklozenge\blacklozenge\lozenge$] Les intégrales de Wallis}
\begin{tcolorbox}[enhanced, width=7in, center, size=fbox, fontupper=\large, drop shadow southwest]
    On définit, pour tout entier $n\in\mathbb{N}$ le nombre
    \begin{equation*}
        W_n = \int_0^{\frac{\pi}{2}}{\sin^n(x)dx}.
    \end{equation*}
    1. À l'aide d'une intégration par parties, montrer que
    \begin{equation*}
        \forall{n\in\mathbb{N}}, \hspace{0.3cm} W_{n+2} = \frac{n+1}{n+2}W_n.
    \end{equation*}
    2. Démontrer les égalités suivantes pour $n\in\mathbb{N}$ :
    \begin{equation*}
        W_{2n}=\frac{(2n)!}{2^{2n}(n!)^2}\cdot\frac{\pi}{2} \hspace{0.5cm} \text{ et } \hspace{0.5cm} W_{2n+1}=\frac{2^{2n}(n!)^2}{(2n+1)!}.
    \end{equation*}
    \begin{center}
        \large{\textbf{Allez voir le DM7}}
    \end{center}
    \qed
\end{tcolorbox}

\addcontentsline{toc}{section}{\protect\numberline{}Exercice 8.14}

\section*{Exercice 8.15 [$\blacklozenge\blacklozenge\lozenge$]}
\begin{tcolorbox}[enhanced, width=7in, center, size=fbox, fontupper=\large, drop shadow southwest]
    Pour tous entiers naturels $p$ et $q$, on note
    \begin{equation*}
        I(p,q):=\int_0^1{t^p(1-t)^qdt}.
    \end{equation*}
    1. Soit $(p,q)\in\mathbb{N}^2$.\\
    Avec un changement de variable, démontrer que $I(p,q)=I(q,p)$.\\
    2. À l'aide de l'intégration par parties, démontrer
    \begin{equation*}
        \forall{p,q\in\mathbb{N}}, \hspace{0.5cm} (p+1)I(p,q+1)=(q+1)I(p+1,q).
    \end{equation*}
    3. (a) Calculer $I(p,0)$ pour un entier $p$ donné.\\
    (b) Démontrer enfin que
    \begin{equation*}
        \forall{p,q\in\mathbb{N}}, \hspace{0.5cm} I(p,q)=\frac{p!q!}{(p+q+1)!}.
    \end{equation*}
    1. Changement de variable : $u=1-t$ :
    \begin{align*}
        \int^1_0{t^p(1-t)^qdt}=-\int_1^0{(1-u)^pu^qdu}=\int_0^1{u^q(1-u)^pdu}
    \end{align*}
    2. On a :
    \begin{align*}
        I(p,q+1)=\int^1_0{t^p(1-t)^{q+1}dt}&=\left[\frac{1}{p+1}t^{p+1}\cdot(1-t)^{q+1}\right]_0^1-\int_0^1{\frac{1}{p+1}t^{p+1}\cdot-(q+1)(1-t)^{q}dt}\\
        &=\frac{q+1}{p+1}\int_0^1{t^{p+1}(1-t)^{q}dt}=\frac{q+1}{p+1}I(p+1,q)
    \end{align*}
    Donc on a bien $(p+1)I(p,q+1)=(q+1)I(p+1,q)$.\\
    3. (a) On a :
    \begin{align*}
        I(p,0)=\int_0^1{t^pdt}=\left[\frac{1}{p+1}t^{p+1}\right]_0^1=\frac{1}{p+1}.
    \end{align*}
    (b) Soit $\mathcal{P}_q$ la proposition $I(p,q)=\frac{p!q!}{(p+q+1)!}$. Montrons que $\mathcal{P}_q$ est vraie pour tout $q\in\mathbb{N}$.\\
    \emph{Initialisation :} Pour $q=0$, on a $I(p,q)=\frac{1}{p+1}$ et $\frac{p!q!}{(p+q+1)!}=\frac{p!}{(p+1)!}=\frac{1}{p+1}$. $\mathcal{P}_0$ est vérifiée.\\
    \emph{Hérédité :} Soit $q\in\mathbb{N}$ fixé tel que $\mathcal{P}_q$ soit vraie. Montrons $\mathcal{P}_{q+1}$.\\
    On a :
    \begin{align*}
        I(p,q+1)&=\frac{q+1}{p+1}I(p+1,q)=\frac{q+1}{p+1}\frac{(p+1)!q!}{(p+q+2)!}\\
        &=\frac{(p+1)!(q+1)!}{(p+1)(p+q+2)!}=\frac{p!(q+1)!}{(p+(q+1)+1)!}
    \end{align*}
    C'est exactement $\mathcal{P}_{q+1}$.\\
    \emph{Conclusion :} Par le principe de récurrence, $\mathcal{P}_q$ est vraie pour tout $q\in\mathbb{N}$.\\
    \qed
\end{tcolorbox}

\addcontentsline{toc}{section}{\protect\numberline{}Exercice 8.15}

\section*{Exercice 8.16 [$\blacklozenge\blacklozenge\lozenge$]}
\begin{tcolorbox}[enhanced, width=7in, center, size=fbox, fontupper=\large, drop shadow southwest]
    Pour tout entier naturel $n$, on pose $I_n=\int_1^e{x(\ln x)^ndx}$.\\
    1. Calculer $I_0$ et $I_1$.\\
    2. Montrer que $J_n=2I_n+nI_{n-1}$ est indépendant de $n$. Déterminer sa valeur.\\
    3.\hspace{0.2cm}Montrer que la suite $(I_n)$ est décroissante puis, en utilisant la question $\mathbf{2.}$, démontrer l'encadrement
    \begin{equation*}
        \frac{e^2}{n+3} \leq I_n \leq \frac{e^2}{n+2}.
    \end{equation*}  
    4. En déduire $\lim\limits_{n\to+\infty}{I_n}$ et $\lim\limits_{n\to+\infty}{nI_n}$.\\[0.25cm]
    1. On a : 
    \begin{align*}
        &I_0=\int_1^e{x\,dx}=\frac{e^2-1}{2}.\\
        &I_1=\int_1^e{x\ln x\,dx}=\left[\frac{1}{2}x^2\ln x\right]_1^e-\frac{1}{2}\int_1^e{x\,dx}=\frac{e^2}{2}-\frac{e^2-1}{4}=\frac{e^2+1}{4}.
    \end{align*}
    2. Soit $n\in\mathbb{N}$. On a :
    \begin{align*}
        I_n &= \int_1^e{x\ln^nx\,dx} = \left[\frac{1}{2}x^2\ln^nx\right]_1^e-\frac{n}{2}\int_1^e{x\ln^{n-1}x\,dx}\\
        &=\frac{e^2}{2}-\frac{n}{2}I_{n-1}
    \end{align*}
    On en déduit que $2I_n=e^2-nI_{n-1}$.\\
    Ainsi, $J_n = 2I_n + nI_{n-1} = e^2$\\
    3. Soit $n\in\mathbb{N}$. On a : 
    \begin{align*}
        I_{n+1} - I_{n} = \int_1^e{x\ln^{n+1}x\,dx} - \int_1^e{x\ln^{n}x\,dx} = \int_1^e{x\ln^nx(\ln x-1)\,dx}
    \end{align*}
    Or, pour $x\in\left[1,e\right]$, $\ln x \in [0,1]$ donc $\ln x - 1 \leq 0$. Ainsi, $x\ln^nx(\ln x-1) \leq 0$.\\
    On en déduit que $I_{n+1}-I_n\leq0$ et donc que $(I_n)$ est décroissante.\\
    Montrons que $I_n \leq \frac{e^2}{n+2}$ :
    \begin{align*}
        &I_n \leq I_{n-1}\\
        \iff&nI_n \leq nI_{n-1}\\
        \iff&(n+2)I_n \leq 2I_n + nI_{n-1}\\
        \iff&I_n \leq \frac{e^2}{n+2}
    \end{align*}
    Montrons que $\frac{e^2}{n+3} \leq I_n$ :
    \begin{align*}
        &I_{n+1} \leq I_n\\
        \iff&(n+1)I_{n+1} \leq (n+1)I_n\\
        \iff&(n+3)I_{n+1} \leq 2I_{n+1} + (n+1)I_n\\
        \iff&I_{n+1} \leq \frac{e^2}{n+3} \iff \frac{e^2}{n+3} \leq I_{n+1} \leq I_n
    \end{align*}
\end{tcolorbox}

\begin{tcolorbox}[enhanced, width=7in, center, size=fbox, fontupper=\large, drop shadow southwest]
    4. On a :
    \begin{equation*}
        \lim\limits_{n\to+\infty}{\frac{e^2}{n+3}}=0 \hspace{1cm} \text{et} \hspace{1cm} \lim\limits_{n\to+\infty}{\frac{e^2}{n+2}}=0
    \end{equation*} 
    Ainsi, d'après le Sandwich Theorem, on a :
    \begin{equation*}
        \lim\limits_{n\to+\infty}{I_n}=0
    \end{equation*}
    On a: 
    \begin{equation*}
        \lim\limits_{n\to+\infty}{\frac{ne^2}{n+3}}=e^2 \hspace{1cm} \text{et} \hspace{1cm} \lim\limits_{n\to+\infty}{\frac{ne^2}{n+2}}=e^2
    \end{equation*} 
    Ainsi, d'après le Théorème de l'Étau, on a :
    \begin{equation*}
        \lim\limits_{n\to+\infty}{I_n}=e^2
    \end{equation*}
    \qed
\end{tcolorbox}
\addcontentsline{toc}{section}{\protect\numberline{}Exercice 8.16}

\section*{Exercice 8.17 [$\blacklozenge\blacklozenge\blacklozenge$]}
\begin{tcolorbox}[enhanced, width=7in, center, size=fbox, fontupper=\large, drop shadow southwest]
    Calculer, pour tout entier naturel $n$, le nombre $I_n=\int_0^1{x^n\sqrt{1-x}dx}$.\\
    On a :
    \begin{align*}
        I_n &= \int_0^1x^n\sqrt{1-x}dx=\left[-\frac{2}{3}x^n(1-x)^{3/2}\right]_0^1-\int_0^1{-\frac{2}{3}nx^{n-1}(1-x)^{3/2}dx}\\
        &=\frac{2n}{3}\int_0^1{x^{n-1}(1-x)\sqrt{1-x}dx}\\
        &=\frac{2n}{3}\int_0^1{x^{n-1}\sqrt{1-x} - x^n\sqrt{1-x}dx} = \frac{2n}{3}(I_{n-1} - I_n)
    \end{align*}
    On obtient que
    \begin{equation*}
        I_n = \frac{2n}{2n+3}I_{n-1}.
    \end{equation*}
    Calculons $I_0$ et $I_1$:
    \begin{align*}
        &I_0 = \int_0^1{\sqrt{1-x}dx} = \left[-\frac{2}{3}(1-x)^{3/2}\right]_0^1 = \frac{2}{3}.\\
        &I_1 = \frac{2}{5}I_0 = \frac{2}{5}\cdot\frac{2}{3}
    \end{align*}
    On a alors :
    \begin{align*}
        I_n &= \frac{2n}{2n+3}I_{n-1} = \frac{2n}{2n+3}\cdot\frac{2(n-1)}{2n+1}\cdot...\cdot I_1\\
        &= \frac{2^{n+1}n!}{\prod\limits_{k=0}^{n}{2k+3}}
    \end{align*}
    On peut donc faire une preuve belle, rigoureuse, et \textbf{triviale} par récurrence mais j'ai la flemme.\\
    \qed
\end{tcolorbox}

\addcontentsline{toc}{section}{\protect\numberline{}Exercice 8.17}
\end{document}
 