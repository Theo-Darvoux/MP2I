\documentclass[11pt]{article}

\def\chapitre{4}
\def\pagetitle{Fonctions de la variable réelle.}

\input{/home/theo/MP2I/setup.tex}

\DeclareMathOperator{\argth}{argth}

\begin{document}

\input{/home/theo/MP2I/title.tex}

\thispagestyle{fancy}

Dans tout ce cours, la lettre $\K$ pourra être remplacée par $\R$ ou $\C$.\n
Les lettres $I$ et $J$ désigneront des intervalles de $\R$ non vides et non réduits à un point.

\section{Vocabulaire sur les fonctions.}

Soit $X$ un partie de $\R$ et $Y$ une partie de $\K$. Une fonction (ou application)
\begin{equation*}
    f:\begin{cases}
        X&\to\quad Y\\
        x&\mapsto f(x)
    \end{cases}
\end{equation*}
est un procédé qui à tout élément $x$ de $X$ associe un unique élément $f(x)$ appartenant à $Y$.\n
Si $x\in X$ et $y=f(x)$, on dit que $y$ est l'\bf{image} de $x$ et que $x$ est un \bf{antécédent} de $y$ par $f$.\n
Puisqu'ici la variable $x$ est un nombre réel, on dit que la fonction est de la \bf{variable réelle}.\n
Lorsque toutes les images par la fonction $f$ sont des nombres réels, alors $f$ est dite \bf{à valeurs réelles}.\n
La notion de fonction (ou d'application) entre deux ensembles quelconques sera étudiée plus formellement dans un cours Ensemble et applications à venir.

\subsection{Ensemble de définition.}

\begin{rappel}{}{}
    L'\bf{ensemble de définition} d'une fonction $f$ est l'ensemble des réels $x$ tel que $f(x)$ a un sens.
\end{rappel}

\begin{ex}{}{}
    Donner l'ensemble de définition de
    \begin{equation*}
        f:x\mapsto\sqrt{x(x-2)},\quad g:x\mapsto\ln(x(x-2)),\quad h:x\mapsto\ln(x)+\ln(x-2).
    \end{equation*}
\end{ex}

\subsection{Représentation graphique d'une fonction à valeurs réelles.}

\begin{defi}{}{}
    Soit $f:X\to\R$ une fonction réelle de la variable réelle et à valeurs réelles.\\
    On appelle \bf{graphe} de $f$ la partie de $\R^2$ suivante:
    \begin{equation*}
        \{(x,f(x)),\quad x\in X\}.
    \end{equation*}
    qui peut aussi s'écrire $\{(x,y)\in\R^2~:~x\in X\et y=f(x)\}$.\\
    Supposons le plan muni d'un repère orthonormé $(O,\v{i}, \v{j})$. À chaque élément du graphe correspond alors un point du plan. Déposons une goutte d'encre sur chacun de ces points. Le résultat est appelé \bf{courbe représentative} de la fonction. Dans la pratique, on confond graphe et courbe représentative.
\end{defi}

\subsection{Somme et produit de fonctions.}

\begin{defi}{}{}
    Soient $f$ et $g$ définies sur un même ensemble $X\subset\R$, à valeurs réelles.\\
    La \bf{somme} de $f$ et $g$ est la fonction définie par
    \begin{equation*}
        f+g:\begin{cases}
            X&\to\quad\R\\
            x&\mapsto\quad f(x)+g(x)
        \end{cases}.
    \end{equation*}
\end{defi}

\begin{defi}{}{}
    Soient $f$ et $g$ définies sur un même ensemble $X\subset\R$, à valeurs réelles.\\
    Le \bf{produit} et le \bf{quotient} de $f$ et $g$ sont les fonctions définies par
    \begin{equation*}
        f\cdot g:\begin{cases}
            X&\to\quad\R\\
            x&\mapsto\quad f(x)g(x)
        \end{cases} \quad \nt{et} \quad f/g : \begin{cases}
            X&\to\quad\R\\
            x&\mapsto\quad f(x)/g(x)
        \end{cases}.
    \end{equation*}
    Pour que la définition du quotient ait un sens, il est nécessaire que $g$ ne s'annule pas sur $X$.
\end{defi}

\subsection{Parité, imparité, périodicité.}

\begin{defi}{}{}
    Soit $X$ un partie de $\R$. Une fonction $f:X\to\K$ est dite
    \begin{center}
        $\bullet$ \bf{paire} si $\forall x \in X$ $\begin{cases}-x\in X\\f(-x)=f(x)\end{cases}$\qquad$\bullet$ \bf{impaire} si $\forall x \in X$ $\begin{cases}-x\in X\\f(-x)=-f(x)\end{cases}$.
    \end{center}
\end{defi}

\begin{ex}{}{}
    Montrer que le produit de deux fonctions impaires est une fonction paire.
    \tcblower
    Soient $f$ et $g$ impaires et définies sur $X\subset\R$.\\
    Soit $x\in X$, on a $(f\cdot g)(-x)=f(-x)g(-x)=(-f(x))(-g(x))=f(x)g(x)=(f\cdot g)(x)$.\\
    Le produit de $f$ et $g$ est donc une fonction paire.
\end{ex}

\begin{ex}{Une preuve par analyse-synthèse.}{}
    Démontrer le résultat ci-dessous
    \begin{center}
        Toute fonction définie sur $\R$ s'écrit de manière unique comme somme\\
        d'une fonction paire et d'une fonction impaire.
    \end{center}
    \tcblower
    Soit $\phi:\R\to\K$ une fonction.\\
    \bf{Analyse.} Supposons qu'il existe $f:\R\to\K$ paire et $g:\R\to\K$ impaire telles que $f+h=\phi$.\\
    Soit $x\in\R$. On a:\\
    --- $\phi(x)=f(x)+g(x)$\\
    --- $\phi(-x)=f(x)-g(x)$\\
    En sommant les deux égalités : $f(x)=\frac{\phi(x)+\phi(-x)}{2}$.\\
    En soustrayant les deux égalités : $g(x)=\frac{\phi(x)-\phi(-x)}{2}$.\\
    On a donc montré l'unicité de la décomposition.\n
    \bf{Synthèse.} Soit $f:\R\to\K$ définie par $f(x)=\frac{\phi(x)+\phi(-x)}{2}$ et $g:\R\to\K$ définie par $g(x)=\frac{\phi(x)-\phi(-x)}{2}$ pour tout $x\in\R$.\\
    On vérifie facilement que $f$ est paire et $g$ est impaire, et que $\phi$ est la somme des deux (on a tout fait pour).\n
    \bf{Conclusion.} $\phi$ s'écrit de manière unique comme somme d'une fonction paire et d'une fonction impaire.
\end{ex}

\begin{defi}{}{}
    Soit $T>0$ et $X\subset\R$. Une fonction $f:X\to\R$ est dite $T$-\bf{périodique} si
    \begin{equation*}
        \forall x \in X~\begin{cases}x+T\in X\\f(x+T)=f(x)\end{cases}
    \end{equation*}
    Une fonction sera dite \bf{périodique} si elle admet une certaine période $T\in\R_+^*$.
\end{defi}

\bf{Exemples.} cos et sin sont $2\pi$-périodiques. La fonction tan et $\pi$-périodique sur son ensemble de définition.\\
Un exemple de fonction à valeurs complexes : $t\mapsto e^{it}$ est $2\pi$-périodique.\n
La fonction $x\mapsto x-\lf x \rf$ est $1$-périodique.

\pagebreak

\begin{ex}{}{}
    Soit $f:\R\to\K$ une fonction $T$-périodique, où $T$ est un réel strictement positif.\\
    Montrer que $g:x\mapsto f(-x)$ est $T$-périodique.\\
    Soit $a>0$, prouver que $h:x\mapsto f(ax)$ est $T'$-périodique, en précisant $T'$.
    \tcblower
    $\bullet$ Soit $x\in\R$, on a $g(x+T)=f(-x-T)=f(-x-T+T)=f(-x)=g(x)$.\\
    $\bullet$ Soit $x\in\R$, on a $h(x+\frac{T}{a})=f(ax+T)=f(ax)=h(x)$.
\end{ex}

\begin{meth}{Réduction de l'intervalle d'étude.}{}
    Soit $f:X\to\R$.
    \begin{itemize}
        \item Si $f$ est $T$-périodique, son graphe est laissé invariant par la translation de vecteur $T\v{i}$.\\
        Il suffit donc d'étudier une fonction $T$-périodique sur un intervalle de longueur $T$, le plus souvent $[0,T]$ ou $[-\frac{T}{2},\frac{T}{2}]$. On obtient le reste du graphe par translations.
        \item Si $f$ est paire, son graphe est symétrique par rapport à l'axe des ordonnées.\\
        Si $f$ est impaire, il est invariant par la symétrie centrale de centre $O$.\\
        Il suffit alors d'étudier $f$ sur $X\cap\R_+$. L'étude sur $X\cap\R_-$ vient par symétrie. 
    \end{itemize}
\end{meth}

\begin{ex}{}{}
    Proposer un intervalle d'étude pour $f:x\mapsto\ch(\sin(3x))$.
    \tcblower
    C'est une fonction paire et $\frac{\pi}{3}$ périodique.\\
    On peut restreindre son intervalle d'étude à $[-\frac{\pi}{6},\frac{\pi}{6}]$ et donc à $[0,\frac{\pi}{6}]$ par parité.
\end{ex}

\subsection{Monotonie.}

\begin{defi}{}{}
    Soit $f:X\to\R$. Elle est dite
    \begin{itemize}
        \item \bf{croissante} si $\forall x,x'\in X\quad x\leq x' \ra f(x)\leq f(x')$.
        \item \bf{décroissante} si $\forall x,x'\in X\quad x\leq x' \ra f(x)\geq f(x')$.
        \item \bf{strictement croissante} si $\forall x,x'\in X\quad x< x' \ra f(x)< f(x')$.
        \item \bf{strictement décroissante} si $\forall x,x'\in X\quad x< x' \ra f(x)> f(x')$.
    \end{itemize}
    Lorsqu'une fonction a l'une de ces propriétés, elle est dite \bf{monotone} (stricteement monotone dans les deux derniers cas).
\end{defi}

\begin{ex}{}{}
    Soit $f$ une fonction paire sur $\R$ et croissante sur $\R_+$, montrer qu'elle est décroissante sur $\R_-$.
    \tcblower
    Soient $x,y\in\R_-$ tels que $x\leq y$, alors $-x\geq -y$.\\
    On applique $f$, croissante sur $\R_+$ : $f(-x)\geq f(-y)$, or $f$ est paire donc $f(x)\geq f(y)$.\\
    On a bien montré que $f$ est décroissante sur $\R_-$. 
\end{ex}

\begin{ex}{Un peu de composition.}{}
    Que dire de la composée de deux fonctions décroissantes ?
    \tcblower
    Soient $f:X\to Y$ et $g: Y\to \R$ décroissantes sur leurs ensembles de définition.\\
    Soient $x,y\in X$ tels que $x\leq y$.\\
    Alors $f(x)\geq f(y)$ donc $g(f(x))\leq g(f(y))$ par décroissance de $g$.\\
    On en conclut que $(g\circ f)(x)\leq (g\circ f)(y)$, la composée est croissante.
\end{ex}

\begin{ex}{Un peu de logique.}{}
    Justifier que si $f$ est strictement monotone, alors les réciproques des implications écrites dans la définition sont vraie.\\
    Que dire si $f$ est seulement monotone ?
    \tcblower
    Soit $f:X\to\R$ strictement croissante et $x,y\in X$ : $x < y \ra f(x) < f(y)$.\\
    La contraposée de la réciproque est $x \geq y \ra f(x) \geq f(y)$, c'est vrai par croissance de $f$.\n
    Supposons $f$ seulement croissante, on a $x\leq y \ra f(x) \leq f(y)$.\\
    On peut avoir $f(2)\leq f(1)$, mais $2\leq 1$ est faux, donc la réciproque est fausse.
\end{ex}

\subsection{Fonctions bornées.}

\begin{defi}{}{}
    Une fonction $f: X\to \R$ est dite
    \begin{itemize}
        \item \bf{majorée} s'il existe $M\in\R$ tel que $\forall x \in X, ~ f(x)\leq M$.
        \item \bf{minorée} s'il existe $m\in\R$ tel que $\forall x \in X, ~ f(x) \geq m$.
        \item \bf{bornée} si elle est majorée et minorée.
    \end{itemize}
\end{defi}

\begin{prop}{Caractérisation des fonctions bornées.}{}
    Soit $f:X\to\R$ une fonction.
    \begin{equation*}
        f \nt{ est bornée} \quad \iff \quad \exists \mu \in \R_+,~\forall x \in X, ~ |f(x)|\leq\mu.
    \end{equation*}
    \tcblower
    On a
    \begin{equation*}
        f \nt{ bornée} \iff \{f(x),~x\in X\} \nt{ est borné}.
    \end{equation*}
    Il reste alors à appliquer la caractérisation des parties bornées à cet ensemble (cours précédent).
\end{prop}

\begin{ex}{$\star$}{}
    Soient deux fonctions bornées $f$ et $g$ définies sur un même ensemble $X$.\\
    Montrer que leur somme $f+g$ et leur produit $fg$ sont bornées.
    \tcblower
    Il existe $M,N\in\R_+$ tels que $\forall x \in X,~|f(x)|\leq M$ et $|g(x)|\leq N$.\\
    Alors pour $x\in X$, on a $|f(x)+g(x)|\leq|f(x)|+|g(x)|\leq M + N$, la somme est bornée.\\
    De même, $|f(x)g(x)|=|f(x)||g(x)|\leq MN$. Le produit est borné. 
\end{ex}

\begin{defi}{}{}
    Soit une fonction $f:X\to\R$ et $a\in X$. On dit que
    \begin{itemize}
        \item $f$ admet un \bf{maximum} en $a$ si $\forall x\in X,~f(x)\leq f(a)$.
        \item $f$ admet un \bf{minimum} en $a$ si $\forall x\in X,~f(x)\geq f(a)$.
        \item un \bf{extremum} est un maximum ou un minimum.
    \end{itemize}
\end{defi}

\subsection{Bijections.}

La notion de bijection entre un ensemble $E$ et un ensemble $F$ sera étudiée ultérieurement.\\
On se contente ici de quelques résultats sur les bijections entre deux intervalles $I$ et $J$ de $\R$.

\begin{defi}{}{}
    On dit qu'une fonction $f:I\to J$ est une \bf{bijection} de $I$ vers $J$ si tout élément de $J$ possède un unique antécédent dans $I$ par $f$, ce qui s'écrit
    \begin{equation*}
        \forall y \in J,~\exists!x\in I\quad y=f(x).
    \end{equation*}
\end{defi}

\begin{defi}{}{}
    Soit $f:I\to J$ une bijection. Tout élément $y\in J$ possède un unique antécédent dans $I$ par $f$; notons-le $f^{-1}(y)$. Ceci définit la fonction \bf{réciproque} de $f$.
    \begin{equation*}
        f^{-1}:\begin{cases}J&\to\quad I\\y&\mapsto\quad f^{-1}(y)\end{cases}.
    \end{equation*}
\end{defi}

\begin{meth}{L'équation reliée à la bijectivité.}{}
    Prouver la bijectivité de $f:I\to J$ revient à montre rque pour tout élément $y\in J$, l'équation
    \begin{equation*}
        y=f(x),
    \end{equation*}
    a une unique solution dans $I$. Calculer $f^{-1}(y)$ c'est exprimer cette solution $x$ en fonction de $y$.
\end{meth}

\begin{ex}{}{}
    Montrer que $f:x\mapsto\sqrt{1+x^2}$ réalise une bijection de $\R_+$ dans $[1,+\infty[$ et expliciter sa réciproque.
    \tcblower
    Soit $y\in[1,+\infty[$, on cherche $x\in\R_+$ tel que $\sqrt{1+x^2}=y$.\\
    On a $1+x^2=y^2$ donc $x^2=y^2-1$ et $x=\sqrt{y^2-1}$ car $x\in\R_+$.\\
    L'équation a donc une unique solution. C'est bien une bijection de $\R_+$ vers $[1,+\infty[$.\\
    Sa réciproque est $f^{-1}:x\mapsto\sqrt{y^2-1}$.
\end{ex}

\begin{prop}{découle de la définition de $f^{-1}$}{}
    Soit $f:I\to J$ une bijection et $f^{-1}:J\to I$ sa réciproque. On a
    \begin{equation*}
        \forall x\in I\quad f^{-1}(f(x))=x\quad\et\quad\forall y\in J\quad f(f^{-1}(y))=y.
    \end{equation*}
\end{prop}

\begin{prop}{}{}
    Soit $f:I\to J$ une bijection.\\
    Le graphe de $f^{-1}:J\to I$ est le symétrique de celui de $f$ par rapport à la droite d'équation $y=x$.
    \tcblower
    Notons $C_f$ et $C_{f^{-1}}$ les graphes de $f$ et $f^{-1}$. Soit $(x,y)\in \R^2$.
    \begin{equation*}
        (x,y)\in C_f \iff x\in I\et y=f(x)\iff y\in J\et x=f^{-1}(y) \iff (y,x)\in C_{f^{-1}}.
    \end{equation*}
\end{prop}

\begin{prop}{}{}
    Soit $f:I\to J$ une bijection.
    \begin{enumerate}
        \item Si $f$ est strictement croissante (resp. strictement décroissante) sur $I$, alors $f^{-1}$ est strictement croissante (resp. strictement décroissante sur $J$).
        \item Si $f$ est impaire, alors $f^{-1}:J\to I$ l'est aussi.
    \end{enumerate}
    \tcblower
    \boxed{1.} Supposons $f$ strictement croissante sur $I$, soient $y,y'\in J$. Par contraposée, supposons $f^{-1}(y)\geq f^{-1}(y')$.\\
    Appliquons $f$ croissante: $f(f^{-1}(y))\geq f(f^{-1}(y'))$ alors $y\geq y'$.\n
    \boxed{2.} Supposons $f$ impaire sur $I$ et soit $y\in J$. On note $x=f^{-1}(y)$. Alors:
    \begin{equation*}
        -y=-f(x)=f(-x)\quad\nt{donc }-y\in J.
    \end{equation*}
    Alors $f^{-1}(-y)=f^{-1}(f(-x))=-x=-f^{-1}(y)$.\\
    On a $f^{-1}$ bien impaire sur $J$.
\end{prop}

\section{Continuité et dérivabilité.}

\subsection{Définitions.}

\begin{defi}{}{}
    Soit $f:I\to\K$ et $a\in I$. La fonction $f$ est \bf{continue} en $a$ si
    \begin{equation*}
        f(x)\xrightarrow[x\to a]{}f(a).
    \end{equation*} 
    Si $f$ est continue en tout point de $I$, elle est dite \bf{continue sur} I.
\end{defi}

\begin{defi}{}{}
    Soit $f:I\to\K$ et $a\in I$. La fonction $f$ est \bf{dérivable} en $a$ si
    \begin{equation*}
        \frac{f(x)-f(a)}{x-a}\quad(x\neq a)
    \end{equation*}
    a une limite finie lorsque $x$ tend vers $a$. On note alors $f'(a)=\lim_{x\to a}\frac{f(x)-f(a)}{x-a}$.\\
    Si $f$ est dérivable en tout point de $I$, elle est dite \bf{dérivable} sur $I$.\\
    La fonction $f':\begin{cases}I&\to\quad\K\\x&\mapsto\quad f'(x)\end{cases}$, est alors appelée \bf{dérivée} de $f$.
\end{defi}

\begin{ex}{}{}
    Continuité et dérivabilité de la fonction $f:x\mapsto\sqrt{x}$.
    \tcblower
    $\bullet$ Montrons que $f$ est dérivable sur $\R_+^*$. Soit $a\in\R_+^*$, $x\in\R_+\setminus\{a\}$.
    \begin{equation*}
        \frac{f(x)-f(a)}{x-a}=\frac{\sqrt{x}-\sqrt{a}}{x-a}=\frac{\sqrt{x}-\sqrt{a}}{(\sqrt{x}-\sqrt{a})(\sqrt{x}+\sqrt{a})}=\frac{1}{\sqrt{x}+\sqrt{a}}\xrightarrow[x\to a]{}\frac{1}{2\sqrt{a}}
    \end{equation*}
    Donc $f$ est dérivable en $a$ et $f'(a)=\frac{1}{2\sqrt{a}}$.\n
    $\bullet$ Montrons que $f$ n'est pas dérivable en 0. Soit $x\in\R_+^*$.
    \begin{equation*}
        \frac{f(x)-f(0)}{x-0}=\frac{\sqrt{x}}{x}=\frac{1}{\sqrt{x}}\xrightarrow[x\to0]{}+\infty
    \end{equation*}
    Ainsi, $f$ n'est pas dérivable en 0.
\end{ex}

\begin{center}
    \includegraphics*[scale=0.9]{derivees_usuelles.png}
\end{center}

\begin{prop}{}{}
    Si une fonction est dérivable, alors elle est continue, la réciproque est fausse.
    \tcblower
    Soit $f:I\to \K$ et supposée dérivable en $a\in I$. Pour $x\in I\setminus\{a\}$, on a
    \begin{equation*}
        f(x)-f(a)=\frac{f(x)-f(a)}{x-a}(x-a)\xrightarrow[x\to a]{} f'(a)\cdot 0 = 0.
    \end{equation*}
    Ainsi, $f(x)\xrightarrow[x\to a]{}f(a)$, donc $f$ est continue en $a$.
\end{prop}

\subsection{Continuité et opérations.}

\begin{prop}{}{}
    Si $f$ et $g$ sont continues sur $I$, alors leur somme et leur produit sont continues sur $I$.\\
    Si $f$ et $g$ sont continues sur $I$ et que $g$ ne s'annule pas sur $I$, leur quotient est continu sur $I$.
\end{prop}

\begin{prop}{}{}
    Soient deux fonctions $f:I\to J$ et $g:J\to\K$.\\
    Si $f$ est continue sur $I$ et $g$ est continue sur $J$, alors la composée $g\circ f$ est continue sur $I$.
\end{prop}

\subsection{Dérivabilité et opérations.}

La plupart des résultats théoriques seront démontrés dans le cours dédié à la dérivabilité.

\begin{prop}{Dérivée d'une somme, d'un produit, d'un quotient.}{}
    Soient deux fonctions $f,g:I\to\K$, dérivables sur l'intervalle $I$. Soit $\l\in\R$.
    \begin{itemize}
        \item La fonction $f+g$ est dérivable sur $I$ et $(f+g)'=f'+g'$.
        \item La fonction $\l f$ est dérivable sur $I$ et $(\l f)'=\l f'$.
        \item La fonction $fg$ est dérivable sur $I$ et $(fg)'=f'g+fg'$.
        \item Si $g$ ne s'annule pas sur $I$, la fonction $f/g$ est dérivable sur $I$ et $(f/g)'=\frac{f'g-fg'}{g^2}$.
    \end{itemize}
\end{prop}

\begin{thm}{Dérivée d'une composée. $\star$}{}
    Soient deux fonctions $f:I\to J$ et $g:J\to\K$.\\
    Si $f$ est dérivable sur $I$ et $g$ dérivable sur $J$, alors $g\circ f$ est dérivable sur $I$ et
    \begin{center}
        \boxed{(g\circ f')=f'\cdot(g'\circ f)} \quad i.e. \quad $\forall x \in I, ~ (g\circ f)'(x)=f'(x)g'(f(x))$.
    \end{center}
\end{thm}

\begin{ex}{}{}
    Dériver les fonctions $A:x\mapsto\cos(\ln(x))$ et $B:x\mapsto(\ch(x))^{\pi}$.
    \tcblower
    $\bullet$ On a $\ln$ dérivable sur $\R_+^*$ et $\cos$ dérivable sur $\R$, alors $A$ est dérivable sur $\R_+^*$. Pour $x\in\R_+^*$:
    \begin{equation*}
        A'(x)=\ln'(x)\cos'(\ln(x))=\frac{1}{x}\sin(\ln(x)).
    \end{equation*}
    $\bullet$ Notons $u:x\mapsto x^\pi$, dérivable sur $\R_+^*$, on a $\ch$ dérivable sur $\R$, alors $B$ est dérivable sur $\R$ et pour $x\in\R$:
    \begin{equation*}
        B'(x)=\\ch'(x)u'(\ch(x))=\sh(x)(\pi\ch(x)^{\pi-1}).
    \end{equation*}
\end{ex}

\pagebreak

\begin{ex}{}{}
    Soit $C:x\mapsto\sqrt{1-x^2}$. Où est-elle définie ? Où est-elle dérivable ? Donner sa dérivée.
    \tcblower
    Elle est définie sur $\{x\in\R\mid 1-x^2\geq0\}=[-1,1]$.\\
    La fonction $\sqrt{\cdot}$ est dérivable sur $\R_+^*$ et $u : x \mapsto 1-x^2$ est dérivable sur $\R$.\\
    C'est est dérivable par composée sur $]-1,1[$. Sa dérivée est pour $x\in]-1,1[$:
    \begin{equation*}
        C'(x)=u'(x)\frac{1}{2u(x)}=\frac{-2x}{2\sqrt{1-x^2}}=\frac{-x}{\sqrt{1-x^2}}.
    \end{equation*} 
\end{ex}

\begin{corr}{Cas particuliers courants.}{}
    Soit $u:I\to\R$, dérivable sur $I$, alors
    \begin{itemize}
        \item la fonction $e^u$ est dérivable sur $I$ et \boxed{(e^u)'=u'e^u}.
        \item $u^n$ est dérivable sur $I$ et $(u^n)'=nu'u^{n-1}$.
    \end{itemize}
    Si de surcroît
    \begin{itemize}
        \item $u:I\to\R^*$, alors $1/u$ est dérivable sur $I$ et \boxed{\left( \frac{1}{u} \right)'=-\frac{u'}{u^2}}.
        \item $u:I\to\R_+^*$, alos pour tout réel $a$, $u^a$ est dérivable sur $I$ et \boxed{(u^a)'=au'u^{a-1}}.\\
        Notamment, \boxed{(\sqrt{u})'=\frac{u'}{2\sqrt{u}}}.
        \item $u:I\to\R_+^*$, la fonction $\ln(u)$ est dérivable sur $I$ et \boxed{(\ln(u))'=\frac{u'}{u}}.
    \end{itemize}
\end{corr}

\bf{Remarque.} À quoi sert de regarder des cas particuliers puisqu'on a une formule simple dans le cas général ? La réponse est à chercher du côté du calcul de primitive : nous aurons besoin de savoir << dériver à l'envers >> : il est donc utile de bien connaître la forme des dérivées de composées dans les cas courants.

\subsection{Dérivée d'une réciproque.}

\begin{thm}{Dérivée d'une réciproque.}{}
    Soit une bijection $f:I\to J$ dérivable sur $I$.\\
    Sa réciproque $f^{-1}$ est dérivable sur $J$ ssi $f'$ ne s'annule pas sur $I$. On a alors
    \begin{center}
        \boxed{(f^{-1})'=\frac{1}{f'\circ f^{-1}}}\quad i.e.\quad $\forall y\in J\quad(f^{-1})'(y)=\frac{1}{f'(f^{-1}(y))}$.
    \end{center}
\end{thm}

\begin{ex}{}{}
    Appliquer le théorème pour retrouver le résultat connu sur la dérivée de ln.
    \tcblower
    La fonction exponentielle est une bijection de $\R$ dans $\R_+^*$, elle est dérivable sur $\R$ et sa dérivée ne s'annule pas sur $\R$, donc par théorème, sa réciproque $\ln$ est dérivable sur $\R_+^*$ et pour $x\in\R_+^*$:
    \begin{equation*}
        \ln'(x)=\frac{1}{\exp'(\ln(x))}=\frac{1}{x}.
    \end{equation*}
\end{ex}

\subsection{Dérivées d'ordre supérieur.}

\begin{defi}{}{}
    Soit $f:I\to\K$ une fonction dérivable sur $I$ telle que $f':I\to\K$ est elle-même dérivable sur $I$.\\
    On appelle \bf{dérivée seconde de} $f$ et on note $f''$ la fonction dérivée de $f'$.
\end{defi}

\begin{defi}{}{}
    Soit $f:I\to\K$. On peut définir par récurrence une dérivée de $f$ à l'ordre $n$, notée $f^{(n)}$.
    \begin{itemize}
        \item On convient que $f^{(0)}=f$.
        \item Soit $n\in\N$, si $f^{(n)}$ est bien définie et dérivable sur $I$, alors \boxed{f^{(n+1)}=(f^{(n)})'}.
    \end{itemize}
\end{defi}

\begin{ex}{$\star$}{}
    Dérivées successives de $f:x\mapsto x^p$ ($p\in\N$) et de $\ln$ sur $\R_+^*$.
    \tcblower
    Elles sont dérivables, de même pour toutes leurs dérivées successives. Soit $x\in\R$.\\
    On a $f'(x)=px^{p-1}$, $f''(x)=p(p-1)x^{p-2}$ ... $f^{(n)}(x)=p(p-1)...(p-(n-1))x^{p-n}$.\\
    Alors si $p<n$, alors $f^{(n)(x)=0}$, sinon, $f^{(n)}(x)=\frac{p!}{(p-n)!}x^{p-n}$.\\
    Soit $x\in\R_+^*$, on a $\ln^0(x)=\ln(x)$, $\ln'(x)=\frac{1}{x}$, $\ln''(x)=-\frac{1}{x^2}$... $\ln^{(n)}(x)=(-1)^{n-1}(n-1)!x^{-n}$. 
\end{ex}

\section{Résultats importants du cours d'analyse.}
Les théorèmes présentés dans cette section étaient connus en Terminale, et seront démontrés plus tard dans l'année dans un cours d'analyse où la notion de limite aura été définie rigoureusement.
\subsection{Théorème des valeurs intermédiaires.}

\begin{thm}{}{}
    Soient deux réels $a\leq b$ et $f:[a,b]\to\R$ continue. Alors, pour tout réel $y$ entre $f(a)$ et $f(b)$,
    \begin{equation*}
        \exists c\in[a,b]\quad y=f(c)
    \end{equation*}
\end{thm}
Autrement dit, toutre valeur intermédiaire entre $f(a)$ et $f(b)$ possède (au moins) un antécédent par $f$. Comme on le voit dans l'exemple ci-dessous, il n'y a pas forcément unicité de l'antécédent.\n
Le TVI pourra donc être utilisé pour prouver l'existence d'une solution à une équation. Ceci est illustré dans le corollaire suivant, qui revient à prouver l'existence d'une solution à une équation du type $f(x)=0$.

\begin{corr}{Changement de signe d'une fonction continue.}{}
    Si une fonction continue sur un intervalle y change de signe, alors elle s'annule sur cet intervalle.
\end{corr}

Voici maintenant un corollaire où l'hypothèse de stricte monotonie est ajoutée, il pourra être utilisé pour prouver l'existence \bf{et} l'unicité d'une solution à une équation.

\begin{corr}{TVI strictement monotone.}{}
    Soit $f:[a,b]\to\R$ continue et strictement monotone sur $[a,b]$. Pour tout réel $y$ entre $f(a)$ et $f(b)$
    \begin{equation*}
        \exists!c\in[a,b]\quad y=f(c).
    \end{equation*}
\end{corr}

\begin{ex}{}{}
    Soient $a$ et $b$ deux nombres réels, on considère l'équation
    \begin{equation*}
        x^3+ax+b=0.
    \end{equation*}
    \begin{enumerate}
        \item Démontrer que l'équation possède une solution et ce quelles que soient les valeurs de $a$ et $b$.
        \item Démontrer l'unicité de la solution dans le cas où $a$ est positif.
    \end{enumerate}
    \tcblower
    \boxed{1.} Posons $f:x\mapsto x^3+ax+b$ polynomiale donc continue sur $\R$.\\
    Elle tend vers $-\infty$ en $-\infty$ et $+\infty$ en $+\infty$, elle change alors de signe, donc s'annule par TVI.\n
    \boxed{2.} Supposons $a\geq0$, $f$ est dérivable sur $\R$ et pour $x\in\R^*$, $f'(x)=3x^2+a>0$.\\
    Ainsi, $\forall x\in]-\infty,0[, ~ f'(x)>0$, $f$ est strictement croissante sur $]-\infty,0[$, de même sur $]0,+\infty[$.\\
    Bilan: $f$ est strictement croissante et continue sur $\R$. Par TVI, $\exists!c\in\R\mid f(c)=0$.
\end{ex}

\begin{thm}{Théorème de la bijection continue.}{}
    Soit $I$ un intervalle de $\R$ et $f:I\to\R$ une fonction.\vspace*{0.2cm}
    \begin{center}
        \fbox{Si $f$ est \bf{strictement monotone} et \bf{continue} sur $I$, alors elle réalise une bijection de $I$ dans $f(I)$.}
    \end{center}\vspace*{0.2cm}
    Précisons que la notation $f(I)$ désigne l'ensemble des images des éléments de $I$ par $f$.
    \begin{equation*}
        f(I)=\{f(x),x\in I\}=\{y\in J\mid \exists x \in I~:~y=f(x)\}.
    \end{equation*}
    De surcroît,
    \begin{itemize}
        \item L'ensemble $f(I)$ est un intervalle.
        \item Sur cet intervalle, $f^{-1}$ est strictement monotone, de même monotonie que $f$.
        \item Sur cet intervalle, la réciproque $f^{-1}$ est continue.
    \end{itemize}
\end{thm}

\begin{ex}{$\star$}{}
    \begin{enumerate}
        \item Justifier que $\sh$ réalise une bijection de $\R$ dans $\R$.
        \item Expliciter sa réciproque, puis calculer la dérivée de cette réciproque.
        \item Retrouver le dernier résultat en appliquant le théorème de dérivation d'une réciproque.
    \end{enumerate}
    \tcblower
    \boxed{1.} $\sh$ est continue et strictement croissante sur $\R$, d'après le théorème de la bijection continue, elle réalise une bijection de $\R$ dans $\sh(R)=\R$.\\
    \boxed{2.} Soient $x,y\in\R$, on pose l'équation $y=\sh(x)$:
    \begin{align*}
        y=\sh(x)&\iff y=\frac{e^x-e^{-x}}{2}\iff e^x-2y-e^{-x}=0\iff e^{2x}-2ye^{x}-1=0\\
        &\iff e^x \nt{ racine de } X^2-2yX-1=0 \iff e^x=y+\sqrt{y^2+1}\iff x=\ln(y+\sqrt{y^2}+1).
    \end{align*}
    La réciproque de $\sh$ est donc argsh$:x\mapsto\ln(x+\sqrt{x^2+1})$.\\
    En dérivant, on trouve pour $x\in\R$ que argsh$'(x)=\frac{1}{\sqrt{x^2+1}}$.\\
    \boxed{3.} $\sh$ est dérivable et ne s'annule pas sur $\R$, $\nt{argsh}=\frac{1}{\ch(\nt{argsh}(x))}=\frac{1}{\sqrt{1+\sh^2(\nt{argsh}(x))}}=\frac{1}{\sqrt{1+x^2}}$.
\end{ex}

\subsection{Variations des fonctions dérivables.}

\begin{thm}{Caractérisation des fonctions monotones parmi les fonctions dérivables.}{}
    Soit $I$ un intervalle et $f:I\to\R$ dérivable sur $I$.
    \begin{itemize}
        \item $f$ est croissante sur $I$ ssi $f'$ est positive sur $I$.
        \item $f$ est décroissante sur $I$ ssi $f'$ est négative sur $I$.
        \item $f$ est constante sur $I$ ssi $f'$ est nulle sur $I$.
    \end{itemize}
\end{thm}

\begin{prop}{Caractérisation des fonctions strictement motones parmi les fonctions dérivables.}{}
    Soit $I$ un intervalle et $f:I\to\R$ dérivable sur $I$.\\
    La fonction $f$ est strictement croissante sur $I$ ssi $f'$ est positive (ou nulle) sur $I$ et n'est identiquement nulle sur aucun intervalle $[a,b]$ inclus dans $I$ avec $a<b$.
\end{prop}

\begin{corr}{Dans la pratique.}{}
    Soit $I$ un intervalle, et $f:I\to\R$ dérivable sur $I$.
    \begin{itemize}
        \item Si $f'$ est strictement positive sur $I$, alors $f$ y est strictement croissante. Réciproque fausse.
        \item Si $f'$ est strictement négative sur $I$, alors $f$ y est strictement décroissante. Réciproque fausse.
    \end{itemize}
    L'implication demeure vraie si $f'$ ne s'annule qu'en un nombre fini de points de $I$.
\end{corr}

\begin{ex}{}{}
    Donner un exemple de fonction...
    \begin{enumerate}
        \item dérivable et strictement croissante sur $\R$, dont la dérivée s'annule.
        \item dérivable sur $\R^*$, et dont la dérivée est négative sans que la fonction soit décroissante sur $\R^*$.
    \end{enumerate}
    \tcblower
    \boxed{1.} La fonction cube est dérivable sur $\R$ et strictement croissante, sa dérivée s'annule en 0.\\
    \boxed{2.} La fonction $f:x\mapsto\frac{1}{x}$ est dérivable sur $\R^*$, de dérivée négative, mais $f$ n'est pas décroissante sur $\R^*$.
\end{ex}

\begin{meth}{Conseils pour l'étude d'une fonction}{}
    \begin{itemize}
        \item Déterminer son ensemble de définition.
        \item Détecter une éventuelle parité/imparité/périodicité et réduire en conséquence le domaine d'étude.
        \item Pour l'étude des variations, on ne se rue pas sur la dérivation si la fonction est une somme ou une composée de fonction croissantes, par exemple !
        \item Dans le cas où on dérive, on justifie sur quel ensemble et pourquoi on peut le faire, soigneusement.
        \item Calcul de la dérivée. Puisque c'est son signe qui nous intéressera, on cherche à la \bf{factoriser} le plus possible!
        \item Étude du signe de la dérivée : il suffira la plupart du temps de résoude l'\bf{inéquation} $f'(x)\geq0$.
        \item Tableau de signe pour la dérivée, de variations pour la fonction.
        \item Calcul des limites aux bords. Si on détecte une incohérence, il est encore temps de se relire !
        \item Esquisser un graphe résumant l'étude. Ne pas hésiter à y souligner des valeurs, ou des tangentes notables.
    \end{itemize}
\end{meth}

\section{Exercices.}

\begin{exercice}{$\blacklozenge\lozenge\lozenge$}{}
    Soit $f:\mathbb{R}\rightarrow\mathbb{R}$ une fonction $2$-périodique et $3$-périodique. Montrer que $f$ est $1$-périodique.
    \tcblower
    On a $\forall{x\in\mathbb{R}}\begin{cases}x-2\in\mathbb{R}\\f(x-2)=f(x)\end{cases} \text{ et } \begin{cases}x+3\in\mathbb{R}\\f(x+3)=f(x)\end{cases}$.\\
    Alors $\forall{x\in\mathbb{R}}\begin{cases}x-2+3\in\mathbb{R}\\f(x-2+3)=f(x-2)=f(x)\end{cases}$
\end{exercice}

\begin{exercice}{$\blacklozenge\blacklozenge\blacklozenge$}{}
    Déterminer toutes les fonctions croissantes $f:\mathbb{R}\rightarrow\mathbb{R}$ telles que
    \begin{equation*}
        \forall{x\in\mathbb{R}} \hspace{0.25cm} f(f(x))=x.
    \end{equation*}
    \tcblower
    Soit $x\in\mathbb{R}$ et $f$ une solution du problème.\\
    On remarque que $f:x\mapsto x$ est solution du problème.\\
    $\bullet$ Supposons $f(x)>x$, on a : $f(f(x))>f(x)$ par croissance de $f$.\\
    Or $f(f(x))=x$ donc $x>f(x)$, ce qui est absurde.\n
    $\bullet$ Supposons $f(x)<x$, on a : $f(f(x))<f(x)$ par croissance de $f$.\\
    Or $f(f(x))=x$ donc $x<f(x)$, ce qui est absurde.\n
    Ainsi, la seule fonction de $\mathbb{R}$ vers $\mathbb{R}$ solution est $f:x\mapsto x$.
\end{exercice}

\begin{exercice}{$\blacklozenge\lozenge\lozenge$}{}
    Pour chacune des fonctions ci-dessous, donner un ou plusieurs intervalles sur lesquels la fonction est dérivable, et préciser sa dérivée.
    \begin{center}
        $A:x\mapsto x^\pi,\hspace{0.5cm}B:x\mapsto\pi^x,\hspace{0.5cm}C:x\mapsto\cos(5x),\hspace{0.5cm}D:x\mapsto\th(\ch(x))$,\\
        $E:x\mapsto\ln\left(1+x^3\right)n\hspace{0.5cm}F:x\mapsto\cos\left(\sqrt{\ln(x)}\right),\hspace{0.5cm}G:x\mapsto\frac{1}{\sqrt{3x-1}},\hspace{0.5cm}H:x\mapsto\sin|x+1|$.
    \end{center}
    \tcblower
    \null
    \begin{itemize}
        \item $A':\begin{cases}\mathbb{R}^*_+\rightarrow\mathbb{R}\\x\mapsto\pi x^{\pi-1}\end{cases}$\hspace{1.8cm}• $D':\begin{cases}\mathbb{R}\rightarrow\mathbb{R}\\x\mapsto\frac{\sh(x)}{\ch^2(\ch(x))}\end{cases}$\hspace{1.6cm}• $G':x\mapsto-\frac{3}{2}(3x-1)^{3/2}$
        \item $B':\begin{cases}\mathbb{R}\rightarrow\mathbb{R}\\x\mapsto\ln(\pi)\pi^x\end{cases}$\hspace{1.5cm}• $E':\begin{cases}\mathbb{R}\setminus\{1\}\rightarrow\mathbb{R}\\x\mapsto\frac{3x^2}{1+x^3}\end{cases}$\hspace{1.6cm}• $H'_-:\begin{cases}]-\infty,-1[\rightarrow\mathbb{R}\\x\mapsto-\cos(-x-1)\end{cases}$
        \item $C':\begin{cases}\mathbb{R}\rightarrow\mathbb{R}\\x\mapsto-5\sin(5x)\end{cases}$\hspace{1.0cm}• $F':\begin{cases}]1,+\infty[\rightarrow\mathbb{R}\\x\mapsto\frac{\sin(\sqrt{\ln(x)})}{2x\sqrt{\ln(x)}}\end{cases}$\hspace{1.6cm}• $H'_+:\begin{cases}]1,+\infty[\rightarrow\mathbb{R}\\x\mapsto\cos(x+1)\end{cases}$
    \end{itemize}
\end{exercice}

\begin{exercice}{$\blacklozenge\lozenge\lozenge$}{}
    Donner le tableau de variations complet de
    \begin{align*}
        f:x\mapsto x^{x\ln(x)}
    \end{align*}
    \tcblower
    On a $f':\begin{cases}\mathbb{R}_+^*\rightarrow\mathbb{R}\\x\mapsto\ln(x)(\ln(x)+2)e^{x\ln^2(x)}\end{cases}$.\\
    Son tableau de variations est donc :\vspace*{0.2cm}
    \begin{center}
        \begin{tikzpicture}
            \tkzTabInit[espcl=3]{$x$/0.6,$f'(x)$/0.6,$f$/1.4}{$0$,$e^{-2}$,$1$,$+\infty$}
            \tkzTabLine{d,+,z,-,z,+}
            \tkzTabVar{-/$1$,+/$e^{4/e^{2}}$,-/$1$,+/$+\infty$}
        \end{tikzpicture}
    \end{center}
\end{exercice}

\begin{exercice}{$\blacklozenge\blacklozenge\lozenge$}{}
    1. Démontrer que
    \begin{equation*}
        \forall x\in]-1,+\infty[, \hspace{0.5cm} \frac{x}{1+x}\leq\ln(1+x)\leq x.
    \end{equation*}
    2. À l'aide du théorème des gendarmes, calculer $\lim\limits_{x\rightarrow0}{\frac{\ln(1+x)}{x}}$.\\
    3. Retrouver ce résultat en faisant apparaître un taux d'accroissement.
    \tcblower
    \boxed{1.} Posons :
    \begin{equation*}
        f:x\mapsto\frac{x}{1+x}-\ln(1+x) \hspace{2cm} g:x\mapsto\ln(1+x)-x
    \end{equation*}
    Elles sont dérivables et tout et tout :
    \begin{equation*}
        f':x\mapsto-\frac{x}{(1+x)^2} \hspace{2cm} g':x\mapsto-\frac{x}{1+x}
    \end{equation*}
    \begin{center}
        \begin{tikzpicture}
            \tkzTabInit[espcl=4]{$x$/0.6,$f'(x)$/0.6,$f$/1.4}{$-1$,0,$+\infty$}
            \tkzTabLine{d,+,z,-}
            \tkzTabVar{D-/$-\infty$, +/$0$, -/$-\infty$}
        \end{tikzpicture}
        \\[0.25cm]
        \begin{tikzpicture}
            \tkzTabInit[espcl=4]{$x$/0.6,$g'(x)$/0.6,$g$/1.4}{$-1$,0,$+\infty$}
            \tkzTabLine{d,+,z,-}
            \tkzTabVar{D-/$-\infty$, +/$0$, -/$-\infty$}
        \end{tikzpicture}
    \end{center}
    L'inégalité est donc vérifiée car ces fonctions prennent des valeurs négatives.\n
    \boxed{2.} Soit $x\in]-1,+\infty[$. On a :
    \begin{equation*}
        \frac{x}{1+x}\leq\ln(1+x)\leq x
    \end{equation*}
    Et :
    \begin{equation*}
        \lim_{x\rightarrow0}\frac{1}{1+x}=1 \hspace{1cm} \text{et} \hspace{1cm} \lim_{x\rightarrow0}1=1
    \end{equation*}
    Donc, d'après le théorème des gendarmes :
   \begin{equation*}
        \lim_{x\rightarrow0}\frac{\ln(1+x)}{x}=1
   \end{equation*}
   \boxed{3.}
   On a :
   \begin{equation*}
        \lim_{x\rightarrow0}\frac{\ln(x+1)}{x}=\lim_{x\rightarrow0}\frac{\ln(x+1)-\ln(1)}{x}=\ln'(1)=\frac{1}{1}=1
   \end{equation*}
\end{exercice}

\begin{exercice}{$\blacklozenge\blacklozenge\lozenge$}{}
    Démontrer l'inégalité
    \begin{equation*}
        \forall{x\in\mathbb{R}_+}, \hspace{1cm} 0\leq x-\sin x\leq\frac{x^3}{6}.
    \end{equation*}
    \tcblower
    Posons :
    \begin{equation*}
        f:x\mapsto x-\sin x \hspace{2cm} g:x\mapsto x-\sin x - \frac{x^3}{6}
    \end{equation*}
    Ces fonctions sont dérivables sur $\mathbb{R}_+$ :
    \begin{equation*}
        f':x\mapsto 1-\cos x \hspace {2cm} g':x\mapsto 1-\cos x - \frac{x^2}{2}
    \end{equation*}
    La première inégalité est triviale car $\cos x\leq1$, ainsi $x-\sin x\geq 0$.\\
    On a :
    \begin{equation*}
        g'':x\mapsto\sin x - x
    \end{equation*}
    Et donc :
    \begin{center}
        \begin{tikzpicture}
            \tkzTabInit[espcl=8]{$x$/0.6,$g''(x)$/0.6,$g'(x)$/1.4,$g$/1.4}{$0$,$+\infty$}
            \tkzTabLine{z,-}
            \tkzTabVar{+/$0$,-/$-\infty$}
            \tkzTabVar{+/$0$,-/$-\infty$}
        \end{tikzpicture}
    \end{center}
    Ainsi, $g$ prend des valeurs négatives sur $\mathbb{R}_+$, donc : $x-\sin x\leq\frac{x^3}{6}$.
\end{exercice}

\begin{exercice}{$\blacklozenge\blacklozenge\lozenge$}{}
    Faire une étude complète de la fonction 
    \begin{equation*}
        f:x\mapsto\ln\left(\left|\frac{1+x}{1-x}\right|\right)
    \end{equation*}
    \tcblower
    $\circledcirc$ Soit $x\in]-1,1[$.\\
    On a :
    \begin{equation*}
        f:x\mapsto\ln\left(\frac{1+x}{1-x}\right) \hspace{2cm} f':x\mapsto\frac{2}{1-x^2}
    \end{equation*}
    $\circledcirc$ Soit $x\in]-\infty,-1[]1,+\infty[$.\\
    On a :
    \begin{equation*}
        f:x\mapsto\ln\left(-\frac{1+x}{1-x}\right) \hspace{2cm} f':x\mapsto\frac{2}{1-x^2}
    \end{equation*}
    Donc :
    \begin{equation*}
        \forall{x\in\mathbb{R}\setminus\{-1,1\}}, \hspace{1cm} f':x\mapsto\frac{2}{1-x^2}.
    \end{equation*}
    \begin{center}
        \begin{tikzpicture}
            \tkzTabInit[espcl=4]{$x$/0.6,$f'(x)$/0.6,$f$/1.4}{$-\infty$,-1,1,$+\infty$}
            \tkzTabLine{,-,d,+,d,-}
            \tkzTabVar{+/$0$,-D-/$-\infty$/$-\infty$,+D+/$+\infty$/$+\infty$,-/$0$}
        \end{tikzpicture}
    \end{center}
    Pour les limites on a factorisé par $x$ :
    \begin{equation*}
        f:x\mapsto\ln\left(\left|\frac{\frac{1}{x}+1}{\frac{1}{x}-1}\right|\right)
    \end{equation*}
\end{exercice}

\begin{exercice}{$\blacklozenge\blacklozenge\lozenge$}{}
    Démontrer l'inégalité
    \begin{equation*}
        \forall{x\in]0,1[}\hspace{1cm}x^x(1-x)^{1-x}\geq\frac{1}{2}.
    \end{equation*}
    \tcblower
    Soit $x\in]0,1[$.\\
    On a :
    \begin{equation*}
        x^x(1-x)^{1-x}=e^{x\ln x}e^{(1-x)\ln(1-x)}=e^{x\ln{x}+(1-x)\ln(1-x)}
    \end{equation*}
    Posons :
    \begin{equation*}
        f:x\mapsto e^{x\ln{x}+(1-x)\ln(1-x)} \hspace{1cm} f':x\mapsto \ln\left(\frac{x}{1-x}\right)e^{x\ln{x}+(1-x)\ln(1-x)}
    \end{equation*}
    \begin{center}
        \begin{tikzpicture}
            \tkzTabInit[espcl=4]{$x$/0.6,$f'(x)$/0.6,$f$/0.9}{$0$,$0.5$,$1$}
            \tkzTabLine{d,-,z,+,d}
            \tkzTabVar{D+/$1$,-/$\frac{1}{2}$,+D/$1$}
        \end{tikzpicture}
    \end{center}
    La fonction est donc toujours supérieur à $\frac{1}{2}$ sur $]0,1[$.
\end{exercice}

\begin{exercice}{$\blacklozenge\blacklozenge\lozenge$}
    1. Étudier les variations de $f:x\mapsto\frac{x}{1+x}$ sur $[0,+\infty[$.\\
    2. Prouver que
    \begin{equation*}
        \forall{x,y\in\mathbb{R}}, \hspace{1cm} \frac{|x+y|}{1+|x+y|}\leq\frac{|x|}{1+|x|}+\frac{|y|}{1+|y|}.
    \end{equation*}
    \tcblower
    \boxed{1.} Soit $x\in\lbrack0,+\infty\lbrack$. On a :
    \begin{equation*}
        f':x\mapsto\frac{1}{(1+x)^2}
    \end{equation*}
    \begin{center}
        \begin{tikzpicture}
            \tkzTabInit[espcl=8]{$x$/0.6,$f'(x)$/0.6,$f$/0.9}{$0$,$+\infty$}
            \tkzTabLine{,+,}
            \tkzTabVar{-/$0$,+/$1$}
        \end{tikzpicture}
    \end{center}
    \boxed{2.} Soient $x,y\in\mathbb{R}$. Par inégalité triangulaire, on a :
    \begin{equation*}
        |x+y|\leq|x|+|y|
    \end{equation*}
    On applique $f$, fonction croissante sur $\mathbb{R}_+$, ce qui ne change pas les inégalités:
    \begin{align*}
        \frac{|x+y|}{1+|x+y|}&\leq\frac{|x|+|y|}{1+|x|+|y|}\\
        &\leq\frac{|x|}{1+|x|+|y|}+\frac{|y|}{1+|x|+|y|}\\
        &\leq\frac{|x|}{1+|x|}+\frac{|y|}{1+|y|}
    \end{align*}
\end{exercice}

\begin{exercice}{$\blacklozenge\blacklozenge\lozenge$}{}
    Soit la fonction $f:x\mapsto\ln\left(\sqrt{x^2+1}-x\right)$.\\
    1. Donner le domaine de définition de $f$.\\
    2. Montrer que $f$ est impaire.\\
    3. Étudier ses variations et donner le tableau correspondant.
    \tcblower
    \boxed{1.} Soit $g:\begin{cases}\mathbb{R}\rightarrow\mathbb{R}\\x\mapsto\sqrt{x^2+1}-x\end{cases}$ On a : $g':\begin{cases}\mathbb{R}\rightarrow\mathbb{R}\\x\mapsto\frac{x}{\sqrt{x^2+1}}-1\end{cases}$
    \begin{center}
        \begin{tikzpicture}
            \tkzTabInit[espcl=8]{$x$/0.6,$g'(x)$/0.6,$g$/0.9}{$-\infty$,$+\infty$}
            \tkzTabLine{,-,}
            \tkzTabVar{+/$+\infty$,-/$0$}
        \end{tikzpicture}
    \end{center}
    $f$ donc définie sur $\mathbb{R}$.\n
    \boxed{2.} Soit $x\in\mathbb{R}$, on a :
    \begin{align*}
        -f(x)&=-\ln\left(\sqrt{x^2+1}-x\right)
        =\ln\left(\frac{1}{\sqrt{x^2+1}-x}\right)
        =\ln\left(\frac{\sqrt{x^2+1}+x}{x^2+1-x^2}\right)
        =\ln\left(\sqrt{x^2+1}+x\right)
        =f(-x)
    \end{align*}
    \boxed{3.} On a $f':\begin{cases}\mathbb{R}\rightarrow\mathbb{R}\\x\mapsto-\frac{1}{\sqrt{x^2+1}}\end{cases}$\vspace*{0.2cm}
    \begin{center}
        \begin{tikzpicture}
            \tkzTabInit[espcl=8]{$x$/0.6,$f'(x)$/0.6,$f$/0.9}{$-\infty$,$+\infty$}
            \tkzTabLine{,-,}
            \tkzTabVar{+/$+\infty$,-/$-\infty$}
        \end{tikzpicture}
    \end{center}
\end{exercice}

\begin{exercice}{$\blacklozenge\blacklozenge\lozenge$}{}
    Notons $a$ le nombre
    \begin{equation*}
        \sqrt[3]{20+14\sqrt{2}}+\sqrt[3]{20-14\sqrt{2}}.
    \end{equation*}
    1. Montrer que $a^3=6a+40$.\\
    2. En déduire la valeur de $a$.
    \tcblower
    \boxed{1.}
    \begin{align*}
        a^3&=40+3\sqrt[3]{(20+14\sqrt{2})^2(20-14\sqrt{2})}+3\sqrt[3]{(20+14\sqrt{2})(20-14\sqrt{2})^2}\\
        &=40+3\sqrt[3]{(20+14\sqrt{2})(400-392)}+3\sqrt[3]{(20-14\sqrt{2})(400-392)}\\
        &=40+6\sqrt[3]{20+14\sqrt{2}}+6\sqrt[3]{20-14\sqrt{2}}\\
        &=6a+40
    \end{align*}
    \boxed{2.} On a :
    \begin{align*}
        a^3-6a-40=0
        \iff(a-4)(a^2+4a+10)=0
        \iff a=4
    \end{align*}
\end{exercice}

\begin{exercice}{$\blacklozenge\lozenge\lozenge$}
    Considérons la fonction
    \begin{equation*}
        f:\begin{cases}\rbrack1,+\infty\lbrack\rightarrow\mathbb{R}\\x\mapsto\exp(-\frac{1}{\ln(x)})\end{cases}.
    \end{equation*}
    1. Démontrer que $f$ réalise une bijection de $\rbrack1,+\infty\lbrack$ dans un intervalle que l'on précisera.\\
    2. Expliciter la réciproque de $f$. Peut-on écrire en conclusion que $f^{-1}=f$ ?
    \tcblower
    \boxed{1.} Soit $x\in\rbrack1,+\infty\lbrack$ et $y\in\mathbb{R}^*_+$. On a :
    \begin{align*}
        &y=f(x)\\
        \iff& y=\exp(-\frac{1}{\ln(x)})\\
        \iff& \ln(y)=-\frac{1}{\ln(x)}\\
        \iff& -\frac{1}{\ln(y)}=\ln(x)\\
        \iff& x=\exp(-\frac{1}{\ln(y)})
    \end{align*}
    L'équation a une unique solution, $f$ réalise donc une bijection de $\rbrack1,+\infty\lbrack$ vers $\mathbb{R}^*_+$.\\
    \boxed{2.} On a :
    \begin{equation*}
        f^{-1}:\begin{cases}\mathbb{R}^*_+\rightarrow\rbrack1,+\infty\lbrack\\x\mapsto\exp(-\frac{1}{\ln(x)})\end{cases}
    \end{equation*}
    $f\neq f^{-1}$ car leurs domaines de définition sont différents.
\end{exercice}

\begin{exercice}{$\blacklozenge\blacklozenge\blacklozenge$}{}
    1. Montrer que $\th$ est une bijection de $\mathbb{R}$ dans $\rbrack-1,1\lbrack$ et déterminer une expression explicite de sa réciproque, qu'on notera $\argth$.\\
    2. De deux façons différentes, montrer que $\argth$ est dérivable sur son intervalle de définition et calculer sa dérivée.\\
    3. Montrer que pour tout $x\in\mathbb{R}$, $\argth\left(\frac{1+3\th x}{3+\th x}\right)=x+\ln\sqrt{2}$.
    \tcblower
    \boxed{1.} Soit $x\in\mathbb{R}$ et $y\in\rbrack-1,1\lbrack$. On a :
    \begin{align*}
        &y=\th(x)\\
        \iff& y=\frac{e^x-e^{-x}}{e^x+e^{-x}}\\
        \iff& y=\frac{e^{2x}-1}{e^{2x}+1}\\
        \iff& e^{2x}(1-y)=y+1\\
        \iff &e^{2x}=\frac{y+1}{1-y}\\
        \iff &x=\frac{1}{2}\ln\left(\frac{y+1}{1-y}\right)
    \end{align*}
    L'équation a une unique solution, $\th$ réalise donc une bijection de $\mathbb{R}$ dans $\rbrack-1,1\lbrack$.\\
    Sa réciproque est $\argth:\begin{cases}\rbrack-1,1\lbrack\rightarrow\mathbb{R}\\x\mapsto\frac{1}{2}\ln\left(\frac{x+1}{1-x}\right)\end{cases}$\\[0.25cm]
    \boxed{2.} On peut montrer que $\argth$ est dérivable sur $]-1,1[$ par le théorème de dérivée des réciproques ou en dérivant $x\mapsto\frac{1}{2}\ln\left(\frac{x+1}{1-x}\right)$ comme composée de fonctions dérivables.\\
    On retrouve dans les deux cas:
    \begin{equation*}
        \argth':\begin{cases}]-1,1[\rightarrow\mathbb{R}\\x\mapsto\frac{1}{1-x^2}\end{cases}
    \end{equation*}
    \boxed{3.} Soit $x\in\mathbb{R}$.
    \begin{align*}
        \argth\left(\frac{1+3\th x}{3+\th x}\right)&=\frac{1}{2}\ln\left(\frac{\frac{1+3\th x}{3+\th x}+1}{1-\frac{1+3\th x}{3+\th x}}\right)\\
        &=\frac{1}{2}\ln\left(\frac{\frac{4+4\th x}{3+\th x}}{\frac{2-2\th x}{3 + \th x}}\right)\\
        &=\frac{1}{2}\ln\left(\frac{2+2\th x}{1-\th x}\right)\\
        &=\frac{1}{2}\ln\left(\frac{\ch(x)+\sh(x)}{\ch(x)-\sh(x)}\right)+\frac{1}{2}\ln(2)\\
        &=\frac{1}{2}\ln\left(e^{2x}\right)+\ln(\sqrt{2})\\
        &=x+\ln\sqrt{2}
    \end{align*}
\end{exercice}

\end{document}