\documentclass[11pt]{article}

\def\chapitre{9}
\def\pagetitle{Petits systèmes linéaires.}

\input{/home/theo/MP2I/setup.tex}

\begin{document}

\input{/home/theo/MP2I/title.tex}

\thispagestyle{fancy}

\section{Droites et plans.}

\fbox{\bf{Droites.}}

\begin{defi}{}{}
    Soit $(a,b)\in\R^2\setminus\{(0,0)\}$ et $c\in\R$.\n
    L'ensemble des couples $(x,y)$ de $\R^2$ qui sont solutions de l'équation linéaire
    \begin{equation*}
        ax+by=c
    \end{equation*}
    est une \bf{droite affine} de $\R^2$.\\
    La droite d'équation $ax+by=0$ est dite \bf{vectorielle}. Parallèle à la première, elle contient $(0,0)$.
\end{defi}

\begin{prop}{}{}
    Soit $(a,b)\in\R^2\setminus\{(0,0)\}$. On considère les droites
    \begin{equation*}
        D=\{(x,y)\in\R^2\mid ax+by=c\}\quad\et\quad D_0=\{(x,y)\in\R^2\mid ax+by=0\}.
    \end{equation*}
    Considérons\\
    --- $\v{u}$ un vecteur (couple) $(\a,\b)$ non nul de $D_0$ (une solution non nulle de $ax+by=0$);\\
    --- $M_p$ un couple $(x_p,y_p)$ de $D$ (une solution particulière de $ax+by=c$).\\
    On a
    \begin{equation*}
        D_0=\{(\l\a,\l\b)\mid \l\in\R\}=\{\l\v{u}\mid\l\in\R\}.
    \end{equation*}
    \begin{equation*}
        D=\{(x_p+\l\a,y_p+\l\b)\mid\l\in\R\}=\{M_p\oplus \l\v{u}\mid\l\in\R\}.
    \end{equation*}
    On appelle ces écritures des \bf{représentations paramétriques} de $D_0$ et $D$, le réel $\l$ étant un paramètre. L'addition $\oplus$ est ici celle des couples, coordonnée par coordonnée.
\end{prop}

\begin{ex}{}{}
    Droite d'équation $x-3y=-6$. Représentations paramétriques. Droite vectorielle associée.
    \tcblower
    La droite : $D = \{(x,y)\in\R^2\mid x-3y=-6\}=\{(3y-6,y)\mid y\in\R\}=\{(-6,0) + y(3,1)\mid y\in\R\}$.
\end{ex}

\begin{ex}{Système linéaire $2\times2$ : l'intersection de droites sous-jacentes.}{}
    Soient $(a,b,c)$ et $(a',b',c')$ deux triplets de réels, tels que $(a,b)\neq(0,0)$ et $(a',b')\neq(0,0)$.\\
    On considère le système linéaire ci-dessous :
    \begin{equation*}
        \begin{cases}
            ax&+\quad by\quad= c\\
            a'x&+\quad b'y\quad=c'
        \end{cases}
    \end{equation*}
    En raisonnant en termes d'intersections de droites, discuter de la forme que peut avoir l'ensemble des solutions dans $\R^2$.
    \tcblower
    Notons $S$ l'ensemble des solutions de ce système.\\
    $\bullet$ Les deux droites sécantes, alors $S=(x,y)$.\\
    $\bullet$ Les deux droites parallèles et non confondues, alors $S=\0$.\\
    $\bullet$ Les deux droites confondues, alors $S=D$.
\end{ex}

\begin{ex}{Notre système linéaire $2\times2$ préféré : somme et différence.}{}
    Soient $(x,y)\in\R^2$ et $(a,b)\in\R^2$. On a
    \begin{equation*}
        \begin{cases}
            x+y\quad=\quad a\\
            x-y\quad=\quad b
        \end{cases}
        \iff
        \begin{cases}
            x&=\quad\frac{a+b}{2}\\
            y&=\quad\frac{a-b}{2}
        \end{cases}
    \end{equation*}
\end{ex}

\fbox{\bf{Plans.}}

\begin{defi}{}{}
    Soit $(a,b,c)\in\R^3\setminus\{(0,0,0)\}$ et $d\in\R$.\\
    L'ensemble des triplets $(x,y,z)$ de $\R^3$ qui sont solutions de l'équation linéaire
    \begin{equation*}
        ax+by+cz=d
    \end{equation*}
    est un \bf{plan affine} de $\R^3$.\\
    Le plan d'équation $ax+by+cz=0$ est dit \bf{vectoriel}. Il contient le triplet $(0,0,0)$.
\end{defi}

\begin{prop}{}{}
    Soit $(a,b,c)\in\R^3\setminus\{(0,0,0)\}$ et $d\in\R$. On considère les plans
    \begin{equation*}
        P=\{(x,y,z)\in\R^3\mid ax+by+cz=d\} \quad\et\quad P_0=\{(x,y,z)\in\R^3\mid ax+by+cz=0\}.
    \end{equation*}
    Considérons\\
    --- $\v{u}$ et $\v{v}$ deux vecteurs (triplets) non colinéaires de $P_0$;\\
    --- $M_p$ un triplet de $P$.\\
    On a
    \begin{equation*}
        P_0=\{\l\v{u}+\mu\v{v}\mid\l,\mu\in\R\}\quad\et\quad P=\{M_p+\l\v{u}+\mu\v{v}\mid\l,\mu\in\R\}.
    \end{equation*}
    On appelle ces écritures des \bf{représentations paramétriques} de $P_0$ et $P$, les réels $\l$ et $\mu$ étant des paramètres. L'addition $+$ est ici celle des triplets, coordonnée par coordonnée.
\end{prop}

\begin{ex}{}{}
    Plan d'équation $x-y-z=3$. Représentation paramétrique. Plan vectoriel associé.
    \tcblower
    Le plan $P=\{(x,y,z)\mid x-y-z=3\}=\{(y+z+3, y, z) \mid y,z\in\R\}=\{(3,0,0)+y(1,1,0)+z(1,0,1)\mid y,z\in\R\}$
\end{ex}

\begin{ex}{Système linéaire $2\times3$ : l'intersection de plans sous-jacente.}{}
    Soient $(a,b,c,d)$ et $(a',b',c',d')$ deux quadruplets de réels, tels que $(a,b,c)\neq(0,0,0)$ et $(a',b',c')\neq(0,0,0)$.\\
    On considère le système linéaire ci-dessous :
    \begin{equation*}
        \begin{cases}
            ax&+\quad by\quad+\quad cz\quad= d\\
            a'x&+\quad b'y\quad+\quad c'z\quad=d'
        \end{cases}
    \end{equation*}
    En raisonnant en termes d'intersections de plans, discuter de la forme que peut avoir l'ensemble des solutions dans $\R^3$.
    \tcblower
    Notons $S$ l'ensemble des solutions de ce système.\\
    $\bullet$ Les deux plans sécants, alors $S$ est une droite de $\R^3$.\\
    $\bullet$ Les deux plans parallèles et non confondus, alors $S=\0$.\\
    $\bullet$ Les deux plans confondus, alors $S=P$.
\end{ex}

\section{L'algorithme du pivot \texorpdfstring{$\star$}{Lg}, par l'exemple.}

\begin{ex}{}{}
    Donner l'ensemble des triplets $(x,y,z)\in\R^3$ solutions de
    \begin{equation*}
        \begin{cases}
            x&+\quad y\quad+\quad z\quad= 1\\
            2x&-\quad y\quad+\quad 11z\quad= -1\\
            3x&+\quad 4y\quad+\quad z\quad= 1
        \end{cases}
    \end{equation*}
    \tcblower
    Soit $(x,y,z)\in\R^3$.
    \begin{align*}
        (x,y,z)\quad\nt{est solution} \iff& \systeme{
            x+y+z=1,
            2x-y+11z=-1,
            3x+4y+z=1
        } \iff \systeme{
            x+y+z=1,
            -3y+9z=-3,
            y-2z=-2
        }\\ \iff& \systeme{
            x+y+z=1,
            y-3z=1,
            y-2z=-2
        } \iff \systeme{
            x+y+z=1,
            y-3z=1,
            z=-3
        }\\ \iff& \begin{cases}
            x=12\\
            y=-8\\
            z=-3
        \end{cases}
    \end{align*}
\end{ex}

\begin{ex}{}{}
    Donner l'ensemble des triplets $(x,y,z)\in\R^3$ solutions de
    \begin{equation*}
        \begin{cases}
            2x&+\quad 3y\quad+\quad 7z\quad= 6\\
            x&+\quad y\quad+\quad 2z\quad= 2\\
            3x&+\quad 4y\quad+\quad 9z\quad= 8
        \end{cases}
    \end{equation*}
    \tcblower
    Soit $(x,y,z)\in\R^3$.
    \begin{align*}
        (x,y,z)\quad\nt{est solution} \iff& \systeme{
            x+y+2z=2,
            2x+3y+7z=6,
            3x+4y+9z=8
        } \iff \systeme{
            x+y+2z=2,
            y+3z=2,
            y+3z=2
        }\\ \iff& \begin{cases}
            x=2-2+3z-2z\\
            y=2-3z\\
            0=0
        \end{cases} \iff \begin{cases}
            x=z\\
            y=2-3z
        \end{cases}
    \end{align*}
    L'ensemble des solutions : $\{(x,y,z)\in\R^3\mid x=z\et y=2-3z\}=\{(0,2,0)+z(1,-3,1)\mid z\in\R\}$.
\end{ex}

\begin{ex}{}{}
    Discuter selon les valeurs de $m\in\R$ la compatibilité et les solutions du système suivant.
    \begin{equation*}
        \begin{cases}
            x&+\quad (m+1)y\quad= (m+2)\\
            mx&-\quad (m+4)y\quad= 8
        \end{cases}
    \end{equation*}
    Interpréter en termes d'intersections de droites.
    \tcblower
    Soit $m\in\R$ et $(x,y)\in\R^2$.
    \begin{align*}
        (x,y)\quad\nt{est solution} \iff& \begin{cases}
            x\quad+\quad(m+1)y&=\quad(m+2)\\
            mx~-\quad(m+4)y&=\quad8
        \end{cases}\\
        \iff& \begin{cases}
            x\quad+\quad(m+1)y&=\quad m+2\\
            \qquad~\quad(4-m^2)y&=\quad8-m(m+2)
        \end{cases}
    \end{align*}
    $\bullet$ $m\notin\{2,-2\}$. Alors $4-m^2\neq0$.\\
    Dans ce cas, $x=m+2-(m+1)y$ et $y=\frac{8-m(m+2)}{4-m^2}$, unique couple solution.\\
    $\bullet$ $m=2$. Alors $x=4-3y$, solutions : $\{(4-3y,y)\mid y\in\R\}=\{(4,0)+y(-3,1)\mid y\in\R\}$.\\
    $\bullet$ $m=-2$. Alors $0=8$... pas de solutions.
\end{ex}

\begin{defi}{}{}
    On appelle \bf{opération élémentaire} sur les lignes d'un système l'une des opérations suivantes :
    \begin{enumerate}
        \item Échange des $i$èmes et $j$èmes lignes. On note $L_i\lra L_j$.
        \item Multiplication d'une ligne par un scalaire $\l\neq0$. On note $L_i\gets\l L_i$.
        \item Ajout à la ligne $L_i$ d'une ligne $L_j$ ($i\neq j$) multipliée par un scalaire $\l$. On note $L_i\gets L_i + \l L_j$.
    \end{enumerate}
\end{defi}

\begin{prop}{admise}{}
    Si on passe d'un système à un autre par un nombre fini d'opérations élémentaires, les deux systèmes ont le même ensemble de solutions.
\end{prop}

\begin{defi}{}{}
    Un système linéaire ayant une unique solution est dit de \bf{Cramer}.
\end{defi}

\pagebreak

\section{Exercices.}

\begin{exercice}{$\bww$ Un système de Cramer bête et méchant.}{}
    Résoudre le système suivant dans $\mathbb{R}^3$.
    \begin{equation*}
        \systeme{
            3x + y- 2z = 10,
            2x - y+ z = 3,
            x - y +2z = 2
        }
    \end{equation*}
    \tcblower
    Soit $(x,y,z)\in\mathbb{R}^3$.
    \begin{align*}
        (x,y,z) \text{ est solution }
        \iff&
        \systeme{
            3x + y -2z = 10,
            2x - y + z = 3,
            x - y + 2z = 2
        }\\ \iff&
        \systeme{
            x - y + 2z = 2,
            2x - y + z = 3,
            3x + y - 2z = 10
        }\\ \iff&
        \systeme{
            x-y+2z=2,
            y-3z=-1,
            4y-8z=4
        }\\ \iff&
        \systeme{
            x-y+2z=2,
            y-3z=-1,
            4z=8
        }\\ \iff&
        \begin{cases}
            x=3\\
            y=5\\
            z=2
        \end{cases}
    \end{align*}
    L'unique solution de système dans $\mathbb{R}^3$ est donc $(3,5,2)$.
\end{exercice}

\begin{exercice}{$\bww$}{}
    Résoudre le système suivant dans $\mathbb{R}^3$.
    \begin{equation*}
        \systeme{
            x+2y-z=2,
            x-2y+3z=-2,
            3x-2y+5z=-2
        }
    \end{equation*}
    \tcblower
    Soit $(x,y,z)\in\mathbb{R}^3$.
    \begin{align*}
        (x,y,z) \text{ est solution}
        \iff&
        \systeme{
            x+2y-z=2,
            x-2y+3z=-2,
            3x-2y+5z=-2
        }\\ \iff&
        \systeme{
            x+2y-z=2,
            -4y+4z=-4,
            -8y+8z=-8
        }\\ \iff&
        \systeme{
            x+2y-z=2,
            y - z = 1,
            z = y - 1
        }\\ \iff&
        \begin{cases}
            y=1-x\\
            z=-x
        \end{cases}
    \end{align*}
    L'ensemble $S$ des solutions est alors 
    \begin{equation*}
        S = \left\{(x, 1-x, -x) \hspace{0.2cm} | \hspace{0.2cm} x \in \mathbb{R}\right\} = \left\{(0, 1, 0) + x(1,-1,-1) \hspace{0.2cm} | \hspace{0.2cm} x\in\mathbb{R}\right\}
    \end{equation*}
\end{exercice}

\pagebreak

\begin{exercice}{$\bbw$}{}
    Soit $(a,b,c)\in\mathbb{R}^3$, $a \neq b$, $a \neq c$, $b \neq c$. Résoudre :
    \begin{equation*}
        \begin{cases}
            x+ay+a^2z=a^3\\
            x+by+b^2z=b^3\\
            x+cy+c^2z=c^3
        \end{cases}
    \end{equation*}
    \tcblower
    Soit $(x,y,z)\in\mathbb{R}^3$.
    \begin{align*}
        (x,y,z) \text{ est solution} \iff&
        \begin{cases}
            x+ay+a^2z=a^3\\
            x+by+b^2z=b^3\\
            x+cy+c^2z=c^3
        \end{cases}
        \iff
        \begin{cases}
            x+ay+a^2z=a^3\\
            (b-a)y+(b^2-a^2)z=b^3-a^3\\
            (c-a)y+(c^2-a^2)z=c^3-a^3
        \end{cases}
        \\\iff &
        \begin{cases}
            x+ay+a^2z=a^3\\
            (b-a)y+(b-a)(b+a)z=(b-a)(a^2+ab+b^2)\\
            (c-a)y+(c-a)(c+a)z=(c-a)(a^2+ac+c^2)
        \end{cases}
        \\ \iff&
        \begin{cases}
            x+ay+a^2z=a^3\\
            y+(b+a)z=a^2+ab+b^2\\
            y+(c+a)z=a^2+ac+b^2
        \end{cases}
        \iff
        \begin{cases}
            x+ay+a^2z=a^3\\
            y+(b+a)z=a^2+ab+b^2\\
            z =a + b + c
        \end{cases}
        \\ \iff& 
        \begin{cases}
            x+ay+a^2z=a^3\\
            y=-bc - ab - ac\\
            z = a + b + c
        \end{cases}
        \iff
        \begin{cases}
            x=abc\\
            y=-(ab + bc + ca)\\
            z = a + b + c
        \end{cases}
    \end{align*}
    L'unique solution est donc $(abc, -(ab + bc + ca), a+b+c)$.
\end{exercice}

\begin{exercice}{$\bbb$}{}
    Soit $\lambda$ un paramètre réel et le système :
    \begin{align*}
        \begin{cases}
            (2-\lambda)x + y + z = 0\\
            x + (2-\lambda)y + z = 0\\
            x + y + (2-\lambda)z = 0
        \end{cases}
    \end{align*}
    Le résoudre, en discutant selon les valeurs de $\lambda$.
    \tcblower
    Soit $(x,y,z)\in\mathbb{R}^3$.
    \begin{align*}
        (x,y,z) \text{ est solution}
        &\iff \begin{cases}
            x + y + (2-\lambda)z = 0\\
            x + (2-\lambda)y + z = 0\\
            (2 - \lambda)x + y + z = 0
        \end{cases}\\
        &\iff \begin{cases}
            x+y+(2-\lambda)z =0\\
            (1-\lambda)y + (\lambda-1)z=0\\
            (\lambda-1)y + (1-(2-\lambda)^2)z=0
        \end{cases}\\
        &\iff \begin{cases}
            x + y + (2-\lambda)z = 0\\
            (1-\lambda)y + (\lambda-1)z = 0\\
            (-\lambda^2 + 5\lambda - 4)z = 0
        \end{cases}
    \end{align*}
    Les racines (évidentes) du polynôme $-\lambda^2+5\lambda-4$ sont $1$ et $4$.\\
    $\circledcirc$ Premier cas : $\lambda\notin\{1,4\}$.
    \begin{equation*}
        (x,y,z) \text{ est solution}
        \iff\begin{cases}
            x=0\\
            y=0\\
            z=0
        \end{cases}
    \end{equation*}
    L'unique solution est le couple $(0,0,0)$.\\
    $\circledcirc$ Deuxième cas : $\lambda=1$.
    \begin{equation*}
        (x,y,z) \text{ est solution}
        \iff\begin{cases}
            x+y+z=0
        \end{cases}
    \end{equation*}
    L'ensemble $S$ des solutions est le plan vectoriel: 
    \begin{equation*}
        S=\{(-y-z,y,z)\,|\, (y,z)\in\mathbb{R}^2\}=\{y(-1,1,0) + z(-1,0,1)\,|\,(y,z)\in\mathbb{R}^2\}
    \end{equation*}
    $\circledcirc$ Dernier cas : $\lambda=4$
    \begin{equation*}
        (x,y,z) \text{ est solution} \iff \begin{cases}
            x+z-2z=0\\
            -3y+3z=0\\
            0=0
        \end{cases}
        \iff\begin{cases}
            x=z\\
            y=z\\
            0=0
        \end{cases}
    \end{equation*}
    L'ensemble S des solutions est la droite passant par l'origine:\\
    $. \hspace{4.5cm} S = \{(z,z,z)\,|\,z\in\mathbb{R}\} = \{(0,0,0) + z(1,1,1)\ \,|\, z\in\mathbb{R}\}$\\
\end{exercice}

\end{document}