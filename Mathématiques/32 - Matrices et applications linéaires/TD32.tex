\documentclass[11pt]{article}

\usepackage[paperheight=15in, left=2cm, right=2cm, top=2cm, bottom=2cm]{geometry}
\usepackage[most]{tcolorbox}
\usepackage{amsmath, amssymb, amsthm, enumitem, stmaryrd, cancel, pifont, dsfont, hyperref, fancyhdr, lastpage, tocloft, changepage}

\def\pagetitle{Matrices et applications linéaires}
\setlength{\headheight}{14pt}

\title{\bf{\pagetitle}\\\large{Corrigé}}

\hypersetup{
    colorlinks=true,
    citecolor=black,
    linktoc=all,
    linkcolor=blue
}

\pagestyle{fancy}
\cfoot{\thepage\ sur \pageref*{LastPage}}

\begin{document}

\newcommand{\providetcbcountername}[1]{%
  \@ifundefined{c@tcb@cnt@#1}{%
    --undefined--%
  }{%
    tcb@cnt@#1%
  }
}

\newcommand{\settcbcounter}[2]{%
  \@ifundefined{c@tcb@cnt@#1}{%
    \GenericError{Error}{counter name #1 is no tcb counter }{}{}%
  }{%
    \setcounter{tcb@cnt@#1}{#2}%
   }%
}%

\newcommand{\displaytcbcounter}[1]{% Wrapper for \the...
  \@ifundefined{thetcb@cnt@#1}{%
    \GenericError{Error}{counter name #1 is no tcb counter }{}{}%
  }{%
    \csname thetcb@cnt@#1\endcsname% 
  }%
}

% MATHS %
\newtcbtheorem{thm}{Théorème}
{
    enhanced,frame empty,interior empty,
    colframe=red,
    after skip = 1cm,
    borderline west={1pt}{0pt}{green!25!red},
    borderline south={1pt}{0pt}{green!25!red},
    left=0.2cm,
    attach boxed title to top left={yshift=-2mm,xshift=-2mm},
    coltitle=black,
    fonttitle=\bfseries,
    colbacktitle=white,
    boxed title style={boxrule=.4pt,sharp corners},
    before lower = {\textbf{Preuve :}\n}
}{thm}

\newtcbtheorem[use counter from = thm]{defi}{Définition}
{
    enhanced,frame empty,interior empty,
    colframe=green,
    after skip = 1cm,
    borderline west={1pt}{0pt}{green},
    borderline south={1pt}{0pt}{green},
    left=0.2cm,
    attach boxed title to top left={yshift=-2mm,xshift=-2mm},
    coltitle=black,
    fonttitle=\bfseries,
    colbacktitle=white,
    boxed title style={boxrule=.4pt,sharp corners},
    before lower = {\textbf{Preuve :}\n}
}{defi}

\newtcbtheorem[use counter from = thm]{prop}{Proposition}
{
    enhanced,frame empty,interior empty,
    colframe=blue,
    after skip = 1cm,
    borderline west={1pt}{0pt}{green!25!blue},
    borderline south={1pt}{0pt}{green!25!blue},
    left=0.2cm,
    attach boxed title to top left={yshift=-2mm,xshift=-2mm},
    coltitle=black,
    fonttitle=\bfseries,
    colbacktitle=white,
    boxed title style={boxrule=.4pt,sharp corners},
    before lower = {\textbf{Preuve :}\n}
}{prop}

\newtcbtheorem[use counter from = thm]{corr}{Corrolaire}
{
    enhanced,frame empty,interior empty,
    colframe=blue,
    after skip = 1cm,
    borderline west={1pt}{0pt}{green!25!blue},
    borderline south={1pt}{0pt}{green!25!blue},
    left=0.2cm,
    attach boxed title to top left={yshift=-2mm,xshift=-2mm},
    coltitle=black,
    fonttitle=\bfseries,
    colbacktitle=white,
    boxed title style={boxrule=.4pt,sharp corners},
    before lower = {\textbf{Preuve :}\n}
}{corr}

\newtcbtheorem[use counter from = thm]{lem}{Lemme}
{
    enhanced,frame empty,interior empty,
    colframe=blue,
    after skip = 1cm,
    borderline west={1pt}{0pt}{green!25!blue},
    borderline south={1pt}{0pt}{green!25!blue},
    left=0.2cm,
    attach boxed title to top left={yshift=-2mm,xshift=-2mm},
    coltitle=black,
    fonttitle=\bfseries,
    colbacktitle=white,
    boxed title style={boxrule=.4pt,sharp corners},
    before lower = {\textbf{Preuve :}\n}
}{lem}

\newtcbtheorem[use counter from = thm]{ex}{Exemple}
{
    enhanced,frame empty,interior empty,
    colframe=orange,
    after skip = 1cm,
    borderline west={1pt}{0pt}{green!25!orange},
    borderline south={1pt}{0pt}{green!25!orange},
    left=0.2cm,
    attach boxed title to top left={yshift=-2mm,xshift=-2mm},
    coltitle=black,
    fonttitle=\bfseries,
    colbacktitle=white,
    boxed title style={boxrule=.4pt,sharp corners},
    before lower = {\textbf{Preuve :}\n}
}{ex}

\newtcbtheorem[use counter from = thm]{meth}{Méthode}
{
    enhanced,frame empty,interior empty,
    colframe=purple,
    after skip = 1cm,
    borderline west={1pt}{0pt}{purple},
    borderline south={1pt}{0pt}{purple},
    left=0.2cm,
    attach boxed title to top left={yshift=-2mm,xshift=-2mm},
    coltitle=black,
    fonttitle=\bfseries,
    colbacktitle=white,
    boxed title style={boxrule=.4pt,sharp corners},
    before lower = {\textbf{Preuve :}\n}
}{meth}

\newtcbtheorem[use counter from = thm]{exercice}{Exercice}
{
    enhanced,frame empty,interior empty,
    colframe=blue,
    after skip = 1cm,
    borderline west={1pt}{0pt}{green!25!blue},
    borderline south={1pt}{0pt}{green!25!blue},
    left=0.2cm,
    attach boxed title to top left={yshift=-2mm,xshift=-2mm},
    coltitle=black,
    fonttitle=\bfseries,
    colbacktitle=white,
    boxed title style={boxrule=.4pt,sharp corners},
    before lower = {\textbf{Preuve :}\n}
}{exercice}

% PHYSIQUE %
\newtcbtheorem[use counter from = thm]{qc}{Question de Cours}
{
    enhanced,frame empty,interior empty,
    colframe=red,
    after skip = 1cm,
    borderline west={1pt}{0pt}{green!25!red},
    borderline south={1pt}{0pt}{green!25!red},
    left=0.2cm,
    attach boxed title to top left={yshift=-2mm,xshift=-2mm},
    coltitle=black,
    fonttitle=\bfseries,
    colbacktitle=white,
    boxed title style={boxrule=.4pt,sharp corners},
    before lower = {\textbf{Preuve :}\n}
}{qc}
\newtcbtheorem[use counter from = thm]{app}{Application}
{
    enhanced,frame empty,interior empty,
    colframe=blue,
    after skip = 1cm,
    borderline west={1pt}{0pt}{green!25!blue},
    borderline south={1pt}{0pt}{green!25!blue},
    left=0.2cm,
    attach boxed title to top left={yshift=-2mm,xshift=-2mm},
    coltitle=black,
    fonttitle=\bfseries,
    colbacktitle=white,
    boxed title style={boxrule=.4pt,sharp corners},
    before lower = {\textbf{Preuve :}\n}
}{app}
% MATHS %
\newcommand*{\K}{\mathbb{K}}
\newcommand*{\C}{\mathbb{C}}
\newcommand*{\R}{\mathbb{R}}
\newcommand*{\Q}{\mathbb{Q}}
\newcommand*{\Z}{\mathbb{Z}}
\newcommand*{\N}{\mathbb{N}}
\newcommand*{\F}{\mathcal{F}}

\newcommand{\0}{\varnothing}
\newcommand*{\e}{\varepsilon}
\newcommand*{\g}{\gamma}
\newcommand*{\s}{\sigma}

\newcommand*{\ra}{\Longrightarrow}
\newcommand*{\la}{\Longleftarrow}
\newcommand*{\rla}{\Longleftrightarrow}
\newcommand*{\lb}{\llbracket}
\newcommand*{\rb}{\rrbracket}
\newcommand*{\n}{\\[0.2cm]}

\newcommand*{\cmark}{\ding{51}}
\newcommand*{\xmark}{\ding{55}}

\newcommand{\rg}[1]{\textrm{rg}(#1)}
\newcommand{\vect}[1]{\textrm{Vect}(#1)}
\newcommand{\tr}[1]{\textrm{Tr}(#1)}

\renewcommand{\dim}[1]{\textrm{dim}~#1}
\renewcommand*{\ker}[1]{\textrm{Ker}(#1)}
\renewcommand{\Im}[1]{\textrm{Im}(#1)}

\renewcommand*{\t}{\tau}
\renewcommand*{\phi}{\varphi}

% PHYSIQUE %
\newcommand{\base}[1]{\overrightarrow{e_{\text{#1}}}}

\renewcommand{\cos}[1]{\text{cos}(#1)}
\renewcommand{\sin}[1]{\text{sin}(#1)}
\renewcommand*{\Vec}[1]{\overrightarrow{\text{#1}}}

\thispagestyle{fancy}
\fancyhead[L]{MP2I Paul Valéry}
\fancyhead[C]{\pagetitle}
\fancyhead[R]{2023-2024}

\hrule
\begin{center}
    \LARGE{\textbf{Chapitre 32}}\\
    \large{\pagetitle}\\
    \rule{0.8\textwidth}{0.5pt}
\end{center}


\vspace{0.5cm}

\begin{exercise}{$\blacklozenge\lozenge\lozenge$}{}
    Soit $u \in \mathcal{L}(\R^{3})$, tel que $u^{2} = 0$ et $u \neq 0$.
    \begin{enumerate}[topsep=0pt,itemsep=-0.9 ex]
        \item Comparer $\ker{u}$ et $\Im{u}$ puis donner leurs dimensions.
        \item Montrer qu'il existe une base de $\R^{3}$ dans laquelle la matrice de u est 
        $\begin{pmatrix}
            0 & 0 & 1 \\
            0 & 0 & 0 \\
            0 & 0 & 0
        \end{pmatrix}$
    \end{enumerate}
    \tcblower\\[0.2cm]
    \boxed{1} On sait que $u \neq 0$ donc $\rg{u} \geq 1$ \\
    D'autres part, on a : $\Im{u} \subset \ker{u}$ ($u^{2} = 0$) \\
    Ainsi on obtient : $\rg{u} \leq \dim{\ker{u}}$ \\
    D'après le théorème du rang, on obtient : $\rg{u} \leq \dim{\R^{3}} -\rg{u}$ \\\\
    $2\rg{u} \leq 3$\\
    $\rg{u} \leq \frac{3}{2}$\\
    Ainsi on a $1 \leq \rg{u} \leq \frac{3}{2}$ et $\rg{u} \in \N$\\
    On en deduis que $\rg{u} = 1$ et avec le théorème du rang que : $\dim{\ker{u}} = 2$\\\\
    \boxed{2} On note ($e_{1}$) une base de $\Im{u}$\\
    On l'a complete en ($e_{1}$, $e_{2}$) afin d'obtenir une base de $\ker{u}$ ($e_{1} \in \ker{u}$ car $\Im{u} \subset \ker{u}$)\\
    Posons $e_{3}$ tq $e_{1} = u{(e_{3})}$ ($e_{1} \in \Im{u}$)\\
    Montrons que ($e_{1}$, $e_{2}$, $e_{3}$) une famille libre de $\R^{3}$ : \\\\
    Soit ($\lambda$, $\beta$, $\gamma$) $\in \R^{3}$\\
    Supposons $\lambda e_{1} + \beta e_{2} + \gamma e_{3} = 0$\\
    $u{(\lambda e_{1} + \beta e_{2} + \gamma e_{3})} = 0$ \\
    $\lambda u{(e_{1})} + \beta u{(e_{2})} + \gamma u{(e_{3})} = 0$ \\
    $\gamma e_{1} = 0$ or $e_{1} \neq 0$ donc $\gamma = 0$\\\\
    Ainsi on a : $\lambda e_{1} + \beta e_{2} = 0$ or il s'agit d'une base de $\ker{u}$\\
    En particulier d'une famille libre de $\ker{u}$ donc aussi d'une famille libre de $\R^{3}$\\\\
    On en deduis que : $\lambda = 0$, $\beta = 0$, $\gamma = 0$\\\\
    $\left\{
        \begin{array}{ll}
            (e_{1}, e_{2}, e_{3})~est~une~famille~libre~de~\R^{3} \\
            \dim{\R^{3}} = 3
        \end{array}
    \right.$\\\\
    Par caractérisation des bases en dimensions finis :\\
    ($e_{1}$, $e_{2}$, $e_{3}$) est une base de $\R^{3}$\\\\
    Ainsi pour finir : $Mat_{(e_{1}, e_{2}, e_{3})}(u) = $ 
    $\begin{pmatrix}
            0 & 0 & 1 \\
            0 & 0 & 0 \\
            \underset{u{(e_{1})}}{0} & \underset{u{(e_{2})}}{0} & \underset{u{(e_{3})}}{0}
    \end{pmatrix}$\\
\end{exercise}

\end{document}
