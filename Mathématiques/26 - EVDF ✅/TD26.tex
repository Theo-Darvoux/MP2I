\documentclass[11pt]{article}

\usepackage[paperheight=63in, left=2cm, right=2cm, top=2cm, bottom=2cm]{geometry}
\usepackage[most]{tcolorbox}
\usepackage{hyperref, fancyhdr, lastpage, tocloft, changepage}
\usepackage{enumitem}
\usepackage{amsmath, amssymb, amsthm, stmaryrd}

\def\pagetitle{Espaces Vectoriels en dimension finie}
\setlength{\headheight}{14pt}
\newcommand*{\F}{\mathcal{F}}
\newcommand*{\R}{\mathbb{R}}
\newcommand*{\C}{\mathbb{C}}
\newcommand*{\K}{\mathbb{K}}
\newcommand*{\N}{\mathbb{N}}
\newcommand*{\m}{\mathcal}

\renewcommand*{\phi}{\varphi}
\renewcommand*{\ker}{\textrm{Ker}}

\DeclareMathOperator*{\vect}{Vect}
\DeclareMathOperator*{\tr}{Tr}
\DeclareMathOperator*{\rg}{rg}

\title{\bf{\pagetitle}\\\large{Corrigé}}

\newtcbtheorem{exercise}{Exercice}
{
    enhanced,frame empty,interior empty,
    colframe=blue,
    after skip = 1cm,
    borderline west={1pt}{0pt}{green!25!blue},
    borderline south={1pt}{0pt}{green!25!blue},
    left=0.2cm,
    attach boxed title to top left={yshift=-2mm,xshift=-2mm},
    coltitle=black,
    fonttitle=\bfseries,
    colbacktitle=white,
    boxed title style={boxrule=.4pt,sharp corners},
    before lower = {\textbf{Solution :}}
}{exercise}

\hypersetup{
    colorlinks=true,
    citecolor=black,
    linktoc=all,
    linkcolor=blue
}

\pagestyle{fancy}
\cfoot{\thepage\ sur \pageref*{LastPage}}

\begin{document}

\thispagestyle{fancy}
\fancyhead[L]{MP2I Paul Valéry}
\fancyhead[C]{\pagetitle}
\fancyhead[R]{2023-2024}

\hrule
\begin{center}
    \LARGE{\textbf{Chapitre 26}}\\
    \large{Espaces de dimension finie}\\
    \small{Marathon du lundi de paques}
    \rule{0.8\textwidth}{0.5pt}
\end{center}


\vspace{0.5cm}

\begin{exercise}{$\blacklozenge\lozenge\lozenge$}{}
    Soit $F=\{M \in M_2(\R) : \tr(M) = 0\}$.\\
    Montrer que $F$ est un s.e.v. de $M_2(\R)$ et calculer sa dimension.
    \tcblower\\[0.2cm]
    La trace est une forme linéaire sur $M_2(\R)$, donc $F = \ker(\tr)$ est un s.e.v. de $M_2(\R)$.\\
    D'après le théorème du rang, on a $\dim(M_2(\R)) = \dim(\ker(\tr)) + \dim(\tr(M_2(\R)))$.\\
    Ainsi, $\dim(\ker(\tr))=\dim(F)=\dim(M_2(\R))-\dim(\R)=3$.
\end{exercise}

\begin{exercise}{$\blacklozenge\lozenge\lozenge$}{}
    Montrer que $(M_1, M_2, M_3, M_4)$ est une base de $M_2(\R)$ avec :
    \begin{equation*}
        M_1 = I_2, ~ M_2 = \begin{pmatrix} 1 & 0 \\ 0 & 0 \end{pmatrix}, ~ M_3 = \begin{pmatrix} 6 & 6 \\ 6 & 0 \end{pmatrix}, ~ M_4 = \begin{pmatrix} 1 & 2 \\ 3 & 4 \end{pmatrix}.
    \end{equation*}
    \tcblower\\[0.2cm]
    Montrons que c'est une famille libre.\\
    Soient $\lambda_1, \lambda_2, \lambda_3, \lambda_4 \in \R$ tels que : $\lambda_1I_2 + \lambda_2M_2 + \lambda_3M_3 + \lambda_4M_4 = 0$.
    Alors :
    \begin{equation*}
        \begin{cases}
            \lambda_1 + \lambda_2 + 6\lambda_3 + \lambda_4 = 0\\
            6\lambda_3 + 2\lambda_4 = 0\\
            6\lambda_3 + 3\lambda_4 = 0\\
            \lambda_1 + 4\lambda_4 = 0
        \end{cases}
        \iff
        \begin{cases}
            \lambda_1 = 0\\
            \lambda_2 = 0\\
            \lambda_3 = 0\\
            \lambda_4 = 0
        \end{cases}
    \end{equation*}
    En résolvant le systeme. Ainsi, $(M_1, M_2, M_3, M_4)$ est une famille libre.\\
    Or $\dim(M_2(\R))=4$, et c'est une famille libre de 4 vecteurs : c'est une base.
\end{exercise}

\begin{exercise}{$\blacklozenge\blacklozenge\lozenge$}{}
    Pour $k\in\llbracket0,n\rrbracket$, on pose $P_k = X^k(1-X)^{n-k}$. Montrer que $(P_0, ..., P_n)$ est base de $\K_n[X]$.
    \tcblower\\[0.2cm]
    On sait déjà que c'est une famille libre (cf 25.13).\\
    C'est une famille libre de $n+1$ vecteurs dans un espace de dimension $n+1$, donc c'est une base.
\end{exercise}

\begin{exercise}{$\blacklozenge\blacklozenge\lozenge$}{}
    Soient $\m{B}=(e_1, ..., e_n)$ et $\m{B}'=(e'_1,...,e'_n)$ deux bases de $E$, $\K$-ev de dimension finie.\\
    Montrer qu'il existe $j\in\llbracket1,n\rrbracket$ tel que $(e_1, ..., e_{n-1}, e'_j)$ est une base de $E$.
    \tcblower\\[0.2cm]
    On sait que $(e_1, ..., e_{n-1})$ est une famille libre de $E$.\\
    Par théorème de la base incomplète, on peut compléter cette famille libre en une base de $E$.\\
    Supposons qu'il n'existe pas de $j$ tel que $(e_1, ..., e_{n-1}, e'_j)$ est une base de $E$.\\
    Alors, pour tout $j$, $(e_1, ..., e_{n-1}, e'_j)$ est liée.\\
    Donc, pour tout $j$, $e'_j$ est combinaison linéaire de $(e_1, ..., e_{n-1})$.\\
    Donc $\m{B}'$ est combinaison linéaire de $\m{B}$, ce qui est absurde.\\
    Donc il existe un $j$ tel que $(e_1, ..., e_{n-1}, e'_j)$ est une base de $E$.
\end{exercise}

\begin{exercise}{$\blacklozenge\lozenge\lozenge$}{}
    Justifier que $\C$ est un $\C$-ev de dimension 1 et un $\R$-ev de dimension 2.
    \tcblower\\[0.2cm]
    $\C$ est un $\C$-ev de dimension 1 car $\forall z \in \C, z = z \cdot 1$.\\
    $\C$ est un $\R$-ev de dimension 2 car $\forall z \in \C, z = \Re(z) \cdot 1 + \Im(z) \cdot i$ avec $\Re(z), \Im(z) \in \R$.
\end{exercise}

\begin{exercise}{$\blacklozenge\blacklozenge\lozenge$}{}
    Soient $n\in\N^*$ et $(\lambda_k)_{0\leq k\leq n}\in\K^{n+1}$ tels que $\sum_{k=0}^n\lambda_k(X + k)^n=0$.
    \begin{enumerate}[topsep=0pt,itemsep=-0.9 ex]
        \item Montrer que $\forall p \in \llbracket 0, n \rrbracket, \sum_{k=0}^n \lambda_k(X+k)^p=0$.
        \item Montrer que $\forall p \in \llbracket 0, n \rrbracket, \sum_{k=0}^n \lambda_kk^p = 0$.
        \item Montrer que $\forall P \in \K_n[X], \sum_{k=0}^n\lambda_k P(k) = 0$.
        \item Déduire que $\left( (X + k)^n, k \in \llbracket 0, n \rrbracket \right)$ est une base de $\K_n[X]$.
    \end{enumerate}
    \tcblower\\[0.2cm]
    On pose $P = \sum_{k=0}^n \lambda_k(X+k)^n = 0$.\\
    \boxed{1.} On a $P' = \sum_{k=0}^n \lambda_k n(X+k)^{n-1} = n\sum_{k=0}^n \lambda_k(X+k)^{n-1} = 0$.\\
    Donc $\sum_{k=0}^n \lambda_k(X+k)^{n-1} = 0$.\\
    En dérivant $n$ fois, on obtient bien l'égalité pour tout $p\in\llbracket 0, n\rrbracket$.\\[0.2cm]
    \boxed{2.} En évaluant en $0$ l'égalité du $\boxed{1.}$, on obtient bien l'égalité.\\[0.2cm]    
    \boxed{3.} Soit $P\in\K_n[X]$. On a $P = \sum_{p=0}^n a_pX^p$.\\
    On a $\sum_{k=0}^n \lambda_k P(k) = \sum_{k=0}^n \lambda_k \sum_{p=0}^n a_pk^p = \sum_{p=0}^n a_p\sum_{k=0}^n \lambda_kk^p = 0$.\\[0.2cm]
    \boxed{4.} On a montré que $\forall P\in\K_n[X], \sum_{k=0}^n \lambda_kP(k) = 0$.\\
    Donc, en particulier pour un polynôme ne s'annulant jamais, on a que les $\lambda_k$ sont nuls.\\
    Donc $\left( (X + k)^n, k \in \llbracket 0, n \rrbracket \right)$ est une famille libre de $\K_n[X]$.\\
    Or, c'est une famille de $n+1$ vecteurs dans un espace de dimension $n+1$, donc c'est une base.
\end{exercise}

\begin{exercise}{$\blacklozenge\blacklozenge\lozenge$}{}
    \begin{enumerate}[topsep=0pt,itemsep=-0.9 ex]
        \item Pour $a \in \R$, on note $f_a : x \mapsto e^{ax}$. Montrer que $(f_a)_{a\in\R}$ est libre dans $\R^\R$.
        \item Déduire que $\R^\R$ n'est pas de dimension finie.
    \end{enumerate}
    \tcblower\\[0.2cm]
    \boxed{1.} Soit $n \in \N^*$ et $a_1 < ... < a_n \in \R$.\\
    Soient $(\lambda_1, ... \lambda_n) \in \R ~ | ~ \sum_{k=1}^n\lambda_k f_{a_k} = 0$.\\
    Alors $\sum_{k=1}^{n-1}\lambda_k f_{a_k} = - \lambda_n f_{a_n}$ et $\sum_{k=1}^{n-1}\lambda_k f_{a_k - a_n} = - \lambda_n$.\\
    Or $\sum_{k=1}^{n-1}\lambda_k f_{a_k - a_n}(x) \xrightarrow[x\to+\infty]{} 0$ donc $\lambda_n = 0$.\\
    En itérant, on obtient que $\lambda_1 = ... = \lambda_n = 0$.\\
    Donc $(f_a)_{a\in\R}$ est une famille libre de $\R^\R$.\\[0.2cm]
    \boxed{2.} Supposons que $\R^\R$ est de dimension finie.\\
    Alors, toute famille libre de $\R^\R$ est de cardinal inférieur ou égal à la dimension de $\R^\R$.\\
    Or, on a montré que $(f_a)_{a\in\R}$ est une famille libre de $\R^\R$ de cardinal infini.\\
    Donc $\R^\R$ n'est pas de dimension finie.
\end{exercise}

\begin{exercise}{$\blacklozenge\blacklozenge\lozenge$}{}
    \begin{enumerate}[topsep=0pt,itemsep=-0.9 ex]
        \item Soit $M \in M_n(\K)$. Justifier l'existence d'un entier $p$ tel que $(I_n, M, M^2, ..., M^p)$ est liée.
        \item Montrer que l'inverse d'une matrice triangulaire supérieure est triangulaire supérieure.
    \end{enumerate}
    \tcblower\\[0.2cm]
    \boxed{1.} L'espace $M_n(\K)$ est de dimension $n^2$, donc toute famille de $n^2+1$ vecteurs est liée.\\
    En particulier, $(I_n, M, M^2, ..., M^{n^2})$ l'est.\\[0.2cm]
    \boxed{2.} Soit $M \in M_n(\K)$ triangulaire supérieure inversible d'inverse $M^{-1}$.\\
    On a que les itérés de $M$ sont triangulaires supérieures.\\
    Soit $p$ tel que $(I_n, M, ..., M^p)$ est liée.\\
    Alors, il existe $\lambda_1, ..., \lambda_p \in \K$ tels que $\sum_{k=1}^p\lambda_k M^k = I_n$.\\
    On multiplie par $M^{-1}$ : $\sum_{k=1}^p\lambda_k M^{k-1} = M^{-1}$.\\
    Ainsi, $M^{-1}$ est combinaison linéaire de MTS, donc est triangulaire supérieure.
\end{exercise}

\begin{exercise}{$\blacklozenge\lozenge\lozenge$}{}
    Sans la formule de Grassmann :
    \begin{enumerate}[topsep=0pt,itemsep=-0.9 ex]
        \item Soient deux plans vectoriels non confondus d'un espace $E$ de dimension 3. Montrer que leur intersection est une droite vectorielle.
        \item Donner un exemple en dimension 4 de deux plans vectoriels supplémentaires.
    \end{enumerate}
    \tcblower\\[0.2cm]
    \boxed{1.} On note $P_1$ et $P_2$ ces deux plans.\\
    Alors $\exists (e_1, e_2) \in E^2 ~ | ~ P_1 = \vect(e_1, e_2)$ et $\exists (e_3, e_4) \in E^2 ~ | ~ P_2 = \vect(e_3, e_4)$.\\
    On suppose ces familles libres. Puisque $E$ est de dimension 3, alors $e_3 \in P_1$ ou $e_4 \in P_1$.\\
    Ainsi, $P_1 \cap P_2 \neq \{0\}$ car $e_3 \neq 0$ et $0<\dim(P_1 \cap P_2)<2$ comme l'intersection est de dimension strictement inférieure à $P_1$ et $P_2$, puisque $P_1 \neq P_2$.\\
    On a donc que $P_1 \cap P_2$ est une droite vectorielle.\\[0.2cm]
    \boxed{2.} On peut prendre $P_1 = \vect(e_1, e_2)$ et $P_2 = \vect(e_3, e_4)$ avec $(e_1, e_2, e_3, e_4)$ base de $\C^4$.
\end{exercise}

\begin{exercise}{$\blacklozenge\blacklozenge\lozenge$}{}
    Soit $E$ un espace vectoriel de dimension égale à $n\in\N$ et $H_1,H_2$ deux hyperplans de $E$ non confondus.\\
    Calculer $\dim(H_1 \cap H_2)$.
    \tcblower\\[0.2cm]
    On a $\dim(H_1) = \dim(H_2) = n-1$.\\
    On a $H_1 \subset H_1 + H_2 \subset E$ donc $n-1 \leq \dim(H_1 + H_2) \leq n$.\\
    Or $H_1 \neq H_2$ donc $\exists x \in H_1 + H_2 ~ | ~ x \notin H_1$. Alors $\dim(H_1 + H_2) > \dim(H_1)$.\\
    Ainsi, $\dim(H_1 + H_2) = n$.\\
    On a $\dim(H_1 \cap H_2) = \dim(H_1) + \dim(H_2) - \dim(H_1 + H_2) = n - 1 + n - 1 - n = n - 2$.
\end{exercise}

\begin{exercise}{$\blacklozenge\blacklozenge\lozenge$}{}
    Calculer $\dim S_n(\R)$. En déduire $\dim A_n(\R)$.
    \tcblower\\[0.2cm]
    On a $\dim S_n(\R) = \frac{n(n+1)}{2}$ car c'est le nombre de coefficients au dessus/dessous de la diagonale.\\
    On a $M_n(\R) = S_n(\R) \oplus A_n(\R)$ donc $\dim A_n(\R) = \dim M_n(\R) - \dim S_n(\R) = n^2 - \frac{n(n+1)}{2} = \frac{n(n-1)}{2}$.
\end{exercise}

\begin{exercise}{$\blacklozenge\blacklozenge\lozenge$}{}
    Soit $F = \{P \in \R_3[X] : P(1) = P(2)\}$.
    \begin{enumerate}[topsep=0pt,itemsep=-0.9 ex]
        \item Prouver que $F$ est un sous-espace vectoriel de $\R_3[X]$ et justifier que $\dim F \leq 3$.
        \item Trouver une base de $F$.
    \end{enumerate}
    \tcblower\\[0.2cm]
    \boxed{1.} On a $F = \{P \in \R_3[X] : P(1) - P(2) = 0\} = \ker(\phi)$ avec $\phi : P \mapsto P(1) - P(2)$.\\
    Or $\phi$ est une forme linéaire sur $\R_3[X]$, donc $F = \ker(\phi)$ est un s.e.v. de $\R_3[X]$.\\
    D'après le théorème du rang, on a $\dim(\R_3[X]) = \rg(\phi) + \dim(\ker(\phi))$.\\
    Donc $\dim F = 4 - 1 = 3$.\\[0.2cm]
    \boxed{2.} On peut prendre $P_1 = 1$, $P_2 = X^2 - 3X + 2$ et $P_3 = X^3 - 3X^2 + 2X$.
\end{exercise}

\begin{exercise}{$\blacklozenge\blacklozenge\lozenge$}{}
    Soit $P \in \R_{n+1}[X]$. Montrer que $\R_{n+1}[X] = \R_n[X] \oplus \vect(P)$.
    \tcblower\\[0.2cm]
    On a que $\R_n[X]$ est un hyperplan de $\R_{n+1}[X]$ et $\vect(P)$ est une droite vectorielle de $\R_{n+1}[X]$.\\
    D'après le chapitre suivant, $\R_{n+1}[X] = \R_n[X] \oplus \vect(P)$.
\end{exercise}

\begin{exercise}{$\blacklozenge\blacklozenge\lozenge$}{}
    Quelle est la condition nécessaire et suffisante sur $\lambda$ pour que $D = \vect((\lambda, \lambda, 1))$ et $P = \vect((1, \lambda, 1), (2, 1, 1))$ soient supplémentaires dans $\R^3$ ?
    \tcblower\\[0.2cm]
    On a que $D$ est une droite vectorielle de $\R^3$ et $P$ est un hyperplan de $\R^3$.\\
    La condition nécessaire et suffisante pour que $D$ et $P$ soient supplémentaires est que $D \cap P = \{0\}$.
\end{exercise}

\begin{exercise}{$\blacklozenge\blacklozenge\blacklozenge$}{}
    Soient $F,G$ deux s.e.v. d'un espace vectoriel $E$ de dimension finie.\\
    Montrer que $(\dim F + G)^2 + (\dim F \cap G)^2 \geq (\dim F)^2 + (\dim G)^2$.
    \tcblower\\[0.2cm]
    On a $\dim F+G = \dim F + \dim G - \dim F \cap G$.\\
    Donc $\dim F+G + \dim F \cap G = \dim F + \dim G$.\\
    Donc $(\dim F+G)^2 + (\dim F \cap G)^2 + 2\dim F + G \dim F \cap G = (\dim F)^2 + (\dim G)^2 + 2\dim F \dim G$.\\
    Montrons que $\dim(F+G)\dim(F \cap G) \geq \dim F \dim G$.\\
    Si $F$ et $G$ sont confondus, il y a égalité.\\
    SPDG, supposons que $\dim F \geq \dim G$.\\
    Alors $\dim(F + G) \geq \dim F + 1$ et $\dim F \cap G \geq \dim G - 1$.\\
    Donc $\dim(F + G)\dim(F \cap G) \geq \dim F \dim G - \dim F + \dim G - 1 \geq \dim F \dim G$.\\
    En remplaçant dans l'égalité, on obtient l'inégalité.
\end{exercise}

\end{document}
 