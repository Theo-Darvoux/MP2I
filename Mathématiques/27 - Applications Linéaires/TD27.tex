\documentclass[11pt]{article}

\usepackage[paperheight=15in, left=2cm, right=2cm, top=2cm, bottom=2cm]{geometry}
\usepackage[most]{tcolorbox}
\usepackage{amsmath, amssymb, amsthm, enumitem, stmaryrd, cancel, pifont, dsfont, hyperref, fancyhdr, lastpage, tocloft, changepage}

\def\pagetitle{Applications linéaires}
\setlength{\headheight}{14pt}

\title{\bf{\pagetitle}\\\large{Corrigé}}

\hypersetup{
    colorlinks=true,
    citecolor=black,
    linktoc=all,
    linkcolor=blue
}

\pagestyle{fancy}
\cfoot{\thepage\ sur \pageref*{LastPage}}

\begin{document}

\newcommand{\providetcbcountername}[1]{%
  \@ifundefined{c@tcb@cnt@#1}{%
    --undefined--%
  }{%
    tcb@cnt@#1%
  }
}

\newcommand{\settcbcounter}[2]{%
  \@ifundefined{c@tcb@cnt@#1}{%
    \GenericError{Error}{counter name #1 is no tcb counter }{}{}%
  }{%
    \setcounter{tcb@cnt@#1}{#2}%
   }%
}%

\newcommand{\displaytcbcounter}[1]{% Wrapper for \the...
  \@ifundefined{thetcb@cnt@#1}{%
    \GenericError{Error}{counter name #1 is no tcb counter }{}{}%
  }{%
    \csname thetcb@cnt@#1\endcsname% 
  }%
}

% MATHS %
\newtcbtheorem{thm}{Théorème}
{
    enhanced,frame empty,interior empty,
    colframe=red,
    after skip = 1cm,
    borderline west={1pt}{0pt}{green!25!red},
    borderline south={1pt}{0pt}{green!25!red},
    left=0.2cm,
    attach boxed title to top left={yshift=-2mm,xshift=-2mm},
    coltitle=black,
    fonttitle=\bfseries,
    colbacktitle=white,
    boxed title style={boxrule=.4pt,sharp corners},
    before lower = {\textbf{Preuve :}\n}
}{thm}

\newtcbtheorem[use counter from = thm]{defi}{Définition}
{
    enhanced,frame empty,interior empty,
    colframe=green,
    after skip = 1cm,
    borderline west={1pt}{0pt}{green},
    borderline south={1pt}{0pt}{green},
    left=0.2cm,
    attach boxed title to top left={yshift=-2mm,xshift=-2mm},
    coltitle=black,
    fonttitle=\bfseries,
    colbacktitle=white,
    boxed title style={boxrule=.4pt,sharp corners},
    before lower = {\textbf{Preuve :}\n}
}{defi}

\newtcbtheorem[use counter from = thm]{prop}{Proposition}
{
    enhanced,frame empty,interior empty,
    colframe=blue,
    after skip = 1cm,
    borderline west={1pt}{0pt}{green!25!blue},
    borderline south={1pt}{0pt}{green!25!blue},
    left=0.2cm,
    attach boxed title to top left={yshift=-2mm,xshift=-2mm},
    coltitle=black,
    fonttitle=\bfseries,
    colbacktitle=white,
    boxed title style={boxrule=.4pt,sharp corners},
    before lower = {\textbf{Preuve :}\n}
}{prop}

\newtcbtheorem[use counter from = thm]{corr}{Corrolaire}
{
    enhanced,frame empty,interior empty,
    colframe=blue,
    after skip = 1cm,
    borderline west={1pt}{0pt}{green!25!blue},
    borderline south={1pt}{0pt}{green!25!blue},
    left=0.2cm,
    attach boxed title to top left={yshift=-2mm,xshift=-2mm},
    coltitle=black,
    fonttitle=\bfseries,
    colbacktitle=white,
    boxed title style={boxrule=.4pt,sharp corners},
    before lower = {\textbf{Preuve :}\n}
}{corr}

\newtcbtheorem[use counter from = thm]{lem}{Lemme}
{
    enhanced,frame empty,interior empty,
    colframe=blue,
    after skip = 1cm,
    borderline west={1pt}{0pt}{green!25!blue},
    borderline south={1pt}{0pt}{green!25!blue},
    left=0.2cm,
    attach boxed title to top left={yshift=-2mm,xshift=-2mm},
    coltitle=black,
    fonttitle=\bfseries,
    colbacktitle=white,
    boxed title style={boxrule=.4pt,sharp corners},
    before lower = {\textbf{Preuve :}\n}
}{lem}

\newtcbtheorem[use counter from = thm]{ex}{Exemple}
{
    enhanced,frame empty,interior empty,
    colframe=orange,
    after skip = 1cm,
    borderline west={1pt}{0pt}{green!25!orange},
    borderline south={1pt}{0pt}{green!25!orange},
    left=0.2cm,
    attach boxed title to top left={yshift=-2mm,xshift=-2mm},
    coltitle=black,
    fonttitle=\bfseries,
    colbacktitle=white,
    boxed title style={boxrule=.4pt,sharp corners},
    before lower = {\textbf{Preuve :}\n}
}{ex}

\newtcbtheorem[use counter from = thm]{meth}{Méthode}
{
    enhanced,frame empty,interior empty,
    colframe=purple,
    after skip = 1cm,
    borderline west={1pt}{0pt}{purple},
    borderline south={1pt}{0pt}{purple},
    left=0.2cm,
    attach boxed title to top left={yshift=-2mm,xshift=-2mm},
    coltitle=black,
    fonttitle=\bfseries,
    colbacktitle=white,
    boxed title style={boxrule=.4pt,sharp corners},
    before lower = {\textbf{Preuve :}\n}
}{meth}

\newtcbtheorem[use counter from = thm]{exercice}{Exercice}
{
    enhanced,frame empty,interior empty,
    colframe=blue,
    after skip = 1cm,
    borderline west={1pt}{0pt}{green!25!blue},
    borderline south={1pt}{0pt}{green!25!blue},
    left=0.2cm,
    attach boxed title to top left={yshift=-2mm,xshift=-2mm},
    coltitle=black,
    fonttitle=\bfseries,
    colbacktitle=white,
    boxed title style={boxrule=.4pt,sharp corners},
    before lower = {\textbf{Preuve :}\n}
}{exercice}

% PHYSIQUE %
\newtcbtheorem[use counter from = thm]{qc}{Question de Cours}
{
    enhanced,frame empty,interior empty,
    colframe=red,
    after skip = 1cm,
    borderline west={1pt}{0pt}{green!25!red},
    borderline south={1pt}{0pt}{green!25!red},
    left=0.2cm,
    attach boxed title to top left={yshift=-2mm,xshift=-2mm},
    coltitle=black,
    fonttitle=\bfseries,
    colbacktitle=white,
    boxed title style={boxrule=.4pt,sharp corners},
    before lower = {\textbf{Preuve :}\n}
}{qc}
\newtcbtheorem[use counter from = thm]{app}{Application}
{
    enhanced,frame empty,interior empty,
    colframe=blue,
    after skip = 1cm,
    borderline west={1pt}{0pt}{green!25!blue},
    borderline south={1pt}{0pt}{green!25!blue},
    left=0.2cm,
    attach boxed title to top left={yshift=-2mm,xshift=-2mm},
    coltitle=black,
    fonttitle=\bfseries,
    colbacktitle=white,
    boxed title style={boxrule=.4pt,sharp corners},
    before lower = {\textbf{Preuve :}\n}
}{app}
\newcommand*{\K}{\mathbb{K}}
\newcommand*{\C}{\mathbb{C}}
\newcommand*{\R}{\mathbb{R}}
\newcommand*{\Q}{\mathbb{Q}}
\newcommand*{\Z}{\mathbb{Z}}
\newcommand*{\N}{\mathbb{N}}
\newcommand*{\F}{\mathcal{F}}

\newcommand{\0}{\varnothing}
\newcommand*{\e}{\varepsilon}
\newcommand*{\g}{\gamma}
\newcommand*{\s}{\sigma}

\newcommand*{\ra}{\Rightarrow}
\newcommand*{\lb}{\llbracket}
\newcommand*{\rb}{\rrbracket}
\newcommand*{\n}{\\[0.2cm]}

\newcommand*{\cmark}{\ding{51}}
\newcommand*{\xmark}{\ding{55}}

\newcommand{\rg}[1]{\textrm{rg}(#1)}
\newcommand{\vect}[1]{\textrm{Vect}(#1)}
\newcommand{\tr}[1]{\textrm{Tr}(#1)}

\renewcommand{\dim}[1]{\textrm{dim}~#1}
\renewcommand*{\ker}[1]{\textrm{Ker}(#1)}
\renewcommand{\Im}[1]{\textrm{Im}(#1)}

\renewcommand*{\t}{\tau}
\renewcommand*{\phi}{\varphi}

\thispagestyle{fancy}
\fancyhead[L]{MP2I Paul Valéry}
\fancyhead[C]{\pagetitle}
\fancyhead[R]{2023-2024}

\hrule
\begin{center}
    \LARGE{\textbf{Chapitre 27}}\\
    \large{Applications linéaires}\\
    \rule{0.8\textwidth}{0.5pt}
\end{center}


\vspace{0.5cm}
\begin{exercise}{$\blacklozenge\lozenge\lozenge$}{}
    Soit $u$ un endomorphisme d'un espace vectoriel $E$. Montrer
    \begin{equation*}
        \ker{(u)} = \ker{(u^{2})} \iff \ker{(u)} \cap \Im{(u)} = \{ 0_{E} \}
    \end{equation*}
    \tcblower\\[0.2cm]
    \fbox{$\Leftarrow$} Supposons que $\ker{(u)} \cap \Im{(u)} = \{ 0_{E} \}$.\\
    On a assez facile que : $\ker{(u)} \subset \ker{(u^{2})}$\\
    Montrons que : $\ker{(u^{2})} \subset \ker{(u)}$\\
    Soit $x \in \ker{(u^{2})}$\\
    Posons $y = u{(x)}$\\
    $u^{2}{(x)} = 0$\\
    $u \circ u{(x)} = 0$\\
    $u{(y)} = 0$\\
    $y = 0$ ($y \in \ker{(u)} \cap \Im{(u)}$)\\
    $u{(x)} = 0$\\
    Ainsi on obtient : $x \in \ker{(u)}$\\\\
    \fbox{$\Rightarrow$} Supposons que $\ker{(u)} = \ker{(u^{2})}$.\\
    Soit $y \in \ker{(u)} \cap \Im{(u)}$\\
    $\exists x \in E ~|~ y = u{(x)}$\\
    Posons $x ~|~ y = u{(x)}$\\
    $u{(y)} = 0$ ($y \in \ker{(u)}$)\\
    $u \circ u{(x)} = 0$\\
    $u^{2}{(x)} = 0$\\
    $u{(x)} = 0$ ($\ker{(u)} = \ker{(u^{2})}$)\\
    Ainsi on obtient : $y = 0_{E}$
\end{exercise}

\begin{exercise}{$\blacklozenge\lozenge\lozenge$}{}
    Soit $u$ un endomorphisme d'un espace vectoriel $E$. Montrer
    \begin{equation*}
        \Im{(u)} = \Im{(u^{2})} \iff E = \ker{(u)} + \Im{(u)}
    \end{equation*}
    \tcblower\\[0.2cm]
    \fbox{$\Leftarrow$} Supposons que $E = \ker{(u)} + \Im{(u)}$.\\
    On a assez facile que : $\Im{(u^{2})} \subset \Im{(u)}$\\
    Montrons que : $\Im{(u)} \subset \Im{(u^{2})}$\\
    Soit $y \in \Im{(u)}$\\
    $\exists x \in E ~|~ y = u{(x)}$ ($y \in \Im{(u)}$)\\
    Posons $x ~|~ y = u{(x)}$\\
    $\exists (x', \widetilde{x}) \in \ker{(u)} \times E ~|~ x = x' + u{(\widetilde{x})}$ ($E = \ker{(u)} + \Im{(u)}$)\\
    Posons $(x', \widetilde{x}) \in \ker{u} \times E ~|~ x = x' + u{(\widetilde{x})}$\\
    $y = u{(x' + u{(\widetilde{x})})}$\\
    $y = u{(x')} + u \circ u{(\widetilde{x})}$\\
    $y = u^{2}{(\widetilde{x})}$ ($x' \in \ker{(u)}$)\\
    Ainsi on obtient que : $y \in \Im{(u^{2})}$\\\\
    \fbox{$\Rightarrow$} Supposons que $\Im{(u)} = \Im{(u^{2})}$.\\
    On a assez facile que : $\ker{(u)} + \Im{(u)} \subset E$\\
    Montrons que : $E = \ker{(u)} + \Im{(u)}$\\
    Soit $x \in E$
    Posons $y ~|~ y = u{(x)}$\\
    $\exists \widetilde{x} \in E ~|~ y = u^{2}{(\widetilde{x})}$ ($\Im{(u)} = \Im{(u^{2})}$)\\
    Posons $\widetilde{x} ~|~ y = u^{2}{(\widetilde{x})}$\\
    $x = x - u{(\widetilde{x})} + u{(\widetilde{x})}$\\
    $u{(x - u{(\widetilde{x})})} = u{(x)} - u^{2}{(\widetilde{x})} = 0$\\
    $u{(\widetilde{x})} \in \Im{(u)}$ et $x - u{(\widetilde{x})} \in \ker{(u)}$\\
    Ainsi on obtient que : $x \in \ker{(u)} + \Im{(u)}$
\end{exercise}

\begin{exercise}{$\blacklozenge\blacklozenge\lozenge$}{}
    Soit $u \in \mathcal{L}(E)$, où $E$ est un espace vectoriel.
     \begin{enumerate}[topsep=0pt,itemsep=-0.9 ex]
        \item Montrer que pour tout $k \geq 0$, on a $\ker{(u^{k})} \subset \ker{(u^{k+1})}$.
        \item Montrer que
        \begin{equation*}
            \forall k \in \N ~~ \ker{(u^{k})} = \ker{(u^{k+1})} \Rightarrow \ker{(u^{k+1})} = \ker{(u^{k+2})}
        \end{equation*}
    \end{enumerate}
    \tcblower\\[0.2cm]
    \boxed{1.}\\
    Soient $k \in \N, x \in \ker{(u^{k})}$\\
    $u^{k+1}{(x)} = u \circ u^{k}{(x)}$\\
    $u^{k+1}{(x)} = u{(0)}$ ($x \in \ker{(u^{k})}$)\\
    $u^{k+1}{(x)} = 0$\\
    Ainsi on obtient que $x \in \ker{(u^{k+1})}$\\\\
    \boxed{2.}\\
    \fbox{$\Rightarrow$} Supposons que $\ker{(u^{k})} = \ker{(u^{k+1})}$.\\
    On obtient avec Q1 : $\ker{(u^{k+1})} \subset \ker{(u^{k+2})}$
    Montrons que : $\ker{(u^{k+2})} \subset \ker{(u^{k+1})}$\\
    Soit $x \in \ker{(u^{k+2})}$\\
    $u^{k+2}{(x)} = u^{k+1} \circ u{(x)} = 0$\\
    $u^{k} \circ u{(x)} = 0$ ($\ker{(u^{k})} = \ker{(u^{k+1})}$)\\
    $u^{k+1}{(x)} = 0$\\
    Ainsi on obtient que : $x \in \ker{(u^{k+1})}$
\end{exercise}

\begin{exercise}{$\blacklozenge\blacklozenge\blacklozenge$}{}
    Soit $E$ un $\K$-espace vectoriel et $p,q$ deux projecteurs.
    \begin{enumerate}[topsep=0pt,itemsep=-0.9 ex]
        \item Montrer que $p+q$ est un projecteur ssi $p \circ q = q \circ p = 0$.
        \item Supposons que $p+q$ est projecteur. Montrer que
    \end{enumerate}
    \begin{equation*}
        \Im(p+q)=\Im(p)\oplus\Im(q) \quad \text{et} \quad \ker(p+q)=\ker(p)\cap\ker(q).
    \end{equation*}
    \tcblower\\[0.2cm]
    \boxed{1.}\\
    \fbox{$\Leftarrow$} Supposons que $p\circ q = q \circ p = 0$.\\
    On a $(p+q)^2=p^2 + pq + qp + q^2 = p^2 + q^2 = p + q$ donc $p+q$ est un projecteur.\\
    \fbox{$\Rightarrow$} Supposons que $p+q$ est un projecteur.\\
    Alors $(p+q)^2 = p + pq + qp + q = p + q$ donc $pq + qp = 0$.\\
    Alors $pq = qp \ra pq = p^2q = -pqp = qp = -qp^2 = qp^2 = qp$.\\
    Donc $pq = qp$, mais aussi $pq = -qp$ donc $pq = qp = 0$.
\end{exercise}

\end{document}
