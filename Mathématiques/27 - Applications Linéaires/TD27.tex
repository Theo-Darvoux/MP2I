\documentclass[11pt]{article}

\usepackage[paperheight=45in, left=2cm, right=2cm, top=2cm, bottom=2cm]{geometry}
\usepackage[most]{tcolorbox}
\usepackage{hyperref, fancyhdr, lastpage, tocloft, changepage}
\usepackage{enumitem}
\usepackage{amsmath, amssymb, amsthm, stmaryrd}

\def\pagetitle{Applications linéaires}
\setlength{\headheight}{14pt}
\newcommand*{\F}{\mathcal{F}}
\newcommand*{\R}{\mathbb{R}}
\newcommand*{\C}{\mathbb{C}}
\newcommand*{\K}{\mathbb{K}}
\newcommand*{\N}{\mathbb{N}}
\newcommand*{\m}{\mathcal}
\newcommand*{\Id}{\textrm{Id}}
\newcommand*{\ra}{\Rightarrow}

\renewcommand*{\phi}{\varphi}
\renewcommand*{\ker}{\textrm{Ker}}
\renewcommand{\Im}{\textrm{Im}}

\DeclareMathOperator*{\vect}{Vect}
\DeclareMathOperator*{\tr}{Tr}
\DeclareMathOperator*{\rg}{rg}

\title{\bf{\pagetitle}\\\large{Corrigé}}

\newtcbtheorem{exercise}{Exercice}
{
    enhanced,frame empty,interior empty,
    colframe=blue,
    after skip = 1cm,
    borderline west={1pt}{0pt}{green!25!blue},
    borderline south={1pt}{0pt}{green!25!blue},
    left=0.2cm,
    attach boxed title to top left={yshift=-2mm,xshift=-2mm},
    coltitle=black,
    fonttitle=\bfseries,
    colbacktitle=white,
    boxed title style={boxrule=.4pt,sharp corners},
    before lower = {\textbf{Solution :}}
}{exercise}

\hypersetup{
    colorlinks=true,
    citecolor=black,
    linktoc=all,
    linkcolor=blue
}

\pagestyle{fancy}
\cfoot{\thepage\ sur \pageref*{LastPage}}

\begin{document}

\thispagestyle{fancy}
\fancyhead[L]{MP2I Paul Valéry}
\fancyhead[C]{\pagetitle}
\fancyhead[R]{2023-2024}

\hrule
\begin{center}
    \LARGE{\textbf{Chapitre 27}}\\
    \large{Applications linéaires}\\
    \rule{0.8\textwidth}{0.5pt}
\end{center}


\vspace{0.5cm}

\begin{exercise}{$\blacklozenge\blacklozenge\blacklozenge$}{}
    Soit $E$ un $\K$-espace vectoriel et $p,q$ deux projecteurs.
    \begin{enumerate}[topsep=0pt,itemsep=-0.9 ex]
        \item Montrer que $p+q$ est un projecteur ssi $p \circ q = q \circ p = 0$.
        \item Supposons que $p+q$ est projecteur. Montrer que
    \end{enumerate}
    \begin{equation*}
        \Im(p+q)=\Im(p)\oplus\Im(q) \quad \text{et} \quad \ker(p+q)=\ker(p)\cap\ker(q).
    \end{equation*}
    \tcblower\\[0.2cm]
    \boxed{1.}\\
    \fbox{$\Leftarrow$} Supposons que $p\circ q = q \circ p = 0$.\\
    On a $(p+q)^2=p^2 + pq + qp + q^2 = p^2 + q^2 = p + q$ donc $p+q$ est un projecteur.\\
    \fbox{$\Rightarrow$} Supposons que $p+q$ est un projecteur.\\
    Alors $(p+q)^2 = p + pq + qp + q = p + q$ donc $pq + qp = 0$.\\
    Alors $pq = qp \ra pq = p^2q = -pqp = qp = -qp^2 = qp^2 = qp$.\\
    Donc $pq = qp$, mais aussi $pq = -qp$ donc $pq = qp = 0$.
\end{exercise}

\end{document}
 