\documentclass[12pt]{article}
\usepackage{ragged2e}
\usepackage[left=2cm, right=2cm, top=2cm, bottom=2cm]{geometry}
\usepackage{color}
\usepackage{amsmath, amssymb}
\usepackage{lastpage}
\usepackage{fancyhdr}
\usepackage[T1]{fontenc}
\usepackage{hyperref}

\title{Chapitre 1\\\large La vérité impuissante, le pouvoir trompeur.\\Arendt et la valeur politique de la vérité.}
\date{}
\author{}

\setlength{\headheight}{15pt}
\pagestyle{fancy}
\cfoot{\thepage\ sur \pageref*{LastPage}}

\hypersetup{
    colorlinks=true,
    citecolor=black,
    linktoc=all,
    linkcolor=blue
}

\renewcommand*\contentsname{Sommaire}

\begin{document}
\maketitle
\thispagestyle{fancy}
\tableofcontents
\fancyhead[L]{Théo DARVOUX}
\fancyhead[R]{MP2I Paul Valéry}
\fancyhead[C]{Français-Philosophie}
\pagebreak
\section*{\color{red}Introduction}
Dans ses réflexions sur la vérité et la politique, Hannah Arendt part de ce qu'elle appelle un bien commun selon lequel vérité et politique ne font pas bon ménage.
En effet, il est tentant de considérer que la politique est avant tout faite de manipulations, de fausses promesses ou de mensonges qu'elle se définit avant tout par la recherche du pouvoir, du rapport de force, et que cela doit se faire bien souvent au détriment de la vérité.
C'est ce qui fait que les mensonges les plus absurdes peuvent infiltrer les plus hautes sphères du pouvoir comme dans l'affaire du rapport du Pentagone sur la guerre du Vietnam.\par
Hannah Arendt a été elle-même confrontée à ce problème au moment du procès d'Eichmann. Elle a souffert de déformation de ses propos en réaction à ses témoignages.
C'est un contexte où vérité et politiques entrent directement en conflit.
Cependant, cette affaire montre aussi qu'on ne peut pas totalement détacher la politique de la question de la vérité et que même si la vérité parrait souvent impuissante, elle reste un enjeu politique en tant que tel: d'abord il y a les vérités historiques qui fournissent le cadre de tout problème politique et dont on attend qu'elles soient relativement fixées.
La politique doit donc tenir compte de la vérité historique.
De façon analogue, on attend de la justice et du pouvoir judiciaire qu'elles établissent la vérité, ce qui la détache quelque peu des enjeux politiques du pouvoir.
C'est ce que montre l'affaire Eichmann.\par
Enfin, il n'est pas exclu que la science ou le savoir doievent jouer un rôle dans le débat public, avec peut-être même une position d'autorité.
Hannah Arendt elle-même en tant que philosophe qui parle de politique parle bien en un sens ou non de la vérité.
On pourrait donc dire que la vérité est une question qui accompagne sans-cesse la politique.
Alors même que la politique ne semble pas s'en préoccuper comme d'une valeur centrale.
\addcontentsline{toc}{section}{Introduction}
\pagebreak
\section*{\color{red}I. Une philosophie politique.}
\addcontentsline{toc}{section}{I. Une philosophie politique}
\subsection*{a) Philosophie et métaphysique\footnote{Science de l'être en tant qu'être}.}
Ce qui fait l'originalité d'Arendt c'est son orientation d'abord politique.
Habituellement, la philosophie politique est considérée comme une branche de la philosophie.
Pour Arendt, la philosophie est centralement politique.
Hannah Arendt développe une philosophie qui échappe aux schémas de la philosophie classiques : dans certains entretiens, elle récuse le titre de philosophe.
La philosophie politique n'est qu'une branche très dérivée de la philosophie.
Pour Hannah Arendt, la politique n'est pas un aspect parmis d'autres de l'existence humaine, elle est au contraire l'experience la plus constitutive de notre sens même de la réalité et donc de l'être.
À notre époque, on dit parfois que tout est politique.
C'est présent par exemple dans les discours féministes pour dire que l'intime est politique.
\addcontentsline{toc}{subsection}{a) Philosophie et métaphysique.}
\subsection*{b) La realité de l'espace public.}
Pour Hannah Arendt, il y a une claire hiérarchisation à établir entre l'espace de la vie privée et l'espace public, c'est-à-dire l'espace de la politique comprise sur le modèle grec antique du rapport entre citoyens égaux.
Les citoyens pouvaient exercer leur faculté politique, leur liberté à condition de pouvoir sortir de leur foyer, de pouvoir se libérer des nécessités économiques. 
Le foyer était donc le lieu des esclaves et de la nécessité sous la forme aussi de la domination.
Mais surtout, ce qui apparaît, ce qui est dit ou ce qui est fait dans l'espace public a, par principe plus d'être ou de réalité que ce qui reste caché dans la pénombre de la vie privée.
La différence entre privé et public est donc une différence métaphysique (ontologique).
Hannah Arendt affirme que l'apparence publique constitue la réalité.
La chose la plus privée qui soit c'est la pensée qui n'apparaît jamais directement en public.
C'est donc en un sens aussi la chose la moins réelle.
Et pourtant, toute la tradition métaphysique occidentale a tendance à affirmer que ce qu'il y a de plus réel est la pensée\footnote{<<Je pense donc je suis>>, Descartes}.
\addcontentsline{toc}{subsection}{b) La réalité de l'espace public.}
\section*{\color{red}II. Le conflit entre vérité et politique.}
\addcontentsline{toc}{section}{II. Le conflit entre vérité et politique}
\subsection*{a) Le procès Eichmann.}
Hannah Arendt a écrit le texte \emph{Vérité et Politique} après avoir affronté une polémique concernant ses articles sur le procès d'Adolf Eichmann.\\
Premier élément polémique : rôle des conseils juifs dans la solution finale.\\
Ce contexte d'écriture montre bien en quoi la vérité est une valeur difficile à défendre dans un contexte politique.\\
Deuxième élément polémique : Sous-titre <<Raport sur la banalité du mal>>.\\
La thèse d'Arendt apparaît comme une déresponsabilisation d'Eichmann.
Ces problématiques judiciaires sont un point de rencontre privilégié entre la question de la vérité et les intérêts politiques.
Hannah Arendt, pour sa part, en tant qu'intellectuelle cherchait à approcher la vérité et se faisant, elle a subi un choc très violent avec la réalité politique de la situation, ce qui explique qu'elle distingue soigneusement dans son texte deux notions différentes : vérité et réalité.
\addcontentsline{toc}{subsection}{a) Le procès Eichmann.}
\subsection*{b) <<Fiat veritas, et pereat mundus>>\footnote{Que la vérité soit, le monde dût-il en périr.}.}
Cette expression latine suggère que la vérité fait partie de ces choses auxquelles nous pouvons attribuer une valeur absolue au même titre, par exemple, que la justice.
Cela implique qu'on peut sacrifier d'autres choses à la vérité et qu'il peut même y avoir un conflit de valeurs entre la recherche de la vérité et la réalité.
Mais en général, la vérité part perdante dans ce conflit.
Politiquement, elle paraît souvent impuissante.
Au début du texte, Hannah Arendt pose clairement le problème du rapport entre vérité et réalité sachant que la réalité se trouve avant tout du côté de la politique et de l'espace public.
Or, le rapport entre la vérité et cet espace est conflictuel et souvent la vérité semble impuissante dans le champ politique.
Hannah Arendt précise en outre que la réalité consiste en la mise en place d'un monde commun et stable qui dépasse nos existences individuelles.
Il est possible que la recherche de la vérité soit incompatible avec la politique, qu'elle suppose de rejeter le monde.
Pourtant, on sent bien que Hannah Arendt va critiquer ce lieu commun, il faut bien que la vérité intervienne à un certain niveau de la vie politique et la vérité a son rôle à jouer dans la stabilité de notre monde
Hannah Arendt va analyser le rôle des vérités scientifiques en se demandant si elles peuvent être menacées par la politique.
Ce sont surtout les vérités historiques qui vont intéresser Arendt: ce sont des vérités plus fragiles, plus factuelles, plus exposées à l'oubli.
Au moment de la guerre en Algérie, personne ne parlait de guerre.
Une fois établies, les vérités historiques prétendent à une certaine objectivité qui encadrent le débat politique.
La vérité peut être apocalyptique lorsqu'elle éclate.
Inversement, les raisonnements politiques se fondent souvent sur une autre valeur considérée comme ultime : la survie de l'individu ou de l'État.
La logique proprement politique, pour Arendt, a toujours tendance à revenir à une logique utilitaire ou instrumentale sur le mode : <<La fin justifie les moyens>>.
La fin ultime de la politique est en général la survie, or le mensonge apparaît toujours comme un moyen politique particulièrement efficace, même si cela implique de sacrifier la vérité.
En outre, le mensonge est un moyen politique relativement inoffensif comparé à des moyens plus violents.

Hobbes est le grand penseur de la sécurité comme valeur politique centrale.
Le point de départ de tout raisonnement politique c'est l'hypothèse de l'État de nature qu'il définit comme une situation où il n'y a pas de pouvoir commun ce qui implique qu'on pourrait retomber à l'État de nature.
Sur cette hypothèse Hobbes formule que tout le monde se méfie de tout le monde, ce qui conduit inévitablement à la violence et à la guerre de chacun contre chacun.
La tâche centrale de la politique est d'éviter cette éventualité.
Pour Hobbes, la mise en place d'un État fort, décrite dans \emph{Le Léviathan} (1651) est le moyen d'éviter cette violence.
Hobbes est donc le tenant d'une logique politique très réaliste, pragmatique et sécuritaire.
C'est pourquoi dans son texte, Hannah Arendt le fait constamment dialoguer avec Platon.
En réalité, Hannah Arendt va tenter dans ce texte de dissiper le malentendu entre vérité et politique pour montrer que la vérité peut effectivement être une valeur politique au moins d'un certain point de vue qu'il ne faut pas systématiquement opposer aux réalisme des raisonnements politiques.
En effet, si le rôle de la politique est d'assurer la survie, et une certaine permanence du monde : la persévérance dans l'être, cette stabilité du monde ne peut pas être détachée de l'etablissement de certaines vérités qu'on peut considérer objectives, notamment les vérités historiques.
C'est le sens de la référence à Hérodote  ($V^{eme}$ siècle avant JC) et son projet : dire ce qui est.
C'est une première forme d'histoire qui était la chronique : faire l'histoire en même temps qu'elle se déroulait puisque les historiens suivant avaient besoin de cette chronique.
Il a eu le réflexe de fixer les évènements de son temps dans un récit.
Cette histoire du temps proche est aussi la plus proche de la politique et aussi la plus difficile à rendre objective.
Hérodote est aussi réputé pour son objectivité\footnote{Hérodote est le reflet de Hannah Arendt}.
Dans la suite du texte, Arendt précise son analyse sur les raisons du malentendu entre vérité et politique en ayant recours aux figures de Platon et de Hobbes.
Il se pourrait que Platon ait trop valorisé la vérité en sous-estimant volontairement le problème politique inversement, Hobbes aurait sous-estimé la question de la vérité en donnant trop d'importance à la conflictualité politique.
En commentant l'image de la caverne de Platon, Arendt remarque que Platon y a effacé toute trace de conflictualité entre les citoyens réduits au statut de spectateurs passifs.
Autrement dit, il a volontairement gommé ce qui fait la difficulté spécifique de la politique.
Hobbes, de son côté, prétendait qu'aucune vérité n'est assez solide pour résister à la politique si elle contrarie des intérêts ou une volonté de domination, ce qu'on constate peut-être aujourd'hui avec le réchauffement climatique.
Les deux auteurs donnent pour cette raison un statut très ambivalent aux mathématiques : les mathématiques ne sont pas menacées par la politique, elles restent stables, mais uniquement parce qu'elles sont inoffensives en elles-mêmes.
Si les mathématiques représentaient un danger politique, elles seraient mises au bûcher.
De son côté, Platon avait constaté lui aussi cette stabilité des mathématiques.
Pour lui, les mathématiques servent alors d'instruments pour faire croire à une vérité plus générale et politique qui devrait mettre tout le monde d'accord sur tous les sujets.
En outre, les mathématiques lui servent d'instruments de séléction sociale.
\addcontentsline{toc}{subsection}{b) <<Fiat veritas et pereat mundus>>.}
\subsection*{c) Les différents types de vérité.}
Tout cela conduit Arendt à distinguer de façon plus précise différents types de vérité qui n'ont pas la même signification et la même puissance du point de vue de la politique.

\begin{center}{Vérités de raison $\neq$ Vérités de fait.\footnote{Leibniz}}\end{center}

Une vérité de raison est une vérité démontrée dont on connaît précisément la raison.
Au contraire, une vérité de fait est une vérité qu'on constate empiriquement ou qui repose sur des témoignages
Les philosphes ont depuis longtemps préférés les vérités rationnelles aux vérités de fait.
Cependant, Arendt va montrer que ce sont précisément les vérités de fait, et notamment les vérités historiques qui sont décisives dans le domaine politique.
Certains philosophes ont formé le projet de réduire toutes les vérités de fait à des vérités de raison, c'est ce que dit notamment Leibniz : <<Dans l'entendement de Dieu, il n'y a que des vérités de raison>>.
Dans l'entendement de Dieu, il n'y a pas de faits contingents : Dieu a la notion complète de chaque être humain. Pour Leibniz tout obéit au principe de la raison : <<rien n'est sans raison>>.
Pour Leibniz, tout le mal qui arrive sur Terre est justifié dans l'esprit de Dieu, tout est rationnalisable\footnote{Théodicée}.
Dans cet univers de vérité nécessaire et rationnelle, il n'y a pas de liberté, tout est déjà prévu.
Le fait que la tradition philosophique, notamment rationnaliste donne un tel privilège aux vérités de raison sur le modèle des mathématiques conduit à une vision du réel où tout est nécessaire, où la liberté et donc la politique au sens fort ne trouve pas forcément sa place.
Au contraire, un acte libre apparaîtra toujours comme un fait non nécessaire.
Arendt remarque, en outre, que le propre de la vérité rationnelle est d'être produite par l'esprit humain, pas simplement donnée ou constatée, c'est une vérité que l'on fabrique.
En physique, on produit aussi la vérité à travers l'éxperimentation, c'est à dire une forme d'experience qui est elle même construite selon des ordres précis.
Cette idée que la vérité est produite, fabriquée a nécessairement pour conséquence du point de vue du sens commun de brouiller la frontière entre vérité et fiction voire entre vérité et mensonge.

La valorisation des vérités rationnelles a entraîné une dévalorisation des vérités simplement factuelles jugées plus faibles parce que contingentes, incertaines, mais ce sont aussi les vérités les plus importantes en politique, notamment quand elles prennent la forme de vérités historiques.
On peut donc penser que dans le développement des sciences modernes, quelque chose a favorisé une certaine méfiance vis-à-vis des faits et de leur solidité en tant que faits.
Les faits ne peuvent pas être justifiés au sens d'une rationnalisation, ils doivent être constatés, mais pour Arendt, ce n'est pas une forme plus faible de vérité.
Arendt suggère aussi que les vérités factuelles sont par nature beaucoup plus fragiles que les vérités rationnelles notamment parce qu'elles sont exposées à la possibilité d'une dissimulation radicale ou d'un oubli définitif.
On aurait pu imaginer qu'elles ne deviennent jamais vraies.
Un fait oublié ne ressurgit jamais.
\addcontentsline{toc}{subsection}{c) Les différents types de vérité.}
\subsection*{d) Vérité et opinion.}
Prendre pour modèle les vérités de raison et la certiture rationnelle a aussi pour conséquence de dévaloriser tout ce qui relève de l'opinion.
L'opinion c'est un jugement qui ne peut pas être entièrement justifié ou qui n'a que des justifications incomplètes.
Dans sa forme la plus faible, c'est la rumeur.
L'opinion peut aussi être une forme de jugement parfaitement légitime dans des domaines où on ne peut pas ésperer une certitude absolue.
Pour Hannah Arendt, l'opinion c'est la forme de pensée proprement politique et la pluralité des opinions dans le champ politique, loin d'être une imperfection est au contraire ce qui constitue notre réalité.
L'opinion a une grande valeur pour Hannah Arendt, c'est aussi une forme de pensée active, évolutive et dans laquelle on s'engage.
Dans la philosophie de Platon, et dans toute la tradition de pensée qui l'a suivi, l'opinion est toujours dévalorisée face à la certitude rationnelle.
Notre modèle dominant de vérité reste envisagé du point de vue de la connaissance.
Qu'il puisse y avoir une vérité de l'action, des intérêts, des intentions, voilà ce qui échappe à cette approche et c'est pourquoi toute cette tradition a sous-estimée le problème du mensonge.
Quel est, en effet, le contraire de la vérité.
Pour Platon c'est l'opinion, ou c'est l'erreur, ou c'est l'ignorance mais pas le mensonge.
Pour Arendt, c'est en général concernant les vérités de fait que le mensonge devient un problème et il s'oppose alors à la sincérité.
C'est pourquoi la question du mensonge, central en politique a été structurellement sous-estimée.
Hannah Arendt remarque de plus que le mensonge paraît être organisé, ce qui n'est pas le cas de l'erreur ou de l'illusion.
Dans la suite du texte, Hannah Arendt remarque cependant que l'opinion a retrouvé une certaine importance à partir du XVIIIème siècle; siècle des Lumières qui valorisait l'exigence de prenser par soi-même et de confronter les opinions dans l'espace public.
C'est à la même époque qu'on commence à parler des limites du rationnalisme.
Notamment, Kant dans \emph{La critique de la raison pure}, qui, comme son nom l'indique, montre les limites du savoir rationnel.
Kant montre qu'on ne peut connaître que des phénomènes et jamais les choses en soi.
Cependant, pour Arendt, le statut de ces vérités de fait reste durablement inchangé, c'est là dessus qu'elle va centrer son propos par la suite.
\addcontentsline{toc}{subsection}{d) Vérité et opinion.}
\subsection*{e) Les vérités de fait.}
Tout l'enjeu, pour Arendt, va être de montrer que ces vérités factuelles, notamment historiques peuvent et doivent atteindre une grande objectivité même si on ne peut pas les extraire complètement du domaine de l'opinion et même si elles sont toujours soumises à la manipulation et au mensonge organisé.
Pour Arendt, il faut admettre que l'existence des camps de concentration est une vérité et non une simple opinion.
Pour Arendt, une vérité de fait, quand elle est importante a forcément une dimension politique et par conséquent, elle contrarie des intérêts, ce qui crée la tentation de la renvoyer à une simple opinion.
Pourtant, selon Arendt, ce sont ces vérités qui constituent le tissu de notre réalité commune. 
\addcontentsline{toc}{subsection}{e) Les vérités de fait.}
\end{document}