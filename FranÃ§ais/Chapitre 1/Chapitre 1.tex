\documentclass[12pt]{article}
\usepackage{ragged2e}
\usepackage[left=2cm, right=2cm, top=2cm, bottom=2cm]{geometry}
\usepackage{color}
\usepackage{amsmath, amssymb}
\usepackage{lastpage}
\usepackage{fancyhdr}
\usepackage[T1]{fontenc}
\usepackage{hyperref}

\title{Chapitre 1\\\large La vérité impuissante, le pouvoir trompeur.\\Arendt et la valeur politique de la vérité.}
\date{}
\author{}


\setlength{\headheight}{15pt}
\pagestyle{fancy}
\cfoot{\thepage\ sur \pageref*{LastPage}}

\hypersetup{
    colorlinks=true,
    citecolor=black,
    linktoc=all,
    linkcolor=blue
}

\renewcommand*\contentsname{Sommaire}

\begin{document}
\maketitle
\thispagestyle{fancy}
\begin{center}
    \LARGE{etienne la feuraude}
\end{center}
\hrule
\tableofcontents
\hrule
\fancyhead[L]{Théo DARVOUX}
\fancyhead[R]{MP2I Paul Valéry}
\fancyhead[C]{Français-Philosophie}
\pagebreak
\section*{Consignes épreuve Français}
Résumé : 45 minutes, $(100 \vee 200 \pm10\%)$ mots au choix.\\
Texte en rapport avec le programme, argumentatif et littéraire, 1 page. Reprendre le mode d'énonciation du texte. Nombre de mots à écrire à la fin. Tout d'un bloc, une barre tous les 10 mots. L'enjeu est de faire apparaître la logique profonde du texte. Respecter l'ordre du texte. 
Ne pas oublier d'idées du texte, orthographe, grammaire.\\
Dissertation : 3h15 sur une citation du texte. Introduction 1 page: une accroche, imagée 4 à 5 lignes. Amener le problème de manière générale, conduire à la citation puis l'expliquer (10-15 lignes). Plan de plus en plus approfondi.
2 ou trois parties 2 ou trois sous parties par parties. Raisonnement au début d'un paragraphe, formuler une idée puis l'illustrer de façon comparative avec les oeuvres au programme. Pas toujours possible de mettre les 3 oeuvres dans chaque paragraphe. Toutes les oeuvres par partie.\\
Conclusion : rappeler les parties, essayer d'apporter une réponse, on peut rappeler un passage d'une oeuvre 
\pagebreak
\section*{\color{red}Introduction}
Dans ses réflexions sur la vérité et la politique, Hannah Arendt part de ce qu'elle appelle un bien commun selon lequel vérité et politique ne font pas bon ménage.
En effet, il est tentant de considérer que la politique est avant tout faite de manipulations, de fausses promesses ou de mensonges qu'elle se définit avant tout par la recherche du pouvoir, du rapport de force, et que cela doit se faire bien souvent au détriment de la vérité.
C'est ce qui fait que les mensonges les plus absurdes peuvent infiltrer les plus hautes sphères du pouvoir comme dans l'affaire du rapport du Pentagone sur la guerre du Vietnam.\par
Hannah Arendt a été elle-même confrontée à ce problème au moment du procès d'Eichmann. Elle a souffert de déformation de ses propos en réaction à ses témoignages.
C'est un contexte où vérité et politiques entrent directement en conflit.
Cependant, cette affaire montre aussi qu'on ne peut pas totalement détacher la politique de la question de la vérité et que même si la vérité parrait souvent impuissante, elle reste un enjeu politique en tant que tel: d'abord il y a les vérités historiques qui fournissent le cadre de tout problème politique et dont on attend qu'elles soient relativement fixées.
La politique doit donc tenir compte de la vérité historique.
De façon analogue, on attend de la justice et du pouvoir judiciaire qu'elles établissent la vérité, ce qui la détache quelque peu des enjeux politiques du pouvoir.
C'est ce que montre l'affaire Eichmann.\par
Enfin, il n'est pas exclu que la science ou le savoir doievent jouer un rôle dans le débat public, avec peut-être même une position d'autorité.
Hannah Arendt elle-même en tant que philosophe qui parle de politique parle bien en un sens ou non de la vérité.
On pourrait donc dire que la vérité est une question qui accompagne sans-cesse la politique.
Alors même que la politique ne semble pas s'en préoccuper comme d'une valeur centrale.
\addcontentsline{toc}{section}{Introduction}
\pagebreak
\section*{\color{red}I. Une philosophie politique.}
\addcontentsline{toc}{section}{I. Une philosophie politique}
\subsection*{a) Philosophie et métaphysique\footnote{Science de l'être en tant qu'être}.}
Ce qui fait l'originalité d'Arendt c'est son orientation d'abord politique.
Habituellement, la philosophie politique est considérée comme une branche de la philosophie.
Pour Arendt, la philosophie est centralement politique.
Hannah Arendt développe une philosophie qui échappe aux schémas de la philosophie classiques : dans certains entretiens, elle récuse le titre de philosophe.
La philosophie politique n'est qu'une branche très dérivée de la philosophie.
Pour Hannah Arendt, la politique n'est pas un aspect parmis d'autres de l'existence humaine, elle est au contraire l'experience la plus constitutive de notre sens même de la réalité et donc de l'être.
À notre époque, on dit parfois que tout est politique.
C'est présent par exemple dans les discours féministes pour dire que l'intime est politique.
\addcontentsline{toc}{subsection}{a) Philosophie et métaphysique.}
\subsection*{b) La realité de l'espace public.}
Pour Hannah Arendt, il y a une claire hiérarchisation à établir entre l'espace de la vie privée et l'espace public, c'est-à-dire l'espace de la politique comprise sur le modèle grec antique du rapport entre citoyens égaux.
Les citoyens pouvaient exercer leur faculté politique, leur liberté à condition de pouvoir sortir de leur foyer, de pouvoir se libérer des nécessités économiques. 
Le foyer était donc le lieu des esclaves et de la nécessité sous la forme aussi de la domination.
Mais surtout, ce qui apparaît, ce qui est dit ou ce qui est fait dans l'espace public a, par principe plus d'être ou de réalité que ce qui reste caché dans la pénombre de la vie privée.
La différence entre privé et public est donc une différence métaphysique (ontologique).
Hannah Arendt affirme que l'apparence publique constitue la réalité.
La chose la plus privée qui soit c'est la pensée qui n'apparaît jamais directement en public.
C'est donc en un sens aussi la chose la moins réelle.
Et pourtant, toute la tradition métaphysique occidentale a tendance à affirmer que ce qu'il y a de plus réel est la pensée\footnote{<<Je pense donc je suis>>, Descartes}.
\addcontentsline{toc}{subsection}{b) La réalité de l'espace public.}
\section*{\color{red}II. Le conflit entre vérité et politique.}
\addcontentsline{toc}{section}{II. Le conflit entre vérité et politique}
\subsection*{a) Le procès Eichmann.}
Hannah Arendt a écrit le texte \emph{Vérité et Politique} après avoir affronté une polémique concernant ses articles sur le procès d'Adolf Eichmann.\\
Premier élément polémique : rôle des conseils juifs dans la solution finale.\\
Ce contexte d'écriture montre bien en quoi la vérité est une valeur difficile à défendre dans un contexte politique.\\
Deuxième élément polémique : Sous-titre <<Raport sur la banalité du mal>>.\\
La thèse d'Arendt apparaît comme une déresponsabilisation d'Eichmann.
Ces problématiques judiciaires sont un point de rencontre privilégié entre la question de la vérité et les intérêts politiques.
Hannah Arendt, pour sa part, en tant qu'intellectuelle cherchait à approcher la vérité et se faisant, elle a subi un choc très violent avec la réalité politique de la situation, ce qui explique qu'elle distingue soigneusement dans son texte deux notions différentes : vérité et réalité.
\addcontentsline{toc}{subsection}{a) Le procès Eichmann.}
\subsection*{b) <<Fiat veritas, et pereat mundus>>\footnote{Que la vérité soit, le monde dût-il en périr.}.}
Cette expression latine suggère que la vérité fait partie de ces choses auxquelles nous pouvons attribuer une valeur absolue au même titre, par exemple, que la justice.
Cela implique qu'on peut sacrifier d'autres choses à la vérité et qu'il peut même y avoir un conflit de valeurs entre la recherche de la vérité et la réalité.
Mais en général, la vérité part perdante dans ce conflit.
Politiquement, elle paraît souvent impuissante.
Au début du texte, Hannah Arendt pose clairement le problème du rapport entre vérité et réalité sachant que la réalité se trouve avant tout du côté de la politique et de l'espace public.
Or, le rapport entre la vérité et cet espace est conflictuel et souvent la vérité semble impuissante dans le champ politique.
Hannah Arendt précise en outre que la réalité consiste en la mise en place d'un monde commun et stable qui dépasse nos existences individuelles.
Il est possible que la recherche de la vérité soit incompatible avec la politique, qu'elle suppose de rejeter le monde.
Pourtant, on sent bien que Hannah Arendt va critiquer ce lieu commun, il faut bien que la vérité intervienne à un certain niveau de la vie politique et la vérité a son rôle à jouer dans la stabilité de notre monde
Hannah Arendt va analyser le rôle des vérités scientifiques en se demandant si elles peuvent être menacées par la politique.
Ce sont surtout les vérités historiques qui vont intéresser Arendt: ce sont des vérités plus fragiles, plus factuelles, plus exposées à l'oubli.
Au moment de la guerre en Algérie, personne ne parlait de guerre.
Une fois établies, les vérités historiques prétendent à une certaine objectivité qui encadrent le débat politique.
La vérité peut être apocalyptique lorsqu'elle éclate.
Inversement, les raisonnements politiques se fondent souvent sur une autre valeur considérée comme ultime : la survie de l'individu ou de l'État.
La logique proprement politique, pour Arendt, a toujours tendance à revenir à une logique utilitaire ou instrumentale sur le mode : <<La fin justifie les moyens>>.
La fin ultime de la politique est en général la survie, or le mensonge apparaît toujours comme un moyen politique particulièrement efficace, même si cela implique de sacrifier la vérité.
En outre, le mensonge est un moyen politique relativement inoffensif comparé à des moyens plus violents.

Hobbes est le grand penseur de la sécurité comme valeur politique centrale.
Le point de départ de tout raisonnement politique c'est l'hypothèse de l'État de nature qu'il définit comme une situation où il n'y a pas de pouvoir commun ce qui implique qu'on pourrait retomber à l'État de nature.
Sur cette hypothèse Hobbes formule que tout le monde se méfie de tout le monde, ce qui conduit inévitablement à la violence et à la guerre de chacun contre chacun.
La tâche centrale de la politique est d'éviter cette éventualité.
Pour Hobbes, la mise en place d'un État fort, décrite dans \emph{Le Léviathan} (1651) est le moyen d'éviter cette violence.
Hobbes est donc le tenant d'une logique politique très réaliste, pragmatique et sécuritaire.
C'est pourquoi dans son texte, Hannah Arendt le fait constamment dialoguer avec Platon.
En réalité, Hannah Arendt va tenter dans ce texte de dissiper le malentendu entre vérité et politique pour montrer que la vérité peut effectivement être une valeur politique au moins d'un certain point de vue qu'il ne faut pas systématiquement opposer aux réalisme des raisonnements politiques.
En effet, si le rôle de la politique est d'assurer la survie, et une certaine permanence du monde : la persévérance dans l'être, cette stabilité du monde ne peut pas être détachée de l'etablissement de certaines vérités qu'on peut considérer objectives, notamment les vérités historiques.
C'est le sens de la référence à Hérodote  ($V^{eme}$ siècle avant JC) et son projet : dire ce qui est.
C'est une première forme d'histoire qui était la chronique : faire l'histoire en même temps qu'elle se déroulait puisque les historiens suivant avaient besoin de cette chronique.
Il a eu le réflexe de fixer les évènements de son temps dans un récit.
Cette histoire du temps proche est aussi la plus proche de la politique et aussi la plus difficile à rendre objective.
Hérodote est aussi réputé pour son objectivité\footnote{Hérodote est le reflet de Hannah Arendt}.
Dans la suite du texte, Arendt précise son analyse sur les raisons du malentendu entre vérité et politique en ayant recours aux figures de Platon et de Hobbes.
Il se pourrait que Platon ait trop valorisé la vérité en sous-estimant volontairement le problème politique inversement, Hobbes aurait sous-estimé la question de la vérité en donnant trop d'importance à la conflictualité politique.
En commentant l'image de la caverne de Platon, Arendt remarque que Platon y a effacé toute trace de conflictualité entre les citoyens réduits au statut de spectateurs passifs.
Autrement dit, il a volontairement gommé ce qui fait la difficulté spécifique de la politique.
Hobbes, de son côté, prétendait qu'aucune vérité n'est assez solide pour résister à la politique si elle contrarie des intérêts ou une volonté de domination, ce qu'on constate peut-être aujourd'hui avec le réchauffement climatique.
Les deux auteurs donnent pour cette raison un statut très ambivalent aux mathématiques : les mathématiques ne sont pas menacées par la politique, elles restent stables, mais uniquement parce qu'elles sont inoffensives en elles-mêmes.
Si les mathématiques représentaient un danger politique, elles seraient mises au bûcher.
De son côté, Platon avait constaté lui aussi cette stabilité des mathématiques.
Pour lui, les mathématiques servent alors d'instruments pour faire croire à une vérité plus générale et politique qui devrait mettre tout le monde d'accord sur tous les sujets.
En outre, les mathématiques lui servent d'instruments de séléction sociale.
\addcontentsline{toc}{subsection}{b) <<Fiat veritas et pereat mundus>>.}
\subsection*{c) Les différents types de vérité.}
Tout cela conduit Arendt à distinguer de façon plus précise différents types de vérité qui n'ont pas la même signification et la même puissance du point de vue de la politique.

\begin{center}{Vérités de raison $\neq$ Vérités de fait.\footnote{Leibniz}}\end{center}

Une vérité de raison est une vérité démontrée dont on connaît précisément la raison.
Au contraire, une vérité de fait est une vérité qu'on constate empiriquement ou qui repose sur des témoignages
Les philosphes ont depuis longtemps préférés les vérités rationnelles aux vérités de fait.
Cependant, Arendt va montrer que ce sont précisément les vérités de fait, et notamment les vérités historiques qui sont décisives dans le domaine politique.
Certains philosophes ont formé le projet de réduire toutes les vérités de fait à des vérités de raison, c'est ce que dit notamment Leibniz : <<Dans l'entendement de Dieu, il n'y a que des vérités de raison>>.
Dans l'entendement de Dieu, il n'y a pas de faits contingents : Dieu a la notion complète de chaque être humain. Pour Leibniz tout obéit au principe de la raison : <<rien n'est sans raison>>.
Pour Leibniz, tout le mal qui arrive sur Terre est justifié dans l'esprit de Dieu, tout est rationnalisable\footnote{Théodicée}.
Dans cet univers de vérité nécessaire et rationnelle, il n'y a pas de liberté, tout est déjà prévu.
Le fait que la tradition philosophique, notamment rationnaliste donne un tel privilège aux vérités de raison sur le modèle des mathématiques conduit à une vision du réel où tout est nécessaire, où la liberté et donc la politique au sens fort ne trouve pas forcément sa place.
Au contraire, un acte libre apparaîtra toujours comme un fait non nécessaire.
Arendt remarque, en outre, que le propre de la vérité rationnelle est d'être produite par l'esprit humain, pas simplement donnée ou constatée, c'est une vérité que l'on fabrique.
En physique, on produit aussi la vérité à travers l'éxperimentation, c'est à dire une forme d'experience qui est elle même construite selon des ordres précis.
Cette idée que la vérité est produite, fabriquée a nécessairement pour conséquence du point de vue du sens commun de brouiller la frontière entre vérité et fiction voire entre vérité et mensonge.

La valorisation des vérités rationnelles a entraîné une dévalorisation des vérités simplement factuelles jugées plus faibles parce que contingentes, incertaines, mais ce sont aussi les vérités les plus importantes en politique, notamment quand elles prennent la forme de vérités historiques.
On peut donc penser que dans le développement des sciences modernes, quelque chose a favorisé une certaine méfiance vis-à-vis des faits et de leur solidité en tant que faits.
Les faits ne peuvent pas être justifiés au sens d'une rationnalisation, ils doivent être constatés, mais pour Arendt, ce n'est pas une forme plus faible de vérité.
Arendt suggère aussi que les vérités factuelles sont par nature beaucoup plus fragiles que les vérités rationnelles notamment parce qu'elles sont exposées à la possibilité d'une dissimulation radicale ou d'un oubli définitif.
On aurait pu imaginer qu'elles ne deviennent jamais vraies.
Un fait oublié ne ressurgit jamais.
\addcontentsline{toc}{subsection}{c) Les différents types de vérité.}
\subsection*{d) Vérité et opinion.}
Prendre pour modèle les vérités de raison et la certiture rationnelle a aussi pour conséquence de dévaloriser tout ce qui relève de l'opinion.
L'opinion c'est un jugement qui ne peut pas être entièrement justifié ou qui n'a que des justifications incomplètes.
Dans sa forme la plus faible, c'est la rumeur.
L'opinion peut aussi être une forme de jugement parfaitement légitime dans des domaines où on ne peut pas ésperer une certitude absolue.
Pour Hannah Arendt, l'opinion c'est la forme de pensée proprement politique et la pluralité des opinions dans le champ politique, loin d'être une imperfection est au contraire ce qui constitue notre réalité.
L'opinion a une grande valeur pour Hannah Arendt, c'est aussi une forme de pensée active, évolutive et dans laquelle on s'engage.
Dans la philosophie de Platon, et dans toute la tradition de pensée qui l'a suivi, l'opinion est toujours dévalorisée face à la certitude rationnelle.
Notre modèle dominant de vérité reste envisagé du point de vue de la connaissance.
Qu'il puisse y avoir une vérité de l'action, des intérêts, des intentions, voilà ce qui échappe à cette approche et c'est pourquoi toute cette tradition a sous-estimée le problème du mensonge.
Quel est, en effet, le contraire de la vérité.
Pour Platon c'est l'opinion, ou c'est l'erreur, ou c'est l'ignorance mais pas le mensonge.
Pour Arendt, c'est en général concernant les vérités de fait que le mensonge devient un problème et il s'oppose alors à la sincérité.
C'est pourquoi la question du mensonge, central en politique a été structurellement sous-estimée.
Hannah Arendt remarque de plus que le mensonge paraît être organisé, ce qui n'est pas le cas de l'erreur ou de l'illusion.
Dans la suite du texte, Hannah Arendt remarque cependant que l'opinion a retrouvé une certaine importance à partir du XVIIIème siècle; siècle des Lumières qui valorisait l'exigence de prenser par soi-même et de confronter les opinions dans l'espace public.
C'est à la même époque qu'on commence à parler des limites du rationnalisme.
Notamment, Kant dans \emph{La critique de la raison pure}, qui, comme son nom l'indique, montre les limites du savoir rationnel.
Kant montre qu'on ne peut connaître que des phénomènes et jamais les choses en soi.
Cependant, pour Arendt, le statut de ces vérités de fait reste durablement inchangé, c'est là dessus qu'elle va centrer son propos par la suite.
\addcontentsline{toc}{subsection}{d) Vérité et opinion.}
\pagebreak
\subsection*{e) Les vérités de fait.}
Tout l'enjeu, pour Arendt, va être de montrer que ces vérités factuelles, notamment historiques peuvent et doivent atteindre une grande objectivité même si on ne peut pas les extraire complètement du domaine de l'opinion et même si elles sont toujours soumises à la manipulation et au mensonge organisé.
Pour Arendt, il faut admettre que l'existence des camps de concentration est une vérité et non une simple opinion.
Pour Arendt, une vérité de fait, quand elle est importante a forcément une dimension politique et par conséquent, elle contrarie des intérêts, ce qui crée la tentation de la renvoyer à une simple opinion.
Pourtant, selon Arendt, ce sont ces vérités qui constituent le tissu de notre réalité commune. 
Le statut des vérités de fait est beaucoup plus ambigu que celui des vérités rationnelles : d'un côté, elles sont plus liées aux opinions, aux témoignages, aux interprétations divergentes, elles sont contingentes\footnote{Les faits n'ont aucune raison décisive d'être ce qu'ils sont}.
Un fait, par définition, aurait toujours pu ne pas arriver, à l'inverse d'une vérité rationnelle, mathématique.
D'un autre côté, cependant, Hannah Arendt tient à affirmer que les vérités de fait, qui ont été largement sous-estimées dans la tradition philosophique sont toutes aussi vraies que les vérités rationnelles.
Pour cette raison, les vérités factuelles doivent elles aussi transcender la divergence des opinions dans une certaine mesure.
Bien sûr, elles sont parfois difficiles à établir, elles peuvent être interprêtées de différentes manières, elles sont matière à opinion.
Néanmoins, pour Arendt, les faits sont aussi la matière des opinions : c'est-à-dire qu'il y a certains faits suffisament objectifs pour échapper à toute divergence d'opinion et pour permettre les débats d'opinions.
Il y a un noyau factuel qui échappe à toute discussion : c'est la vérité de fait.
Hannah Arendt caractérise ce noyau de vérité factuelle comme un ensemble de données élémentaires brutales qui doit échapper à toute discussion.
Il faut donc admettre qu'il existe des vérités factuelles nombreuses qui transcende le registre de l'opinion.
Pourtant, on peut encore imaginer que ces vérités factuelles soient attaquées, déformées ou manipulées voire détruites pour servir des intérêts politiques, mais dans ce cas, on aurait affaire à un mensonge de masse organisé.
Un tel mensonge est cependant bien plus qu'une opinion divergente. Contrairement à l'opinion, le mensonge est une action délibérée.
C'est face à cette possibilité que les vérités de fait sont fragiles.
Hannah Arendt a pris conscience de la possibilité de tels mensonges en analysant le totalitarisme (de l'URSS ou de l'Allemagne Nazie).
Un exemple classique de mensonge politique est l'incendie du Reichstag.
Arendt précise que le mensonge est bien une modalité d'action politique au sens où le propre de toute action est de modifier la réalité.
\addcontentsline{toc}{subsection}{e) Les vérités de fait.}
\pagebreak
\section*{\color{red}III. Le despotisme de la vérité}
\addcontentsline{toc}{section}{III. Le despotisme de la vérité}
Le despotisme est un régime politique autoritaire qui justifie son pouvoir par la volonté de faire le bien de ses sujets en sachant mieux qu'eux où réside ce bien : sur le modèle du père de famille (\emph{grec: despotes}) qui décide pour ses enfants.
Le despotisme se distincte donc de la tyrannie: le tyran se définissant comme quelqu'un qui cherche à imposer tous ses désirs et à régner par la crainte.
Il est donc légitime de parler d'un despotisme de la vérité s'il est vrai qu'elle se présente comme un pouvoir qui légitimement devrait s'imposer à tous de façon contraignante.
La partie 3 du texte d'Arendt traite de ce despotisme.
\subsection*{a) La vérité transcendante}
Au début du III, Hannah Arendt passe en revue les différents types de vérités qu'elle a examinés pour montrer que dans tous les cas, lorsqu'une vérité est établie, elle a tendance à effectivement exercer un pouvoir despotique.
Arendt soutient que les gens habitués à cotoyer la vérité ont une tendance à développer des caractères despotiques.
Arendt peut donc affirmer à propos de toutes les vérités même factuelles qu'elles ont un <<caractère despotique>>.
La conséquence paradoxale est que pour cette raison, la vérité est haïe des tyrans.
C'est ce qui fait que la vérité factuelle, bien qu'elle dépasse les opinions, peut toujours être menacée par les mensonges politiques.
\addcontentsline{toc}{subsection}{a) La vérité transcendante}
\subsection*{b) Opinion, représentativité, <<pensée élargie>>}
Par contraste avec ce despotisme de la vérité qui est inévitable, Hannah Arendt peut ensuite revenir à la question de l'opinion pour y voir un mode de pensée qui a sa valeur propre et qui n'est pas simplement un mode de jugement incertain, inférieur au savoir.
Pour Arendt, dans l'opinion, il y a un exercice de jugement beaucoup plus actif que dans le savoir puisqu'il s'agit de se former un avis en situation d'incertitude.
En général, l'opinion porte sur des sujets beaucoup plus riches et complexes que le savoir.
Surtout, l'opinion implique un rapport à autrui qui est beaucoup plus riche que le savoir, puisque le savoir, de son côté, rend inutile le dialogue.
Pour Arendt, quand on essaye véritablement de former une opinion, notamment sur un sujet politique, on va tenir compte dans notre pensée du point de vue des autres, pas seulement de leur avis, mais de leur position, de leurs intérêts, motivations, etc... On doit adopter le point de vue de l'autre.
Ce travail est ce que Hannah Arendt appelle, à la suite de Kant, la <<pensée élargie>>.
C'est ainsi qu'Hannah Arendt comprend l'idée de représentativité en politique : cette capacité dans l'opinion à se mettre à la place d'autrui, en tenant compte de toute la complexité de cette place. 
\addcontentsline{toc}{subsection}{b) Opinion, représentativité, <<pensée élargie>>}
\pagebreak
\subsection*{c) Les vérités philosophiques}
Ce sont des vérités rationnelles, au même titre que les vérités mathématiques, avec comme différence que les concepts utilisés sont moins nettement définies sur tous les aspects de la vie : <<Il vaut mieux subir l'injustice que la commettre>>.
C'est une vérité souvent utilisée par Arendt et énoncée par Platon. Socrate a été condamné injustement à mort, contrairement à Gorgias, le rhéteur qui gagne tous ses procès. D'après Socrate, <<nul n'est méchant volontairement>>: on ne peut vouloir profondément le mal, par définition même du bien et du mal, c'est toujours comme une faiblesse ou une exception.
Aux yeux de Platon, ces vérités sont logiques et rationnelles, comme en mathématique, simplement, elles heurtent tellement l'opinion, elles sont tellement paradoxales qu'elles ne peuvent pas vraiment se diffuser dans la société. 
C'est pour cette raison que les vérités philosophiques, bien qu'elles se présentent comme objectives et universelles, et en même temps restent souvent présentées comme les opinions d'un seul homme
\addcontentsline{toc}{subsection}{c) Les vérités philosophiques}
\section*{\color{red}IV. Le pouvoir du mensonge}
\addcontentsline{toc}{section}{IV. Le pouvoir du mensonge}
Les réflexions d'Arendt sur la vérité ont commencées avec l'idée que la vérité était impuissante.
Celui qui dit la vérité ne fait que constater ce qui existe, il se soumet au despotisme de la vérité, il a donc rarement un rôle actif, notamment au niveau politique.
À l'inverse, le menteur est toujours quelqu'un qui modifie la réalité par définition, et en ce sens, mentir c'est agir, le mensonge est puissant, c'est un instrument de pouvoir.
Le mensonge demande beaucoup plus d'efforts et d'investissements que la vérité
\subsection*{a) Mensonge et action}
Au début du $IV$, Hannah Arendt met au premier plan cette vision du mensonge pour montrer que, contrairement à l'erreur ou à l'illusion, le mensonge peut avoir une grande signification politique.
En effet, le mensonge étant délibéré, cherchant par définition à changer la réalité, ou à réecrire l'histoire est toujours très actif.
Au contraire, décrire la vérité est une attitude en général beaucoup plus passive.
Hannah Arendt joue sur le double sens du mot <<acteur>>.
Pour Arendt, le mensoge, comme capacité à s'éloigner de la réalité, voire à la transformer est bien une manifestation de notre liberté.
Au contraire, le respect de la vérité nous inscrit dans un universe déterministe.
L'historien recherche des causes et nie implicitement la liberté.
L'exemple que donne Arendt peut surprendre: dire que le soleil brille quand il pleut. C'est un exemple de mensonge où tout le monde voit la vérité où il pourrait presque paraître ridicule de mentir.
Et pourtant, c'est précisément là que le mensonge est le plus impressionnant, lorsqu'il parvient à tenir, à avoir une efficacité, à changer le réel, alors que tout le monde connaît la vérité.
C'est une antiphrase. Il y a un seul type de situation où dire la vérité devient actif: c'est lorsque tout le monde ment sur un sujet donné.
\addcontentsline{toc}{subsection}{a) Mensonge et action}
\subsection*{b) Mensonge et rationnalité}
En plus d'avoir une grande affinité avec la liberté, le mensonge est séduisant parce qu'il peut souvent être plus rationnel que la vérité, surtout la vérité de fait.
Les vérités de fait sont souvent surprenantes, dérangentes, contingentes, on dit qu'une vérité factuelle fait mentir nos discours.
Le mensonge apparaît souvent plus rationnel que les faits, plus cohérent, tout cela permet à Arendt d'interroger le rapport entre mensonge et rationnalité.
Ce faisant, elle vise peut-être la philosophie de Kant et ses thèses sur le mensonge\footnote{Fondements de la métaphysique des moeurs}. 
Pour Kant, mentir est immoral, mais pour une raison rationnelle : le menteur prend pour principe que l'autre croît à la vérité de ce qu'il dit.
Pour cette raison, le principe du mensonge ne peut jamais devenir une règle universelle.
Dans un univers où tout le monde ment à tout moment dans toute situation, on ne peut même plus parler.
Le mensonge se contredit dans son principe.
Quand Arendt décrit le mensonge organisé ou le mensonge de masse, elle semble renverser la perspective de Kant en décrivant des formes de mensonge qui semblent avoir tendance à s'universaliser.
\addcontentsline{toc}{subsection}{b) Mensonge et rationnalité}
\subsection*{c) Le mensonge organisé}
Pour Arendt, ce qui caractérise le mensonge de masse, c'est sa capacité à construire une réalité alternative pour concurrencer une réalité qui est connue de tout le monde, c'est ce dont elle va parler à propos du Vietnam, la politique est entièrement centrée sur la construction d'une image de la réalité.
Auparavant, les mensonges politiques portaient sur des choses dissimulées, soit des secrets, soit des intentions.
On peut s'interroger sur la nouveauté qui caractérise ce type de mensonge.
Arendt les explique par l'influence des médias de masse, mais il y a peut-être des raisons plus profondes.
Notre civilisation donne beaucoup de poids à l'idée de réalités alternatives et cela tient peut-être à l'influence de certains discours scientifiques.
Les théories sur la relativité du temps et de l'espace entretiennent dans les cultures l'idée qu'on peut, par exemple, voyager dans le temps : Retour vers le Futur.
Le problème de tels mensonges à grande échelle est qu'il menace toujours de se retourner contre leurs auteurs qui finissent en général par se mentir à eux-mêmes.
Dans \emph{Du mensonge en politique}, Hannah Arendt suggère que le président des États-Unis lui-même était la principale victime du mensonge organisé autour de la guerre du Vietnam, ce qui suggère que personne n'est en position de maîtrise vis-à-vis de tels mensonges.
Ce ne sont plus des mensonges qui sont destinés à des ennemis, à des publics bien délimités.
En cherchant à se substituer au réel, ils finissent fatalement par se retourner contre les menteurs eux-mêmes. 
\addcontentsline{toc}{subsection}{c) Le mensonge organisé}
\subsection*{d) La fonction politique de la vérité}
La dernière partie de vérité et politique, après avoir abordé la question du mensonge, Hannah Arendt en revient à son point de départ pour le nuancer quelque peu, s'il est vrai que la vérité est extérieure à la politique, qu'elle est en ce sens impuissante comme principe d'action à la différence du mensonge, qui peut toujours changer le monde, il n'en demeure pas moins que l'on peut attribuer à la vérité et aux diseurs de vérité une <<fonction politique>>. Il s'agit d'une fonction critique, celle qui consiste, par exemple pour l'historien, à rappeler les faits, pour le journaliste, à informer des faits, pour le philosophe, à interpréter leur signification, ce qui suppose toujours une <<impartialité>> qui par principe les fait sortir du champ politique.
Les diseurs de vérité ont donc une fonction politique qu'ils exercent de l'extérieur du champ politique.
Hannah Arendt suggère que leur impartialité les coupe nécessairement d'un certain rapport à autrui, les maintient dans une certaine solitude.
Le juge, le journaliste ou l'historien peuvent difficilement concilier avec leur activité un engagement politique trop marqué.
Cependant, de cette posiiton relativement extérieure au champ politique, les diseurs de vérité peuvent maintenir une objectivité qui sert de référence au débat politique et qui fait exister un monde objectivement commun <<conceptuellement nous pouvons appeler la vérité ce qu'on ne peut pas changer, métaphoriquement, elle est la sol sur lequel nous nous tenons et le ciel qui s'étend au dessus de nous>>.
À la fin du texte, de façon assez surprenante, Hannah Arendt mentionne parmis les diseurs de vérité, non seulement le juge ou le chercheur, mais aussi l'artiste et notamment le romancier
La catharsis c'est la purification des passions grâce au déroulement des faits et au scénario c'est ce que Arendt appelle la transfiguration artistique des émotions privées qui en fait des vérités objectives malgré tout
Pour Hannah Arendt, le romancier et l'historien ont en commun de raconter des histoires, de mettre des évènements en récit, soit fictifs ou réels, donc de donner un certain sens objectif à ces évènements.
Un grand roman, comme un mythe devient un élément de notre monde commun.
Hannah Arendt dit même que la poursuite désintéressée de la vérité est apparue en occident avec les récits d'Homère (H-O-M-È-R-E) dans l'Ilyade et l'Odyssée.
Dans l'Ilyade, Homère raconte la guerre entre les grecs et les troyens en vantant de façon égale les exploits des uns et des autres, donc avec impartialité.
Bien sûr, c'est une fiction ou un mythe, mais il y a déjà l'exigence d'impartialité <<dire ce qui est.
On peut comparer cette attitude de vérité concernant une guerre fictive avec le mensonge générale qui concerne une guerre réelle dans le cas de la guerre du VIETNAM.
Cette attitude de vérité est liée à la notion de récit alors que le mensonge va toujours être associé à la notion d'<<image>>.
Un récit véritable, va toujours vers la complexité, la multiplicité des points de vue.
L'image réduit toujours tout à un point de vue, à une intention, à une face des choses.
C'est en général ce que recherche le menteur 
\addcontentsline{toc}{subsection}{d) La fonction politique de la vérité}
\section*{\color{red}V. Le mensonge et la guerre}
Hannah Arendt décrit souvent le mensonge comme un instrument de violence.
Il est donc logique qu'il soit souvent utilisé dans la guerre d'autant qu'il est relativement inoffensif par rapport à d'autres armes.
Toute guerre comporte son lot de mensonges déstinés principalement à tromper l'ennemi, mais dans le cas des documents du Pentagone et de la guerre du Vietnam, on a affaire à un mensonge d'une autre nature au sens où les motivations mêmes de la guerre reposent sur un mensonge, sur le fait de vouloir entretenir un mensonge, pour entretenir une certaine image des États-Unis dans la lutte contre le communisme.
Ici, la guerre devient un instrument au service du mensonge.
En outre, ce mensonge ne vise pas à tromper l'ennemi, mais il est tellement profond, tellement organisé qu'il se propage dans les plus hautes sphères de l'administration américaine.
Hannah Arendt dit même que la principale victime est le président des États-Unis.
Il ne s'agit pas de tromper l'ennemi ou la population, il s'agit de se tromper soi-même.
Pour Arendt cela reste  un mensonge.\\[0.5cm]
Jusqu'en 1954, le Vietnam était une colonie Française, coupé en deux nord : chine communiste; sud : etats-unis militaire; guerres civiles dans le sud. Les États-Unis soutiennent le gouvernement du sud contre les rebelles, dans un contexte de guerre froide.
Doctrine du <<containment>> : contenir la contagion communiste. Les services de renseignement américains donnaient un compte rendu assez fidèle de la réalité en décrivant une guerre civile à visée nationaliste, une guerre post-coloniale.
L'image qui fait le fond du mensonge politique est de transformer cette guerre en guerre contre le communisme.
C'est cela qu'appelle Arendt l'image qu'on a cherché à substituer à la réalité. 
C'est une image comparable à l'image de la France gagnant la 2nde guerre mondiale.
En 1965, sous la présidence de ..., président démocrate qui a suivi Kennedy, les États-Unis bombardent la route ... qui ravitaillait les rebelles du sud.
Retrait progressif des troupes à partir de 1970 et accord de Paris en 1973.
La question se pose donc de savoir en quoi on a affaire ici à un véritable mensonge, qui a menti à qui et comment ce mensonge a pu prendre une telle ampleur.
Hannah Arendt souligne plus particulièrement l'intervention de deux facteurs nouveaux, le rôle des conseillers en communication ou en relations publiques et le rôle des technocrates qu'elle appelle des experts en solution de problèmes
\addcontentsline{toc}{section}{V. Le mensonge et la guerre}
\subsection*{a) La communication politique}
Pour Hannah Arendt, il y a un problème lié à la conception même de l'espace public dans une économie de marché et dans la société de consommation.
Les conseillers en communication qui sont entrés massivement dans l'administration américaine dans les années 60 sont issus en droite ligne de Madison Avenue, c'est à dire du quartier des publicitaires.
Le propre d'un publicitaire est de se détacher de la réalité de son produit, de toute exigence d'objectivité, pour se soucier uniquement de la perception du consommateur et donc d'une certaine image du produit qui a effectivement pour objectif de se substituer à la réalité.
Toute publicité est mensongère.
Pour Arendt, le principe de la publicité a ouvert un champ d'exploration infini et modifier les perceptions qu'on a. Les publicitaires devenus conseillers en communication sont appellés les spin doctors.
L'image des États-Unis est devenue un enjeu politique en soi, avec une dévalorisation de la réalité.
Pour Arendt, les principaux enjeux politiques pour défendre les Etats-Unis est de vendre une image.
Ce mode de pensée publicitaire qui envahit les pratiques de gouvernement selon Arendt a pour conséquence de faire de l'opinion et de sa manipulation un enjeu indépendant rapprochant la politique de la psychologie.
Pour Arendt cela se produit par une distortion de la notion d'espace public.
Normalement, elle considère que c'est le lieu central de notre rapport au réel dans la mesure où des jugements, des actions, et des réalités s'y exposent au regard de tous.
À la fin du texte de $1971$, Hannah Arendt souligne d'ailleurs que l'espace public Américain a finalement bien joué son rôle pendant la guerre du Vietnam notamment grâce à la presse permettant à l'opinion de saisir la réalité du conflit.
Paradoxalement, ce sont les milieux de pouvoir eux-mêmes qui se sont coupés de la réalité en raison de leur mentalité publicitaire, vers la fin du texte, Arendt disait qu'ils s'adressaient à des publics largement fantasmés et non pas aux publics qui auraient pu servir de contrepouvoir.
\addcontentsline{toc}{subsection}{a) La communication politique}
\subsection*{b) Les experts en solution des problèmes}
Hannah Arendt désigne de cette façon une catégorie nouvelle d'acteurs politiques qui interviennent dans l'administration américaine comme experts ou conseillers, qui orientent la décision politique en fonction de théories, notamment géopolitiques, inspirés du modèle scientifique.
C'est un des premiers grands exemples de l'application des sciences politiques à la réalité et c'est une autre façon en fait de s'éloigner du réel.
En effet, les sciences politiques, comme les sciences de la nature modélisent leur objet, simplement, dans les sciences de la nature, on applique ces modèles à des faits objectifs, résistants, nécessaires, qui obligent à corriger les modèles.
Les modèles appliqués à la politique rencontrent par définition des faits beaucoup plus contingents, évolutifs et sont utilisés comme principe d'action.
Ils modifient donc eux-mêmes la réalité qu'ils décrivent.
Pour Hannah Arendt, ces experts en solution de problèmes, théoriciens, technocrates ne sont pas des menteurs purs et simples mais ils ont en commum avec eux de vouloir se débarasser des faits qui les dérangent.
Ces deux facteurs nouveaux de l'action politique se sont conjugués selon Arendt pour justifier une stratégie militaire qui ne tenait plus compte de la réalité (les rapports des services des renseignements) et qui surtout n'avait même plus d'objectif réel.
Pour Arendt, l'objectif était d'entretenir une image des États-Unis et une théorie de leur rôle dans le monde au détriment de tout enjeu réellement atteignable.
\addcontentsline{toc}{subsection}{b) Les experts en solution des problèmes}
\end{document}