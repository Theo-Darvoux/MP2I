\documentclass[12pt]{article}
\usepackage{ragged2e}
\usepackage[left=2cm, right=2cm, top=2cm, bottom=2cm]{geometry}
\usepackage{color}
\usepackage{amsmath, amssymb, amsthm}
\usepackage{lastpage}
\usepackage{fancyhdr}
\usepackage[T1]{fontenc}
\usepackage{hyperref}

\title{Chapitre 3\\\large L'illusion des Liaisons}
\date{}
\author{}


\setlength{\headheight}{15pt}
\pagestyle{fancy}
\cfoot{\thepage\ sur \pageref*{LastPage}}

\hypersetup{
    colorlinks=true,
    citecolor=black,
    linktoc=all,
    linkcolor=blue
}

\renewcommand*\contentsname{Sommaire}

\begin{document}
\maketitle
\thispagestyle{fancy}

\hrule
\tableofcontents
\hrule
\fancyhead[L]{Théo DARVOUX}
\fancyhead[R]{MP2I Paul Valéry}
\fancyhead[C]{Français-Philosophie}
\pagebreak
\section*{\color{red}Introduction}
L'illusion est d'abord dans l'intrigue elle-même puisque les personnages entre-eux multiplient les mensonges, les manipulations mais c'est aussi le dispositif de l'oeuvre, qui se présente simplement comme un recueil de lettres.
Les personnages sont d'ailleurs fictifs : c'est une illusion produite par l'auteur, on pourrait dire que c'est une illusion d'illusion.
Pourtant, il faut bien que derrière ces illusions, il y ait un fond de vérité.
Même si tous les personnages mentent ou dissimulent, il faut quand même expliquer ce qui les pousse à écrire toutes ces lettres.
Il leur arrive parfois d'être très sincères, notamment, chaque personnage a son confident ou sa confidente, c'est un dépositaire des secrets.
Ce qui apparaît est la vérité de leurs désirs.
On peut se poser la question à propos de Laclos lui-même : qu'est-ce qui le pousse à écrire, quelle vérité recherche-t-il ?
Question d'autant plus intéressante qu'il ne parle jamais en son nom propre, il pourrait y avoir une raison morale, ou juste pour le plaisir, ce qui le rapprocherait des libertins qu'il met en scène.
Il ne donne jamais clairement son point de vue, l'auteur est un peu comme dieu vis-à-vis de sa création.
Il est absent et on ne sait pas très bien ce qu'il en pense et pourquoi il a créé tout ça.
Les jeux d'illusion, l'idée que le monde est un grand théâtre, c'est un thème baroque.
Dans l'esthétique baroque, derrière les illusions, il peut y avoir une vérité, derrière le désordre apparent, il peut y avoir un ordre caché.
C'est ce que le philosophie Leibniz a appelé l'harmonie préétablie.
Chaque être est comme un instrument qui joue sa partition, mais le créateur a tout écrit de façon harmonieuse.
Les liaisons dangereuses évoquent cette esthétique du contrepoint, chaque personnage déroule sa parition, affirme son point de vue, sans connaître celui des autre et c'est Laclos qui maîtrise la cohérence de l'ensemble.
Le lecteur du roman se trouve lui aussi placé dans cette position quasi divine.
Mais, on peut constater qu'aucun ordre caché, qu'aucune leçon morale particulièrement claire n'émerge de la confrontation de ces lettres.
Ce qui en ressort c'est surtout une impression de jubilation et la vérité de Laclos, comme la vérité de ses personnages, c'est peut-être la vérité de son désir.
\addcontentsline{toc}{section}{Introduction}

\section*{\color{red}I) Les <<projets>>.}
\subsection*{a) L'effacement du Créateur.}
Au début du roman, les projets ne manquent pas, Mme de Volanges projette de marier sa fille Cécile, qui sort du couvent.
Mme de Merteuil veut la faire corrompre par Valmont.
Valmont veut séduire la présidente de Tourvel.
Le projet de Laclos: pourquoi écrit-il ce roman ?
Au début du roman, Laclos met en scène son propre effacement en tant qu'auteur, faisant comme si il n'avait pas écrit le roman, comme si il avait simplement recueilli et compilé les lettres.
Le premier personnage du roman est l'éditeur.
Dans un second temps, Laclos se présente sous le masque du rédacteur.
Dans la préface du rédacteur, il prétend que son rôle d'élaguer les lettres inutiles.
Non seulement, il n'aurait pas produit le texte, mais en plus il l'aurait réduit <<ma mission ne s'étendait pas plus loin>>.
Le terme de mission est significatif, c'est un terme qu'utilisera Valmont, cela pose la question du but de toute cette entreprise.
Le rédacteur prétend qu'il a eu une intention morale.
Dans son agrément entre la variété des styles.

\addcontentsline{toc}{subsection}{a) L'effacement du Créateur.}
\pagebreak
\addcontentsline{toc}{section}{I) Les <<projets>>.}
\subsection*{b) Le <<bel objet>>}
Cécile de Volanges sort du couvent pour être mariée, c'est le projet de sa mère, mais elle est aussi au coeur du projet de Mme de Merteuil, qui veut la former, d'abord avec Danceny puis avec Valmont dans une sorte d'éducation libertine, c'est Merteuil qui l'appelle le <<bel objet>> dans la lettre II.
Les deux projet sont symétriques, c'est une victime sacrificielle pour sa mère et pour Merteuil.
Pourtant, Cécile va montrer qu'elle est aussi un sujet de désir. On devine qu'elle n'est pas complètement naïve par rapport à l'amour, elle a déjà des envies et des idées par elle-même.
Peut-être le projet de Merteuil va-t-il l'aider à mieux s'approprier son désir dans une société où règne le mariage arrangé.
Dans la lettre I, le désir est là même s'il n'a pas encore d'objet, immédiatement, la société resserre son emprise sur ce désir là à travers le système des mariages arrangés.
Cécile se rapproche des héroïnes de contes de fées qui sont au début de leur désir, qui est imprégné de stéréotypes, comme Blanche Neige de Walt Disney, la représentation du désir y est complètement narcissique, elle le trouve en elle-même.
C'est un désir purement intérieur, il n'y a pas d'objet. Il y a une sorte de déséquilibre entre la pensée et le réel.
Cécile se confie à son amie de couvent, qui est la seule personne qui reçoit des lettres mais n'en écrit pas.
\subsection*{c) Le projet de Merteuil.}
Merteuil dit à Valmont qu'il servira l'amour et la vengeance, Merteuil veut que cette histoire soit imprimée dans les mémoires de Valmont. Le <<sort inévitable (Lettre 2)>> est de devenir cocu.
Il y a une claire référence à l'École des Femmes de Molière, dans laquelle Arnolph, qui a la phobie d'être trompé par sa femme.
\subsection*{d) La mission d'amour de Valmont}
Valmont ne peut pas répondre tout de suite à la demande de Merteuil parce qu'il a son propre projet, qui est de séduire la présidente de Tourvel.
Ce qui l'attire chez elle, c'est un désir particulièrement transgressif, parce qu'elle est non seulement mariée mais aussi dévote, cette entreprise de séduction va prendre la forme d'une lutte, Valmont cherche à la convertir à l'amour, et elle cherche à le convertir, à le ramener dans le droit chemin.
Cela donne des lettres où le vocabulaire de la religion et de l'amour sont constamment mélangés de façon transgressive et blasphématoire.
L'amour est présenté  comme une religion <<Elle ne se doute pas de la divinité que j'y adore>>.
La religion et le désir se mélangent pour Valmont, il érotise la religion et le désir prend une dimension sacrée.
\section*{II) Les liens et les liaisons.}
Les liaisons sont des relations de séduction et de désir, qui sont très instables, en général illégitimes, Danceny-Cécile, Valmont-Cécile.
Mais à côté des liaisons, il y a les liens, c'est-à-dire les relations sociales légitimes qui unissent les personnages et qui sont justement menacées par les liaisons.
Par exemple, la famille, sans père ni mari, le mariage, les amitiés, la réputation. Tous ces liens forment un réseau, dont un des principaux buts est d'empêcher le développement des liaisons.
\pagebreak
\subsection*{a) La religion.}
Religion vient du Latin Religare, qui veut dire relier, c'est un lien social, qui garantit d'autres liens comme le mariage.
C'est Mme de Tourvel qui est le personnage religieux dans cette histoire, une dévote, elle félicite Cécile de son prochain mariage.
Mme de Tourvel a aussi été mariée par Mme de Volanges.
Lettres 21 et 22 : Valmont va faire une mise en scène en allant aider une famille qui se fait confisquer ses biens, pour redorer son blason aux yeux de la présidente de Tourvel, sachant qu'il est suivi.
\end{document}