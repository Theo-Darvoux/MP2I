\documentclass[12pt]{article}
\usepackage{ragged2e}
\usepackage[left=2cm, right=2cm, top=2cm, bottom=2cm]{geometry}
\usepackage{color}
\usepackage{amsmath, amssymb, amsthm}
\usepackage{lastpage}
\usepackage{fancyhdr}
\usepackage[T1]{fontenc}
\usepackage{hyperref}

\title{Chapitre 2\\\large Faire croire, pour faire : dissimulation, simulation et action dans \emph{Lorenzaccio} de Musset}
\date{}
\author{}


\setlength{\headheight}{15pt}
\pagestyle{fancy}
\cfoot{\thepage\ sur \pageref*{LastPage}}

\hypersetup{
    colorlinks=true,
    citecolor=black,
    linktoc=all,
    linkcolor=blue
}

\renewcommand*\contentsname{Sommaire}

\begin{document}
\maketitle
\thispagestyle{fancy}
\begin{center}
    \LARGE{apagn}
\end{center}
\hrule
\tableofcontents
\hrule
\fancyhead[L]{Théo DARVOUX}
\fancyhead[R]{MP2I Paul Valéry}
\fancyhead[C]{Français-Philosophie}
\pagebreak
\section*{\color{red}Introduction}
On a vu avec Arendt la grande proximité qu'il peut y avoir entre le mensonge et l'action politique.
L'intrigue de \emph{Lorenzaccio} de Musset se construit entièrement autour d'un acte à réaliser d'un projet brûlant à accomplir : l'assassinat du tyran.
Pour y parvenir, Lorenzo a choisi de se cacher, de simuler pour gagner sa confiance.
Habituellement, faire croire en politique, c'est dissimuler sous un masque légitime ou vertueux des intérêts et du vice.
Chez Lorenzo, c'est l'inverse, il dissimule des intentions vertueuses sous un masque de vice.
Le problème est qu'à force de porter de masque, il risque de réellement se corrompre : <<Le vice a été pour moi un vêtement, maintenant il est collé à ma peau.>> (III. 3).
Ce qui doit tout justifier à la fin, c'est un seul acte décisif et qui devrait favoriser la cause Républicaine.

Cependant, la dimension Républicaine et politique de cet acte ne va pas de soi, il s'agit aussi d'une vengeance personnelle, le fait que Lorenzo agisse seul et ne semble pas du tout croire aux capacités d'action des Républicains entretient un climat de grand pessimisme qui correspond aussi à ce que Musset pense de son époque.
Faire Croire dans cette pièce, cela pourrait aussi s'appliquer à ce problème : comment faire croire à la République, à cet idéal politique qui semble irréalisable.
Cela se redouble d'une autre question plus esthétique: comment faire croire au théàtre à ce qu'on fait représenter, comment recréer la Florence de la Renaissance sur scène, comment évoquer tous ces personnages et tous ces enjeux ?
De ce point de vue là, Musset s'est sans doute représenter dans le personnage de Tebaldeo : quel est le rôle de l'artiste, son rôle politique, à quoi doit-il faire croire ?
On peut s'interroger sur le détour historique choisi par Musset pour parler clairement de sa propre époque.
On pourrait dire que Musset, lui aussi chosit de porter un masque, de dissimuler ses intentions, mais en ayant tout de même en vue un acte politique dans l'écriture de sa pièce.
\addcontentsline{toc}{section}{Introduction}

\section*{\color{red}I) Entrer dans l'histoire.}
Cette expression désigne d'abord ce qu'on attend d'un acte d'exposition: exposer les enjeux de l'intrigue, présenter les personnages, présenter le contexte, surtout quand on est dans une époque complètement différente.
Entrer dans l'histoire, c'est aussi ce que veut faire Lorenzo, à sa façon, influencer par son acte héroïque le cours des évènements historiques.
Cependant, il y a une grande hésitation dans cette pièce sur la vision de l'Histoire.
D'un côté, elle pourrait être un processus qui mène vers le progrès, vers la réalisation des idéaux Républicains.
D'un autre côté, elle est perçue de façon beaucoup plus pessimiste, comme un processus chaotique dans le quel tout vient se corrompre, il est difficile de placer le personnage de Lorenzo entre ces deux visions de l'Histoire.
La République elle-même est dans la pièce aussi bien un enjeu de désir pour l'avenir qu'un souvenir nostalgique d'un passé perdu
\addcontentsline{toc}{section}{I) Entrer dans l'histoire}
\subsection*{a) Le mal du siècle}
On désigne ainsi le malaise de cette génération qui est arrivée après la Révolution française, après l'empire et pour qui la République était à la fois un souvenir et un idéal inatteignable : une génération très désenchantée.
La \underline{Liberté Guidant Le Peuple} : Allégorie qui mène les Républicains dans les rues de Paris, cependant, c'est aussi un idéal qui exige des sacrifices : la liberté chevauche un tas de cadavres, son visage est frois et inflexible.\\
L'enfant est inconscient, c'est la première victime du sacrifice Républicain. Dans ce tableau, la Liberté pourrait bien être une forme d'Hallucination Collective. Ce tableau exprime lui aussi un profond pessimisme, ou une incertitude sur la liberté républicaine.
\addcontentsline{toc}{subsection}{a) Le mal du siècle}
\subsection*{b) Le masque du vice}
Musset entre très vite dans l'intrigue du sujet: les intrigues du Duc et le rôle actif et particulièrement trouble de Lorenzo, à ce moment là, le spectateur n'a aucun moyen de connaître les véritables intentions de Lorenzo.
Dès sa première réplique, Lorenzo va très loin dans le vice.
Le portrait que Lorenzo fait de la jeune fille est aussi un portrait de lui-même car lui aussi se prostitue d'une certaine manière auprès du Duc pour gagner sa confiance, dès le départ, presque toutes les répliques de Lorenzo envers le Duc sont à double sens, ce double sens, Lorenzo le dit pour lui-même : <<\underline{Le vrai mérite est de frapper juste}>>.
Lorenzaccio est une grande pièce de l'implicite et du double sens.
Dès cette première réplique, l'acte qui constitue le fond ultime des propos se comprend d'un double sens : d'un côté le meurtre et de l'autre côté, l'acte sexuel.
À la fin, c'est en croyant aller retrouver une fille que le Duc aura rendez-vous avec la mort.
Dans les sociétés humaines, ce sont les deux grandes figures de l'acte : soit la sexualité soit la violence.
Cela correspond à ce que Freud désignait comme les deux grandes pulsions fondamentales : Éros et Thanatos.
La scène 2 est une scène de contexte où l'auteur plante le décor de Florence.
\addcontentsline{toc}{subsection}{b) Le masque du vice}
\subsection*{c) Les masques de la vertu}
Le cardinal représente un sorte d'image inversée de Lorenzo : il dissimule du vice sous un habit de vertu.
En réalité, il dissimule à peine.
La marquise Cibo est un personnage encore plus complexe.
Elle a une première apparence vertueuse qui est d'être fidèle à son mari.
En réalité, elle va le tromper avec le Duc, qui lui fait une cour assidüe et menaçante, et plus profondément encore, elle veut s'en servir pour faire avancer des idées Républicaines.
Sa position ressemble un peu à celle de Lorenzo puisqu'elle se compromet aussi auprès du Duc.
Pour sa part, sa stratégie va complètement échouer.
C'est le cardinal qui a la fin sera le personnage triomphant. 
L'implicite est du côté de l'agressivité.
\addcontentsline{toc}{subsection}{c) Les masques de la vertu}
\subsection*{d) Le masque de la lâcheté}
Si Lorenzo n'était qu'un personnage vicieux et violent comme tous ceux qui l'entourent, il pourraît apparaître comme une menace pour le pouvoir et on ne comprendrait pas complètement sa proximité avec le Duc, il faut donc que Lorenzo soit aussi lâche, il doit être une femmelette.
Le soupçon d'homosexualité est permanent dans la pièce, même s'il n'est jamais explicitement formulé.
Cela pourrait expliquer pourquoi le duc protège à ce point Lorenzo.
Les autres personnages perçoivent forcément quelque chose de ce genre et ne manquent pas de l'insinuer.
La décapitation des statues de Constantin est un élément du passé qui poursuit Lorenzo, qui montre sa capacité à agir, ça annonce l'acte de tuer le tyran.
Lorenzo lisait Plutarque. <<Le peuple appelle Lorenzo "Lorenzaccio", on sait qu'il dirige vos plaisirs>>. Le sous-entendu d'homosexualité, qui n'est jamais rendu explicite, explique pourtant le titre de la pièce.
<<Il se fourre partout et me dit tout>>. Dans la scène 4, sire Maurice provoque Lorenzo, l'insulte et veut le faire chasser de la cour, finit par sortir son épée.
Lorenzo ne peut pas se battre et finit par s'évanouir devant l'épée. Le texte ne permet pas de dire à quel point cet évanouissement est simulé, cela sert bien les intérêts de Lorenzo, il va devenir <<la fable de Florence>>.
On peut aussi imaginer que Lorenzo a réellement peur devant cette arme, elle évoque aussi peut-être pour lui la perspective du meurtre final.
Le fond de l'implicite de la pièce est libre à l'interprétation.
La scène de l'épée marque l'instant culminant de l'acte I. C'est un quasi passage à l'acte qui préfigure le meurtre de la fin, sauf que Lorenzo, justement n'agit pas pour cultiver son image de lâcheté.
Il est remarquable que cette scène hésite fortement entre le comique et le tragique à l'image du Duc dont on se sait pas s'il est sérieux ou s'il plaisante.
Cette hésitation entre comédie et tragédie est clairement inspirée du style de Shakespeare, de son fameux mélange des genres. La tragédie se caractérise par son sérieux, le comique est en général beaucoup plus distancié, ici, on ne bascule pas complètement dans le tragique mais on n'en est pas loin.
\addcontentsline{toc}{subsection}{d) Le masque de la lâcheté}
\subsection*{e) <<La Fable de Florence>>}
C'est ainsi que la mère de Lorenzo le qualifie dans la scène 6, en référence directe à l'épisode de l'épée, c'est celui dont tout le monde se moque.
La scène entre Marie et Catherine permet de compléter le portrait du personnage de Lorenzo en évoquant sa jeunesse et des traits plus authentiques de sa personnalité. Mais pour elles, il est d'autant plus douloureux de voir ce qu'il est devenu.
Lorenzo jeune était un <<soleil levant>>. La scène se déroule au soleil couchant.
C'est cette ambiance crépusculaire qui marque la fin de l'acte I, avec les ombres des bannis qui s'éloignent de la ville.
Marie décrit son fils comme un spectre hideux. La même expression est ensuite utilisée par un banni pour désigner la ville de Florence, elle aussi devenue un spectre hideux.
Le personnage de Lorenzo, image d'une espérance déçue, se confond donc avec le destin de Florence : <<adieu Florence la batarde, spectre hideux de l'antique Florence>>.
Lorenzo est effectivement devenu un homme illustre, mais dans le mauvais sens du terme, il est un anti-héros, comme Deadpool, une fable, mais ce terme suggère aussi qu'il ne s'agit que d'une fiction.
\addcontentsline{toc}{subsection}{e) <<La Fable de Florence>>}
\section*{\color{red}II) Rêves d'action}
L'acte I se termine donc dans une ambiance desespérée, l'acte II pour sa part, va présenter toute une gallerie de personnages qui ont en commun un désir de changement mais qui le conçoivent sur des modes bien différents et sans trop savoir comment le concrétiser: Philippe Strozzi, le vieux républicain idéaliste et rêveur, par contraste, son fils Pierre Strozzi, uniquement dans l'impulsivité, Tebaldeo le peintre se demandant comment peindre Florence, le Cardinal rêve aussi de changements mais sur le mode de l'ambition, enfin la Marquise de Cibo qui se demande quoi faire de sa liaison dangereuse avec le Duc.
On est dans un moment où les choses se précipitent. C'est caractéristique du temps théâtral, il faut que les évènements de précipitent.
\addcontentsline{toc}{section}{II) Rêves d'action}
\subsection*{a) Le rêve républicain}
Philippe Strozzi se présente lui-même comme un philosophe, un vieux rêveur mais il est décalé par rapport aux exigences de l'action.
Philippe Strozzi se demande ce qu'on peut faire avec des mots, ce que peut faire un philosophe, sachant que la République est avant tout un mot ou une idée <<La République, il nous faut ce mot là, et quand ce ne serait qu'un mot, c'est quelque chose>>.
C'est sans doute aussi une réflexion de Musset sur le pouvoir de l'écrivain.
Le vieux Strozzi s'exalte en prononçant ce mot.
Il s'exprime dans un monologue, le monologue signifie par convention qu'on entend ses pensées, mais Musset joue sur la convention, puisqu'à la fin il est interrompu par son fils qui arrive.
On a vraiment affaire à un vieillard qui parle tout seul. 
La suite de la scène fait intervenir les deux fils de Philippe Strozzi, confrontés à la provocation de Salviatti. À ce moment là, le personnage de leur père s'efface du dialogue.
Il y a un effet de symétrie entre le mot de République agité au début de la scène par le vieux Strozzi comme un idéal peut-être vide et le mot rapporté par Léon Strozzi à propos de sa sœur qui déclenche une action immédiate <<ta sœur la pute>>. D'un côté, l'idéalisme creux du vieux Strozzi, de l'autre, l'impulsivité irréfléchie de son fils.
D'un côté, on aimerait faire croire à un idéal politique difficile à réaliser, d'un autre côté, on arrive trop facilement à faire croire aux pires rumeurs ou aux médisances
\addcontentsline{toc}{subsection}{a) Le rêve républicain}
\subsection*{b) Les rêves d'artiste}
La scène 2 introduit le personnage de Tebaldeo, jeune peintre florentin qui incarne de façon plus générale la figure de l'artiste avec la question de ce qu'il peut apporter à la société, à la différence de Lorenzo, Tebaldeo n'est pas un homme d'action, et pourtant, il a un certain nombre de points communs avec Lorenzo : c'est aussi le reflet de Musset lui-même, l'enjeu de la scène est de savoir comment <<peindre Florence>>, ce qui est aussi la préoccupation de Musset avec sa pièce.
Mais au début de la scène, l'art est d'abord pensé en lien avec la religion dans le grand monologue de Valori: la force du catholicisme est d'avoir su mettre l'art à son service, plus que les autres religions, le catholicisme est la religion de l'incarnation : il ne s'agit pas juste de faire croire, mais de séduire, aider à croire.
Pour Valori, le catholicisme est un ensemble mondain, inscrit dans le monde et qui se met au service de l'art. Derrière cette remarque, il y a aussi l'ambition politique des catholiques.
Le peintre Tebaldeo va a la rencontre de Valori et de Lorenzo, ce qui semble suggérer qu'il a lui aussi des ambitions politiques et qu'il voudrait mettre son art au service de la religion.
Dans cette scène, Tebaldeo est sûrement ironique <<ce ne serait qu'un parfum stérile, si elle ne montait à Dieu>>.
L'ambivalence de Tebaldeo reste plus ouverte que celle de Lorenzo.
Valori propose à Tebaldeo de l'engager à son service, mais Tebaldeo esquive <<c'est trop d'honneur>>, plus la scène avance et plus il va se révéler comme un esprit libre.
Pour Tebaldeo, le rôle de l'artiste est de réaliser des rêves. C'est un thème récurrent de la pièce.
Dans le cas du théâtre, cette formule a un sens particulier, on matérialise des rêves.
Cette réalisation ne va pas jusqu'à une action, ce qui le distingue de Lorenzo.
On pourrait représenter Florence de manière idéalisée <<Mère>>, comme ce qu'on veut quelle soit ou telle qu'elle est <<Catin>>, dans le rabaissement.
Cette dialectique de l'idéalisation et du rabaissement est ensuite traduite en différentes façons de considérer les femmes, c'est soit la mère soit la prostituée.
Il n'y a pas d'équivalent du côté masculin.
Freud montre que cette dualité de la figure féminine vient des différentes façons de considérer l'amour dans les étapes de notre vie psychique. Le premier modèle d'amour, c'est celui que l'enfant reçoit de ses parents, peut-être plus particulièrement de la mère, souvent présenté comme un modèle d'amour pur, inconditionnel, désinteressé, et surtout coupé de toute référence explicite à la sexualité: voir les représentations de la vierge marie par Raphaël.
Dans la suite du developpement psychique, cette idéalisation de la figure féminine maternelle peut avoir comme conséquence un rejet ou un rabaissement des autres figures féminines associées à la sexualité qu'on va donc désigner comme des prostituées.
Pour Freud, lorsque ce mépris de la sexualité féminine peut aussi être une explication de l'homosexualité masculine.
Le personnage de Lorenzo semble parfaitement bien s'inscrire dans ce schéma.
\addcontentsline{toc}{subsection}{b) Les rêves d'artiste}
\subsection*{c) Rêves de pouvoir}
Cette scène est un dialogue entre la Marquise Cibo et le Cardinal, qui se caractérisent tous les deux par leurs désirs et leur ambitions liées à la proximité avec le Duc.
Dans cette scène, le cardinal entend la Marquise en confession. Dans une pièce où tout le monde ment, la confession représenterait un moment de vérité complètement dévoilée, sauf que le cardinal veut s'en servir politiquement.
La marquise ne dit pas tout dans sa confession.
\addcontentsline{toc}{subsection}{c) Rêves de pouvoir}
\end{document}