\documentclass[12pt]{article}
\usepackage{ragged2e}
\usepackage[left=2cm, right=2cm, top=2cm, bottom=2cm]{geometry}
\usepackage{color}
\usepackage{amsmath, amssymb, amsthm}
\usepackage{lastpage}
\usepackage{fancyhdr}
\usepackage[T1]{fontenc}
\usepackage{hyperref}

\title{Chapitre 2\\\large Faire croire, pour faire : dissimulation, simulation et action dans \emph{Lorenzaccio} de Musset}
\date{}
\author{}


\setlength{\headheight}{15pt}
\pagestyle{fancy}
\cfoot{\thepage\ sur \pageref*{LastPage}}

\hypersetup{
    colorlinks=true,
    citecolor=black,
    linktoc=all,
    linkcolor=blue
}

\renewcommand*\contentsname{Sommaire}

\begin{document}
\maketitle
\thispagestyle{fancy}
\begin{center}
    \LARGE{apagn}
\end{center}
\hrule
\tableofcontents
\hrule
\fancyhead[L]{Théo DARVOUX}
\fancyhead[R]{MP2I Paul Valéry}
\fancyhead[C]{Français-Philosophie}
\pagebreak
\section*{\color{red}Introduction}
On a vu avec Arendt la grande proximité qu'il peut y avoir entre le mensonge et l'action politique.
L'intrigue de \emph{Lorenzaccio} de Musset se construit entièrement autour d'un acte à réaliser d'un projet brûlant à accomplir : l'assassinat du tyran.
Pour y parvenir, Lorenzo a choisi de se cacher, de simuler pour gagner sa confiance.
Habituellement, faire croire en politique, c'est dissimuler sous un masque légitime ou vertueux des intérêts et du vice.
Chez Lorenzo, c'est l'inverse, il dissimule des intentions vertueuses sous un masque de vice.
Le problème est qu'à force de porter de masque, il risque de réellement se corrompre : <<Le vice a été pour moi un vêtement, maintenant il est collé à ma peau.>> (III. 3).
Ce qui doit tout justifier à la fin, c'est un seul acte décisif et qui devrait favoriser la cause Républicaine.

Cependant, la dimension Républicaine et politique de cet acte ne va pas de soi, il s'agit aussi d'une vengeance personnelle, le fait que Lorenzo agisse seul et ne semble pas du tout croire aux capacités d'action des Républicains entretient un climat de grand pessimisme qui correspond aussi à ce que Musset pense de son époque.
Faire Croire dans cette pièce, cela pourrait aussi s'appliquer à ce problème : comment faire croire à la République, à cet idéal politique qui semble irréalisable.
Cela se redouble d'une autre question plus esthétique: comment faire croire au théàtre à ce qu'on fait représenter, comment recréer la Florence de la Renaissance sur scène, comment évoquer tous ces personnages et tous ces enjeux ?
De ce point de vue là, Musset s'est sans doute représenter dans le personnage de Tebaldeo : quel est le rôle de l'artiste, son rôle politique, à quoi doit-il faire croire ?
On peut s'interroger sur le détour historique choisi par Musset pour parler clairement de sa propre époque.
On pourrait dire que Musset, lui aussi chosit de porter un masque, de dissimuler ses intentions, mais en ayant tout de même en vue un acte politique dans l'écriture de sa pièce.
\addcontentsline{toc}{section}{Introduction}

\section*{\color{red}I) Entrer dans l'histoire.}
Cette expression désigne d'abord ce qu'on attend d'un acte d'exposition: exposer les enjeux de l'intrigue, présenter les personnages, présenter le contexte, surtout quand on est dans une époque complètement différente.
Entrer dans l'histoire, c'est aussi ce que veut faire Lorenzo, à sa façon, influencer par son acte héroïque le cours des évènements historiques.
Cependant, il y a une grande hésitation dans cette pièce sur la vision de l'Histoire.
D'un côté, elle pourrait être un processus qui mène vers le progrès, vers la réalisation des idéaux Républicains.
D'un autre côté, elle est perçue de façon beaucoup plus pessimiste, comme un processus chaotique dans le quel tout vient se corrompre, il est difficile de placer le personnage de Lorenzo entre ces deux visions de l'Histoire.
La République elle-même est dans la pièce aussi bien un enjeu de désir pour l'avenir qu'un souvenir nostalgique d'un passé perdu
\addcontentsline{toc}{section}{I) Entrer dans l'histoire}
\subsection*{a) Le mal du siècle}
On désigne ainsi le malaise de cette génération qui est arrivée après la Révolution française, après l'empire et pour qui la République était à la fois un souvenir et un idéal inatteignable : une génération très désenchantée.
La \underline{Liberté Guidant Le Peuple} : Allégorie qui mène les Républicains dans les rues de Paris, cependant, c'est aussi un idéal qui exige des sacrifices : la liberté chevauche un tas de cadavres, son visage est frois et inflexible.\\
L'enfant est inconscient, c'est la première victime du sacrifice Républicain. Dans ce tableau, la Liberté pourrait bien être une forme d'Hallucination Collective. Ce tableau exprime lui aussi un profond pessimisme, ou une incertitude sur la liberté républicaine.
\addcontentsline{toc}{subsection}{a) Le mal du siècle}
\subsection*{b) Le masque du vice}
Musset entre très vite dans l'intrigue du sujet: les intrigues du Duc et le rôle actif et particulièrement trouble de Lorenzo, à ce moment là, le spectateur n'a aucun moyen de connaître les véritables intentions de Lorenzo.
Dès sa première réplique, Lorenzo va très loin dans le vice.
Le portrait que Lorenzo fait de la jeune fille est aussi un portrait de lui-même car lui aussi se prostitue d'une certaine manière auprès du Duc pour gagner sa confiance, dès le départ, presque toutes les répliques de Lorenzo envers le Duc sont à double sens, ce double sens, Lorenzo le dit pour lui-même : <<\underline{Le vrai mérite est de frapper juste}>>.
Lorenzaccio est une grande pièce de l'implicite et du double sens.
Dès cette première réplique, l'acte qui constitue le fond ultime des propos se comprend d'un double sens : d'un côté le meurtre et de l'autre côté, l'acte sexuel.
À la fin, c'est en croyant aller retrouver une fille que le Duc aura rendez-vous avec la mort.
Dans les sociétés humaines, ce sont les deux grandes figures de l'acte : soit la sexualité soit la violence.
Cela correspond à ce que Freud désignait comme les deux grandes pulsions fondamentales : Éros et Thanatos.
La scène 2 est une scène de contexte où l'auteur plante le décor de Florence.
\addcontentsline{toc}{subsection}{b) Le masque du vice}
\subsection*{c) Les masques de la vertu}
Le cardinal représente un sorte d'image inversée de Lorenzo : il dissimule du vice sous un habit de vertu.
En réalité, il dissimule à peine.
La marquise Cibo est un personnage encore plus complexe.
Elle a une première apparence vertueuse qui est d'être fidèle à son mari.
En réalité, elle va le tromper avec le Duc, qui lui fait une cour assidüe et menaçante, et plus profondément encore, elle veut s'en servir pour faire avancer des idées Républicaines.
Sa position ressemble un peu à celle de Lorenzo puisqu'elle se compromet aussi auprès du Duc.
Pour sa part, sa stratégie va complètement échouer.
C'est le cardinal qui a la fin sera le personnage triomphant. 
L'implicite est du côté de l'agressivité.
\addcontentsline{toc}{subsection}{c) Les masques de la vertu}
\subsection*{d) Le masque de la lâcheté}
Si Lorenzo n'était qu'un personnage vicieux et violent comme tous ceux qui l'entourent, il pourraît apparaître comme une menace pour le pouvoir et on ne comprendrait pas complètement sa proximité avec le Duc, il faut donc que Lorenzo soit aussi lâche, il doit être une femmelette.
Le soupçon d'homosexualité est permanent dans la pièce, même s'il n'est jamais explicitement formulé.
Cela pourrait expliquer pourquoi le duc protège à ce point Lorenzo.
Les autres personnages perçoivent forcément quelque chose de ce genre et ne manquent pas de l'insinuer.
La décapitation des statues de Constantin est un élément du passé qui poursuit Lorenzo, qui montre sa capacité à agir, ça annonce l'acte de tuer le tyran.
Lorenzo lisait Plutarque. <<Le peuple appelle Lorenzo "Lorenzaccio", on sait qu'il dirige vos plaisirs>>. Le sous-entendu d'homosexualité, qui n'est jamais rendu explicite, explique pourtant le titre de la pièce.
<<Il se fourre partout et me dit tout>>. Dans la scène 4, sire Maurice provoque Lorenzo, l'insulte et veut le faire chasser de la cour, finit par sortir son épée.
Lorenzo ne peut pas se battre et finit par s'évanouir devant l'épée. Le texte ne permet pas de dire à quel point cet évanouissement est simulé, cela sert bien les intérêts de Lorenzo, il va devenir <<la fable de Florence>>.
On peut aussi imaginer que Lorenzo a réellement peur devant cette arme, elle évoque aussi peut-être pour lui la perspective du meurtre final.
Le fond de l'implicite de la pièce est libre à l'interprétation.
La scène de l'épée marque l'instant culminant de l'acte I. C'est un quasi passage à l'acte qui préfigure le meurtre de la fin, sauf que Lorenzo, justement n'agit pas pour cultiver son image de lâcheté.
Il est remarquable que cette scène hésite fortement entre le comique et le tragique à l'image du Duc dont on se sait pas s'il est sérieux ou s'il plaisante.
Cette hésitation entre comédie et tragédie est clairement inspirée du style de Shakespeare, de son fameux mélange des genres. La tragédie se caractérise par son sérieux, le comique est en général beaucoup plus distancié, ici, on ne bascule pas complètement dans le tragique mais on n'en est pas loin.
\addcontentsline{toc}{subsection}{d) Le masque de la lâcheté}
\subsection*{e) <<La Fable de Florence>>}
C'est ainsi que la mère de Lorenzo le qualifie dans la scène 6, en référence directe à l'épisode de l'épée, c'est celui dont tout le monde se moque.
La scène entre Marie et Catherine permet de compléter le portrait du personnage de Lorenzo en évoquant sa jeunesse et des traits plus authentiques de sa personnalité. Mais pour elles, il est d'autant plus douloureux de voir ce qu'il est devenu.
Lorenzo jeune était un <<soleil levant>>. La scène se déroule au soleil couchant.
C'est cette ambiance crépusculaire qui marque la fin de l'acte I, avec les ombres des bannis qui s'éloignent de la ville.
Marie décrit son fils comme un spectre hideux. La même expression est ensuite utilisée par un banni pour désigner la ville de Florence, elle aussi devenue un spectre hideux.
Le personnage de Lorenzo, image d'une espérance déçue, se confond donc avec le destin de Florence : <<adieu Florence la batarde, spectre hideux de l'antique Florence>>.
Lorenzo est effectivement devenu un homme illustre, mais dans le mauvais sens du terme, il est un anti-héros, comme Deadpool, une fable, mais ce terme suggère aussi qu'il ne s'agit que d'une fiction.
\addcontentsline{toc}{subsection}{e) <<La Fable de Florence>>}
\section*{\color{red}II) Rêves d'action}
L'acte I se termine donc dans une ambiance desespérée, l'acte II pour sa part, va présenter toute une gallerie de personnages qui ont en commun un désir de changement mais qui le conçoivent sur des modes bien différents et sans trop savoir comment le concrétiser: Philippe Strozzi, le vieux républicain idéaliste et rêveur, par contraste, son fils Pierre Strozzi, uniquement dans l'impulsivité, Tebaldeo le peintre se demandant comment peindre Florence, le Cardinal rêve aussi de changements mais sur le mode de l'ambition, enfin la Marquise de Cibo qui se demande quoi faire de sa liaison dangereuse avec le Duc.
On est dans un moment où les choses se précipitent. C'est caractéristique du temps théâtral, il faut que les évènements de précipitent.
\addcontentsline{toc}{section}{II) Rêves d'action}
\subsection*{a) Le rêve républicain}
Philippe Strozzi se présente lui-même comme un philosophe, un vieux rêveur mais il est décalé par rapport aux exigences de l'action.
Philippe Strozzi se demande ce qu'on peut faire avec des mots, ce que peut faire un philosophe, sachant que la République est avant tout un mot ou une idée <<La République, il nous faut ce mot là, et quand ce ne serait qu'un mot, c'est quelque chose>>.
C'est sans doute aussi une réflexion de Musset sur le pouvoir de l'écrivain.
Le vieux Strozzi s'exalte en prononçant ce mot.
Il s'exprime dans un monologue, le monologue signifie par convention qu'on entend ses pensées, mais Musset joue sur la convention, puisqu'à la fin il est interrompu par son fils qui arrive.
On a vraiment affaire à un vieillard qui parle tout seul. 
La suite de la scène fait intervenir les deux fils de Philippe Strozzi, confrontés à la provocation de Salviatti. À ce moment là, le personnage de leur père s'efface du dialogue.
Il y a un effet de symétrie entre le mot de République agité au début de la scène par le vieux Strozzi comme un idéal peut-être vide et le mot rapporté par Léon Strozzi à propos de sa sœur qui déclenche une action immédiate <<ta sœur la pute>>. D'un côté, l'idéalisme creux du vieux Strozzi, de l'autre, l'impulsivité irréfléchie de son fils.
D'un côté, on aimerait faire croire à un idéal politique difficile à réaliser, d'un autre côté, on arrive trop facilement à faire croire aux pires rumeurs ou aux médisances
\addcontentsline{toc}{subsection}{a) Le rêve républicain}
\subsection*{b) Les rêves d'artiste}
La scène 2 introduit le personnage de Tebaldeo, jeune peintre florentin qui incarne de façon plus générale la figure de l'artiste avec la question de ce qu'il peut apporter à la société, à la différence de Lorenzo, Tebaldeo n'est pas un homme d'action, et pourtant, il a un certain nombre de points communs avec Lorenzo : c'est aussi le reflet de Musset lui-même, l'enjeu de la scène est de savoir comment <<peindre Florence>>, ce qui est aussi la préoccupation de Musset avec sa pièce.
Mais au début de la scène, l'art est d'abord pensé en lien avec la religion dans le grand monologue de Valori: la force du catholicisme est d'avoir su mettre l'art à son service, plus que les autres religions, le catholicisme est la religion de l'incarnation : il ne s'agit pas juste de faire croire, mais de séduire, aider à croire.
Pour Valori, le catholicisme est un ensemble mondain, inscrit dans le monde et qui se met au service de l'art. Derrière cette remarque, il y a aussi l'ambition politique des catholiques.
Le peintre Tebaldeo va a la rencontre de Valori et de Lorenzo, ce qui semble suggérer qu'il a lui aussi des ambitions politiques et qu'il voudrait mettre son art au service de la religion.
Dans cette scène, Tebaldeo est sûrement ironique <<ce ne serait qu'un parfum stérile, si elle ne montait à Dieu>>.
L'ambivalence de Tebaldeo reste plus ouverte que celle de Lorenzo.
Valori propose à Tebaldeo de l'engager à son service, mais Tebaldeo esquive <<c'est trop d'honneur>>, plus la scène avance et plus il va se révéler comme un esprit libre.
Pour Tebaldeo, le rôle de l'artiste est de réaliser des rêves. C'est un thème récurrent de la pièce.
Dans le cas du théâtre, cette formule a un sens particulier, on matérialise des rêves.
Cette réalisation ne va pas jusqu'à une action, ce qui le distingue de Lorenzo.
On pourrait représenter Florence de manière idéalisée <<Mère>>, comme ce qu'on veut quelle soit ou telle qu'elle est <<Catin>>, dans le rabaissement.
Cette dialectique de l'idéalisation et du rabaissement est ensuite traduite en différentes façons de considérer les femmes, c'est soit la mère soit la prostituée.
Il n'y a pas d'équivalent du côté masculin.
Freud montre que cette dualité de la figure féminine vient des différentes façons de considérer l'amour dans les étapes de notre vie psychique. Le premier modèle d'amour, c'est celui que l'enfant reçoit de ses parents, peut-être plus particulièrement de la mère, souvent présenté comme un modèle d'amour pur, inconditionnel, désinteressé, et surtout coupé de toute référence explicite à la sexualité: voir les représentations de la vierge marie par Raphaël.
Dans la suite du developpement psychique, cette idéalisation de la figure féminine maternelle peut avoir comme conséquence un rejet ou un rabaissement des autres figures féminines associées à la sexualité qu'on va donc désigner comme des prostituées.
Pour Freud, lorsque ce mépris de la sexualité féminine peut aussi être une explication de l'homosexualité masculine.
Le personnage de Lorenzo semble parfaitement bien s'inscrire dans ce schéma.
\addcontentsline{toc}{subsection}{b) Les rêves d'artiste}
\subsection*{c) Rêves de pouvoir}
Cette scène est un dialogue entre la Marquise Cibo et le Cardinal, qui se caractérisent tous les deux par leurs désirs et leur ambitions liées à la proximité avec le Duc.
Dans cette scène, le cardinal entend la Marquise en confession. Dans une pièce où tout le monde ment, la confession représenterait un moment de vérité complètement dévoilée, sauf que le cardinal veut s'en servir politiquement.
La marquise ne dit pas tout dans sa confession, après la confession incomplète avec le Cardinal, la Marquise reste seule pour un long monologue, qui est peut-être sa véritable confession.
\addcontentsline{toc}{subsection}{c) Rêves de pouvoir}
\subsection*{d) Dans l'intimité de Lorenzo}
Plusieurs parties : première partie avec Marie et Catherine, la seconde avec l'arrivée de l'oncle Bindo et Venturi, troisième partie où le Duc fait une entrée tonitruante.
Cette scène est une sorte de triptyque qui se passe chez Lorenzo avec sa mère et sa tante.
Ce qui pose problème est l'arrivée du Duc.
Le début de la scène met en place cette sorte d'intimité lorsque Marie et Catherine rappellent à Lorenzo d'où il vient et qui il était.
Dans le dialogue avec Marie et Catherine, il est question du vrai Lorenzo, du Lorenzino d'autre fois, de Lorenzino, de celui qu'on aimerait voir revenir : <<Comme tu reviens de bonne heure>>, c'est justement comme un revenant qu'il est évoqué par Marie, comme un spectre, un double positif du Lorenzo monstrueux que l'on connaît, l'intensité émotionelle de cette scène parvient à fendre l'armure de Lorenzo, qui montre pour la première fois son vrai visage.
Le Lorenzo d'autrefois aimait l'histoire de Brutus qui a vengé Lucrèce en tuant Tarquin et en permettant l'établissement de la République à Rome.
C'est clairement un modèle pour l'assassinat d'Alexandre.
Les deux personnages de femmes, Marie et Catherine, sont ici directement confrontées à la vulgarité de Lorenzo, qui rabaisse le personnage de Lucrèce : <<Si vous méprisez les femmes, pourquoi affectez-vous de les rabaisser devant votre mère et votre sœur ?>>.
On retrouve ici la dualité de la condition féminin, d'un côté la mère et la soeur, et d'un autre côté, les autres sont rabaissées.
Dans cette dualité, le personnage de Catherine est intéressant, Musset met ici en place un univers intime pour Lorenzo, que malgré tous ses mensonges et sa violence, il veut protéger du reste du monde.
La deuxième partie de la scène fait justement intervenir le reste du monde, avec l'entrée de deux personnages masculins, qui vont ramener les préoccupations politiques au premier plan : Bindo, l'oncle de Lorenzo qui veut le ramener à la cause Républicaine et Venturi un marchant, qui pour Lorenzo est méprisable.
Bindo se présente comme Républicain, mais veut le rétablissement des privilièges de sa famille.
Venturi représente les puissances d'argent, ce qui suscite clairement l'ironie de Lorenzo.
La troisième partie de la scène est marquée par l'entrée du Duc qui est une intrusion, c'est la première fois qu'il vient chez Lorenzo.
Dans cette scène, le Duc est particulièrement répugnant et monstrueux.
Bindo et Venturi font tout de même partie dans une certaine mesure de l'intimité de Lorenzo, de sa famille.
Lorenzo va les compromettre auprès du Duc, ils sont obligés d'accepter le poste et de retourner leur veste.
Lorenzo, en même temps, cherche peut-être à les protéger de la violence du Duc.
Le véritable enjeu dans le dernier moment va être de protéger Catherine contre la convoitise du Duc.
Dans cette scène, le Duc fait peser une menace physique sur l'intimité de Lorenzo et c'est cela qui va accélérer les évènements.
\addcontentsline{toc}{subsection}{d) Dans l'intimité de Lorenzo}
\subsection*{e) Le passage à l'action}
Dans le palais du Duc, le peintre Tebaldeo fait le portrait du Duc.
C'est la scène où Lorenzo va faire le premier geste concret de préparation de son acte, en subtilisant la cotte de maille du Duc.
C'est un geste qui ne ment pas.
Lorenzo se sert du portrait pour que le duc pose le cou découvert, peindre le Duc, c'est l'immortaliser, en faire une oeuvre d'art, cela aussi renvoie au meurtre que Lorenzo comparera à une oeuvre d'art
\addcontentsline{toc}{subsection}{e) Le passage à l'action}
\section*{\textcolor{red}{III. Le sens du sacrifice.}}
Plus on s'approche de l'acte, plus celui-ci prend la dimension d'un sacrifice de la part de Lorenzo.
Etymologie sacrifice: faire le sacré. On peut donner à la notion de sacrifice un sens faible dans un registre utilitaire : par exemple, on sacrifie son fou pour voler un cavalier, on sacrifie ses vacances pour préparer un concours << La fin justifie les moyens >>.
Sauf qu'un sacrifice au sens fort est quelque chose qui renvoie aux limites de ce genre de calculs quand on accepte un renoncement particulièrement important au profit d'une finalité particulièrement incertaine.
Dans un vrai sacrifice, la finalité reste profondément incertaine.
Exemple Isaac et Abraham : abraham ne pose pas de questions et le fait, dans ce cas là, il y a un non sens du sacrifice, il n'en est que plus signifiant du point de vue de la foi.
Dans Lorenzaccio, cela pose la question des motivations de Lorenzo,  qui restent extrêmement floues, mais il compare souvent son projet à des mythes religieux, une sorte d'appel divin, sans en préciser le contenu.
Surtout, Lorenzo montre de plus en plus que le véritable sens de son geste appartient aux autres. <<Je jette la nature humaine à pile ou face sur la tombe d'Alexandre>>.
La grande scène avec Philippe Strozzi est une réflexion sur le sens de ce sacrifice.
C'est aussi cela qui va donner au meurtre une dimension sacrée.
\addcontentsline{toc}{section}{III) Le sens du sacrifice.}
\subsection*{a) Répétition générale}
Lorenzo répète le meurtre avec son domestique.
L'acte est la répétition des actes du passé (Brutus), cette scène est une sorte de mise en abyme du théâtre.
Lorenzo répète son acte, mais en faisant semblant, il se laisse emporter dans l'action, il y croit tellement qu'il se voit en train de le faire.
C'est à ce moment là qu'il parle de vengeance, se dévoilant aux yeux de son domestique. Cette scène est très charnelle.
Elle prend presque une dimension érotique.
Lorenzo va même jusqu'au fantasme de manger le duc, ici c'est la dimension physique du sacrifice.
Néanmoins, Lorenzo lui donne une dimension spirituellen, il compare le meurtre à un baptême et avec le sacrifice du Christ <<pour toi je remettrais le Christ en croix>>.
Dans l'acte 3, les choses se préparent secrètement du côté de Lorenzo et cela bouge aussi du côté des Républicains, plus précisément chez les Strozzi qui vont aller progressivement vers la révolte.
D'un côté, l'acte individuel, de l'autre côté le climat politique et la grande scène entre Philippe et Lorenzo est la rencontre de ces deux éléménts, l'acte isolé de Lorenzo, trouvera-t-il un prolongement politique ?
\addcontentsline{toc}{subsection}{a) Répétition générale}
\subsection*{b) La cause de l'humanité}
La grande scène 3 de l'acte 3 est une scène centrale de la pièce.
C'est là que Lorenzo s'explique sur les raisons de son acte et qu'il enlève vraiment son masque pour faire part à Philippe Strozzi de ses intentions.
Cette scène est aussi la jonction de l'acte (Lorenzo) et de la parole (Philippe). Lorenzo va demander à Philippe d'agir et Philippe demande à Lorenzo de parler.
Lorenzo va donc en dire plus sur sa vocation qu'il compare à l'annonciation, Lorenzo reste flou sur cet appel qu'il a entendu, il se maintient dans une sorte de religiosité diffuse <<Prend garde à toi Philippe, tu as pensé au bonheur de l'humanité>>.
Peut-on croire en l'humanité ? Lorenzo et le problème qu'il renferme est comme un condensé de cette question de l'humanité.
L'humanité qui se condense ddans le personnage de Lorenzo, c'est aussi l'humanité historique, Lorenzo a ressenti son appel dans les ruines du Colisée, il est le successeur des héros de l'antiquité, ce qui veut dire qu'il est à la fois très jeune et très vieux.
Tous les tueurs de tyrans ont insipiré Lorenzo, il se compare lui-même à une statue antique qui descend de son socle.
La cause de l'humanité apparaît dans cette scène comme une cause très contradictoire, l'humanité c'est d'un côté, pour Lorenzo, un idéal symbolisé par le héros antique, mais d'un autre côté, c'est la corruption, la prostitution et les illusions perdues et c'est pourquoi Lorenzo ne peut agir que seul, néanmoins, il va remettre le sens de son action entre les mains des autres.
\end{document}