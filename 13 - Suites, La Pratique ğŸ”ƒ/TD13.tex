\documentclass[10pt]{article}

\usepackage[T1]{fontenc}
\usepackage[left=2cm, right=2cm, top=2cm, bottom=2cm, paperheight=31cm]{geometry}
\usepackage[skins]{tcolorbox}
\usepackage{hyperref, fancyhdr, lastpage, tocloft, ragged2e, multicol, changepage}
\usepackage{amsmath, amssymb, amsthm, stmaryrd}
\usepackage{tkz-tab}
\usepackage{systeme}

\def\pagetitle{Suites, La Pratique}
\setlength{\headheight}{13pt}

\title{\bf{\pagetitle}\\\large{Corrigé}}
\date{Novembre 2023}
\author{DARVOUX Théo}

\DeclareMathOperator{\ch}{ch}

\hypersetup{
    colorlinks=true,
    citecolor=black,
    linktoc=all,
    linkcolor=blue
}

\pagestyle{fancy}
\cfoot{\thepage\ sur \pageref*{LastPage}}


\begin{document}
\renewcommand*\contentsname{Exercices.}
\renewcommand*{\cftsecleader}{\cftdotfill{\cftdotsep}}
\maketitle

\hrule
\tableofcontents
\vspace{0.5cm}
\hrule


\thispagestyle{fancy}
\fancyhead[L]{MP2I Paul Valéry}
\fancyhead[C]{\pagetitle}
\fancyhead[R]{2023-2024}
\allowdisplaybreaks

\pagebreak


\section*{Exercice 13.1 [$\blacklozenge\lozenge\lozenge$]}
\begin{tcolorbox}[enhanced, width=7.6in, center, size=fbox, fontupper=\large, drop shadow southwest]
    Une suite croissante est une fonction croissante sur $\mathbb{N}$.\\
    Démontrer que le titre de l'exercice dit vraie, c'est à dire, pour une suite réelle $(u_n)_{n\in\mathbb{N}}$ l'équivalence entre\\
    1. $\forall{n\in\mathbb{N}} ~ u_{n+1} \geq u_n$.\\
    2. $\forall{(n,p)\in\mathbb{N}^2} ~ n \leq p \Longrightarrow u_n \leq u_p$.\\[0.1cm]
    Supposons 2, montrons 1.\\
    Soit $n\in\mathbb{N}$\\
    On a $n\leq n+1$. D'après 2, $u_n \leq u_{n+1}$. ez\\[0.1cm]
    Supposons 1, montrons 2.\\
    Soit $(n, p)\in\mathbb{N}^2$ tels que $n \leq p$. On sait que $u_{n+1} \geq u_n$, $u_{n+2} \geq u_{n+1}$, $u_{n+3} \geq u_{n+2}$, etc...\\
    Par récurrence triviale et par transitivité, pour tout entier $q\geq n$, $u_q \geq u_n$.\\
    En particulier, $u_p \geq u_n$\\
    \qed
\end{tcolorbox}
\addcontentsline{toc}{section}{Avant de parler de convergence.}
\addcontentsline{toc}{section}{\protect\numberline{}Exercice 13.1}

\section*{Exercice 13.2 [$\blacklozenge\blacklozenge\lozenge$]}
\begin{tcolorbox}[enhanced, width=7.6in, center, size=fbox, fontupper=\large, drop shadow southwest]
    Soit $a$ un réel supérieur à 1 et $(u_n)_{n\geq0}$ la suite définie par $\forall n \in \mathbb{N} ~ u_n = \frac{a^n}{n!}$.\\
    Démontrer que l'ensemble des termes de la suite possède un maximum, qu'on exprimera en fonction de $a$.\\
    $(u_n)$ est strictement positive sur $\mathbb{N}$.\\
    Soit $n \in \mathbb{N}$.\\
    On peut donc écrire : $\frac{u_{n+1}}{u_n}=\frac{a}{n+1}$.\\
    Ainsi, $(u_n)$ est croissante ($a\geq n+1$) puis décroissante ($a\leq n+1$), ce qui implique qu'un maximum existe.\\
    Ce maximum est atteint lorsque $a=n+1$ c'est à dire quand $n=\lfloor a \rfloor$.\\
    Ainsi, le maximum de la suite $u$ est : $\frac{a^{\lfloor a \rfloor}}{\lfloor a \rfloor!}$\\
    \qed
\end{tcolorbox}

\addcontentsline{toc}{section}{\protect\numberline{}Exercice 13.2}

\section*{Exercice 13.3 [$\blacklozenge\lozenge\lozenge$]}
\begin{tcolorbox}[enhanced, width=7.6in, center, size=fbox, fontupper=\large, drop shadow southwest]
    Pour $n\in\mathbb{N}$, on pose
    \begin{equation*}
        u_n = \sum_{k=n+1}^{2n}\frac{k\sin k}{k^2+1}.
    \end{equation*}
    Prouver que la suite $(u_n)$ est bornée.\\
    Soit $n\in\mathbb{N}$, on a : $-1 \leq \sin n \leq 1$. Donc :
    \begin{align*}
        \left|\sum_{k=n+1}^{2n}\frac{k\sin k}{k^2+1}\right| &\leq \sum_{k=n+1}^{2n}\frac{k}{k^2+1}\\
        &\leq \sum_{k=n+1}^{2n}\frac{n+1}{(n+1)^2+1}\\
        &\leq \frac{n^2 + n}{n^2 + 2n + 2}\\
        &\leq 1
    \end{align*}
    Majorér en valeur absolue c'est borner\\
    \qed
\end{tcolorbox}

\addcontentsline{toc}{section}{\protect\numberline{}Exercice 13.3}

\end{document}
 