\documentclass[10pt]{article}

\usepackage[T1]{fontenc}
\usepackage[left=2cm, right=2cm, top=2cm, bottom=2cm, paperheight=31cm]{geometry}
\usepackage[skins]{tcolorbox}
\usepackage{hyperref, fancyhdr, lastpage, tocloft, ragged2e, multicol, changepage}
\usepackage{amsmath, amssymb, amsthm, stmaryrd}
\usepackage{tkz-tab}
\usepackage{systeme}

\def\pagetitle{Suites, La Pratique}
\setlength{\headheight}{13pt}

\title{\bf{\pagetitle}\\\large{Corrigé}}
\date{Novembre 2023}
\author{DARVOUX Théo}

\DeclareMathOperator{\ch}{ch}

\hypersetup{
    colorlinks=true,
    citecolor=black,
    linktoc=all,
    linkcolor=blue
}

\pagestyle{fancy}
\cfoot{\thepage\ sur \pageref*{LastPage}}


\begin{document}
\renewcommand*\contentsname{Exercices.}
\renewcommand*{\cftsecleader}{\cftdotfill{\cftdotsep}}
\maketitle

\hrule
\tableofcontents
\vspace{0.5cm}
\hrule


\thispagestyle{fancy}
\fancyhead[L]{MP2I Paul Valéry}
\fancyhead[C]{\pagetitle}
\fancyhead[R]{2023-2024}
\allowdisplaybreaks

\pagebreak


\section*{Exercice 13.1 [$\blacklozenge\lozenge\lozenge$]}
\begin{tcolorbox}[enhanced, width=7.5in, center, size=fbox, fontupper=\large, drop shadow southwest]
    Une suite croissante est une fonction croissante sur $\mathbb{N}$.\\
    Démontrer que le titre de l'exercice dit vraie, c'est à dire, pour une suite réelle $(u_n)_{n\in\mathbb{N}}$ l'équivalence entre\\
    1. $\forall{n\in\mathbb{N}} ~ u_{n+1} \geq u_n$.\\
    2. $\forall{(n,p)\in\mathbb{N}^2} ~ n \leq p \Longrightarrow u_n \leq u_p$.\\[0.1cm]
    Supposons 2, montrons 1.\\
    Soit $n\in\mathbb{N}$\\
    On a $n\leq n+1$. D'après 2, $u_n \leq u_{n+1}$. ez\\[0.1cm]
    Supposons 1, montrons 2.\\
    Soit $(n, p)\in\mathbb{N}^2$ tels que $n \leq p$. On sait que $u_{n+1} \geq u_n$, $u_{n+2} \geq u_{n+1}$, $u_{n+3} \geq u_{n+2}$, etc...\\
    Par récurrence triviale et par transitivité, pour tout entier $q\geq n$, $u_q \geq u_n$.\\
    En particulier, $u_p \geq u_n$\\
    \qed
\end{tcolorbox}
\addcontentsline{toc}{section}{Avant de parler de convergence.}
\addcontentsline{toc}{section}{\protect\numberline{}Exercice 13.1}


\end{document}
 