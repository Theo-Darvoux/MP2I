\documentclass[10pt]{article}

\usepackage[T1]{fontenc}
\usepackage[left=2cm, right=2cm, top=2cm, bottom=2cm]{geometry}
\usepackage[skins]{tcolorbox}
\usepackage{hyperref, fancyhdr, lastpage, tocloft, ragged2e, multicol}
\usepackage{amsmath, amssymb, amsthm, stmaryrd}

\def\pagetitle{Équations Algébriques}
\setlength{\headheight}{13pt}

\title{\bf{\pagetitle}\\\large{Corrigé}}
\date{Novembre 2023}
\author{DARVOUX Théo}

\hypersetup{
    colorlinks=true,
    citecolor=black,
    linktoc=all,
    linkcolor=blue
}

\pagestyle{fancy}
\cfoot{\thepage\ sur \pageref*{LastPage}}


\begin{document}
\renewcommand*\contentsname{Exercices.}
\renewcommand*{\cftsecleader}{\cftdotfill{\cftdotsep}}
\maketitle
\begin{center}
    \LARGE{Crédits : Etienne pour les exercices 9.25 et 9.26}\\
\end{center}
\hrule
\tableofcontents
\vspace{0.5cm}
\hrule


\thispagestyle{fancy}
\fancyhead[L]{MP2I Paul Valéry}
\fancyhead[C]{\pagetitle}
\fancyhead[R]{2023-2024}
\allowdisplaybreaks

\begin{center}
    \LARGE{Soit $z\in\mathbb{C}$}.\\
    \LARGE{$\Re(z)$ est la partie réelle de $z$}.\\
    \LARGE{$\Im(z)$ est la partie imaginaire de $z$}.
\end{center}

\pagebreak

\section*{Exercice 10.17 [$\blacklozenge\lozenge\lozenge$]}
\begin{tcolorbox}[enhanced, width=7in, center, size=fbox, fontupper=\large, drop shadow southwest]
    1. Calculer les racines carrées du nombre $-8i$.\\
    On donnera ces nombres sous forme algébrique et sous forme trigonométrique.\\
    2. Résoudre dans $\mathbb{C}$ l'équation
    \begin{equation*}
        z^2 - 4z + 4 + 2i = 0
    \end{equation*}
    Notons $\delta$ une racine de $-8i$ :
    \begin{equation*}
        \delta = \sqrt{8}e^{-i\frac{\pi}{4}}= 2\sqrt{2}\left(\cos\left( -\frac{\pi}{4} \right) + i\sin\left( -\frac{\pi}{4} \right)\right) = 2\sqrt{2}\left(\frac{\sqrt{2}}{2}-\frac{\sqrt{2}}{2}i\right)=2-2i
    \end{equation*}
    2. Le discriminant $\Delta$ vaut $-8i$. Ses racines carrées sont donc $2 - 2i$ et $-2+2i$.\\
    L'ensemble des solutions de l'équation est donc : $\left\{3-i, 1 + i\right\}$.\\
    \qed
\end{tcolorbox}

\addcontentsline{toc}{section}{\protect\numberline{}Exercice 10.17}

\section*{Exercice 10.18 [$\blacklozenge\lozenge\lozenge$]}
\begin{tcolorbox}[enhanced, width=7in, center, size=fbox, fontupper=\large, drop shadow southwest]
    Soit $n\in\mathbb{N}$, $n\geq2$. Calcul de
    \begin{equation*}
        \sum_{z\in\mathbb{U}_n}{z} \hspace{0.5cm} \text{et} \hspace{0.5cm} \prod_{z\in\mathbb{U}_n}z
    \end{equation*}
    On a :
    \begin{align*}
        \sum_{z\in\mathbb{U}_n}{z}&=\sum_{k=0}^{n-1}{e^{i\frac{2k\pi}{n}}}=\frac{1-e^{i2\pi}}{1-e^{i\frac{2\pi}{n}}}=0
    \end{align*}
    Et :
    \begin{align*}
        \prod_{z\in\mathbb{U}_n}{z}&=\prod_{k=0}^{n-1}{e^{i\frac{2k\pi}{n}}}=\exp\left( \sum_{k=0}^{n-1}{i\frac{2k\pi}{n}} \right) = \exp\left( i\frac{2\pi}{n}\sum_{k=0}^{n-1}{k} \right)=e^{i\pi(n-1)}=(-1)^{n-1}
    \end{align*}
    \qed
\end{tcolorbox}

\addcontentsline{toc}{section}{\protect\numberline{}Exercice 10.18}

\section*{Exercice 10.19 [$\blacklozenge\blacklozenge\lozenge$]}
\begin{tcolorbox}[enhanced, width=7in, center, size=fbox, fontupper=\large, drop shadow southwest]
    Donner une expression du périmètre du polygone régulier formé par les nombres de $\mathbb{U}_n$.\\
    Que conjecture-t-on sur la limite lorsque $n\to+\infty$ ? Essayer de prouver votre conjecture.\\[0.2cm]
    Soit $n\in\mathbb{N}$. Le périmètre du polygone régulier formé par les nombres de $\mathbb{U}_n$ est :
    \begin{equation*}
        \sum_{k=0}^{n-1}{|e^{i\frac{2k\pi}{n}} - e^{i\frac{2(k+1)\pi}{n}}|}=\sum_{k=0}^{n-1}{|e^{\frac{(2k+1)\pi}{n}}||e^{-\frac{\pi}{n}} - e^{\frac{\pi}{n}}|}=2n\sin\left( \frac{\pi}{n} \right)
    \end{equation*}
    Et, puisque $\lim_{x\to0}{\frac{\sin(x)}{x}}=1$, alors :
    \begin{equation*}
        \lim_{n\to+\infty}{2n\sin\left(\frac{\pi}{n}\right)}=\lim_{n\to+\infty}2\pi\frac{\sin\frac{\pi}{n}}{\frac{\pi}{n}}=2\pi
    \end{equation*}
    \qed
\end{tcolorbox}

\addcontentsline{toc}{section}{\protect\numberline{}Exercice 10.19}

\section*{Exercice 10.20 [$\blacklozenge\lozenge\lozenge$]}
\begin{tcolorbox}[enhanced, width=7in, center, size=fbox, fontupper=\large, drop shadow southwest]
    Soit $\omega\in\mathbb{U}_7$, une racine 7e de l'unité différente de 1.\\
    1. Justifier que $1+\omega+\omega^2+\omega^3+\omega^4+\omega^5+\omega^6=0$.\\
    2. Calculer le nombre $\frac{\omega}{1+\omega^2} + \frac{\omega^2}{1+\omega^4} + \frac{\omega^3}{1+\omega^6}$.\\[0.2cm]
    1. On a déjà montré que $\forall{n\in\mathbb{N}},n>2,\sum\limits_{z\in\mathbb{U}_n}z=0$ dans le 10.18.\\
    2. On a :
    \begin{align*}
        \frac{\omega}{1+\omega^2} + \frac{\omega^2}{1+\omega^4} + \frac{\omega^3}{1+\omega^6} &=
        \frac{2+2\omega+2\omega^2+2\omega^3+2\omega^4+2\omega^5}{\omega^6}=-\frac{2\omega^6}{\omega^6}=-2
    \end{align*}
    \qed
\end{tcolorbox}

\addcontentsline{toc}{section}{\protect\numberline{}Exercice 10.20}

\section*{Exercice 10.21 [$\blacklozenge\blacklozenge\lozenge$]}
\begin{tcolorbox}[enhanced, width=7in, center, size=fbox, fontupper=\large, drop shadow southwest]
    1. Quand dit-on qu'un nombre réel $\theta$ est un argument d'un nombre complexe $z$ ?\\
    2. Soit $k\in\llbracket0,n-1\rrbracket$. Donner le module et un argument de $e^{\frac{2ik\pi}{n}}-1$.\\
    3. Établir l'égalité
    \begin{equation*}
        \sum_{z\in\mathbb{U}_n}{|z-1|}=\frac{2}{\tan\left( \frac{\pi}{2n} \right)}.
    \end{equation*}
    1. $\theta$ est un argument de $z\neq0$ ssi $z=|z|e^{i\theta}$.\\
    2. On a :
    \begin{align*}
        e^{\frac{2ik\pi}{n}}-1=2i\sin\left( \frac{k\pi}{n} \right)e^{\frac{ik\pi}{n}}=2\sin\left( \frac{k\pi}{n} \right)e^{i\frac{\pi(2k+n)}{2n}}
    \end{align*}
    Ainsi son module est $2\sin\left( \frac{k\pi}{n} \right)$ et l'un de ses arguments est $\frac{\pi(2k+n)}{2n}$.\\
    3. Soit $n\in\mathbb{N}$. On a :
    \begin{align*}
        \sum_{z\in\mathbb{U}_n}{|z-1|}&=\sum_{k=0}^{n-1}{|e^{\frac{2ik\pi}{n}}-1|} = \sum_{k=0}^{n-1}{|e^{\frac{ik\pi}{n}}\left( e^{\frac{ik\pi}{n}} - e^{-\frac{ik\pi}{n}} \right)|}\\
        &=\sum_{k=0}^{n-1}{\left|2i\sin\left( \frac{k\pi}{n} \right)\right|} = 2\sum_{k=0}^{n-1}{\left|\sin\left( \frac{k\pi}{n} \right)\right|}
    \end{align*}
    Or, $\forall{k\in\llbracket0,n-1\rrbracket}, \sin\left( \frac{k\pi}{n} \right) \geq 0$. Ainsi (formule du cours) :
    \begin{align*}
        \sum_{z\in\mathbb{U}_n}{|z-1|} &= 2\sum_{k=0}^{n-1}{\sin\left( \frac{k\pi}{n} \right)} = 2 \cdot \frac{ \sin \left( \frac{(n+1) \frac{\pi}{n}}{2} \right)}{\sin \left( \frac{\pi}{2n} \right)}\\
        &= 2 \cdot \frac{\sin\left( \frac{\pi}{2n} + \frac{\pi}{2} \right)}{\sin \left( \frac{\pi}{2n} \right)} = 2 \cdot \frac{\cos\left( \frac{\pi}{2n} \right)}{\sin\left( \frac{\pi}{2n} \right)}\\
        &= \frac{2}{\tan\left( \frac{\pi}{2n} \right)}
    \end{align*}
    \qed
\end{tcolorbox}

\addcontentsline{toc}{section}{\protect\numberline{}Exercice 10.21}

\section*{Exercice 10.22 [$\blacklozenge\blacklozenge\lozenge$]}
\begin{tcolorbox}[enhanced, width=7in, center, size=fbox, fontupper=\large, drop shadow southwest]
    Soit $\theta$ un nombre réel appartenant à $]0,\pi[$. Résoudre l'équation
    \begin{equation*}
        z^2 - 2e^{i\theta}z + 2ie^{i\theta}\sin\theta=0.
    \end{equation*}
    On écrira les solutions sous forme algébrique \underbar{et} sous forme trigonométrique.
    \begin{align*}
        \Delta&=4e^{2i\theta}-8ie^{i\theta}\sin\theta = 4e^{i\theta}\left(\cos\theta + i\sin\theta - 2i\sin\theta \right) \\
        &= 4e^{i\theta}(\cos\theta-i\sin\theta) = 4e^{i\theta}e^{-i\theta}\\
        &=4
    \end{align*}
    On a alors :
    \begin{align*}
        &x_1 = e^{i\theta} + 1 = 2\cos\left( \frac{\theta}{2} \right)e^{\frac{i\theta}{2}}=2\cos\left( \frac{\theta}{2} \right)\left( \cos\left( \frac{\theta}{2} \right) + i\sin\left( \frac{\theta}{2} \right) \right)\\
        &x_2 = e^{i\theta} - 1 = 2i\sin\left( \frac{\theta}{2} \right)e^{\frac{i\theta}{2}}=2i\sin\left( \frac{\theta}{2} \right)\left( \cos\left( \frac{\theta}{2} \right) + i\sin\left( \frac{\theta}{2} \right) \right)\\
    \end{align*}
    \qed
\end{tcolorbox}
\addcontentsline{toc}{section}{\protect\numberline{}Exercice 10.22}

\section*{Exercice 10.23 [$\blacklozenge\blacklozenge\lozenge$]}
\begin{tcolorbox}[enhanced, width=7in, center, size=fbox, fontupper=\large, drop shadow southwest]
    Soit $n\in\mathbb{N}^*$.\\
    1. Résoudre dans $\mathbb{C}$ l'équation $z^2 - 2\cos(\theta)z + 1 = 0$.\\
    2. Résoudre dans $\mathbb{C}$ l'équation $z^{2n} - 2\cos(\theta)z^n + 1 = 0$.\\
    1. $\Delta = 4\cos^2(\theta)-4=4(\cos^2(\theta)-1)=-4\sin^2(\theta) \leq 0$.
    \begin{align*}
        &x_1 = \frac{2\cos(\theta)+i\sqrt{4\sin^2(\theta)}}{2}=\cos(\theta)+i\sin(\theta)=e^{i\theta}\\
        &x_2 = \cos(\theta) - i\sin(\theta) = e^{-i\theta}
    \end{align*}
    2. Posons $z' = z^n$.\\
    On sait que $z'$ est solution de $z'^2-2\cos(\theta)z'+1=0$.\\
    Ainsi, $z'_1 = e^{i\theta}$ et $z'_2=e^{-i\theta}$.\\
    On en déduit :
    \begin{align*}
        &z_1 = {z'_1}^{\frac{1}{n}}=e^{\frac{i\theta}{n}}\\
        &z_2={z'_2}^{\frac{1}{n}}=e^{-\frac{i\theta}{n}}
    \end{align*}
    \qed
\end{tcolorbox}
\addcontentsline{toc}{section}{\protect\numberline{}Exercice 10.23}

\section*{Exercice 10.24 [$\blacklozenge\blacklozenge\lozenge$]}
\begin{tcolorbox}[enhanced, width=7in, center, size=fbox, fontupper=\large, drop shadow southwest]
    Résoudre.
    \begin{equation*}
        \left( \frac{z+i}{z-i} \right)^3 + \left( \frac{z+i}{z-i} \right)^2 + \left( \frac{z+i}{z-i} \right) + 1 = 0.
    \end{equation*}
    Posons $\omega=\left( \frac{z+i}{z-i} \right)$. On a : $\omega^3 + \omega^2 + \omega + 1 = 0$.\\
    On a alors $\omega\in\mathbb{U}_4 \setminus \{1\}$.\\
    Ainsi, $\left( \frac{z+i}{z-i} \right)=i$ ou $\left( \frac{z+i}{z-i} \right) = -1$ ou $\left( \frac{z+i}{z-1} \right) = -i$.\\
    1. $\left( \frac{z+i}{z-i} \right)=i\iff z+i = iz+1 \iff z(1 - i) = 1 - i \iff z = 1$.\\
    2. $\left( \frac{z+i}{z-i} \right)=-1 \iff z+i = i - z \iff z = -z \iff z = 0$.\\
    3. $\left( \frac{z+i}{z-i} \right)=-i \iff z+i = -1 - zi \iff z(1+i) = -1 - i \iff z=-\frac{1+i}{1+i}=-1$\\
    L'ensemble des solutions est donc : $\{-1, 0, 1\}$.\\
    \qed
\end{tcolorbox}
\addcontentsline{toc}{section}{\protect\numberline{}Exercice 10.24}

\section*{Exercice 10.25 [$\blacklozenge\blacklozenge\blacklozenge$]}
\begin{tcolorbox}[enhanced, width=7in, center, size=fbox, fontupper=\large, drop shadow southwest]
    Résoudre dans $\mathbb{C}$ l'équation $(z+1)^n=z^n$.\\
    Soit $z\in\mathbb{C}^*$. On a :
    \begin{align*}
        z^n=(z+1)^n &\iff \left(1+\frac{1}{z}\right)^n=1\\
        &\iff(1+\frac{1}{z})\in\mathbb{U}_n\\
        &\iff\exists k\in\llbracket1,n-1\rrbracket \hspace{0.2cm} | \hspace{0.2cm} 1+\frac{1}{z}=e^{i\frac{2k\pi}{n}}\\
        &\iff\frac{1}{z}=e^{i\frac{2k\pi}{n}}-1\\
        &\iff z=\frac{1}{e^{i\frac{2k\pi}{n}}-1}\\
        &\iff z=\frac{e^{-i\frac{k\pi}{n}}}{2i\sin(\frac{k\pi}{n})}\\
        &\iff z=\frac{\cos(\frac{k\pi}{n})-i\sin(\frac{k\pi}{n})}{2i\sin(\frac{k\pi}{n})}\\
        &\iff z=-\frac{1}{2}-\frac{i}{2\tan(\frac{k\pi}{n})}
    \end{align*}
    Ainsi, l'ensemble des solutions est : $\{-\frac{1}{2}-\frac{i}{2\tan(\frac{k\pi}{n})} \, | \, k\in\llbracket1,n-1\rrbracket\}$.\\
    \qed
\end{tcolorbox}
\addcontentsline{toc}{section}{\protect\numberline{}Exercice 10.25}

\section*{Exercice 10.26 [$\blacklozenge\blacklozenge\blacklozenge$]}
\begin{tcolorbox}[enhanced, width=7in, center, size=fbox, fontupper=\large, drop shadow southwest]
    Résoudre dans $\mathbb{C}^2$ le système
    \begin{equation*}
        \begin{cases}
            u^2 + v^2 = -1\\
            uv = 1
        \end{cases}
    \end{equation*}
    On peut prendre un couple dans $(C^*)^2$ car le système impose que les membres soient non nuls.\\
    Soit $(u,v)\in(\mathbb{C}^*)^2$. Soit $(r,\rho)\in(\mathbb{R}_+^*)^2$ et $(\theta, \pi)\in\mathbb{R}^2$ tels que $u=re^{i\theta}$ et $v=\rho e^{i\varphi}$
    \begin{align*}
        (u,v) \text{ est solution } &\iff \begin{cases}
            u^2 + v^2 = -1\\
            uv = 1
        \end{cases}\\
        &\iff u^2 \text{ et } v^2 \text{ racines de } X^2 + X + 1\\
        &\iff(u^2, v^2)\in\left\{\frac{-1-i\sqrt{3}}{2}, \frac{-1+i\sqrt{3}}{2}\right\}\\
        &\iff\begin{cases}
            u^2 = e^{i\frac{4\pi}{3}}\\
            v^2 = e^{i\frac{2\pi}{3}}
        \end{cases}\\
        &\iff\begin{cases}
            r = 1\\
            \theta = \frac{2\pi}{3}[\pi]\\
            \rho = 1\\
            \varphi = \frac{\pi}{3}[\pi]
        \end{cases}
    \end{align*}
    L'ensemble des solutions est donc :
    \begin{equation*}
        \left\{(e^{i\frac{2\pi}{3}}, e^{i\frac{4\pi}{3}}), (e^{i\frac{5\pi}{3}}, e^{i\frac{\pi}{3}}), (e^{i\frac{\pi}{3}}, e^{i\frac{5\pi}{3}}), (e^{i\frac{4\pi}{3}}, e^{i\frac{2\pi}{3}})\right\}
    \end{equation*}
    \qed
\end{tcolorbox}
\addcontentsline{toc}{section}{\protect\numberline{}Exercice 10.26}

\section*{Exercice 10.27 [$\blacklozenge\blacklozenge\blacklozenge$]}
\begin{tcolorbox}[enhanced, width=7in, center, size=fbox, fontupper=\large, drop shadow southwest]
    Soient $n\in\mathbb{N}^*$ et $z\in\mathbb{C}$ tels que $z^n=(1+z)^n=1$.\\
    Montrer que $n$ est un multiple de $6$ et que $z^3=1$.\\[0.25cm]
    \emph{Analyse.}\\[0.1cm]
    On a $z^n = (1+z)^n = 1$. Ainsi, $|z|=|1+z|=1$ et $z\in\mathbb{U}$.\\[0.1cm]
    Puisque $|z|=|1+z|$ et que $\Im(z) = \Im(1+z)$, on a :
    \begin{align*}
        &\sqrt{\Re(z)^2 + \Im(z)^2} = \sqrt{\Re(z+1)^2 + \Im(z+1)^2}\\
        \Longrightarrow&\Re(z)^2=\Re(z+1)^2\\
        \Longrightarrow&\Re(z)^2=(1 + \Re(z))^2\\
        \Longrightarrow&\Re(z)=-\frac{1}{2}
    \end{align*}
    Ainsi, $\exists\theta\in\mathbb{R} \, | \, z=e^{i\theta}$, et : $\Re\left(e^{i\theta}\right)=-\frac{1}{2}$, donc $\cos(\theta)=-\frac{1}{2}$.\\[0.1cm]
    On obtient que $\theta\in\{\frac{2\pi}{3} + 2k\pi, \frac{4\pi}{3} + 2k\pi \, | \, k\in\mathbb{Z}\}$.\\[0.1cm]
    Ainsi, $z\in\{e^{i\frac{2\pi}{3}}, e^{i\frac{4\pi}{3}}\}$ et $z^3 = 1$.\\[0.1cm]
    On a $z^n\in\{e^{i\frac{2n\pi}{3}}, e^{i\frac{4n\pi}{3}}\}$, or $z^n=1$ donc $\frac{2n\pi}{3}\equiv0[2\pi]$ et $\frac{4n\pi}{3}\equiv0[2\pi]$.\\[0.1cm]
    Ainsi, $n \equiv 0[3]$ et $2n \equiv 0[3]$. $n$ est donc multiple de 6.\\[0.2cm]
    \emph{Synthèse}.\\[0.1cm]
    On a $z\in\{e^{i\frac{2\pi}{3}}, e^{i\frac{4\pi}{3}}\}$ et $k\in\mathbb{N}$ tel que $n=6k$.\\[0.1cm]
    On a que $z^3=1$.\\[0.1cm]
    De plus, $z^n\in\{e^{i4k\pi}, e^{i8k\pi}\}$, or $e^{i4k\pi} = e^{i8k\pi} = 1$. Ainsi, $z^n = 1$.\\[0.1cm]
    Enfin, $(1+e^{i\frac{2\pi}{3}})^n = (2e^{i\frac{\pi}{3}}\cos(\frac{\pi}{3}))^{6k}=(64e^{2i\pi}\frac{1}{64})^k=1$.\\[0.1cm]
    Et : $(1+e^{i\frac{4\pi}{3}})^n=(2e^{i\frac{2\pi}{3}}\cos(\frac{2\pi}{3}))^{6k}=(64e^{4i\pi}\frac{1}{64})^k=1$.\\
    \qed
\end{tcolorbox}
\addcontentsline{toc}{section}{\protect\numberline{}Exercice 10.27}

\end{document}
 