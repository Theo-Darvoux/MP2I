\documentclass[10pt]{article}

\usepackage[T1]{fontenc}
\usepackage[left=2cm, right=2cm, top=2cm, bottom=2cm]{geometry}
\usepackage[skins]{tcolorbox}
\usepackage{hyperref, fancyhdr, lastpage, tocloft, ragged2e, multicol}
\usepackage{amsmath, amssymb, amsthm, stmaryrd}

\def\pagetitle{Équations Algébriques}

\title{\bf{\pagetitle}\\\large{Corrigé}}
\date{Novembre 2023}
\author{DARVOUX Théo}

\hypersetup{
    colorlinks=true,
    citecolor=black,
    linktoc=all,
    linkcolor=blue
}

\pagestyle{fancy}
\cfoot{\thepage\ sur \pageref*{LastPage}}


\begin{document}
\renewcommand*\contentsname{Exercices.}
\renewcommand*{\cftsecleader}{\cftdotfill{\cftdotsep}}
\maketitle
\begin{center}
    \LARGE{Crédits : Etienne pour les exercices 9.25 et 9.26}\\
\end{center}
\hrule
\tableofcontents
\vspace{0.5cm}
\hrule


\thispagestyle{fancy}
\fancyhead[L]{MP2I Paul Valéry}
\fancyhead[C]{\pagetitle}
\fancyhead[R]{2023-2024}
\allowdisplaybreaks

\pagebreak

\section*{Exercice 10.17 [$\blacklozenge\lozenge\lozenge$]}
\begin{tcolorbox}[enhanced, width=7in, center, size=fbox, fontupper=\large, drop shadow southwest]
    1. Calculer les racines carrées du nombre $-8i$.\\
    On donnera ces nombres sous forme algébrique et sous forme trigonométrique.\\
    2. Résoudre dans $\mathbb{C}$ l'équation
    \begin{equation*}
        z^2 - 4z + 4 + 2i = 0
    \end{equation*}
    Notons $\delta$ une racine de $-8i$ :
    \begin{equation*}
        \delta = \sqrt{8}e^{-i\frac{\pi}{4}}= 2\sqrt{2}\left(\cos\left( -\frac{\pi}{4} \right) + i\sin\left( -\frac{\pi}{4} \right)\right) = 2\sqrt{2}\left(\frac{\sqrt{2}}{2}-\frac{\sqrt{2}}{2}i\right)=2-2i
    \end{equation*}
    2. Le discriminant $\Delta$ vaut $-8i$. Ses racines carrées sont donc $2 - 2i$ et $-2+2i$.\\
    L'ensemble des solutions de l'équation est donc : $\left\{3-i, 1 + i\right\}$.\\
    \qed
\end{tcolorbox}

\addcontentsline{toc}{section}{\protect\numberline{}Exercice 10.17}

\section*{Exercice 10.18 [$\blacklozenge\lozenge\lozenge$]}
\begin{tcolorbox}[enhanced, width=7in, center, size=fbox, fontupper=\large, drop shadow southwest]
    Soit $n\in\mathbb{N}$n $n\geq2$. Calcul de
    \begin{equation*}
        \sum_{z\in\mathbb{U}_n}{z} \hspace{0.5cm} \text{et} \hspace{0.5cm} \prod_{z\in\mathbb{U}_n}z
    \end{equation*}
    On a :
    \begin{align*}
        \sum_{z\in\mathbb{U}_n}{z}&=\sum_{k=0}^{n-1}{e^{i\frac{2k\pi}{n}}}=\frac{1-e^{i2\pi}}{1-e^{i\frac{2\pi}{n}}}=0
    \end{align*}
    Et :
    \begin{align*}
        \prod_{z\in\mathbb{U}_n}{z}&=\prod_{k=0}^{n-1}{e^{i\frac{2k\pi}{n}}}=\exp\left( \sum_{k=0}^{n-1}{i\frac{2k\pi}{n}} \right) = \exp\left( i\frac{2\pi}{n}\sum_{k=0}^{n-1}{k} \right)=e^{i\pi(n-1)}=(-1)^{n-1}
    \end{align*}
    \qed
\end{tcolorbox}

\addcontentsline{toc}{section}{\protect\numberline{}Exercice 10.18}

\section*{Exercice 10.19 [$\blacklozenge\blacklozenge\lozenge$]}
\begin{tcolorbox}[enhanced, width=7in, center, size=fbox, fontupper=\large, drop shadow southwest]
    Donner une expression du périmètre du polygone régulier formé par les nombres de $\mathbb{U}_n$.\\
    Que conjecture-t-on sur la limite lorsque $n\to+\infty$ ? Essayer de prouver votre conjecture.\\[0.2cm]
    Soit $n\in\mathbb{N}$. Le périmètre du polygone régulier formé par les nombres de $\mathbb{U}_n$ est :
    \begin{equation*}
        \sum_{k=0}^{n-1}{|e^{i\frac{2k\pi}{n}} - e^{i\frac{2(k+1)\pi}{n}}|}=\sum_{k=0}^{n-1}{|e^{\frac{(2k+1)\pi}{n}}||e^{-\frac{\pi}{n}} - e^{\frac{\pi}{n}}|}=2n\sin\left( \frac{\pi}{n} \right)
    \end{equation*}
    Et, puisque $\lim_{x\to0}{\frac{\sin(x)}{x}}=1$, alors :
    \begin{equation*}
        \lim_{n\to+\infty}{2n\sin\left(\frac{\pi}{n}\right)}=\lim_{n\to+\infty}2\pi\frac{\sin\frac{\pi}{n}}{\frac{\pi}{n}}=2\pi
    \end{equation*}
    \qed
\end{tcolorbox}

\addcontentsline{toc}{section}{\protect\numberline{}Exercice 10.19}

\end{document}
 