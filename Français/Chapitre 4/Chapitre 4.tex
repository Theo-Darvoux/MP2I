\documentclass[12pt]{article}
\usepackage{ragged2e}
\usepackage[left=2cm, right=2cm, top=2cm, bottom=2cm]{geometry}
\usepackage{color}
\usepackage{amsmath, amssymb, amsthm}
\usepackage{lastpage}
\usepackage{fancyhdr}
\usepackage[T1]{fontenc}
\usepackage{hyperref}

\title{Chapitre 4\\\large{Individu et communauté}}
\date{}
\author{}


\setlength{\headheight}{15pt}
\pagestyle{fancy}
\cfoot{\thepage\ sur \pageref*{LastPage}}

\hypersetup{
    colorlinks=true,
    citecolor=black,
    linktoc=all,
    linkcolor=blue
}

\renewcommand*\contentsname{Sommaire}

\begin{document}
\maketitle
\thispagestyle{fancy}

\fancyhead[L]{}
\fancyhead[R]{MP2I Paul Valéry}
\fancyhead[C]{Français-Philosophie}

\section*{\color{red}Introduction}
Il nous semble aller de soi qu'une communauté ou qu'une société puisse se concevoir comme un rassemblement d'individus, d'un point de vue logique, l'individu précéderait la communauté, c'est ce principe qu'on retrouve notamment dans toutes les philosophies, c'est ce qui est en jeu dans les philosophies contractualistes (Hobbes et Rousseau).
Les deux versions voient la société comme le résultat d'un contrat ou d'un pacte entre les individus, ce qui dote les individus de la capacité de chosiir d'appartenir à la société.
D'ailleurs, l'état de nature, qui précède le contrat social est pour Hobbes un état hyper-individualiste, la guerre de chacun contre chacun.
Cependant, il existe des modèles de communautés et de sociétés qui ne se conçoivent pas ainsi et où l'individu n'a pas d'existence hors de la communauté, la communauté précède l'individu : la famille.
Au contraire, avant même d'être un individu, chaque être humain se définit comme membre d'une famille.
Nous vivons dans une époque très individualiste où l'individu est souvent vu comme centre du réel. 
Cela veut dire qu'il y a autant de centre que d'individus, et surtout, la notion même d'individu est une construction historique et culturelle.
Dans l'antiquité grecque ou romaine, la notion même d'individu n'existait pas.
L'individu était négligeable par rapport à la communauté.
\subsection*{a) Les différents types de communautés.}
Le terme de communauté a un sens assez fort, plus fort que le terme de groupe, ou de société. Pour une communauté, il faut quelque chose en commun : une religion, la famille, amicale, nationale et politique, la cité pour les grecs et l'Etat pour l'époque moderne.
On peut s'interroger sur la place qu'occupe l'individu en tant que tel dans ces différents modèles de communautés.
Par exemple, dans la famille, l'individu ne choisit pas d'en faire parti, et en même temps, il se définit profondément dans son identité par son appartenance à sa famille, la famille n'est donc pas un rassemblement d'individus.
Au contraire, l'individu émerge peut-être par arrachement à sa famille.
Rousseau dans le contrat social dit que la famille est la première des communautés, la seule communauté naturelle que connaisse les Hommes, mais quand les enfants deviennent suffisamment grands, <<ce lien naturel ce dissout>>, et s'il continue à subsister, il devient contractuel.
On est toujours rattrapés par le lien familial.
On peut appeler holistes des communautés où le tout prime sur l'individu par opposition à l'individualisme. La famille et la religion sont probablement les deux exemples les plus forts. La communauté politique est traditionnellement présentée comme la communauté qui rassemble toutes les autres. On peut se demander ce qui fait la communauté propre.
La notion de société est elle aussi une notion englobante, mais son sens est plus flou, une société ne repose pas forcément sur la présence d'un élément commun. L'existence de communautés peut représenter une menace pour la société. On dit souvent que les sociétés modernes sont atomisées ou individualistes.
\subsection*{b) Les racines de l'individualisme.}
L'individualisme moderne trouve ses origines vers le XVe - XVIIe siècle, une époque où l'emprise de la religion se relache, en raison des guerres de religion. C'est le moment des grands progrès scientifiques de la renaissance, du rationnalisme qui repose sur une exigence de tout vérifier par soi-même. Au moment du doute cartésien, la seule chose qui existe est le moi.
Le XVIe siècle voit la naissance des grands États modernes, c'est à ce moment là que se développent les sociétés contractualistes. À l'état de nature, chaque individu a un droit absolu sur toute chose, y compris sur la personne d'autrui. L'individualisme est d'un côté angoissant, et un désir radical. 
\end{document}