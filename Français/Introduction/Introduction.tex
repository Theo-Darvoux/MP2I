\documentclass[12pt]{article}
\usepackage{ragged2e}
\usepackage[left=2cm, right=2cm, top=2cm, bottom=2cm]{geometry}
\usepackage{color}
\usepackage{lastpage}
\usepackage{fancyhdr}
\usepackage[T1]{fontenc}

\title{Faire Croire\\\large Cours d'introduction}
\date{}
\author{}

\setlength{\headheight}{15pt}
\pagestyle{fancy}
\cfoot{\thepage\ sur \pageref*{LastPage}}

\begin{document}
\maketitle
\thispagestyle{fancy}


La différence entre croire et savoir est en soi un problème philosphique fondamental.
Rares sont les choses que l'on peut prétendre connaître avec une certitude absolue ou même avec une certitude suffisante.
La plupart du temps, nous devons nous contenter de croire soit à ce qu'on nous dit, soit à des opinions ou à des croyances, notamment religieuses, ce qui n'empêche pas d'y apporter de forts investissements subjectifs.
Cette importance de la croyance est notamment présente dans les rapports sociaux qui reposent sur la parole d'autrui (engagements, promesses, \dots).
C'est pour cela que la croyance est un enjeu de pouvoir et que le problème lié au verbe \emph{croire} se redouble puisqu'on peut aussi \emph{faire croire} (mensonges, manipulations, persuasions, bluffs, fiction, \dots) pas forcément pour de mauvaises intentions mais plutôt par nécessité pour les relations sociales\footnote{W. Lippman, \emph{The Manufacture Of Consent}}.\par
Ce pouvoir de \emph{faire croire} est d'autant plus précieux qu'au fond nos rapports sociaux sont avant tout dominés par la méfiance.\footnote{Thomas Hobbes, \emph{Le Léviathan}, 1651}\par
D'où vient alors la crédulité humaine ? Comment se développe le pouvoir de faire croire ?
\section*{\color{red}A. Les différentes figures de la croyance.}
\subsection*{Texte 2: Spinoza, \emph{Traité de la réforme de l'entendement}, 1677}
Spinoza distingue 3 modes de connaissance:
\begin{enumerate}
    \item La croyance d'opinion, acquise par <<ouï-dire>> ou par <<expérience vague>>
    \item Le savoir, acquis par déduction et conclusions de faits avérés auparavant.
    \item L'intuition, une connaissance directe, quasiment divine.
\end{enumerate}
\par
Ce texte est caractéristique du rationalisme de l'époque moderne qui présuppose de distinguer soigneusement le savoir et la foi.
Un rationaliste cherche à tout vérifier/démontrer.
Ce faisant, il est condamné à constater que la plupart des vérités auxquelles il adhère sont en fait des croyances au premier rang desquelles est la croyance en le pouvoir de la raison.
Ce problème est d'autant plus embêtant que le rationaliste doit compter sur sa propre pensée : <<Penser par soi-même>>, ce qui l'expose à une solitude radicale.\par 
\pagebreak
\subsection*{Texte 1: Kant, \emph{Critique de la raison pure}, 1781}
Kant énonce le pricipe de <<créance>> (donner du crédit à quelque chose), et en distingue 3 types:
\begin{enumerate}
    \item L'opinion, ni subjective, ni objective par insuffisance.
    \item La croyance, entièrement subjective, de l'ordre de la persuasion.
    \item Le savoir, suffisant subjectivement ainsi qu'objectivement, de l'ordre de la conviction.
\end{enumerate}
Dans l'Évangile, se pose souvent la question de savoir si le Christ peut faire des miracles et s'il peut les faire publiquement.
Dans ce cas, il lui serait facile de faire croire.
Il commence toujours par refuser de faire des miracles car cela dispense les gens de croire.
Dans cette mesure, la croyance est supérieure au savoir car elle impose un acte.

\section*{\color{red}B. Les faiseurs de croyance.}
Dans le livre VII de \emph{La République}, qui est un des deux textes politiques de Platon, Platon élabore une image sensée rendre compte de notre condition humaine relativement à ce que nous savons et ne savons pas.\par
L'image de la caverne se présente comme une image totale et définitive de la réalité.
Elle anticipe le principe du cinéma.
Au centre du dispositif, il y a la présence incontournable de puissances aussi bien divines qu'humaines qui influencent voire produisent nos croyances.

\section*{\color{red}C. Les rhéteurs}
Le rhéteur est un personnage central de la démocratie athénienne dans laquelle il y avait une valorisation extrême de la parole.
Ils interviennent aux assemblées et aux procès.
\subsection*{Texte 7: Platon, \emph{Gorgias}}
Pour Platon, un principe d'organisation central pour la cité est la division du travail.
Cela pose la question de savoir qui est le spécialiste de la politique. 
C'est ainsi que se présente Gorgias.
Il est le spécialiste de la persuasion et c'est ce qui le place au dessus de toutes les autres compétences.
D'où la comparaiso, avec le médecin, qui, malgré tout son savoir, n'arrive pas à persuader le patient.\par
La rhétorique représente une autonomisation du pouvoir de la parole commune, instrument de persuasion indépendamment de tout savoir et de toute maîtrise objective de ce dont on parle.
Mais le point essentiel est qu'aux yeux de Gorgias, c'est une compétence indispensable.\par
Du point de vue de Platon, il faudrait que ce soit le savant qui prenne la position du rhéteur pour prendre les commandes de la cité.

\section*{\color{red}D. Sophistes.}
Le sophiste est davantage un intellectuel qui prétend éxercer une influence sur la jeunesse et sur les esprites, sur l'élite cultivée.
C'est une influence qui développe une fausse logique, qui trafique la langue dans la logique avec des sophismes\footnote{Sophias: sagesse.}.

\section*{\color{red}E. Les Artistes.}
Ce sont des producteurs de croyance car ils ont la prétention de pouvoir recréer la réaliter voir de pouvoir reproduite intégralement des mondes de fiction.
Pour Platon, ils se prennent pour des dieux.
Platon a une réaction beaucoup plus violente envers les artistes: il veut les chasser de la cité.
Dans le livre X de \emph{La République}, Platon compare le peintre au menuisier.
Le menuisier est un artisan qui produit quelque chose de concret (un lit) en imitant son Idée. 
L'artiste, lui, va imiter l'apparence du lit pour en créer une simple image.
Il crée donc une apparence d'apparence.
Cela produit un schéma avec de multiples niveaux de réalités et d'apparences.\footnote{Magritte : Trahison des Images}\par
On pourrait objecter à Platon que face à une \oe{uvre} d'art, on est en général parfaitement conscient d'avoir affaire à une fiction ou a une simple apparence.
L'artiste est précisément celui qui nous permet de mettre à distance le réel pour l'interroger.\par
Laclos ressemble à Platon dans la façon de mettre en scène ses personnages.

\section*{\color{red}F. Les Savants}
C'est l'horizon ultime de la philosophie de Platon: donner le pouvoir au savoir et à ceux qui savent.
C'est son projet politique.
Cela pose d'insurmontables problèmes car par définition, le savoir est quelque chose de politiquement invisible.
On peut penser que le véritable objectif de Platon était plutôt d'essayer de préserver le mode de vie qu'il privilégiait le plus, à savoir le mode de vie académique contre l'influence néfaste de la Politique.\par
Platon valorise la science.
Le problème est que les savants sont peu nombreux et le savoir sur lequel repose leur autorité est difficilement partageable avec le reste de la population.
Là aussi les savants devront faire croire à leur expertise et légitimité quand bien même ils sont capables pour leur part de différencier le savoir et la croyance.
Autre difficulté: il n'est pas exclu que des savants placés en position de pouvoir soient tentés d'en abuser et de s'en servir pour leurs propres intérêts.
\color{red}Dialectique: art du dialogue.\color{black}\par
Hannah Arendt voit dans la philosophie politique de Platon une mise en concurrence entre deux espaces relativement incompatibles: l'espace académique de la science tourné vers la vérité et l'espace politique de la cité dominé par les opinions et les luttes de pouvoir.
Pour Platon, le seul moyen de résoudre cette incompatibilité est de mettre la politique au service de l'Académie.\par
Tout le problème est que lestitle savants peuvent aussi abuser de leur autorité soit en faisant croire qu'ils savent soit en instrumentalisant leur autorité scientifique au profit de leur intérêt.
\pagebreak
\section*{\color{red}G. La Religion}
Il peut sembler essentiel à toute religion dans la mesure où elles ressemblent à des dogmes et des croyances d'essayer de faire croire.\par
Dans certains cas, la religion est tellement présente dans la société qu'elle se transmet par tradition, éducation, rituels omniprésents.
La situation se complique lorsque plusieurs religions se rencontrent ou lorsqu'on cherche à propager une religion (croisades, évangélisations, \dots).
Les monothéismes ont peut-être davantage ce genre de prétentions, notamment dans le christianisme.
Le problème est que la croyance est un acte intime et volontaire qui ne peut pas être provoqué.
Ce problème se pose de façon particulièrement nette en théologie à propos de la question des miracles.\par
À priori, rien n'est plus efficace pour susciter la croyance qu'un miracle.
Cependant, le miracle a quelque chose d'insatisfaisant pour susciter la croyance.
Le Christ refuse d'abord de faire le miracle, il reproche les demandes : si Dieu fait des miracles, la foi devient superflue.
Le miracle peut faire croire mais crée une sorte de court-circuit dans la logique de la croyance.\par
La signification du miracle est décuplée par le fait qu'il ait lieu en public devant une foule nombreuse : les témoins jouent un rôle décisif.
Cela donne au miracle une dimension politique.
Dans L'Évangile, cet épisode décide les grands prêtres à demander l'éxécution de Jesus.\par
La transfiguration est un autre épisode de l'Évangile dans lequel le Christ révèle sa vraie nature à 3 de ses disciples.
Ce tableau\footnote{Raphaël, \emph{La Transfiguration}} met en scène la révélation surnaturelle et ultime de la vérité, celle qui mettrait tout le monde d'accord.
La partie inférieure du tableau représente l'épisode suivant: une foule réunie autour d'un jeune garçon possédé.
Lorsqu'il descend de la montagne, le Christ rencontre cette foule.
On lui demande de soigner le garçon. Elle pourrait représenter la politique et la société dominée par les désaccords et les opinions.
Le génie de Raphaël est d'avoir imaginé de coller les deux scènes ensemble.\par
Bien sur, la question des miracles reste une question théologique, dans le fonctionnement habituel des sociétés, ce sont les autorités religieuses qui sont chargées d'entretenir et de préserver la croyance.
Mais là aussi, elles peuvent également l'instrumentaliser.
\pagebreak
\section*{\color{red}H. Faire-croire, dans un monde désenchanté ?}
Notre époque, dans ses orientations philosophiques et métaphysiques les plus profondes, se veut rationaliste.
Cela va de pair avec ce qu'on appelle le désenchantement du monde\footnote{Max Weber}, c'est-à-dire un certain recul des croyances religieuses au profit de recherches de certitudes rationnelles.
C'est ce qui a conduit Nietzsche à affirmer de façon provocatrice <<Dieu est mort>>.
Ce désenchantement prend sans-doute sa source dans la philosophie des Lumières.
Marquée par le rejet des superstitions et l'exigence de penser par soi-même.
Dans un tel monde, l'attitude la plus fondamentale à l'égard du réel n'est plus l'admiration naïve mais le doute.\par
Descartes, dans \emph{Les Méditations Métaphysiques}, (1641) a joué un rôle central dans cette évolution, son projet est rationaliste: rejeter toutes les croyances, toutes les opinions infondées pour repartir depuis les <<fondements>>.
Il veut tout ramener à des évidences absolues: des idées claires et distinctes.
Le paradoxe est que cette démarche ultra-rationaliste le conduit très vite au doute tellement radical qu'il s'approche vite de la folie.\par
Il commence par douter de ce que lui enseignent ses sens (argument du rêve). Cela fait apparaître que ce qui nous semble le plus évident est aussi une croyance.\par
Même si tout est un rêve, les idées qui composent un rêve conservent une certaine consistence et une certaine logique, et, dans le rêve, les vérités mathématiques restent vraies.
<<Que je veille ou que je dorme, 2 plus 3 font 5>>.
La vérité des mathématiques a un plus grand degré de certitude que les sens.\par
Cependant, Descartes va là aussi trouver une raison de douter qui est la fiction du Dieu trompeur ou du malin génie: <<Mon esprit aurait pu être fait par un Dieu qui aurait voulu que je me trompe sur les choses les plus évidentes>>.
Le malin génie ressemble au montreur de marionnettes de Platon.
Mais ici, la caverne est notre propre pensée manipulée de l'intérieur jusque dans les règles les plus élémentaires de la logique.
Le malin génie trafique la raison elle-même.
Tout cela montre qie dans notre époque rationaliste, notre confiance dans la raison humaine est elle aussi une croyance et une croyance fabriquée.\par
Avec Descartes, on s'est rendus compte que le doute n'est pas seulement une attitue négative.
Il est aussi générateur de découvertes et de connaissances.
On ouvre de nouveaux chemins pour le savoir.\par
Dans un monde où on croit avant tout à la raison et à l'exigence de penser par soi-même; l'individualisme devient une valeur centrale, ou plutôt, c'est peut-être le dernier refuge de nos valeurs: <<Moi donc, à tout le moi, ne suis-je pas quelque chose?>>.\par
Au delà du questionnement théorique ou philosophique, la croyance dans le moi peut aussi être l'objet d'un intense investissement affectif: narcissisme.
C'est sans doute un des principaux enjeux de la croyance de notre époque.
Le <<moi>> est devenu une idole.\par
Le paradoxe est que cette croyance a aussi très largement une dimension socialge au sens où certains sociologues parlent d'une culture du narcissisme (réseaux sociaux, publicités, \dots).
Cette culture nous fournit sans-cesse des modèles identificatoires.
Les fonctions sont souvent centrées sur un héros omniprésent et qui contrôle tout.\par
Dans \emph{Lorenzaccio}, on a déjà affaire à quelque chose de ce genre puisque toute l'intrigue politique de la pièce se retrouve centrée sur un individu et sur la façon dont cet individu masque son intériorité.
Cette exhaltation de l'intériorité de l'individu est un thème typiquement romantique.\par
Dans \emph{Les Liaisons Dangereuses}, on a une intrigue plus collective: les personnages ont un plus grand contrôle sur le <<moi>>, ils se font écho, sans argumentation.

\end{document}