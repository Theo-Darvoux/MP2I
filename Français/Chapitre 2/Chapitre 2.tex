\documentclass[12pt]{article}
\usepackage{ragged2e}
\usepackage[left=2cm, right=2cm, top=2cm, bottom=2cm]{geometry}
\usepackage{color}
\usepackage{amsmath, amssymb}
\usepackage{lastpage}
\usepackage{fancyhdr}
\usepackage[T1]{fontenc}
\usepackage{hyperref}

\title{Chapitre 2\\\large Faire croire, pour faire : dissimulation, simulation et action dans \emph{Lorenzaccio} de Musset}
\date{}
\author{}


\setlength{\headheight}{15pt}
\pagestyle{fancy}
\cfoot{\thepage\ sur \pageref*{LastPage}}

\hypersetup{
    colorlinks=true,
    citecolor=black,
    linktoc=all,
    linkcolor=blue
}

\renewcommand*\contentsname{Sommaire}

\begin{document}
\maketitle
\thispagestyle{fancy}
\begin{center}
    \LARGE{etienne la feuraude}
\end{center}
\hrule
\tableofcontents
\hrule
\fancyhead[L]{Théo DARVOUX}
\fancyhead[R]{MP2I Paul Valéry}
\fancyhead[C]{Français-Philosophie}
\pagebreak
\section*{\color{red}Introduction}
On a vu avec Arendt la grande proximité qu'il peut y avoir entre le mensonge et l'action politique.
L'intrigue de \emph{Lorenzaccio} de Musset se construit entièrement autour d'un acte à réaliser d'un projet brûlant à accomplir : l'assassinat du tyran.
Pour y parvenir, Lorenzo a choisi de se cacher, de simuler pour gagner sa confiance.
Habituellement, faire croire en politique, c'est dissimuler sous un masque légitime ou vertueux des intérêts et du vice.
Chez Lorenzo, c'est l'inverse, il dissimule des intentions vertueuses sous un masque de vice.
Le problème est qu'à force de porter de masque, il risque de réellement se corrompre : <<Le vice a été pour moi un vêtement, maintenant il est collé à ma peau.>> (III. 3).
Ce qui doit tout justifier à la fin, c'est un seul acte décisif et qui devrait favoriser la cause Républicaine.

Cependant, la dimension Républicaine et politique de cet acte ne va pas de soi, il s'agit aussi d'une vengeance personnelle, le fait que Lorenzo agisse seul et ne semble pas du tout croire aux capacités d'action des Républicains entretient un climat de grand pessimisme qui correspond aussi à ce que Musset pense de son époque.
Faire Croire dans cette pièce, cela pourrait aussi s'appliquer à ce problème : comment faire croire à la République, à cet idéal politique qui semble irréalisable.
Cela se redouble d'une autre question plus esthétique: comment faire croire au théàtre à ce qu'on fait représenter, comment recréer la Florence de la Renaissance sur scène, comment évoquer tous ces personnages et tous ces enjeux ?
De ce point de vue là, Musset s'est sans doute représenter dans le personnage de Tebaldeo : quel est le rôle de l'artiste, son rôle politique, à quoi doit-il faire croire ?
On peut s'interroger sur le détour historique choisi par Musset pour parler clairement de sa propre époque.
On pourrait dire que Musset, lui aussi chosit de porter un masque, de dissimuler ses intentions, mais en ayant tout de même en vue un acte politique dans l'écriture de sa pièce.
\addcontentsline{toc}{section}{Introduction}

\section*{\color{red}I) Entrer dans l'histoire.}
Cette expression désigne d'abord ce qu'on attend d'un acte d'exposition: exposer les enjeux de l'intrigue, présenter les personnages, présenter le contexte, surtout quand on est dans une époque complètement différente.
Entrer dans l'histoire, c'est aussi ce que veut faire Lorenzo, à sa façon, influencer par son acte héroïque le cours des évènements historiques.
Cependant, il y a une grande hésitation dans cette pièce sur la vision de l'Histoire.
D'un côté, elle pourrait être un processus qui mène vers le progrès, vers la réalisation des idéaux Républicains.
D'un autre côté, elle est perçue de façon beaucoup plus pessimiste, comme un processus chaotique dans le quel tout vient se corrompre, il est difficile de placer le personnage de Lorenzo entre ces deux visions de l'Histoire.
La République elle-même est dans la pièce aussi bien un enjeu de désir pour l'avenir qu'un souvenir nostalgique d'un passé perdu
\subsection*{a) Le mal du siècle}
On désigne ainsi le malaise de cette génération qui est arrivée après la Révolution française, après l'empire et pour qui la République était à la fois un souvenir et un idéal inatteignable : une génération très désenchantée.
La \underline{Liberté Guidant Le Peuple} : Allégorie qui mène les Républicains dans les rues de Paris, cependant, c'est aussi un idéal qui exige des sacrifices : la liberté chevauche un tas de cadavres, son visage est frois et inflexible.\\
L'enfant est inconscient, c'est la première victime du sacrifice Républicain. Dans ce tableau, la Liberté pourrait bien être une forme d'Hallucination Collective. Ce tableau exprime lui aussi un profond pessisme, ou une incertitude sur la liberté républicaine.
\addcontentsline{toc}{subsection}{a) Le mal du siècle}
\subsection*{b) Le masque du vice}
Musset entre très vite dans l'intrigue du sujet: les intrigues du Duc et le rôle actif et particulièrement trouble de Lorenzo, à ce moment là, le spectateur n'a aucun moyen de connaître les véritables intentions de Lorenzo.
Dès sa première réplique, Lorenzo va très loin dans le vice.
Le portrait que Lorenzo fait de la jeune fille est aussi un portrait de lui-même car lui aussi se prostitue d'une certaine manière auprès du Duc pour gagner sa confiance, dès le départ, presque toutes les répliques de Lorenzo envers le Duc sont à double sens, ce double sens, Lorenzo le dit pour lui-même : <<\underline{Le vrai mérite est de frapper juste}>>.
Lorenzaccio est une grande pièce de l'implicite et du double sens.
\addcontentsline{toc}{subsection}{b) Le masque du vice}
\addcontentsline{toc}{section}{I) Entrer dans l'histoire}
\end{document}