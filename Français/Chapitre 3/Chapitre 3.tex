\documentclass[12pt]{article}
\usepackage{ragged2e}
\usepackage[left=2cm, right=2cm, top=2cm, bottom=2cm]{geometry}
\usepackage{color}
\usepackage{amsmath, amssymb, amsthm}
\usepackage{lastpage}
\usepackage{fancyhdr}
\usepackage[T1]{fontenc}
\usepackage{hyperref}

\title{Chapitre 3\\\large L'illusion des Liaisons}
\date{}
\author{}


\setlength{\headheight}{15pt}
\pagestyle{fancy}
\cfoot{\thepage\ sur \pageref*{LastPage}}

\hypersetup{
    colorlinks=true,
    citecolor=black,
    linktoc=all,
    linkcolor=blue
}

\renewcommand*\contentsname{Sommaire}

\begin{document}
\maketitle
\thispagestyle{fancy}

\hrule
\tableofcontents
\hrule
\fancyhead[L]{Théo DARVOUX}
\fancyhead[R]{MP2I Paul Valéry}
\fancyhead[C]{Français-Philosophie}
\pagebreak
\section*{\color{red}Introduction}
L'illusion est d'abord dans l'intrigue elle-même puisque les personnages entre-eux multiplient les mensonges, les manipulations mais c'est aussi le dispositif de l'oeuvre, qui se présente simplement comme un recueil de lettres.
Les personnages sont d'ailleurs fictifs : c'est une illusion produite par l'auteur, on pourrait dire que c'est une illusion d'illusion.
Pourtant, il faut bien que derrière ces illusions, il y ait un fond de vérité.
Même si tous les personnages mentent ou dissimulent, il faut quand même expliquer ce qui les pousse à écrire toutes ces lettres.
Il leur arrive parfois d'être très sincères, notamment, chaque personnage a son confident ou sa confidente, c'est un dépositaire des secrets.
Ce qui apparaît est la vérité de leurs désirs.
On peut se poser la question à propos de Laclos lui-même : qu'est-ce qui le pousse à écrire, quelle vérité recherche-t-il ?
Question d'autant plus intéressante qu'il ne parle jamais en son nom propre, il pourrait y avoir une raison morale, ou juste pour le plaisir, ce qui le rapprocherait des libertins qu'il met en scène.
Il ne donne jamais clairement son point de vue, l'auteur est un peu comme dieu vis-à-vis de sa création.
Il est absent et on ne sait pas très bien ce qu'il en pense et pourquoi il a créé tout ça.
Les jeux d'illusion, l'idée que le monde est un grand théâtre, c'est un thème baroque.
Dans l'esthétique baroque, derrière les illusions, il peut y avoir une vérité, derrière le désordre apparent, il peut y avoir un ordre caché.
C'est ce que le philosophie Leibniz a appelé l'harmonie préétablie.
Chaque être est comme un instrument qui joue sa partition, mais le créateur a tout écrit de façon harmonieuse.
Les liaisons dangereuses évoquent cette esthétique du contrepoint, chaque personnage déroule sa parition, affirme son point de vue, sans connaître celui des autre et c'est Laclos qui maîtrise la cohérence de l'ensemble.
Le lecteur du roman se trouve lui aussi placé dans cette position quasi divine.
Mais, on peut constater qu'aucun ordre caché, qu'aucune leçon morale particulièrement claire n'émerge de la confrontation de ces lettres.
Ce qui en ressort c'est surtout une impression de jubilation et la vérité de Laclos, comme la vérité de ses personnages, c'est peut-être la vérité de son désir.
\addcontentsline{toc}{section}{Introduction}

\section*{\color{red}I) Les <<projets>>.}
\addcontentsline{toc}{section}{I) Les <<projets>>.}
\subsection*{a) L'effacement du Créateur.}
Au début du roman, les projets ne manquent pas, Mme de Volanges projette de marier sa fille Cécile, qui sort du couvent.
Mme de Merteuil veut la faire corrompre par Valmont.
Valmont veut séduire la présidente de Tourvel.
Le projet de Laclos: pourquoi écrit-il ce roman ?
Au début du roman, Laclos met en scène son propre effacement en tant qu'auteur, faisant comme si il n'avait pas écrit le roman, comme si il avait simplement recueilli et compilé les lettres.
Le premier personnage du roman est l'éditeur.
Dans un second temps, Laclos se présente sous le masque du rédacteur.
Dans la préface du rédacteur, il prétend que son rôle d'élaguer les lettres inutiles.
Non seulement, il n'aurait pas produit le texte, mais en plus il l'aurait réduit <<ma mission ne s'étendait pas plus loin>>.
Le terme de mission est significatif, c'est un terme qu'utilisera Valmont, cela pose la question du but de toute cette entreprise.
Le rédacteur prétend qu'il a eu une intention morale.
Dans son agrément entre la variété des styles.
\pagebreak

\addcontentsline{toc}{subsection}{a) L'effacement du Créateur.}
\subsection*{b) Le <<bel objet>>}
Cécile de Volanges sort du couvent pour être mariée, c'est le projet de sa mère, mais elle est aussi au coeur du projet de Mme de Merteuil, qui veut la former, d'abord avec Danceny puis avec Valmont dans une sorte d'éducation libertine, c'est Merteuil qui l'appelle le <<bel objet>> dans la lettre II.
Les deux projet sont symétriques, c'est une victime sacrificielle pour sa mère et pour Merteuil.
Pourtant, Cécile va montrer qu'elle est aussi un sujet de désir. On devine qu'elle n'est pas complètement naïve par rapport à l'amour, elle a déjà des envies et des idées par elle-même.
Peut-être le projet de Merteuil va-t-il l'aider à mieux s'approprier son désir dans une société où règne le mariage arrangé.
Dans la lettre I, le désir est là même s'il n'a pas encore d'objet, immédiatement, la société resserre son emprise sur ce désir là à travers le système des mariages arrangés.
Cécile se rapproche des héroïnes de contes de fées qui sont au début de leur désir, qui est imprégné de stéréotypes, comme Blanche Neige de Walt Disney, la représentation du désir y est complètement narcissique, elle le trouve en elle-même.
C'est un désir purement intérieur, il n'y a pas d'objet. Il y a une sorte de déséquilibre entre la pensée et le réel.
Cécile se confie à son amie de couvent, qui est la seule personne qui reçoit des lettres mais n'en écrit pas.
\addcontentsline{toc}{subsection}{b) Le <<bel objet>>}
\subsection*{c) Le projet de Merteuil.}
Merteuil dit à Valmont qu'il servira l'amour et la vengeance, Merteuil veut que cette histoire soit imprimée dans les mémoires de Valmont. Le <<sort inévitable (Lettre 2)>> est de devenir cocu.
Il y a une claire référence à l'École des Femmes de Molière, dans laquelle Arnolph, qui a la phobie d'être trompé par sa femme.
\addcontentsline{toc}{subsection}{c) Le projet de Merteuil}
\subsection*{d) La mission d'amour de Valmont}
Valmont ne peut pas répondre tout de suite à la demande de Merteuil parce qu'il a son propre projet, qui est de séduire la présidente de Tourvel.
Ce qui l'attire chez elle, c'est un désir particulièrement transgressif, parce qu'elle est non seulement mariée mais aussi dévote, cette entreprise de séduction va prendre la forme d'une lutte, Valmont cherche à la convertir à l'amour, et elle cherche à le convertir, à le ramener dans le droit chemin.
Cela donne des lettres où le vocabulaire de la religion et de l'amour sont constamment mélangés de façon transgressive et blasphématoire.
L'amour est présenté  comme une religion <<Elle ne se doute pas de la divinité que j'y adore>>.
La religion et le désir se mélangent pour Valmont, il érotise la religion et le désir prend une dimension sacrée.
\section*{II) Les liens et les liaisons.}
Les liaisons sont des relations de séduction et de désir, qui sont très instables, en général illégitimes, Danceny-Cécile, Valmont-Cécile.
Mais à côté des liaisons, il y a les liens, c'est-à-dire les relations sociales légitimes qui unissent les personnages et qui sont justement menacées par les liaisons.
Par exemple, la famille, sans père ni mari, le mariage, les amitiés, la réputation. Tous ces liens forment un réseau, dont un des principaux buts est d'empêcher le développement des liaisons.
\addcontentsline{toc}{subsection}{d) La mission d'amour de Valmont}
\pagebreak
\subsection*{a) La religion.}
Religion vient du Latin Religare, qui veut dire relier, c'est un lien social, qui garantit d'autres liens comme le mariage.
C'est Mme de Tourvel qui est le personnage religieux dans cette histoire, une dévote, elle félicite Cécile de son prochain mariage.
Mme de Tourvel a aussi été mariée par Mme de Volanges.
Lettres 21 et 22 : Valmont va faire une mise en scène en allant aider une famille qui se fait confisquer ses biens, pour redorer son blason aux yeux de la présidente de Tourvel, sachant qu'il est suivi.
Il s'agit de travailler sur sa reputation, Mme de Tourvel pense qu'il a été corrompu par de mauvaises fréquentations : <<exemple de plus du danger des liaisons>>.
\addcontentsline{toc}{subsection}{a) La religion.}
\subsection*{b) Le mariage.}
C'est une autre lien garanti par la religion et qui a un lien complexe avec le désir érotique, mais on est dans un milieu social où le mariage arrangé est la règle.
S'il y a peut-être un message dans cette oeuvre, c'est la critique de cette pratique sociale.
Dans la lettre 8, Mme de Tourvel se réjouit de l'établissement de Cécile, le mariage apparaît comme ce qui vient refermer le désir au moment où il commence à s'ouvrir, Mme de Tourvel a aussi été mariée par Mme de Volanges.
Le paradoxe est que, dans cette histoire dominée par la loi du mariage, aucun mari n'est représenté.
\addcontentsline{toc}{subsection}{b) Le mariage.}
\subsection*{c) La famille.}
Les liens familiaux sont surtout représentés par Mme de Volanges et la relation qu'elle a avec sa fille, pas frère, soeur ou père.
Il y a une grande opposition entre l'amour familial et l'amour érotique c'est directement lié à l'interdit de l'inceste, l'érotisme est hors de la famille, le mariage est entre les deux malgré tout.
Un élément important est que Merteuil ne veut pas que Gercourt devienne son cousin.
\addcontentsline{toc}{subsection}{c) La famille.}
\subsection*{d) La réputation.}
C'est un élément essentiel dans les rapports entre les personnages puisque chacun d'entre-eux est accompagné et précédé par sa réputation. Mme de Tourvel est une dévote, Valmont un libertin, on peut remarquer que cette réputation sulfureuse ne le dérange pas tant que ça, de la même façon, Gercourt a une réputation douteuse aussi.
Les hommes souffrent moins d'une mauvaise réputation, mais pour les femmes, la réputation est beaucoup plus fragile.
La réputation de la marquise de Merteuil reste irréprochable, elle aurait su tenir sa réputation notamment car elle aurait résisté à Valmont.
\addcontentsline{toc}{subsection}{d) La réputation}
\subsection*{e) L'amitié.}
L'amitié apparaît comme un lien ambivalent dans les Liaisons Dangereuses, d'un côté, elle peut apparaître comme précieuse et protectrice pour des personnages qui veillent les uns sur les autres, par exemple Cécile et Sophie, Mme de Volanges et de Tourvel, Danceny et Valmont ou Merteuil et Cécile.
Il y a aussi les amitiés manipulées, entre Merteuil et Cécile ou Merteuil et Mme de Volanges.
Au centre de tout ça, il y a l'amitié entre Valmont et Merteuil, le lien érotique, la complicité, d'importantes confidences et de pouvoir, peu à peu, on sent que Valmont essaie de l'impressionner, dans les lettres 47 et 48.
\addcontentsline{toc}{subsection}{e) L'amitié}
\section*{III) Désirs et illusions}
La croyance dans les liaisons dangereuses est toujours liée au désir, c'est une puissance hallucinatoire.
Par définition, l'objet du désir est un objet absent, qui manque, qui se dérobe, c'est donc un objet qu'on doit imaginer, symboliser, idéaliser ou diviniser au point de produire une vraie religion du désir.
L'objet du désir n'est jamais un objet réel. En réalité, c'est un objet qu'on rencontre surtout en soi-même.
\addcontentsline{toc}{section}{III) Désirs et illusions.}
\subsection*{a) L'idéalisation}
C'est le fait de surestimer l'objet de son désir, de lui prêter des qualités qu'il n'a pas.
<<Qui pourrait arrêter une femme qui fait sans s'en douter l'éloge de ce qu'elle aime>>, elle est déjà en pleine idéalisation, elle compose une image, qui symbolise le travail de l'idéalisation, elle ne s'en rend pas compte.
Laclos arrive a rendre le désir de quelqu'un qui ne se rend pas compte qu'il désire.
Le tableau de Valmont faisant la charité pour les pauvres est un exemple typique d'idéalisation artificiellement provoquée.
C'est aussi un exemple de sublimation de l'esprit érotique, la transformation d'un désir sexuel en un désir plus noble.
Mais l'arrière plan érotique de cette image reste malgré tout présent.
Le psychanaliste Lacan utilise les symboles $\$$ comme sujet barré du désir, $\lozenge$ pour l'écran du fantasme, l'imaginaire, c'est un obstacle qui emêche de voir le vrai objet, et $a$, l'objet petit a qui est le registre du symbolique, le langage.
\addcontentsline{toc}{subsection}{a) L'idéalisation}
\subsection*{b) Le narcissisme du désir}
Une autre grande motivation des personnages dans leur désir, indépendante de tout objet est l'amour qu'ils se portent à eux-mêmes et à leur réputation, c'est notamment ce qui caractérise Merteuil et Valmont : <<me voir confondue avec les femmes que vous méprisez>> (lettre 26).
Cette dimension narcissique du désir est très présente chez Valmont et en particulier dans sa rivalité avec Mme de Merteuil dans un grand nombre de lettres, il manifeste le souci de marquer la postérité par ses exploits et c'est particulièrement auprès de Mme de Merteuil qu'il s'en vante.
A l'inverse, c'est sur ce ressort narcissique qu'elle ne cesse d'appuyer pour le manipuler en lui faisant sentir que ses exploits ne sont pas si admirables que cela en minimisant son aventure avec la présidente de Tourvel.
Une dimension essentielle de ce narcissisme est la jalousie : désirer ce que possède quelqu'un d'autre, avoir peur de perdre ce qu'on possède au profit d'un autre.
Dans la jalousie, on s'éloigne très facilement de la réalité, et cela montre à quel point le sujet du désir à peu de certitudes sur la réalisation de son propre désir.
La force de la jaloufilesie est indépendante de l'objet du désir, elle vient de la présence d'un autre sujet du désir, réel ou fantasmé, qui apparaît comme un rival menaçant.
Dans le cas de Valmont, ce sera Prévan. Il y a aussi la jalousie de Gercourt, qu'on cherche à provoquer, c'est ce que le philosophe René Girard appelle le désir mimétique dans <<Mensonges romantiques et vérités romanesques>>.
C'est l'illusion que la jalousie serait la rencontre immédiate entre un sujet du désir et l'objet du désir, c'est le mensonge romantique.
La vérité romanesque c'est que nous formons toujours notre désir en fonction de ce que désirent les autres.
C'est une vérité qu'on rencontre dans les romans, dans <<Le Rouge et le Noir>> de Stendhal, il appelait ça la cristallisation.
La publicité joue sur ce désir mimétique, on désire ce que les autres désirent.
Lettre 70, c'est la rivalité avec Prévan qui l'oblige à séduire Cécile de Volanges, par pression.
La jalousie est aussi présente du côté de Mme de Merteuil vis-à-vis de la présidente de Tourvel, surtout quand celle-ci commence à séduire Valmont (141).
\addcontentsline{toc}{subsection}{b) Le narcissisme du désir.}
\subsection*{c) La dissimulation}
Lettre 26 : <<Je ne sais ni dissimuler ni combattre les émotions que j'éprouve>>.
À un moment ou un autre, le désir finit toujours par se révéler.
\addcontentsline{toc}{subsection}{c) La dissimulation}
\section*{IV) Le danger des liaisons.}
Dans cette oeuvre, le désir apparaît comme structurellement illusoire et la réalité va résider du côté de la souffrance et de la violence. C'est évidemment ce qui domine à la fin de l'oeuvre, où presque tous les personnages tombent de très haut, y compris Valmont et Merteuil, qui, cependant ne sont pas naïfs vis-à-vis de cette réalité, au contraire, pour eux, c'est l'enjeu suprême : se rendre maîtres de cette violence, dépasser toutes les illusions du désir, devenir des <<divinités cruelles de l'amour>>, mais peut-être est-ce là l'illusion la plus trompeuse tant cette position de maîtrise est impossible à atteindre : même Valmont et Merteuil vont se laisser surprendre par l'amour. C'est un thème chez Marivaux.
Mais avant cela, ils ont tendance à essayer de concilier leur désir avec leur lucidité sur la cruauté des liaisons, cette cruauté fait même parti de leur désir, c'est même la définition de cruauté : malveillance avec plaisir gratuit.
\addcontentsline{toc}{section}{IV) Le danger des liaisons.}
\subsection*{a) La violence}
Lettre 96 : Le plaisir du récit prolonge la satisfaction du viol et vient lui donner son supplément de cruauté.
Il le fait passer comme un sujet secondaire, après Mme de Tourvel. \\
Lettre 141 : C'est évidemment aussi dans les ruptures que la cruauté des personnages se fait sentir. La lettre de rupture que Merteuil à écrit pour Valmont à l'adresse de Mme de Tourvel est sans doute une des plus cruelles qu'on puisse imaginer précisément parce qu'il ne l'écrit pas lui-même
\addcontentsline{toc}{subsection}{a) La violence}
\subsection*{b) La guerre}
Le vocabulaire guerrier est présent dans tout le roman d'abord comme une métaphore notamment lorsque Valmont parle de ses conquêtes, de ses victoires et de ses assauts.
La violence prend le dessus notamment entre Valmont et Merteuil, qui se lancent réciproquement des ultimatums <<Je serais ou votre amant ou votre ennemi>>, <<Et bien, la guerre.>>.
Une fois cette guerre déclarée, le roman se termine très vite, comme si c'était toute la violence accumulée depuis le début qui se libérait d'un seul coup.
La liaison entre Merteuil et Valmont est certainement la liaison dangereuse par excellence.
\subsection*{c) La fatalité} 
C'est le mot qu'utilisent les personnages à la fin pour désigner les malheurs qui leur arrivent.\\
Ce terme de fatalité indique que les liaisons finissent toujours mal en général.
Il renvoie aussi à une idée de destin et à une tonalité tragique.
Le destin prend pour instrument le désir des personnages, les liaisons dangereuses ne sont pas une tragédie, il n'y a pas de Dieu derrière la fatalité qui s'abbat sur les personnages, il n'y a justement que leurs désirs irrésistibles
Il est constamment question d'amour dans les liaisons dangereuses, c'est un roman qui parle d'amour tout le temps, et en même temps, on peut se demander s'il y a vraiment de l'amour dans cette histoire.
\section*{III. L'amour}
Il y a les amours innocents, entre Cécile et sa mère, mais aussi les amours plus troubles, entre Valmont et Mme de Tourvel, entre Merteuil et Valmont, entre Danceny et Cécile.
\subsection*{a) Valmont et Tourvel}
En quittant les bras de Mme de Tourvel, Valmont se vante immédiatement de son succès auprès de Merteuil, il lui parle de son plaisir, il se dit trop plein de son bonheur. Après le triomphe, en principe, il devrait déjà en avoir assez.
Valmont est lui-même étonné de son propre plaisir, de son intensité et surtout du fait qu'il ne s'éteint pas.
Ce surcroît de plaisir a plusieurs explications : le plaisir sexuel, Valmont l'attribue à la gloire, la vanité, il y a peut-être une quatrième raison: la naissance d'un sentiment amoureux. Valmont s'en défend tellement qu'il semble dire l'inverse : <<Vous le niez bien de cent façons, mais vous le prouvez de milles>>
A travers cette question se joue aussi leur amour à eux deux. Valmont assimile l'amour à de l'esclavage et il veut rester libre.
Valmont pense que l'amour est pour les faibles, pareil pour Mme de Merteuil. 
Le plaisir, c'est ce qui vient s'ajouter à une experience comme un supplément, comme quelque chose qui au fond ne correspond à aucune utilité précise.
Valmont a du mal à voir l'amour comme un sentiment qui ne serait pas une forme d'esclavage.
Pour lui, être amoureux, c'est être <<Subjugué>>, être <<maîtrisé comme un écolier>>.
Valmont confond amour et domination. Ce qui va tout faire basculer chez Valmont, c'est l'attitude de Mme de Tourvel après qu'elle a cédé : chez elle, le basculement est total, elle assume presque entièrement le plaisir qu'elle a ressenti et un sentiment amoureux qui va complètement surprendre Valmont, d'autant qu'il est simultanément surpris par son propre plaisir.
On peut comparer la façon d'aimer de Mme de Tourvel à ce qu'Aristote appelait, dans l'Ethique à Nicomaque la philia vertueuse. On voit que la vertu est là quand on ne sait pas ce qui réunit.
Le mot vertueux renvoie à un amour qui rend plus noble, qui élève, ça s'entretient soi-même.
<<Ce bonheur qu'on fait naître est le plus fort lien>>.
\subsection*{b) Merteuil et Valmont}
On peut penser qu'entre ces deux personnages, il y a aussi quelque chose qui ressemble à de l'amour, qui les réunit depuis longtemps, qui a donc une certaine puissance, mais qui va se rompre très brutalement.
Cet amour n'a rien de comparable avec celui qui s'esquisse entre Valmont et Tourvel.
C'est un amour vaniteux dominé par la jalousie, dans lequel chacun est le reflet narcissique de l'autre.
Merteuil exige des sacrifices de Valmont, loin d'éprouver pour lui de la bienveillance, c'est une liaison longuement dangereuse.
Ils ne s'écrivent jamais franchement, par exemple, Madame de Merteuil ne demande jamais franchement à Valmont de quitter Mme de Tourvel, alors qu'elle fait tout pour cela, ce serait reconnaître qu'elle est jalouse.
Inversement, la façon dont Valmont parle de ses sentiments n'est jamais sincère et finalement, chez lui aussi, c'est la vanité qui l'emporte quand il rompt avec Tourvel.
\end{document}