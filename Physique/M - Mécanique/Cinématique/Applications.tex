\documentclass[11pt]{article}

\usepackage[paperheight=15in, left=2cm, right=2cm, top=2cm, bottom=2cm]{geometry}
\usepackage[most]{tcolorbox}
\usepackage{amsmath, amssymb, amsthm, enumitem, stmaryrd, cancel, pifont, dsfont, hyperref, fancyhdr, lastpage, tocloft, changepage}

\def\pagetitle{Cinématique}
\setlength{\headheight}{14pt}

\title{\bf{\pagetitle}\n\large{Corrigé}}

\hypersetup{
    colorlinks=true,
    citecolor=black,
    linktoc=all,
    linkcolor=blue
}

\pagestyle{fancy}
\cfoot{\thepage\ sur \pageref*{LastPage}}

\begin{document}

\newcommand{\providetcbcountername}[1]{%
  \@ifundefined{c@tcb@cnt@#1}{%
    --undefined--%
  }{%
    tcb@cnt@#1%
  }
}

\newcommand{\settcbcounter}[2]{%
  \@ifundefined{c@tcb@cnt@#1}{%
    \GenericError{Error}{counter name #1 is no tcb counter }{}{}%
  }{%
    \setcounter{tcb@cnt@#1}{#2}%
   }%
}%

\newcommand{\displaytcbcounter}[1]{% Wrapper for \the...
  \@ifundefined{thetcb@cnt@#1}{%
    \GenericError{Error}{counter name #1 is no tcb counter }{}{}%
  }{%
    \csname thetcb@cnt@#1\endcsname% 
  }%
}

% MATHS %
\newtcbtheorem{thm}{Théorème}
{
    enhanced,frame empty,interior empty,
    colframe=red,
    after skip = 1cm,
    borderline west={1pt}{0pt}{green!25!red},
    borderline south={1pt}{0pt}{green!25!red},
    left=0.2cm,
    attach boxed title to top left={yshift=-2mm,xshift=-2mm},
    coltitle=black,
    fonttitle=\bfseries,
    colbacktitle=white,
    boxed title style={boxrule=.4pt,sharp corners},
    before lower = {\textbf{Preuve :}\n}
}{thm}

\newtcbtheorem[use counter from = thm]{defi}{Définition}
{
    enhanced,frame empty,interior empty,
    colframe=green,
    after skip = 1cm,
    borderline west={1pt}{0pt}{green},
    borderline south={1pt}{0pt}{green},
    left=0.2cm,
    attach boxed title to top left={yshift=-2mm,xshift=-2mm},
    coltitle=black,
    fonttitle=\bfseries,
    colbacktitle=white,
    boxed title style={boxrule=.4pt,sharp corners},
    before lower = {\textbf{Preuve :}\n}
}{defi}

\newtcbtheorem[use counter from = thm]{prop}{Proposition}
{
    enhanced,frame empty,interior empty,
    colframe=blue,
    after skip = 1cm,
    borderline west={1pt}{0pt}{green!25!blue},
    borderline south={1pt}{0pt}{green!25!blue},
    left=0.2cm,
    attach boxed title to top left={yshift=-2mm,xshift=-2mm},
    coltitle=black,
    fonttitle=\bfseries,
    colbacktitle=white,
    boxed title style={boxrule=.4pt,sharp corners},
    before lower = {\textbf{Preuve :}\n}
}{prop}

\newtcbtheorem[use counter from = thm]{corr}{Corrolaire}
{
    enhanced,frame empty,interior empty,
    colframe=blue,
    after skip = 1cm,
    borderline west={1pt}{0pt}{green!25!blue},
    borderline south={1pt}{0pt}{green!25!blue},
    left=0.2cm,
    attach boxed title to top left={yshift=-2mm,xshift=-2mm},
    coltitle=black,
    fonttitle=\bfseries,
    colbacktitle=white,
    boxed title style={boxrule=.4pt,sharp corners},
    before lower = {\textbf{Preuve :}\n}
}{corr}

\newtcbtheorem[use counter from = thm]{lem}{Lemme}
{
    enhanced,frame empty,interior empty,
    colframe=blue,
    after skip = 1cm,
    borderline west={1pt}{0pt}{green!25!blue},
    borderline south={1pt}{0pt}{green!25!blue},
    left=0.2cm,
    attach boxed title to top left={yshift=-2mm,xshift=-2mm},
    coltitle=black,
    fonttitle=\bfseries,
    colbacktitle=white,
    boxed title style={boxrule=.4pt,sharp corners},
    before lower = {\textbf{Preuve :}\n}
}{lem}

\newtcbtheorem[use counter from = thm]{ex}{Exemple}
{
    enhanced,frame empty,interior empty,
    colframe=orange,
    after skip = 1cm,
    borderline west={1pt}{0pt}{green!25!orange},
    borderline south={1pt}{0pt}{green!25!orange},
    left=0.2cm,
    attach boxed title to top left={yshift=-2mm,xshift=-2mm},
    coltitle=black,
    fonttitle=\bfseries,
    colbacktitle=white,
    boxed title style={boxrule=.4pt,sharp corners},
    before lower = {\textbf{Preuve :}\n}
}{ex}

\newtcbtheorem[use counter from = thm]{meth}{Méthode}
{
    enhanced,frame empty,interior empty,
    colframe=purple,
    after skip = 1cm,
    borderline west={1pt}{0pt}{purple},
    borderline south={1pt}{0pt}{purple},
    left=0.2cm,
    attach boxed title to top left={yshift=-2mm,xshift=-2mm},
    coltitle=black,
    fonttitle=\bfseries,
    colbacktitle=white,
    boxed title style={boxrule=.4pt,sharp corners},
    before lower = {\textbf{Preuve :}\n}
}{meth}

\newtcbtheorem[use counter from = thm]{exercice}{Exercice}
{
    enhanced,frame empty,interior empty,
    colframe=blue,
    after skip = 1cm,
    borderline west={1pt}{0pt}{green!25!blue},
    borderline south={1pt}{0pt}{green!25!blue},
    left=0.2cm,
    attach boxed title to top left={yshift=-2mm,xshift=-2mm},
    coltitle=black,
    fonttitle=\bfseries,
    colbacktitle=white,
    boxed title style={boxrule=.4pt,sharp corners},
    before lower = {\textbf{Preuve :}\n}
}{exercice}

% PHYSIQUE %
\newtcbtheorem[use counter from = thm]{qc}{Question de Cours}
{
    enhanced,frame empty,interior empty,
    colframe=red,
    after skip = 1cm,
    borderline west={1pt}{0pt}{green!25!red},
    borderline south={1pt}{0pt}{green!25!red},
    left=0.2cm,
    attach boxed title to top left={yshift=-2mm,xshift=-2mm},
    coltitle=black,
    fonttitle=\bfseries,
    colbacktitle=white,
    boxed title style={boxrule=.4pt,sharp corners},
    before lower = {\textbf{Preuve :}\n}
}{qc}
\newtcbtheorem[use counter from = thm]{app}{Application}
{
    enhanced,frame empty,interior empty,
    colframe=blue,
    after skip = 1cm,
    borderline west={1pt}{0pt}{green!25!blue},
    borderline south={1pt}{0pt}{green!25!blue},
    left=0.2cm,
    attach boxed title to top left={yshift=-2mm,xshift=-2mm},
    coltitle=black,
    fonttitle=\bfseries,
    colbacktitle=white,
    boxed title style={boxrule=.4pt,sharp corners},
    before lower = {\textbf{Preuve :}\n}
}{app}
% MATHS %
\newcommand*{\K}{\mathbb{K}}
\newcommand*{\C}{\mathbb{C}}
\newcommand*{\R}{\mathbb{R}}
\newcommand*{\Q}{\mathbb{Q}}
\newcommand*{\Z}{\mathbb{Z}}
\newcommand*{\N}{\mathbb{N}}
\newcommand*{\F}{\mathcal{F}}

\newcommand{\0}{\varnothing}
\newcommand*{\e}{\varepsilon}
\newcommand*{\g}{\gamma}
\newcommand*{\s}{\sigma}

\newcommand*{\ra}{\Longrightarrow}
\newcommand*{\la}{\Longleftarrow}
\newcommand*{\rla}{\Longleftrightarrow}
\newcommand*{\lb}{\llbracket}
\newcommand*{\rb}{\rrbracket}
\newcommand*{\n}{\\[0.2cm]}

\newcommand*{\cmark}{\ding{51}}
\newcommand*{\xmark}{\ding{55}}

\newcommand{\rg}[1]{\textrm{rg}(#1)}
\newcommand{\vect}[1]{\textrm{Vect}(#1)}
\newcommand{\tr}[1]{\textrm{Tr}(#1)}

\renewcommand{\dim}[1]{\textrm{dim}~#1}
\renewcommand*{\ker}[1]{\textrm{Ker}(#1)}
\renewcommand{\Im}[1]{\textrm{Im}(#1)}

\renewcommand*{\t}{\tau}
\renewcommand*{\phi}{\varphi}

% PHYSIQUE %
\newcommand{\base}[1]{\overrightarrow{e_{\text{#1}}}}

\renewcommand{\cos}[1]{\text{cos}(#1)}
\renewcommand{\sin}[1]{\text{sin}(#1)}
\renewcommand*{\Vec}[1]{\overrightarrow{\text{#1}}}

\thispagestyle{fancy}
\fancyhead[L]{MP2I Paul Valéry}
\fancyhead[C]{\pagetitle}
\fancyhead[R]{2023-2024}

\hrule
\begin{center}
    \LARGE{\textbf{Chapitre M1}}\n
    \large{\pagetitle}\n
    \rule{0.8\textwidth}{0.5pt}
\end{center}


\vspace{0.5cm}

\begin{application}{}{}
    On considère un point $M$ dont les coordonnées cartésiennes dépendent du temps, avec \n
    $x(t) = 2t^{2}$, $y(t) = 4t + 7$ et $z(t) = t(2-t)$
    \begin{enumerate}
        \item Calculer les coordonnées du vecteur vitesse $\Vec{v}$, ainsi que sa norme.
        \item En déduire l’expression du vecteur déplacement élémentaire $d \Vec{OM}$
        \item Calculer les coordonnées du vecteur accélération $\Vec{a}$, ainsi que sa norme.
        \item Calculer l’angle que fait le vecteur vitesse avec l’axe ($Ox$) à l’instant $t = 1$.
    \end{enumerate}
    \tcblower\n
    \boxed{1} Dans la base cartésienne, le vecteur $\Vec{v}$ s'exprime $\Vec{v} = \Dot{x} \base{x} + \Dot{y} \base{y} + \Dot{z} \base{z}$ avec ($\base{x}$, $\base{y}$, $\base{z}$) les vecteurs de la base.\n
    $\Dot{x} = 4t$, $\Dot{y} = 4$ et $\Dot{z} = -2t + 2$\n
    Ainsi on obtient $\Vec{v} = 4t \base{x} + 4 \base{y} + (-2t + 2) \base{z}$\n
    $||\Vec{v}|| = \sqrt{\Dot{x}^{2} + \Dot{y}^{2} + \Dot{z}^{2}}$\n
    $~~~~~~~~ = \sqrt{(4t)^{2} + 4^{2} + (-2t + 2)^{2}}$\n
    $~~~~~~~~ = \sqrt{16t^{2} + 16 + 4t^{2} - 8t + 4}$\n
    $~~~~~~~~ = \sqrt{20t^{2} - 8t + 20}$\\\\
    \boxed{2} On a la relation suivant : $\Vec{v} = \frac{d \Vec{OM}}{dt}$\n
    $d \Vec{OM} = \Vec{v} dt$\n
    $~~~~~~~~ = (\Dot{x} \base{x} + \Dot{y} \base{y} + \Dot{z} \base{z}) dt$\n
    $~~~~~~~~ = (\frac{dx}{dt} \base{x} + \frac{dy}{dt} \base{y} + \frac{dz}{dt} \base{z}) dt$\n
    $~~~~~~~~ = dx \base{x} + dy \base{y} + dz \base{z}$\\\\
    \boxed{3} Dans la base cartésienne, le vecteur $\Vec{a}$ s'exprime $\Vec{a} = \Ddot{x} \base{x} + \Ddot{y} \base{y} + \Ddot{z} \base{z}$ avec ($\base{x}$, $\base{y}$, $\base{z}$) les vecteurs de la base.\n
    $\Ddot{x} = 4$, $\Ddot{y} = 0$ et $\Ddot{z} = -2$\n
    Ainsi on obtient $\Vec{a} = 4 \base{x} -2 \base{z}$\n
    $||\Vec{a}|| = \sqrt{\Ddot{x}^{2} + \Ddot{y}^{2} + \Ddot{z}^{2}}$\n
    $~~~~~~~~ = \sqrt{4^{2} (-2)^{2}}$\n
    $~~~~~~~~ = \sqrt{20} = 2\sqrt{5}$\\\\
    \boxed{4} En passant dans une base cylindrinque, on obtient que $x(t) = r\cos{\theta}$ avec $r = ||\Vec{v}||$\n
    $\theta = cos^{-1}(\frac{x(t)}{||\Vec{v}||})$\n
    A l'instant $t = 1$ : $\theta(1) = cos^{-1}(\frac{x(1)}{||\Vec{v}(1)||})$\n
    $\theta(1) = cos^{-1}(\frac{2}{\sqrt{32}})$\n
    $~~~~~~~ = cos^{-1}(\frac{2}{4\sqrt{2}})$\n
    $~~~~~~~ = cos^{-1}(\frac{\sqrt{2}}{4})$\n
    $~~~~~~~ \approx 69.3^{\circ}$\n
\end{application}

\begin{application}{Changement de coordonnées}{}
    \begin{enumerate}
        \item Sur un schéma, représenter les vecteurs $\base{x}$ et $\base{y}$
    \end{enumerate}
    \tcblower\n
    \boxed{1} TODO 
\end{application}
\end{document}
