\documentclass[10pt]{article}

\usepackage[T1]{fontenc}
\usepackage[left=2cm, right=2cm, top=2cm, bottom=2cm]{geometry}
\usepackage[skins]{tcolorbox}
\usepackage{hyperref, fancyhdr, lastpage, tocloft, ragged2e}
\usepackage{amsmath, amssymb, amsthm}

\def\pagetitle{Propriétés de $\mathbb{R}$}

\title{\bf{\pagetitle}\\\large{Corrigé}}
\date{Septembre 2023}
\author{DARVOUX Théo}

\hypersetup{
    colorlinks=true,
    citecolor=black,
    linktoc=all,
    linkcolor=blue
}

\pagestyle{fancy}
\cfoot{\thepage\ sur \pageref*{LastPage}}


\begin{document}
\renewcommand*\contentsname{Exercices.}
\renewcommand*{\cftsecleader}{\cftdotfill{\cftdotsep}}
\maketitle
\hrule
\tableofcontents
\vspace{0.5cm}
\hrule

\thispagestyle{fancy}
\fancyhead[L]{MP2I Paul Valéry}
\fancyhead[C]{\pagetitle}
\fancyhead[R]{2023-2024}


\section*{Exercice 2.1 [$\blacklozenge\lozenge\lozenge$]}
\begin{tcolorbox}[enhanced, width=6in, center, size=fbox, fontupper=\large, drop shadow southwest]
    Soient $a$ et $b$ deux nombres réels strictement positifs. Démontrer l'inégalité
    \begin{equation*}
        \frac{a^2}{b}+\frac{b^2}{a} \geq a + b
    \end{equation*}
    On a :
    \begin{align*}
        &\hspace{1.2cm}\frac{a^2}{b}+\frac{b^2}{a} \geq a+b\\
        &\iff\frac{a^3-a^2b+b^3-ab^2}{ab}\geq0\\
        &\iff\frac{a^2(a-b)+b^2(b-a)}{ab}\geq0\\
        &\iff\frac{(a-b)(a^2-b^2)}{ab}\geq0\\
        &\iff\frac{(a-b)^2(a+b)}{ab}\geq0
    \end{align*}
    Or $(a-b)^2\geq0$, $(a+b)\geq0$ et $ab\geq0$.\\
    Ainsi, cette inégalité est vraie pour tout $(a,b)\in\mathbb{R}^*_+$.
\end{tcolorbox}
\addcontentsline{toc}{section}{\protect\numberline{}Exercice 2.1}

\section*{Exercice 2.2 [$\blacklozenge\lozenge\lozenge$]}
\begin{tcolorbox}[enhanced, width=6in, center, size=fbox, fontupper=\large, drop shadow southwest]
    1. Montrer que $\forall(a,b)\in(\mathbb{R}_+)^2$ $\sqrt{a+b}\leq\sqrt{a}+\sqrt{b}$.\\
    Soit $(a,b)\in(\mathbb{R}_+)^2$.
    \begin{align*}
        &\hspace{0.9cm}\sqrt{a+b}\leq\sqrt{a}+\sqrt{b}\\
        &\iff a+b\leq a + 2\sqrt{ab} + b\\
        &\iff 2\sqrt{ab} \geq 0
    \end{align*}
    Ainsi, $\forall(a,b)\in(\mathbb{R}_+)^2$ $\sqrt{a+b}\leq\sqrt{a}+\sqrt{b}$.\\[0.5cm]
    2. Montrer que $\forall(a,b)\in(\mathbb{R}_+)^2$ $|\sqrt{a}-\sqrt{b}|\leq\sqrt{|a-b|}$.\\
    Soit $(a,b)\in(\mathbb{R}_+)^2$.\\
    Considérons $a\geq b$, alors $|a-b| = a-b$.
    \begin{align*}
        &\hspace{1.2cm}|\sqrt{a}-\sqrt{b}|\leq\sqrt{a-b}\\
        &\iff a - 2\sqrt{ab} + b \leq a-b\\
        &\iff 2b \leq 2\sqrt{ab}\\
        &\iff b^2 \leq ab & \text{Or $a\geq b$ donc $ab \geq b^2$}
    \end{align*}
    Le raisonnement est symétrique lorsque $b\geq a$.\\
    Ainsi, $\forall(a,b)\in(\mathbb{R}_+)^2$ $|\sqrt{a}-\sqrt{b}|\leq\sqrt{|a-b|}$.

\end{tcolorbox}
\addcontentsline{toc}{section}{\protect\numberline{}Exercice 2.2}

\end{document}