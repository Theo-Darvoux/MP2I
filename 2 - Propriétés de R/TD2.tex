\documentclass[10pt]{article}

\usepackage[T1]{fontenc}
\usepackage[left=2cm, right=2cm, top=2cm, bottom=2cm]{geometry}
\usepackage[skins]{tcolorbox}
\usepackage{hyperref, fancyhdr, lastpage, tocloft, ragged2e}
\usepackage{amsmath, amssymb, amsthm}

\def\pagetitle{Propriétés de $\mathbb{R}$}

\title{\bf{\pagetitle}\\\large{Corrigé}}
\date{Septembre 2023}
\author{DARVOUX Théo}

\hypersetup{
    colorlinks=true,
    citecolor=black,
    linktoc=all,
    linkcolor=blue
}

\pagestyle{fancy}
\cfoot{\thepage\ sur \pageref*{LastPage}}


\begin{document}
\renewcommand*\contentsname{Exercices.}
\renewcommand*{\cftsecleader}{\cftdotfill{\cftdotsep}}
\maketitle
\hrule
\tableofcontents
\vspace{0.5cm}
\hrule

\thispagestyle{fancy}
\fancyhead[L]{MP2I Paul Valéry}
\fancyhead[C]{\pagetitle}
\fancyhead[R]{2023-2024}


\addcontentsline{toc}{section}{Inégalités.}

\section*{Exercice 2.1 [$\blacklozenge\lozenge\lozenge$]}
\begin{tcolorbox}[enhanced, width=6in, center, size=fbox, fontupper=\large, drop shadow southwest]
    Soient $a$ et $b$ deux nombres réels strictement positifs. Démontrer l'inégalité
    \begin{equation*}
        \frac{a^2}{b}+\frac{b^2}{a} \geq a + b
    \end{equation*}
    On a :
    \begin{align*}
        &\hspace{1.2cm}\frac{a^2}{b}+\frac{b^2}{a} \geq a+b\\
        &\iff\frac{a^3-a^2b+b^3-ab^2}{ab}\geq0\\
        &\iff\frac{a^2(a-b)+b^2(b-a)}{ab}\geq0\\
        &\iff\frac{(a-b)(a^2-b^2)}{ab}\geq0\\
        &\iff\frac{(a-b)^2(a+b)}{ab}\geq0
    \end{align*}
    Or $(a-b)^2\geq0$, $(a+b)\geq0$ et $ab\geq0$.\\
    Ainsi, cette inégalité est vraie pour tout $(a,b)\in\mathbb{R}^*_+$.
\end{tcolorbox}
\addcontentsline{toc}{section}{\protect\numberline{}Exercice 2.1}

\section*{Exercice 2.2 [$\blacklozenge\lozenge\lozenge$]}
\begin{tcolorbox}[enhanced, width=6in, center, size=fbox, fontupper=\large, drop shadow southwest]
    1. Montrer que $\forall(a,b)\in(\mathbb{R}_+)^2$ $\sqrt{a+b}\leq\sqrt{a}+\sqrt{b}$.\\
    Soit $(a,b)\in(\mathbb{R}_+)^2$.
    \begin{align*}
        &\sqrt{a+b}\leq\sqrt{a}+\sqrt{b}\\
        \iff&a+b\leq a + 2\sqrt{ab} + b\\
        \iff&2\sqrt{ab} \geq 0\\
        \iff&\sqrt{ab} \geq 0\\
        \iff&ab \geq 0
    \end{align*}
    Ainsi, $\forall(a,b)\in(\mathbb{R}_+)^2$ $\sqrt{a+b}\leq\sqrt{a}+\sqrt{b}$.\\[0.5cm]
    2. Montrer que $\forall(a,b)\in(\mathbb{R}_+)^2$ $|\sqrt{a}-\sqrt{b}|\leq\sqrt{|a-b|}$.\\
    Soit $(a,b)\in(\mathbb{R}_+)^2$.\\
    Considérons $a\geq b$, alors $|a-b| = a-b$.
    \begin{align*}
        &|\sqrt{a}-\sqrt{b}|\leq\sqrt{a-b}\\
        \iff& a - 2\sqrt{ab} + b \leq a-b\\
        \iff& 2b \leq 2\sqrt{ab}\\
        \iff& b^2 \leq ab\\
        \iff&b \leq a
    \end{align*}
    Le raisonnement est symétrique lorsque $b\geq a$.\\
    Ainsi, $\forall(a,b)\in(\mathbb{R}_+)^2$ $|\sqrt{a}-\sqrt{b}|\leq\sqrt{|a-b|}$.
\end{tcolorbox}
\addcontentsline{toc}{section}{\protect\numberline{}Exercice 2.2}

\section*{Exercice 2.3 [$\blacklozenge\lozenge\lozenge$] \emph{Manipuler la notion de distance}}
\begin{tcolorbox}[enhanced, width=6in, center, size=fbox, fontupper=\large, drop shadow southwest]
    En utilisant la notion de distance sur $\mathbb{R}$, écrire comme réunion d'intervalles l'ensemble
    \begin{equation*}
        E=\{x\in\mathbb{R} \hspace{0.25cm} | \hspace{0.25cm} |x+3| \leq 6 \text{ et } |x^2-1| > 3\}
    \end{equation*}
    On a :
    \begin{equation*}
        x\in[-9,3] \text{ et } x\in]-\infty, -2[\cup]2,+\infty[
    \end{equation*}
    Donc :
    \begin{equation*}
        x\in[-9,-2]\cup[2,3]
    \end{equation*}
\end{tcolorbox}
\addcontentsline{toc}{section}{\protect\numberline{}Exercice 2.3}

\section*{Exercice 2.4 [$\blacklozenge\blacklozenge\lozenge$] \emph{Plusieurs façons de définir une moyenne}}
\begin{tcolorbox}[enhanced, width=6in, center, size=fbox, fontupper=\large, drop shadow southwest]
    Soient $a$ et $b$ deux réels tels que $0 < a \leq b$. On définit les nombres $m,g,h$ par
    \begin{equation*}
        m=\frac{a+b}{2}, \hspace{1.5cm} g=\sqrt{ab}, \hspace{1.5cm} \frac{1}{h}=\frac{1}{2}\left(\frac{1}{a}+\frac{1}{b}\right).
    \end{equation*}
    Et on les appelle respectivement moyenne arithmétique, géométrique et harmonique de $a$ et $b$.\\
    Démontrer l'encadrement
    \begin{equation*}
        a \leq h \leq g \leq m \leq b
    \end{equation*}
    Montrons les inégalités une par une :
    \begin{itemize}
        \item $m \leq b \iff \frac{a+b}{2}-b\leq 0 \iff \frac{a-b}{2} \leq 0 \iff a - b \leq 0 \iff a \leq b$.
        \item $g \leq m \iff \sqrt{ab} \leq \frac{a+b}{2} \iff \frac{a - 2\sqrt{ab} + b}{2} \geq 0 \iff \frac{(\sqrt{a}-\sqrt{b})^2}{2} \geq 0$.
        \item $h \leq g \iff \frac{1}{h} \geq \frac{1}{g} \iff \frac{1}{2a}+\frac{1}{2b}-\frac{1}{\sqrt{ab}} \geq 0 \iff \frac{a-2\sqrt{ab}+b}{2ab} \geq 0 \iff \frac{(\sqrt{a}-\sqrt{b})^2}{2ab}\geq0$.
        \item $a \leq h \iff \frac{1}{a} \geq \frac{1}{h} \iff \frac{1}{a}-\frac{1}{2a}-\frac{1}{2b}\geq 0 \iff \frac{b-a}{2ab}\geq0 \iff b-a\geq 0 \iff a \leq b$
    \end{itemize}
    Ainsi, toutes les inégalités sont vraies et $a\leq h \leq g \leq m \leq b$.
\end{tcolorbox}
\addcontentsline{toc}{section}{\protect\numberline{}Exercice 2.4}

\addcontentsline{toc}{section}{Valeurs absolues.}

\section*{Exercice 2.5 [$\blacklozenge\lozenge\lozenge$]}
\begin{tcolorbox}[enhanced, width=6in, center, size=fbox, fontupper=\large, drop shadow southwest]
    Résoudre l'équation
    \begin{equation*}
        \ln|x|+\ln|x+1|=0
    \end{equation*}
    Soit $x\in\mathbb{R}^*_+$.
    \begin{align*}
        &\ln|x|+\ln|x+1|=0\\
        \iff&\ln\left(|x(x+1|\right)=0\\
        \iff&|x(x+1)|=1\\
        \iff&x(x+1)=1\\
        \iff&x^2+x-1=0\\
        \iff&x=\frac{1\pm\sqrt{5}}{2}
    \end{align*}
    L'ensemble des solutions de l'équation est : $\{\frac{1-\sqrt{5}}{2},\frac{1+\sqrt{5}}{2}\}$
\end{tcolorbox}
\addcontentsline{toc}{section}{\protect\numberline{}Exercice 2.5}

\section*{Exercice 2.6 [$\blacklozenge\lozenge\lozenge$]}
\begin{tcolorbox}[enhanced, width=6in, center, size=fbox, fontupper=\large, drop shadow southwest]
    Résoudre l'équation
    \begin{equation*}
        |x-2|=6-2x
    \end{equation*}
    Soit $x\in\mathbb{R}$.\\
    Considérons $x\geq2$
    \begin{align*}
        &|x-2|=6-2x\\
        \iff&x-2=6-2x\\
        \iff&x=\frac{8}{3}
    \end{align*}
    Considérons $x\leq2$
    \begin{align*}
        &|x-2|=6-2x\\
        \iff&2-x=6-2x\\
        \iff&x=4
    \end{align*}
    Seul la solution $x=\frac{8}{3}$ convient.
    Ainsi, l'unique solution à l'équation est $\frac{8}{3}$.
\end{tcolorbox}
\addcontentsline{toc}{section}{\protect\numberline{}Exercice 2.6}

\addcontentsline{toc}{section}{Entiers, rationnels.}

\section*{Exercice 2.7 [$\blacklozenge\blacklozenge\blacklozenge$]}
\begin{tcolorbox}[enhanced, width=6in, center, size=fbox, fontupper=\large, drop shadow southwest]
    Démontrer l'égalité $\lfloor\frac{\lfloor{nx}\rfloor}{n}\rfloor=\lfloor{x}\rfloor$ pour tout entier $n\in\mathbb{N}^*$ et tout réel $x$.\\
    Soient $(x,n)\in\mathbb{R}\times\mathbb{N}^*$.\\
    Notons $r$ la partie fractionnaire de $x$, ainsi $x=\lfloor{x}\rfloor+r$.\\
    On a alors $nx=n\lfloor{x}\rfloor+nr$ et $\lfloor{nx}\rfloor=\lfloor{n\lfloor{x}\rfloor+nr}\rfloor=n\lfloor{x}\rfloor+\lfloor{nr}\rfloor$.\\
    Conséquemment, $\frac{\lfloor{nx}\rfloor}{n}=\lfloor{x}\rfloor+\frac{\lfloor{nr}\rfloor}{n}$.\\
    Or, $0\leq\frac{\lfloor{nr}\rfloor}{n}<1$ car $0\leq r<1$, donc $\lfloor{x}\rfloor\leq\lfloor{x}\rfloor+\frac{\lfloor{nr}\rfloor}{n}<\lfloor{x}\rfloor+1$.\\
    Ainsi, $\lfloor{x}\rfloor\leq\lfloor\frac{\lfloor{nx}\rfloor}{n}\rfloor<\lfloor{x}+1\rfloor$.\\
    Par conséquent, $\lfloor\frac{\lfloor{nx}\rfloor}{n}\rfloor = \lfloor{x}\rfloor$.
\end{tcolorbox}
\addcontentsline{toc}{section}{\protect\numberline{}Exercice 2.7}

\end{document}