\documentclass[10pt]{article}

\usepackage[T1]{fontenc}
\usepackage[left=2cm, right=2cm, top=2cm, bottom=2cm]{geometry}
\usepackage[skins]{tcolorbox}
\usepackage{hyperref, fancyhdr, lastpage, tocloft, ragged2e}
\usepackage{amsmath, amssymb, amsthm}

\def\pagetitle{Propriétés de $\mathbb{R}$}

\title{\bf{\pagetitle}\\\large{Corrigé}}
\date{Septembre 2023}
\author{DARVOUX Théo}

\hypersetup{
    colorlinks=true,
    citecolor=black,
    linktoc=all,
    linkcolor=blue
}

\pagestyle{fancy}
\cfoot{\thepage\ sur \pageref*{LastPage}}


\begin{document}
\renewcommand*\contentsname{Exercices.}
\renewcommand*{\cftsecleader}{\cftdotfill{\cftdotsep}}
\maketitle
\hrule
\tableofcontents
\vspace{0.5cm}
\hrule

\thispagestyle{fancy}
\fancyhead[L]{MP2I Paul Valéry}
\fancyhead[C]{\pagetitle}
\fancyhead[R]{2023-2024}


\section*{Exercice 2.1 [$\blacklozenge\lozenge\lozenge$]}
\begin{tcolorbox}[enhanced, width=7in, center, size=fbox, fontupper=\large, drop shadow southwest]
    Soient $a$ et $b$ deux nombres réels strictement positifs. Démontrer l'inégalité
    \begin{equation*}
        \frac{a^2}{b}+\frac{b^2}{a} \geq a + b
    \end{equation*}
    On a :
    \begin{align*}
        &\hspace{1.2cm}\frac{a^2}{b}+\frac{b^2}{a} \geq a+b\\
        &\iff\frac{a^3-a^2b+b^3-ab^2}{ab}\geq0\\
        &\iff\frac{a^2(a-b)+b^2(b-a)}{ab}\geq0\\
        &\iff\frac{(a-b)(a^2-b^2)}{ab}\geq0\\
        &\iff\frac{(a-b)^2(a+b)}{ab}\geq0
    \end{align*}
    Or $(a-b)^2\geq0$, $(a+b)\geq0$ et $ab\geq0$.\\
    Ainsi, cette inégalité est vraie pour tout $(a,b)\in\mathbb{R}^*_+$.\\
    \qed
\end{tcolorbox}

\addcontentsline{toc}{section}{Inégalités.}
\addcontentsline{toc}{section}{\protect\numberline{}Exercice 2.1}

\section*{Exercice 2.2 [$\blacklozenge\lozenge\lozenge$]}
\begin{tcolorbox}[enhanced, width=7in, center, size=fbox, fontupper=\large, drop shadow southwest]
    1. Montrer que $\forall(a,b)\in(\mathbb{R}_+)^2$ $\sqrt{a+b}\leq\sqrt{a}+\sqrt{b}$.\\
    Soit $(a,b)\in(\mathbb{R}_+)^2$.
    \begin{align*}
        &\sqrt{a+b}\leq\sqrt{a}+\sqrt{b}\\
        \iff&a+b\leq a + 2\sqrt{ab} + b\\
        \iff&2\sqrt{ab} \geq 0\\
        \iff&\sqrt{ab} \geq 0\\
        \iff&ab \geq 0
    \end{align*}
    Ainsi, $\forall(a,b)\in(\mathbb{R}_+)^2$ $\sqrt{a+b}\leq\sqrt{a}+\sqrt{b}$.\\
    \qed\\
    2. Montrer que $\forall(a,b)\in(\mathbb{R}_+)^2$ $|\sqrt{a}-\sqrt{b}|\leq\sqrt{|a-b|}$.\\
    Soit $(a,b)\in(\mathbb{R}_+)^2$.\\
    Considérons $a\geq b$, alors $|a-b| = a-b$.
    \begin{align*}
        &|\sqrt{a}-\sqrt{b}|\leq\sqrt{a-b}\\
        \iff& a - 2\sqrt{ab} + b \leq a-b\\
        \iff& 2b \leq 2\sqrt{ab}\\
        \iff& b^2 \leq ab\\
        \iff&b \leq a
    \end{align*}
    Le raisonnement est symétrique lorsque $b\geq a$.\\
    Ainsi, $\forall(a,b)\in(\mathbb{R}_+)^2$ $|\sqrt{a}-\sqrt{b}|\leq\sqrt{|a-b|}$.\\
    \qed
\end{tcolorbox}
\addcontentsline{toc}{section}{\protect\numberline{}Exercice 2.2}

\section*{Exercice 2.3 [$\blacklozenge\lozenge\lozenge$] \emph{Manipuler la notion de distance}}
\begin{tcolorbox}[enhanced, width=7in, center, size=fbox, fontupper=\large, drop shadow southwest]
    En utilisant la notion de distance sur $\mathbb{R}$, écrire comme réunion d'intervalles l'ensemble
    \begin{equation*}
        E=\{x\in\mathbb{R} \hspace{0.25cm} | \hspace{0.25cm} |x+3| \leq 6 \text{ et } |x^2-1| > 3\}
    \end{equation*}
    On a :
    \begin{equation*}
        x\in[-9,3] \text{ et } x\in]-\infty, -2[\cup]2,+\infty[
    \end{equation*}
    Donc :
    \begin{equation*}
        x\in[-9,-2]\cup[2,3]
    \end{equation*}
\end{tcolorbox}
\addcontentsline{toc}{section}{\protect\numberline{}Exercice 2.3}

\section*{Exercice 2.4 [$\blacklozenge\blacklozenge\lozenge$] \emph{Plusieurs façons de définir une moyenne}}
\begin{tcolorbox}[enhanced, width=7in, center, size=fbox, fontupper=\large, drop shadow southwest]
    Soient $a$ et $b$ deux réels tels que $0 < a \leq b$. On définit les nombres $m,g,h$ par
    \begin{equation*}
        m=\frac{a+b}{2}, \hspace{1.5cm} g=\sqrt{ab}, \hspace{1.5cm} \frac{1}{h}=\frac{1}{2}\left(\frac{1}{a}+\frac{1}{b}\right).
    \end{equation*}
    Et on les appelle respectivement moyenne arithmétique, géométrique et harmonique de $a$ et $b$.\\
    Démontrer l'encadrement
    \begin{equation*}
        a \leq h \leq g \leq m \leq b
    \end{equation*}
    Montrons les inégalités une par une :
    \begin{itemize}
        \item $m \leq b \iff \frac{a+b}{2}-b\leq 0 \iff \frac{a-b}{2} \leq 0 \iff a - b \leq 0 \iff a \leq b$.
        \item $g \leq m \iff \sqrt{ab} \leq \frac{a+b}{2} \iff \frac{a - 2\sqrt{ab} + b}{2} \geq 0 \iff \frac{(\sqrt{a}-\sqrt{b})^2}{2} \geq 0$.
        \item $h \leq g \iff \frac{1}{h} \geq \frac{1}{g} \iff \frac{1}{2a}+\frac{1}{2b}-\frac{1}{\sqrt{ab}} \geq 0 \iff \frac{a-2\sqrt{ab}+b}{2ab} \geq 0 \iff \frac{(\sqrt{a}-\sqrt{b})^2}{2ab}\geq0$.
        \item $a \leq h \iff \frac{1}{a} \geq \frac{1}{h} \iff \frac{1}{a}-\frac{1}{2a}-\frac{1}{2b}\geq 0 \iff \frac{b-a}{2ab}\geq0 \iff b-a\geq 0 \iff a \leq b$
    \end{itemize}
    Ainsi, $a\leq h \leq g \leq m \leq b$.\\
    \qed
\end{tcolorbox}
\addcontentsline{toc}{section}{\protect\numberline{}Exercice 2.4}

\section*{Exercice 2.5 [$\blacklozenge\lozenge\lozenge$]}
\begin{tcolorbox}[enhanced, width=7in, center, size=fbox, fontupper=\large, drop shadow southwest]
    Résoudre l'équation
    \begin{equation*}
        \ln|x|+\ln|x+1|=0
    \end{equation*}
    Soit $x\in\mathbb{R}\setminus\{-1,0\}$.
    \begin{align*}
        &\ln|x|+\ln|x+1|=0\\
        \iff&\ln\left(|x(x+1|\right)=0\\
        \iff&|x(x+1)|=1\\
    \end{align*}
    Supposons $x\in]-\infty,-1[\cup]0,+\infty[$.\\
    On a :
    \begin{align*}
        &|x(x+1)|=1\\
        \iff&x(x+1)=1\\
        \iff&x^2+x-1=0\\
        \iff&x=\frac{1\pm\sqrt{5}}{2}
    \end{align*}
    Supposons $x\in]-1,0[$.
    \begin{align*}
        &|x(x+1)|=1\\
        \iff&-x^2-x-1=0
    \end{align*}
    Il n'y a donc pas de solutions dans $]-1,0[$.\\
    L'ensemble des solutions de l'équation est : $\{\frac{1-\sqrt{5}}{2},\frac{1+\sqrt{5}}{2}\}$
\end{tcolorbox}
\addcontentsline{toc}{section}{Valeurs absolues.}
\addcontentsline{toc}{section}{\protect\numberline{}Exercice 2.5}

\section*{Exercice 2.6 [$\blacklozenge\lozenge\lozenge$]}
\begin{tcolorbox}[enhanced, width=7in, center, size=fbox, fontupper=\large, drop shadow southwest]
    Résoudre l'équation
    \begin{equation*}
        |x-2|=6-2x
    \end{equation*}
    Soit $x\in\mathbb{R}$.\\
    Considérons $x\geq2$
    \begin{align*}
        &|x-2|=6-2x\\
        \iff&x-2=6-2x\\
        \iff&x=\frac{8}{3}
    \end{align*}
    Considérons $x\leq2$
    \begin{align*}
        &|x-2|=6-2x\\
        \iff&2-x=6-2x\\
        \iff&x=4
    \end{align*}
    Seul la solution $x=\frac{8}{3}$ convient.
    Ainsi, l'unique solution à l'équation est $\frac{8}{3}$.
\end{tcolorbox}
\addcontentsline{toc}{section}{\protect\numberline{}Exercice 2.6}


\section*{Exercice 2.7 [$\blacklozenge\blacklozenge\blacklozenge$]}
\begin{tcolorbox}[enhanced, width=7in, center, size=fbox, fontupper=\large, drop shadow southwest]
    Démontrer l'égalité $\lfloor\frac{\lfloor{nx}\rfloor}{n}\rfloor=\lfloor{x}\rfloor$ pour tout entier $n\in\mathbb{N}^*$ et tout réel $x$.\\[0.2cm]
    Soient $(x,n)\in\mathbb{R}\times\mathbb{N}^*$.\\[0.2cm]
    Notons $r$ la partie fractionnaire de $x$, ainsi $x=\lfloor{x}\rfloor+r$.\\[0.2cm]
    On a alors $nx=n\lfloor{x}\rfloor+nr$ et $\lfloor{nx}\rfloor=\lfloor{n\lfloor{x}\rfloor+nr}\rfloor=n\lfloor{x}\rfloor+\lfloor{nr}\rfloor$.\\[0.2cm]
    Conséquemment, $\frac{\lfloor{nx}\rfloor}{n}=\lfloor{x}\rfloor+\frac{\lfloor{nr}\rfloor}{n}$.\\[0.2cm]
    Or, $0\leq\frac{\lfloor{nr}\rfloor}{n}<1$ car $0\leq r<1$, donc $\lfloor{x}\rfloor\leq\lfloor{x}\rfloor+\frac{\lfloor{nr}\rfloor}{n}<\lfloor{x}\rfloor+1$.\\[0.2cm]
    Ainsi, $\lfloor{x}\rfloor\leq\lfloor\frac{\lfloor{nx}\rfloor}{n}\rfloor<\lfloor{x}+1\rfloor$.\\[0.2cm]
    Par conséquent, $\lfloor\frac{\lfloor{nx}\rfloor}{n}\rfloor = \lfloor{x}\rfloor$.\\
    \qed
\end{tcolorbox}

\addcontentsline{toc}{section}{Entiers, rationnels.}
\addcontentsline{toc}{section}{\protect\numberline{}Exercice 2.7}

\section*{Exercice 2.8 [$\blacklozenge\blacklozenge\lozenge$] mal de crane tier}
\begin{tcolorbox}[enhanced, width=7in, center, size=fbox, fontupper=\large, drop shadow southwest]
    1. Démontrer : 
    \begin{equation*}
        \forall{x\in\mathbb{R}^*_+}\hspace{0.5cm}\frac{1}{2\sqrt{x+1}}<\sqrt{x+1}-\sqrt{x}<\frac{1}{2\sqrt{x}}.
    \end{equation*}
    Soit $x\in\mathbb{R}^*_+$.\\
    On a :
    \begin{align*}
        &\sqrt{x+1}-\sqrt{x}<\frac{1}{2\sqrt{x}}\\
        \iff&2\sqrt{x(x+1)}-2x<1\\
        \iff&(2\sqrt{x(x+1)})^2<(1+2x)^2\\
        \iff&4x(x+1)<4x^2+4x+1\\
        \iff&4x^2+4x-4x^2-4x<1\\
        \iff&0<1
    \end{align*}
    Et :
    \begin{align*}
        &\frac{1}{2\sqrt{x+1}}<\sqrt{x+1}-\sqrt{x}\\
        \iff&1<2\sqrt{(x+1)^2}-2\sqrt{x(x+1)}\\
        \iff&1<2|x+1|-2\sqrt{x(x+1)}\\
        \iff&(2x+1)^2>(2\sqrt{x(x+1)})^2\\
        \iff&4x^2+4x+1>4x^2+4x\\
        \iff&1>0
    \end{align*}
    \qed
\end{tcolorbox}
\begin{tcolorbox}[enhanced, width=7in, center, size=fbox, fontupper=\large, drop shadow southwest]
    2. Soit $p$ un entier supérieur à $2$. Que vaut la partie entière de
    \begin{equation*}
        \sum_{k=1}^{p^2-1}{\frac{1}{\sqrt{k}}}
    \end{equation*}
    Soit $x\in\mathbb{R}^*_+$\\
    On a :
    \begin{equation*}
        \frac{1}{2\sqrt{x+1}}<\sqrt{x+1}-\sqrt{x}<\frac{1}{2\sqrt{x}}
    \end{equation*}
    Donc, en remplaçant $x$ par $x-1$ :
    \begin{equation*}
        \frac{1}{2\sqrt{x}}<\sqrt{x}-\sqrt{x-1}<\frac{1}{2\sqrt{x-1}}
    \end{equation*}
    Ainsi,
    \begin{equation*}
        \sqrt{x+1}-\sqrt{x}<\frac{1}{2\sqrt{x}}<\sqrt{x}-\sqrt{x-1}
    \end{equation*}
    $\mathbb{MAIS}$ $\mathbb{ALORS}$ :
    \begin{align*}
        &\sum^{p^2-1}_{k=1}{\left(\sqrt{k+1}-\sqrt{k}\right)}<\sum^{p^2-1}_{k=1}{\frac{1}{2\sqrt{k}}}<\sum^{p^2-1}_{k=1}{\left(\sqrt{k}-\sqrt{k-1}\right)}\\
        \iff&\sqrt{p^2}-\sqrt{1}<\frac{1}{2}\sum^{p^2-1}_{k=1}{\frac{1}{\sqrt{k}}}<\sqrt{p^2-1}-\sqrt{0}\\
        \iff&2p-2<\sum^{p^2-1}_{k=1}{\frac{1}{\sqrt{k}}}<2\sqrt{p^2-1}\\
        \iff&2p-2<\sum^{p^2-1}_{k=1}{\frac{1}{\sqrt{k}}}<\lfloor{2\sqrt{p^2-1}}\rfloor
    \end{align*}
    Or $2p-2<2\sqrt{p^2-1}<2p$ donc $\lfloor{2\sqrt{p^2-2}}\rfloor=2p-2$\\
    On en conclut : 
    \begin{equation*}
        \lfloor{\sum^{p^2-1}_{k=1}{\frac{1}{\sqrt{k}}}}\rfloor = 2p-2
    \end{equation*}
\end{tcolorbox}
\addcontentsline{toc}{section}{\protect\numberline{}Exercice 2.8}

\section*{Exercice 2.9 [$\blacklozenge\blacklozenge\blacklozenge$]}
\begin{tcolorbox}[enhanced, width=7in, center, size=fbox, fontupper=\large, drop shadow southwest]
    Prouver que $\frac{\ln(2)}{\ln(3)}$ est un nombre irrationnel.\\
    Supposons que $\frac{\ln{2}}{\ln{3}}\in\mathbb{Q}$. Alors il existe $p\in\mathbb{N}$ et $q\in\mathbb{N}^*$ premiers entre eux tels que :
    \begin{equation*}
        \frac{\ln{2}}{\ln{3}}=\frac{p}{q}
    \end{equation*}
    Alors :
    \begin{align*}
        &p\ln{3}=q\ln{2}\\
        \iff&\ln(3^p)=\ln(2^q)\\
        \iff&e^{\ln(3^p)}=e^{\ln{2^q}}\\
        \iff&3^p=2^q
    \end{align*}
    Or $3^p$ est toujours impair et $2^q$ est toujours pair, donc cela est absurde.\\
    Ainsi, $\frac{\ln2}{\ln3}$ est irrationnel.\\
    \qed
\end{tcolorbox}
\addcontentsline{toc}{section}{\protect\numberline{}Exercice 2.9}

\section*{Exercice 2.10 [$\blacklozenge\blacklozenge\blacklozenge$]}
\begin{tcolorbox}[enhanced, width=7in, center, size=fbox, fontupper=\large, drop shadow southwest]
    Soient $x$ et $y$ deux rationnels positifs tels que \\$\sqrt{x}$ et $\sqrt{y}$ soient irrationnels.\\
    Montrer que $\sqrt{x} + \sqrt{y}$ est irrationnel.
    Supposons $\sqrt{x}+\sqrt{y}\in\mathbb{Q}$.\\
    On a :
    \begin{align*}
        &(\sqrt{x}+\sqrt{y})(\sqrt{x}-\sqrt{y})=x-y\\
        \iff&\sqrt{x}-\sqrt{y}=\frac{x-y}{\sqrt{x}+\sqrt{y}}
    \end{align*}
    Or $x-y\in\mathbb{Q}$ et $\sqrt{x}+\sqrt{y}\in\mathbb{Q}$ par hypothèse. Donc $\sqrt{x}-\sqrt{y}\in\mathbb{Q}$.\\
    D'autre part,
    \begin{align*}
        &\sqrt{x}+\sqrt{y}+\sqrt{x}-\sqrt{y}=2\sqrt{x}\\
    \end{align*}
    $\sqrt{x}$ est donc la somme de deux rationnels, et est donc rationnel.\\
    C'est absurde. On en conclut que $\sqrt{x}+\sqrt{y}$ est irrationnel.\\
    \qed
\end{tcolorbox}
\addcontentsline{toc}{section}{\protect\numberline{}Exercice 2.10}

\section*{Exercice 2.11 [$\blacklozenge\blacklozenge\lozenge$]}
\begin{tcolorbox}[enhanced, width=7in, center, size=fbox, fontupper=\large, drop shadow southwest]
    Soit l'ensemble 
    \begin{equation*}
        A = \left\{\frac{ n-\frac{1}{n} }{ n+\frac{1}{n}}, n\in\mathbb{N}^* \right\}
    \end{equation*}
    Cette partie de $\mathbb{R}$ est-elle bornée ? Possède-t-elle un maximum ? Un minimum ?\\
    Soit $(u_n)$ une suite telle que $\forall{n\in\mathbb{N}^*}, u_n=\frac{n-\frac{1}{n}}{n+\frac{1}{n}}$.\\
    Soit $n\in\mathbb{N}^*$.\\
    On a :
    \begin{align*}
        u_n 
        &= \frac{n-\frac{1}{n}}{n+\frac{1}{n}} = \frac{n^2-1}{n} \cdot \frac{n}{n^2+1}\\
        &= \frac{n^3-n}{n^3+n} = \frac{n^3+n}{n^3+n}-\frac{2n}{n^3+n}\\
        &= 1 - \frac{2}{n^2+1}
    \end{align*}
    Étudions le signe de $u_{n+1}-u_n$.
    \begin{align*}
        u_{n+1}-u_n
        &= 1 - \frac{2}{(n+1)^2+1}-1+\frac{2}{n^2+1}\\
        &= \frac{2}{n^2+1} - \frac{2}{n^2+2n+2}\\
        &= \frac{4n+2}{(n^2)(n^2+2n+2)}
    \end{align*}
    C'est toujours positif : on en déduit que, $(u_n)$ est croissante sur $\mathbb{N}^*$.\\
    Elle admet donc un minimum en $1$, qui est $0$.\\
    Elle admet aussi un majorant lorsque $n$ tend vers l'infini :
    \begin{align*}
        \lim_{n\rightarrow+\infty}u_n=1
    \end{align*}
    Ainsi, $A$ admet $0$ comme minimum, n'a pas de maximum et est majorée par $1$.
\end{tcolorbox}

\addcontentsline{toc}{section}{Parties bornées}
\addcontentsline{toc}{section}{\protect\numberline{}Exercice 2.11}

\section*{Exercice 2.12 [$\blacklozenge\blacklozenge\lozenge$]}
\begin{tcolorbox}[enhanced, width=7in, center, size=fbox, fontupper=\large, drop shadow southwest]
    1. Montrer que
    \begin{equation*}
        \forall(a,b)\in(\mathbb{R}^*_+)^2\hspace{0.5cm}:\hspace{0.5cm}\frac{a^2}{a+b}\geq\frac{3a-b}{4}.
    \end{equation*}
    Étudier le cas d'égalité.\\
    2. En déduire que l'ensemble
    \begin{equation*}
        E=\left\{\frac{a^2}{a+b}+\frac{b^2}{b+c}+\frac{c^2}{c+a} \text{ | } (a,b,c)\in(\mathbb{R^*_+})^3 \text{ et } a+b+c\geq2 \right\}
    \end{equation*}
    admet un minimum et le calculer.\\[0.5cm]
    1. Soit $(a,b)\in(\mathbb{R}^*_+)^2$\\
    On a :
    \begin{align*}
        &\frac{a^2}{a+b}-\frac{3a-b}{4}\geq0\\
        \iff&\frac{a^2-2ab+b^2}{4(a+b)}\geq0\\
        \iff&(a-b)^2\geq0\\
    \end{align*}
    D'autre part,
    \begin{align*}
        &\frac{a^2}{a+b}=\frac{3a-b}{4}\\
        \iff&(a-b)^2=0\\
        \iff&a=b
    \end{align*}
    2. Soient $(a,b,c)\in\mathbb{R^*_+}^3$ tels que $a+b+c\geq2$.\\
    On a :
    \begin{align*}
        \frac{a^2}{a+b}+\frac{b^2}{b+c}+\frac{c^2}{c+a}
        &\geq\frac{3a-b}{4}+\frac{3b-c}{4}+\frac{3c-a}{4}\\
        &\geq\frac{2a+2b+2c}{4}\\
        &\geq\frac{a+b+c}{2}\\
        &\geq1
    \end{align*}
    Or, lorsque $a=b=c=\frac{2}{3}$, on a $a+b+c\geq2$ et:
    \begin{align*}
        \frac{a^2}{a+b}+\frac{b^2}{b+c}+\frac{c^2}{c+a}
        &=3\frac{a}{2}=3\cdot\frac{2}{3}\cdot\frac{1}{2}=1
    \end{align*}
    Ainsi, $1\in E$ et $\forall{x\in E}$, $x\geq1$ donc $1$ est minimum de $E$.
\end{tcolorbox}
\addcontentsline{toc}{section}{\protect\numberline{}Exercice 2.12}

\begin{center}\LARGE{FIN DU TD 2 VU QU'ON A PAS FAIT LES BORNES INF ET SUP}\end{center}
\end{document}