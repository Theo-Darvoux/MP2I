\documentclass[10pt]{article}

\usepackage[T1]{fontenc}
\usepackage[left=2cm, right=2cm, top=2cm, bottom=2cm]{geometry}
\usepackage[skins]{tcolorbox}
\usepackage{hyperref, fancyhdr, lastpage, tocloft, ragged2e, multicol}
\usepackage{amsmath, amssymb, amsthm, stmaryrd}
\usepackage{tkz-tab}

\def\pagetitle{Forme Algébrique}

\title{\bf{\pagetitle}\\\large{Corrigé}}
\date{Octobre 2023}
\author{DARVOUX Théo}

\hypersetup{
    colorlinks=true,
    citecolor=black,
    linktoc=all,
    linkcolor=blue
}

\pagestyle{fancy}
\cfoot{\thepage\ sur \pageref*{LastPage}}


\begin{document}
\renewcommand*\contentsname{Exercices.}
\renewcommand*{\cftsecleader}{\cftdotfill{\cftdotsep}}
\maketitle
\hrule
\tableofcontents
\vspace{0.5cm}
\hrule

\thispagestyle{fancy}
\fancyhead[L]{MP2I Paul Valéry}
\fancyhead[C]{\pagetitle}
\fancyhead[R]{2023-2024}
\allowdisplaybreaks

\pagebreak

\section*{Exercice 6.1 [$\blacklozenge\lozenge\lozenge$]}
\begin{tcolorbox}[enhanced, width=7in, center, size=fbox, fontupper=\large, drop shadow southwest]
    Résoudre $4z^2 + 8|z|^2 - 3 = 0$.\\
    Soit $z \in \mathbb{C}$ et $(a,b)\in\mathbb{R}^2$ tels que $z=a+ib$. On a :
    \begin{align*}
        &4z^2+8|z|^2-3=0\\
        \iff&4(a+ib)^2+8(a^2+b^2)-3=0\\
        \iff&4a^2+8aib-4b^2+8a^2+8b^2-3=0\\
        \iff&(12a^2+4b^2-3)+i(8ab) = 0\\
        \iff&\begin{cases}12a^2+4b^2-3=0\\8ab=0\end{cases}\\
        \iff&\begin{cases}12a^2+4b^2-3=0\\a=0\end{cases} \text{ ou } \begin{cases}12a^2+4b^2-3=0\\b=0\end{cases}\\
        \iff&4b^2-3=0 \text{ ou } 12a^2-3=0\\
        \iff&b^2=\frac{3}{4} \text{ ou } a^2 = \frac{1}{4}\\
        \iff&b=\pm\frac{\sqrt{3}}{2} \text{ ou } a=\pm\frac{1}{2}
    \end{align*}
    Les solutions sont donc :
    \begin{equation*}
        \left\{-\frac{1}{2}, \frac{1}{2}, -i\frac{\sqrt{3}}{2}, i\frac{\sqrt{3}}{2}\right\}
    \end{equation*}
    \qed
\end{tcolorbox}

\addcontentsline{toc}{section}{\protect\numberline{}Exercice 6.1}

\section*{Exercice 6.2 [$\blacklozenge\lozenge\lozenge$]}
\begin{tcolorbox}[enhanced, width=7in, center, size=fbox, fontupper=\large, drop shadow southwest]
    Soient $a$ et $b$ deux nombres complexes non nuls. Montrer que :
    \begin{equation*}
        \left|\frac{a}{|a|^2}-\frac{b}{|b|^2}\right|=\frac{|a-b|}{|a||b|}.
    \end{equation*}
    On a : 
    \begin{align*}
        \left|\frac{a}{|a|^2}-\frac{b}{|b|^2}\right| &= \left|\frac{a|b|^2-b|a|^2}{|a|^2|b|^2}\right| = \frac{|ab\overline{b}-ba\overline{a}|}{||ab|^2|}\\
        &=\frac{\left|ab(\overline{b}-\overline{a})\right|}{||ab|^2|} =\frac{|ab||\overline{a}-\overline{b}|}{|ab|^2}\\
        &=\frac{|a-b|}{|ab|} = \frac{|a-b|}{|a||b|}
    \end{align*}
    \qed
\end{tcolorbox}

\addcontentsline{toc}{section}{\protect\numberline{}Exercice 6.2}

\section*{Exercice 6.3 [$\blacklozenge\blacklozenge\lozenge$]}
\begin{tcolorbox}[enhanced, width=7in, center, size=fbox, fontupper=\large, drop shadow southwest]
    Soit $z \in \mathbb{C} \setminus \{1\}$, montrer que :
    \begin{equation*}
        \frac{1+z}{1-z} \in i\mathbb{R} \iff |z| = 1.
    \end{equation*}
    Supposons $\frac{1+z}{1-z} \in i\mathbb{R}$. Montrons $|z|=1$.\\
    Soit $b\in\mathbb{R}$, on a :
    \begin{align*}
        &\frac{1+z}{1-z}=ib\iff1+z=ib-zib\iff z(1+ib)=ib-1\iff z=\frac{ib-1}{1+ib}
    \end{align*}
    Ainsi, $|z|=|\frac{ib-1}{1+ib}|=\frac{\sqrt{1+b^2}}{\sqrt{1+b^2}}=1$.\\
    Supposons $|z|=1$, montrons $\frac{1+z}{1-z} \in i\mathbb{R}$.\\
    Soient $(a,b)\in\mathbb{R}$ tels que $z=a+ib$. Par supposition, $a^2+b^2=1$. On a :
    \begin{align*}
        \frac{1+z}{1-z}&=\frac{1+a+ib}{1-a-ib}=\frac{(1+a+ib)(1-a+ib)}{(1-a-ib)(1-a+ib)}=\frac{1+2ib-a^2-b^2}{1-2a+a^2+b^2}\\
        &= \frac{2ib}{2-2a} = \frac{ib}{1-a}=i\frac{b}{1-a}
    \end{align*}
    \qed
\end{tcolorbox}

\addcontentsline{toc}{section}{\protect\numberline{}Exercice 6.3}

\section*{Exercice 6.4 [$\blacklozenge\lozenge\lozenge$]}
\begin{tcolorbox}[enhanced, width=7in, center, size=fbox, fontupper=\large, drop shadow southwest]
    Soient $z_1,z_2,\dots,z_n$ des nombres complexes non nuls de mêmes module. Démontrer que
    \begin{equation}
        \frac{(z_1 + z_2)(z_2 + z_3)\dots(z_{n-1}+z_n)(z_n + z_1)}{z_1z_2\dots z_n} \in \mathbb{R}.
    \end{equation}
    Commençons par énoncer que :
    \begin{equation*}
        \forall{(i,j)\in\llbracket1,n\rrbracket^2}, \hspace{1cm} \frac{\overline{z_i}}{\overline{z_j}}=\frac{z_j}{z_i}.
    \end{equation*}
    En effet,
    \begin{equation*}
        \frac{z_i}{z_j}\cdot\frac{\overline{z_i}}{\overline{z_j}}=\left|\frac{z_i}{z_j}\right|^2=1 \iff \frac{\overline{z_i}}{\overline{z_j}}=\frac{z_j}{z_i}.
    \end{equation*}
    Le conjugué de (1) est :
    \begin{align*}
        &\frac{(\overline{z_1} + \overline{z_2})(\overline{z_2} + \overline{z_3})\dots(\overline{z_{n-1}}+\overline{z_n})(\overline{z_n} + \overline{z_1})}{\overline{z_1}\overline{z_2}\dots\overline{z_n}}=(1+\frac{\overline{z_2}}{\overline{z_1}})(1+\frac{\overline{z_3}}{\overline{z_2}})\dots(1+\frac{\overline{z_n}}{\overline{z_{n-1}}})(1+\frac{\overline{z_1}}{\overline{z_n}})
    \end{align*}
    Ainsi :
    \begin{align*}
        &\frac{(\overline{z_1} + \overline{z_2})(\overline{z_2} + \overline{z_3})\dots(\overline{z_{n-1}}+\overline{z_n})(\overline{z_n} + \overline{z_1})}{\overline{z_1}\overline{z_2}\dots\overline{z_n}}=(1+\frac{z_1}{z_2})\dots(1+\frac{z_n}{z_1})\\
        =&\frac{z_1+z_2}{z_2}\dots\frac{z_n+z_1}{z_1}=\frac{(z_1+z_2)(z_2+z_3)\dots(z_{n-1}+z_n)(z_n+z_1)}{z_1z_2\dots z_n}
    \end{align*}
    Puisque (1) est égal à son conjugué, $(1) \in \mathbb{R}$.
    \qed
\end{tcolorbox}

\addcontentsline{toc}{section}{\protect\numberline{}Exercice 6.4}


\section*{Exercice 6.5 [$\blacklozenge\blacklozenge\lozenge$]}
\begin{tcolorbox}[enhanced, width=7in, center, size=fbox, fontupper=\large, drop shadow southwest]
    Soient $a,b$ deux nombres complexes tels que $\overline{a}b\neq1$ et $c=\frac{a-b}{1-\overline{a}b}$. Montrer que
    \begin{equation*}
        (|c|=1) \iff (|a| = 1 \text{ ou } |b| = 1).
    \end{equation*}
    Supposons $|c|=1$. Montrons que $|a|=1$ ou $|b|=1$.\\
    On a :
    \begin{align*}
        &|c|=1\\
        \iff&|c|^2=\frac{(a-b)(\overline{a}-\overline{b})}{(1-\overline{a}b)(1-a\overline{b})}=\frac{|a|^2-a\overline{b}-b\overline{a}+|b|^2}{1-a\overline{b}-\overline{a}b+|a|^2|b|^2}=1\\
        \iff&|a|^2-a\overline{b}-\overline{a}b+|b|^2=1-a\overline{b}-\overline{a}b+|a|^2|b|^2\\
        \iff&|a|^2+|b|^2-|a|^2|b|^2=1\\
        \iff&|a|^2(1-|b|^2)=1-|b|^2
    \end{align*}
    Si on suppose $|b|\neq1$, on obtient : $|c|=1\iff|a|^2=\frac{1-|b|^2}{1-|b|^2}=1$ donc $|a|=1$.\\
    Si on suppose $|a|\neq1$, on obtient : $|c|=1\iff|b|^2=\frac{1-|a|^2}{1-|a|^2}=1$ donc $|b|=1$.\\
    Supposons $|a|=1$. 
    On a :
    \begin{align*}
        |c|=\left|\frac{a-b}{1-\overline{a}b}\right|=\left|\frac{a-b}{\overline{a}a-\overline{a}b}\right|=\left|\frac{1}{\overline{a}}\right|\left|\frac{a-b}{a-b}\right|=|a|=1
    \end{align*}
    Supposons $|b|=1$. 
    On a :
    \begin{align*}
        |c|=\left|\frac{a-b}{1-\overline{a}b}\right|=\left|\frac{a-b}{\overline{b}b-\overline{a}b}\right|=\left|\frac{1}{b}\right|\left|\frac{a-b}{\overline{b}-\overline{a}}\right|=|b|\frac{|a-b|}{|a-b|}=|b|=1
    \end{align*}
    \qed
\end{tcolorbox}

\addcontentsline{toc}{section}{\protect\numberline{}Exercice 6.5}

\section*{Exercice 6.6 [$\blacklozenge\blacklozenge\blacklozenge$]}
\begin{tcolorbox}[enhanced, width=7in, center, size=fbox, fontupper=\large, drop shadow southwest]
    Pour $n\in\mathbb{N}^*$, calculer $R^2+S^2$ où
    \begin{equation*}
        R = \sum_{0\leq 2k \leq n}{(-1)^k\binom{n}{2k}} \hspace{1cm} \text{et} \hspace{1cm} S = \sum_{0 \leq 2k+1 \leq n}{(-1)^k\binom{n}{2k+1}}.
    \end{equation*}
    On a :
    \begin{equation*}
        (1+i)^n = \sum_{k=0}^n{\binom{n}{k}i^k}=\sum_{0 \leq 2k \leq n}{\binom{n}{2k}i^{2k}}+\sum_{0\leq 2k+1 \leq n}{\binom{n}{2k+1}i^{2k}\cdot i}=R+iS
    \end{equation*}
    Ainsi :
    \begin{equation*}
        \begin{cases}
            R = \text{Re}\left((1+i)^n\right)=2^\frac{n}{2}\cos(\frac{n\pi}{4})\\
            S = \text{Im}\left((1+i)^n\right)=2^\frac{n}{2}\sin(\frac{n\pi}{4})
        \end{cases}
    \end{equation*}
    Finalement, $R^2 + S^2$ = $2^n(\cos^2(\frac{n\pi}{4})+\sin^2(\frac{n\pi}{4}))=2^n$.\\\qed
\end{tcolorbox}

\addcontentsline{toc}{section}{\protect\numberline{}Exercice 6.6}

\section*{Exercice 6.7 [$\blacklozenge\blacklozenge\blacklozenge$]}
\begin{tcolorbox}[enhanced, width=7in, center, size=fbox, fontupper=\large, drop shadow southwest]
    Soit $ABCD$ un parallélogramme.\\
    Montrer que $AC^2+BD^2=AB^2+BC^2+CD^2+DA^2$\\
    Soient $(z,z')\in\mathbb{R}$. Les points $A,B,C,D$ d'affixes $0,z,z+z',z'$ forment un parallélogramme.\\
    Alors :
    \begin{equation*}
        \begin{cases}
            AC^2 = |z+z'|^2\\
            BD^2 = |z-z'|^2\\
            AB^2 = CD^2 = |z|^2\\
            BC^2 = DA^2 = |z'|^2
        \end{cases}
    \end{equation*}
    On a :
    \begin{align*}
        AC^2 + BD^2 &= |z+z'|^2 + |z-z'|^2 = (z+z')(\overline{z}+\overline{z'})+(z-z')(\overline{z}-\overline{z'})\\
        &=z\overline{z}+z\overline{z'}+z'\overline{z}+z'\overline{z'}+z\overline{z}-z\overline{z'}-z'\overline{z}+z'\overline{z'}\\
        &=|z|^2+|z'|^2+|z|^2+|z'|^2\\
        &= AB^2 + BC^2 + CD^2 + DA^2
    \end{align*}
    \qed
\end{tcolorbox}

\addcontentsline{toc}{section}{\protect\numberline{}Exercice 6.7}


\end{document}
 