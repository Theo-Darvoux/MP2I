\documentclass[10pt]{article}

\usepackage[T1]{fontenc}
\usepackage[left=2cm, right=2cm, top=2cm, bottom=2cm]{geometry}
\usepackage[skins]{tcolorbox}
\usepackage{hyperref, fancyhdr, lastpage, tocloft, ragged2e, multicol}
\usepackage{amsmath, amssymb, amsthm, stmaryrd}
\usepackage{tkz-tab}

\def\pagetitle{Forme Algébrique}

\title{\bf{\pagetitle}\\\large{Corrigé}}
\date{Octobre 2023}
\author{DARVOUX Théo}

\hypersetup{
    colorlinks=true,
    citecolor=black,
    linktoc=all,
    linkcolor=blue
}

\pagestyle{fancy}
\cfoot{\thepage\ sur \pageref*{LastPage}}


\begin{document}
\renewcommand*\contentsname{Exercices.}
\renewcommand*{\cftsecleader}{\cftdotfill{\cftdotsep}}
\maketitle
\hrule
\tableofcontents
\vspace{0.5cm}
\hrule

\thispagestyle{fancy}
\fancyhead[L]{MP2I Paul Valéry}
\fancyhead[C]{\pagetitle}
\fancyhead[R]{2023-2024}
\allowdisplaybreaks

\pagebreak

\section*{Exercice 6.1 [$\blacklozenge\lozenge\lozenge$]}
\begin{tcolorbox}[enhanced, width=7in, center, size=fbox, fontupper=\large, drop shadow southwest]
    Résoudre $4z^2 + 8|z|^2 - 3 = 0$.\\
    Soit $z \in \mathbb{C}$ et $(a,b)\in\mathbb{R}^2$ tels que $z=a+ib$. On a :
    \begin{align*}
        &4z^2+8|z|^2-3=0\\
        \iff&4(a+ib)^2+8(a^2+b^2)-3=0\\
        \iff&4a^2+8aib-4b^2+8a^2+8b^2-3=0\\
        \iff&(12a^2+4b^2-3)+i(8ab) = 0\\
        \iff&\begin{cases}12a^2+4b^2-3=0\\8ab=0\end{cases}\\
        \iff&\begin{cases}12a^2+4b^2-3=0\\a=0\end{cases} \text{ ou } \begin{cases}12a^2+4b^2-3=0\\b=0\end{cases}\\
        \iff&4b^2-3=0 \text{ ou } 12a^2-3=0\\
        \iff&b^2=\frac{3}{4} \text{ ou } a^2 = \frac{1}{4}\\
        \iff&b=\pm\frac{\sqrt{3}}{2} \text{ ou } a=\pm\frac{1}{2}
    \end{align*}
    Les solutions sont donc :
    \begin{equation*}
        \left\{-\frac{1}{2}, \frac{1}{2}, -i\frac{\sqrt{3}}{2}, i\frac{\sqrt{3}}{2}\right\}
    \end{equation*}
    \qed
\end{tcolorbox}

\addcontentsline{toc}{section}{\protect\numberline{}Exercice 6.1}

\section*{Exercice 6.2 [$\blacklozenge\lozenge\lozenge$]}
\begin{tcolorbox}[enhanced, width=7in, center, size=fbox, fontupper=\large, drop shadow southwest]
    Soient $a$ et $b$ deux nombres complexes non nuls. Montrer que :
    \begin{equation*}
        \left|\frac{a}{|a|^2}-\frac{b}{|b|^2}\right|=\frac{|a-b|}{|a||b|}.
    \end{equation*}
    On a : 
    \begin{align*}
        \left|\frac{a}{|a|^2}-\frac{b}{|b|^2}\right| &= \left|\frac{a|b|^2-b|a|^2}{|a|^2|b|^2}\right| = \frac{|ab\overline{b}-ba\overline{a}|}{||ab|^2|}\\
        &=\frac{\left|ab(\overline{b}-\overline{a})\right|}{||ab|^2|} =\frac{|ab||\overline{a}-\overline{b}|}{|ab|^2}\\
        &=\frac{|a-b|}{|ab|} = \frac{|a-b|}{|a||b|}
    \end{align*}
    \qed
\end{tcolorbox}

\addcontentsline{toc}{section}{\protect\numberline{}Exercice 6.2}


\end{document}
