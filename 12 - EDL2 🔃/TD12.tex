\documentclass[10pt]{article}

\usepackage[T1]{fontenc}
\usepackage[left=2cm, right=2cm, top=2cm, bottom=2cm, paperheight=31cm]{geometry}
\usepackage[skins]{tcolorbox}
\usepackage{hyperref, fancyhdr, lastpage, tocloft, ragged2e, multicol}
\usepackage{amsmath, amssymb, amsthm, stmaryrd}
\usepackage{tkz-tab}
\usepackage{systeme}

\def\pagetitle{Équations Différentielles Linéaires d'ordre 2}
\setlength{\headheight}{13pt}

\title{\bf{\pagetitle}\\\large{Corrigé}}
\date{Novembre 2023}
\author{DARVOUX Théo}

\DeclareMathOperator{\ch}{ch}

\hypersetup{
    colorlinks=true,
    citecolor=black,
    linktoc=all,
    linkcolor=blue
}

\pagestyle{fancy}
\cfoot{\thepage\ sur \pageref*{LastPage}}


\begin{document}
\renewcommand*\contentsname{Exercices.}
\renewcommand*{\cftsecleader}{\cftdotfill{\cftdotsep}}
\maketitle
\hrule
\tableofcontents
\vspace{0.5cm}
\hrule


\thispagestyle{fancy}
\fancyhead[L]{MP2I Paul Valéry}
\fancyhead[C]{\pagetitle}
\fancyhead[R]{2023-2024}
\allowdisplaybreaks

\pagebreak

\section*{Exercice 12.1 [$\blacklozenge\lozenge\lozenge$]}
\begin{tcolorbox}[enhanced, width=7in, center, size=fbox, fontupper=\large, drop shadow southwest]
    Résoudre le problème de Cauchy ci-dessous :
    \begin{equation*}
        \begin{cases}
            y'' + 2y' + 10y = 5\\
            y(0) = 1 \quad y'(0) = 0
        \end{cases}
    \end{equation*}
    Polynome caractéristique : $r^2 + 2r + 10$. $\Delta = -36$. $r_{\pm}=-1\pm3i$.\\
    Solutions de l'équation homogène : $S_0 = \{x\mapsto e^{-x}\left(\alpha\cos(3x) + \beta\sin(3x)\right) ~ | ~ (\alpha, \beta)\in\mathbb{R}^2\}$\\
    Solution particulière : $S_p : x\mapsto\frac{1}{2}$.\\
    Solution générale : $S = \{x\mapsto\frac{1}{2} + e^{-x}\left( \alpha\cos(3x) + \beta\sin(3x) \right) ~ | ~ (\alpha, \beta) \in \mathbb{R}^2\}$.\\
    Conditions initiales.\\
    Soit $(\alpha, \beta)\in\mathbb{R}^2 ~ | ~ \forall{x\in\mathbb{R}, ~ y(x)=\frac{1}{2} + e^{-x}\left( \alpha\cos(3x) + \beta\sin(3x) \right)}$.\\
    On a $y(0)=1 \iff \frac{1}{2} + \alpha = 1 \iff \alpha = \frac{1}{2}$.\\
    On a $y'(0)=0 \iff -\frac{1}{2}+3\beta = 0 \iff \beta = \frac{1}{6}$.\\
    L'unique solution de ce problème de Cauchy est : $x\mapsto\frac{1}{2} + e^{-x}\left( \frac{1}{2}\cos(3x) + \frac{1}{6}\sin(3x) \right)$\\
    \qed
\end{tcolorbox}
\addcontentsline{toc}{section}{\protect\numberline{}Exercice 12.1}

\section*{Exercice 12.2 [$\blacklozenge\lozenge\lozenge$]}
\begin{tcolorbox}[enhanced, width=7in, center, size=fbox, fontupper=\large, drop shadow southwest]
    Résoudre :
    \begin{equation*}
        y'' - y' - 2y = 2\ch(x)
    \end{equation*}
    On réecrit d'abord cette équation comme : $y'' - y' - 2y = e^{x} + e^{-x}$.\\
    Polynome caractéristique : $r^2 - r - 2$. $\Delta = 9$. $r_1 = -1$ et $r_2 = 2$.\\
    Solutions de l'équation homogène : $S_0 = \{x \mapsto \lambda e^{-x} + \mu e^{2x} ~ | ~ (\lambda, \mu) \in \mathbb{R}^2\}$.\\
    Équation auxiliaire 1 : $y'' - y' - 2y = e^x$. Solution particulière : $S_{p,1} : x\mapsto Be^{x} ~ | ~ B\in\mathbb{R}$.\\
    Soit $x\in\mathbb{R}$, $B\in\mathbb{R}$ et $y:x\mapsto Be^x$.\\
    On a $y''(x) - y'(x) - 2y(x) = e^x \iff -2Be^x = e^x \iff B = -\frac{1}{2}$.\\
    Ainsi, $S_{p,1}:x\mapsto -\frac{1}{2}e^x$.\\
    Équation auxiliaire 2 : $y'' - y' - 2y = e^{-x}$. Solution particulière : $S_{p,2} : x\mapsto Cxe^{-x} ~ | ~ C\in\mathbb{R}$.\\
    Soit $x\in\mathbb{R}$, $C\in\mathbb{R}$ et $y:x\mapsto Cxe^{-x}$.\\
    On a $y''(x) - y'(x) - 2y(x) = e^{-x} \iff -3Ce^{-x} = e^{-x} \iff C = -\frac{1}{3}$.\\
    Ainsi, $S_{p,2}: x\mapsto -\frac{1}{3}xe^{-x}$.\\
    Par superposition, l'ensemble des solutions est :
    \begin{equation*}
        \{x\mapsto \lambda e^{-x} + \mu e^{2x} - \frac{1}{2}e^x - \frac{1}{3}xe^{-x} ~ | ~ (\lambda, \mu)\in\mathbb{R}^2\}
    \end{equation*}
    \qed
\end{tcolorbox}
\addcontentsline{toc}{section}{\protect\numberline{}Exercice 12.2}

\section*{Exercice 12.3 [$\blacklozenge\lozenge\lozenge$]}
\begin{tcolorbox}[enhanced, width=7in, center, size=fbox, fontupper=\large, drop shadow southwest]
    Résoudre :
    \begin{equation*}
        y'' + 2y' + y = \cos(2t) \quad (E).
    \end{equation*}
    Polynome caractéristique : $r^2 + 2r + 1$. $\Delta = 0$. $r = -1$.\\
    Solutions de l'équation homogène : $S_0 = \{x \mapsto \lambda x e^{-x} + \mu e ^{-x} ~ | ~ (\lambda, \mu) \in \mathbb{R}^2\}$.\\
    Équation auxiliaire : $y'' + 2y' + y = e^{2ix}$. Solution particulière : $S_{p,aux} : x\mapsto Be^{2ix}$ avec $B\in\mathbb{R}$.\\
    Soit $x\in\mathbb{R}$, $B\in\mathbb{R}$ et $y:x\mapsto Be^{2ix}$.\\
    On a : $y''(x) + 2y'(x) + y(x) = e^{2ix} \iff Be^{2ix}(-3+4i)= e^{2ix} \iff B = \frac{1}{-3+4i} = \frac{-3-4i}{25}$.\\
    Passage à la partie réelle : $\Re(y(x)) = \Re\left( -\frac{3+4i}{25}\left( \cos(2x) + i\sin(2x) \right) \right) = -\frac{3}{25}\cos(2x) + \frac{4}{25}\sin(2x)$.\\
    Solution générale : $S = \{x \mapsto \lambda x e^{-x} + \mu e^{-x} - \frac{3}{25}\cos(2x) + \frac{4}{25}\sin(2x) ~ | ~ (\lambda, \mu) \in \mathbb{R}^2\}$.\\
    \qed
\end{tcolorbox}
\addcontentsline{toc}{section}{\protect\numberline{}Exercice 12.3}

\section*{Exercice 12.4 [$\blacklozenge\blacklozenge\lozenge$] Résonance... ou pas}
\begin{tcolorbox}[enhanced, width=7in, center, size=fbox, fontupper=\large, drop shadow southwest]
    1. \underbar{Excitation à une pulsation quelconque}. Résoudre
    \begin{equation*}
        y'' + 4y = \cos t
    \end{equation*}
    2. \underbar{Excitation à la pulsation propre : résonance}. Résoudre
    \begin{equation*}
        y'' + 4y = \cos(2t)
    \end{equation*}
    1. Polynome caractéristique : $r^2 + 4$. $\Delta=-16$. $r_1 = 2i$, $r_2=-2i$.\\
    Solutions de l'équation homogène : $S_0 = \{x\mapsto \lambda\cos(2x) + \mu\sin(2x) ~ | ~ (\lambda, \mu) \in \mathbb{R}^2\}$.\\
    Équation auxiliaire : $y'' + 4y = e^{it}$. Solution particulière : $S_{p,aux}:x\mapsto Be^{ix}$ avec $B\in\mathbb{R}$.\\
    Soit $x,B\in\mathbb{R}$, et $y:x\mapsto Be^{ix}$.\\
    On a : $y''(x) + 4y(x) = e^{ix} \iff 3Be^{ix} = e^{ix} \iff B = \frac{1}{3}$.\\
    Passage à la partie réelle : $\Re(y(x)) = \frac{1}{3}\cos(x)$.\\
    Solution générale : $S = \{x\mapsto \lambda \cos(2x) + \mu \sin(2x) + \frac{1}{3}\cos(x) ~ | ~ (\lambda, \mu) \in \mathbb{R}^2\}$\\[0.2cm]
    2. L'ensemble des solutions de l'équation homogène est encore $S_0$.\\
    Équation auxiliaire : $y'' + 4y + e^{2it}$. Solution particulière : $S_{p,aux}:x\mapsto Bxe^{2ix}$ avec $B\in\mathbb{R}$.\\
    Soit $x,B\in\mathbb{R}$ et $y:x\mapsto Bxe^{2ix}$.\\
    On a : $y''(x) + 4y(x) = e^{2ix} \iff Be^{2ix}(4i-4x) + 4Bxe^{2ix} = e^{2ix} \iff B=\frac{1}{4i}=-\frac{i}{4}$\\
    Passage à la partie réelle : $\Re(y(x)) = \Re\left( -\frac{i}{4}x(\cos(2x) + i\sin(2x)) \right)=\frac{1}{4}x\sin(2x)$.\\
    Solution générale : $S=\{x\mapsto \lambda\cos(2x) + \mu\cos(2x) + \frac{1}{4}x\sin(2x) ~ | ~ (\lambda, \mu)\in\mathbb{R}^2\}$.\\
    \qed
\end{tcolorbox}
\addcontentsline{toc}{section}{\protect\numberline{}Exercice 12.4}
\end{document}
 