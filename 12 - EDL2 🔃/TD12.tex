\documentclass[10pt]{article}

\usepackage[T1]{fontenc}
\usepackage[left=2cm, right=2cm, top=2cm, bottom=2cm, paperheight=29cm]{geometry}
\usepackage[skins]{tcolorbox}
\usepackage{hyperref, fancyhdr, lastpage, tocloft, ragged2e, multicol}
\usepackage{amsmath, amssymb, amsthm, stmaryrd}
\usepackage{tkz-tab}
\usepackage{systeme}

\def\pagetitle{Équations Différentielles Linéaires d'ordre 2}
\setlength{\headheight}{13pt}

\title{\bf{\pagetitle}\\\large{Corrigé}}
\date{Novembre 2023}
\author{DARVOUX Théo}

\hypersetup{
    colorlinks=true,
    citecolor=black,
    linktoc=all,
    linkcolor=blue
}

\pagestyle{fancy}
\cfoot{\thepage\ sur \pageref*{LastPage}}


\begin{document}
\renewcommand*\contentsname{Exercices.}
\renewcommand*{\cftsecleader}{\cftdotfill{\cftdotsep}}
\maketitle
\hrule
\tableofcontents
\vspace{0.5cm}
\hrule


\thispagestyle{fancy}
\fancyhead[L]{MP2I Paul Valéry}
\fancyhead[C]{\pagetitle}
\fancyhead[R]{2023-2024}
\allowdisplaybreaks

\pagebreak

\section*{Exercice 12.1 [$\blacklozenge\lozenge\lozenge$]}
\begin{tcolorbox}[enhanced, width=7in, center, size=fbox, fontupper=\large, drop shadow southwest]
    Résoudre le problème de Cauchy ci-dessous :
    \begin{equation*}
        \begin{cases}
            y'' + 2y' + 10y = 5\\
            y(0) = 1 \quad y'(0) = 0
        \end{cases}
    \end{equation*}
    Polynome caractéristique : $r^2 + 2r + 10$. $\Delta = -36$. $r_{\pm}=-1\pm3i$.\\
    Solutions de l'équation homogène : $S_0 = \{x\mapsto e^{-x}\left(\alpha\cos(3x) + \beta\sin(3x)\right) ~ | ~ (\alpha, \beta)\in\mathbb{R}^2\}$\\
    Solution particulière : $S_p : x\mapsto\frac{1}{2}$.\\
    Solution générale : $S = \{x\mapsto\frac{1}{2} + e^{-x}\left( \alpha\cos(3x) + \beta\sin(3x) \right) ~ | ~ (\alpha, \beta) \in \mathbb{R}^2\}$.\\
    Conditions initiales.\\
    Soit $(\alpha, \beta)\in\mathbb{R}^2 ~ | ~ \forall{x\in\mathbb{R}, ~ y(x)=\frac{1}{2} + e^{-x}\left( \alpha\cos(3x) + \beta\sin(3x) \right)}$.\\
    On a $y(0)=1 \iff \frac{1}{2} + \alpha = 1 \iff \alpha = \frac{1}{2}$.\\
    On a $y'(0)=0 \iff -\frac{1}{2}+3\beta = 0 \iff \beta = \frac{1}{6}$.\\
    L'unique solution de ce problème de Cauchy est : $x\mapsto\frac{1}{2} + e^{-x}\left( \frac{1}{2}\cos(3x) + \frac{1}{6}\sin(3x) \right)$\\
    \qed
\end{tcolorbox}
\addcontentsline{toc}{section}{\protect\numberline{}Exercice 12.1}

\end{document}
 