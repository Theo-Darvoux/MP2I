\documentclass[10pt]{article}

\usepackage[T1]{fontenc}
\usepackage[left=2cm, right=2cm, top=2cm, bottom=2cm]{geometry}
\usepackage[skins]{tcolorbox}
\usepackage{hyperref, fancyhdr, lastpage, tocloft, ragged2e, multicol}
\usepackage{amsmath, amssymb, amsthm, stmaryrd}
\usepackage{tkz-tab}
\usepackage{systeme}

\def\pagetitle{Petits Systèmes Linéaires}

\title{\bf{\pagetitle}\\\large{Corrigé}}
\date{Novembre 2023}
\author{DARVOUX Théo}

\hypersetup{
    colorlinks=true,
    citecolor=black,
    linktoc=all,
    linkcolor=blue
}

\pagestyle{fancy}
\cfoot{\thepage\ sur \pageref*{LastPage}}


\begin{document}
\renewcommand*\contentsname{Exercices.}
\renewcommand*{\cftsecleader}{\cftdotfill{\cftdotsep}}
\maketitle

\hrule
\tableofcontents
\vspace{0.5cm}
\hrule


\thispagestyle{fancy}
\fancyhead[L]{MP2I Paul Valéry}
\fancyhead[C]{\pagetitle}
\fancyhead[R]{2023-2024}
\allowdisplaybreaks

\pagebreak

\section*{Exercice 9.1 [$\blacklozenge\lozenge\lozenge$] [Un système de Cramer bête et méchant]}
\begin{tcolorbox}[enhanced, width=7in, center, size=fbox, fontupper=\large, drop shadow southwest]
    Résoudre le système suivant dans $\mathbb{R}^3$.
    \begin{equation*}
        \systeme{
            3x + y- 2z = 10,
            2x - y+ z = 3,
            x - y +2z = 2
        }
    \end{equation*}
    Soit $(x,y,z)\in\mathbb{R}^3$.
    \begin{align*}
        &(x,y,z) \text{ est solution } \\
        \iff&
        \systeme{
            3x + y -2z = 10,
            2x - y + z = 3,
            x - y + 2z = 2
        }\\ \iff&
        \systeme{
            x - y + 2z = 2,
            2x - y + z = 3,
            3x + y - 2z = 10
        }\\ \iff&
        \systeme{
            x-y+2z=2,
            y-3z=-1,
            4y-8z=4
        }\\ \iff&
        \systeme{
            x-y+2z=2,
            y-3z=-1,
            4z=8
        }\\ \iff&
        \begin{cases}
            x=3\\
            y=5\\
            z=2
        \end{cases}
    \end{align*}
    L'unique solution de système dans $\mathbb{R}^3$ est donc $(3,5,2)$.\\
    \qed
\end{tcolorbox}

\addcontentsline{toc}{section}{\protect\numberline{}Exercice 9.1}

\end{document}
 