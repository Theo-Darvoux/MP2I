\documentclass[10pt]{article}

\usepackage[T1]{fontenc}
\usepackage[left=2cm, right=2cm, top=2cm, bottom=2cm, paperheight=31cm]{geometry}
\usepackage[skins]{tcolorbox}
\usepackage{hyperref, fancyhdr, lastpage, tocloft, ragged2e, multicol, changepage}
\usepackage{amsmath, amssymb, amsthm, stmaryrd}
\usepackage{tkz-tab}
\usepackage{systeme}

\def\pagetitle{Applications}
\setlength{\headheight}{13pt}

\title{\bf{\pagetitle}\\\large{Corrigé}}
\date{Décembre 2023}
\author{DARVOUX Théo}

\DeclareMathOperator{\ch}{ch}

\hypersetup{
    colorlinks=true,
    citecolor=black,
    linktoc=all,
    linkcolor=blue
}

\pagestyle{fancy}
\cfoot{\thepage\ sur \pageref*{LastPage}}

\begin{document}
\renewcommand*\contentsname{Exercices.}
\renewcommand*{\cftsecleader}{\cftdotfill{\cftdotsep}}
\maketitle

\hrule
\tableofcontents
\vspace{0.5cm}
\hrule

\thispagestyle{fancy}
\fancyhead[L]{MP2I Paul Valéry}
\fancyhead[C]{\pagetitle}
\fancyhead[R]{2023-2024}
\allowdisplaybreaks

\pagebreak


\section*{Exercice 15.1 [$\blacklozenge\lozenge\lozenge$]}
\begin{tcolorbox}[enhanced, width=7.6in, center, size=fbox, fontupper=\large, drop shadow southwest]
    Soit $f:E\to F$ une application. Soient deux parties $A \subset E$ et $B \subset F$. Montrer l'égalité
    \begin{equation*}
        f(A) \cap B = f(A \cap f^{-1}(B)).
    \end{equation*}
    Procédons par double inclusion.\\
    $\circledcirc$ Soit $y\in f(A) \cap B$. Montrons que $y\in f(A \cap f^{-1}(B))$.\\
    On a $y\in f(A)$ et $y\in B$.\\
    $\exists x\in A ~ | ~ y = f(x)$ donc $x\in A$ et $x\in f^{-1}(B)$ car $y\in B$.\\
    Ainsi $x\in A\cap f^{-1}(B)$ et $f(x) = y \in f(A \cap f^{-1}(B))$\\[0.15cm]
    $\circledcirc$ Soit $y\in f(A \cap f^{-1}(B))$ Montrons que $y\in f(A) \cap B$.\\
    $\exists x \in A \cap f^{-1}(B) ~ | ~ y = f(x)$ donc $x\in A$ et $x \in f^{-1}(B)$.\\
    Ainsi, $f(x) = y \in f(A)$ et $f(x) = y \in B$ : $y\in f(A)\cap B$.\\
    \qed
\end{tcolorbox}
\addcontentsline{toc}{section}{Images directes, images réciproques.}
\addcontentsline{toc}{section}{\protect\numberline{}Exercice 15.1}

\section*{Exercice 15.2 [$\blacklozenge\blacklozenge\lozenge$]}
\begin{tcolorbox}[enhanced, width=7.6in, center, size=fbox, fontupper=\large, drop shadow southwest]
    Soit $f:E\to F$ une application. Soit $A$ une partie de $E$ et $B$ une partie de $F$.\\
    1. (a) Montrer que $A \subset f^{-1}(f(A))$.\\
    (b) Montrer que si $f$ est injective, la réciproque est vraie.\\
    2. (a) Montrer que $f(f^{-1}(B)) \subset B$.\\
    (b) Démontrer que si $f$ est surjective, la réciproque est vraie.\\
    3. Montrer que $f(f^{-1}(f(A))) = f(A)$.\\
    4. Montrer que $f^{-1}(f(f^{-1}(B)))=f^{-1}(B)$.\\[0.15cm]
    1.\\
    a) Soit $x\in A$. Montrons que $x\in f^{-1}(f(A))$.\\
    On a $x\in A$ alors $f(x) \in f(A)$ et $x\in f^{-1}(f(A))$.\\
    b) On suppose $f$ injective, soit $x \in f^{-1}(f(A))$.\\
    On applique $f$ : $f(x) \in f(A)$. Par injectivité de $f$, $x \in A$.\\[0.2cm]
    2.\\
    a) Soit $y \in f(f^{-1}(B))$.\\
    On a $\exists x \in f^{-1}(B) ~ | ~ y = f(x)$. Ainsi, $f(x)\in B$ : $y\in B$.\\
    b) Supposons $f$ surjective, soit $y\in B$.\\
    On a $\exists x \in f^{-1}(B) ~ | ~ y = f(x)$ et $f(x) = y \in f(f^{-1}(B))$.\\[0.2cm]
    3) Soit $y\in f(f^{-1}(f(A)))$. Montrons que $y\in f(A)$.\\
    On a $\exists x \in f^{-1}(f(A)) ~ | ~ y = f(x)$ et $f(x) \in f(A)$ donc $y \in f(A)$.\\
    Soit $y\in f(A)$. Montrons que $y\in f(f^{-1}(f(A)))$.\\
    On a $\exists x \in A ~ | ~ y = f(x)$ alors $f(x) \in f(A)$ et $x\in f^{-1}(f(A))$. Donc $f(x) = y \in f(f^{-1}(f(A)))$.\\[0.2cm]
    4) Soit $y \in f^{-1}(f(f^{-1}(B)))$. Montrons que $y \in f^{-1}(B)$.\\
    On a $f(y) \in f(f^{-1}(B))$ alors $y \in f^{-1}(B)$.\\
    Soit $y\in f^{-1}(B)$. Montrons que $y\in f^{-1}(f(f^{-1}(B)))$.\\
    On a $f(y) \in f(f^{-1}(B))$ donc $y \in f^{-1}(f(f^{-1}(B)))$.\\
    \qed
\end{tcolorbox}
\addcontentsline{toc}{section}{\protect\numberline{}Exercice 15.2}

\section*{Exercice 15.3 [$\blacklozenge\blacklozenge\blacklozenge$]}
\begin{tcolorbox}[enhanced, width=7.6in, center, size=fbox, fontupper=\large, drop shadow southwest]
    Soit $f:E\to F$ une application. Montrer que
    \begin{equation*}
        f \text{ est injective } \iff [\forall A,B \in \mathcal{P}(E) ~ f(A \cap B) = f(A) \cap f(B)]
    \end{equation*}
    $\circledcirc$ Supposons $f$ injective. Soient $A,B \in \mathcal{P}(E)$.\\
    On sait déjà que $f(A \cap B) \subset f(A) \cap f(B)$.\\
    Montrons alors que $f(A) \cap f(B) \subset f(A \cap B)$.\\
    Soit $y \in f(A) \cap f(B)$. On a que $y \in f(A) \wedge y \in f(B)$.\\
    Ainsi, $\exists x_A \in A ~ | ~ y = f(x_A)$ et $\exists x_B \in B ~ | ~ y = f(x_B)$.\\
    Or $f$ est injective : $x_A = x_B$, ainsi $x_A \in A \cap B$.\\
    On a enfin que $f(x_A) \in f(A \cap B)$, alors $y \in f(A \cap B)$.\\[0.2cm]
    $\circledcirc$ Supposons $[\forall A,B \in \mathcal{P}(E) ~ f(A \cap B) = f(A) \cap f(B)]$. Montrons que $f$ est injective.\\
    Soient $A,B \in \mathcal{P}(E)$.\\
    Soient $x,x' \in E$. On suppose que $f(x) = f(x')$. Montrons que $x = x'$.\\
    On a que $\{x\}$ et $\{x'\} \in \mathcal{P}(E)$.\\
    Ainsi : $f(\{x\} \cap \{x'\}) = f(\{x\}) \cap f(\{x'\})$.\\
    Supposons que $x \neq x'$. On a alors : $f(\varnothing) = f(\{x\}) \cap f(\{x'\})$ : $\varnothing = \{f(x)\} \cap \{f(x')\}$.\\
    Or $f(x) = f(x')$ donc $\{f(x)\} \cap \{f(x')\} \neq \varnothing$. C'est absurde : $x = x'$.\\
    On a bien montré que $f$ est injective.\\
    \qed 
\end{tcolorbox}
\addcontentsline{toc}{section}{\protect\numberline{}Exercice 15.3}

\section*{Exercice 15.4 [$\blacklozenge\lozenge\lozenge$]}
\begin{tcolorbox}[enhanced, width=7.6in, center, size=fbox, fontupper=\large, drop shadow southwest]
    Soient
    \begin{equation*}
        f : \begin{cases}\mathbb{N}^2 \to \mathbb{Z} \\ (n,p) \mapsto (-1)^np \end{cases} \hspace{1.25cm} \text{et} \hspace{1.25cm} g : \begin{cases}\mathbb{R} \to \mathbb{C} \\ x\mapsto \frac{1+ix}{1-ix} \end{cases}
    \end{equation*}
    Ces fonctions sont-elles injectives ? Surjectives ?\\[0.2cm]
    On a que $f$ n'est pas injective : $f(0,1) = f(2,1) = 1$.\\
    Montrons que $f$ est surjective.\\
    Soit $y\in\mathbb{Z}$. Montrons que $\exists (n,p)\in\mathbb{N}^2 ~ | ~ f(n,p) = y$.\\
    Si $y \geq 0$, on prend $n = 0$ et $p = |y|$.\\
    Si $y \leq 0$, on prend $n = 1$ et $p = |y|$.\\[0.2cm]
    On a que $g$ n'est pas surjective : $0$ n'a aucun antécédent par $g$.\\
    Montrons que $g$ est injective.\\
    Soient $x,x' \in \mathbb{R}$, supposons $g(x) = g(x')$. Montrons que $x=x'$.\\
    On a :
    \begin{align*}
        g(x) = g(x') &\iff \frac{1+ix}{1-ix} = \frac{1+ix'}{1-ix'}\\
        &\iff (1+ix)(1-ix') = (1+ix')(1-ix)\\
        &\iff 1 - ix' + ix + xx' = 1 - ix + ix' + xx'\\
        &\iff 2ix = 2ix'\\
        &\iff x = x' 
    \end{align*}
    On a bien que $g$ est injective.\\
    \qed
\end{tcolorbox}
\addcontentsline{toc}{section}{Applications injectives, surjectives.}
\addcontentsline{toc}{section}{\protect\numberline{}Exercice 15.4}

\section*{Exercice 15.5 [$\blacklozenge\lozenge\lozenge$]}
\begin{tcolorbox}[enhanced, width=7.6in, center, size=fbox, fontupper=\large, drop shadow southwest]
    Dans cet exercice, on admet que $\pi$ est irrationnel.\\
    Démontrer que $\cos_{|\mathbb{Q}}$ n'est pas injective et que $\sin_{|\mathbb{Q}}$ l'est.\\[0.2cm]
    On sait que $\cos$ est paire : $\cos_{|\mathbb{Q}}$ l'est aussi.\\
    Alors $\cos_{|\mathbb{Q}}(\frac{1}{2}) = \cos_{|\mathbb{Q}}(-\frac{1}{2})$. Or $\frac{1}{2} \neq -\frac{1}{2}$ : $\cos_{|\mathbb{Q}}$ n'est pas injective.\\[0.2cm]
    Soient $x,x'\in\mathbb{Q}^2$. Supposons que $\sin_{|\mathbb{Q}}(x) = \sin_{|\mathbb{Q}}(x')$. Montrons que $x=x'$.\\
    On a :
    \begin{align*}
        \sin_{|\mathbb{Q}}(x) = \sin_{|\mathbb{Q}}(x') &\iff x \equiv x' [2\pi] ~ (2\pi\text{-périodicité}) \\
        &\iff x = x' + 2k\pi ~ (k\in\mathbb{Z})\\
    \end{align*}
    Or, $\forall{k\in\mathbb{Z}^*}, ~ x' + 2k\pi \notin \mathbb{Q}$. On a alors que $k=0$ :
    \begin{equation*}
        \sin_{|\mathbb{Q}}(x) = \sin_{|\mathbb{Q}}(x') \iff x = x' + 2\cdot0\pi \iff x = x'
    \end{equation*}
    \qed
\end{tcolorbox}
\addcontentsline{toc}{section}{\protect\numberline{}Exercice 15.5}

\end{document}
 
