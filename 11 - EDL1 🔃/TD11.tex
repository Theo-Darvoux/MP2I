\documentclass[10pt]{article}

\usepackage[T1]{fontenc}
\usepackage[left=2cm, right=2cm, top=2cm, bottom=2cm]{geometry}
\usepackage[skins]{tcolorbox}
\usepackage{hyperref, fancyhdr, lastpage, tocloft, ragged2e, multicol}
\usepackage{amsmath, amssymb, amsthm, stmaryrd}
\usepackage{tkz-tab}
\usepackage{systeme}

\def\pagetitle{Équations Différentielles Linéaires d'ordre 1}
\setlength{\headheight}{13pt}

\title{\bf{\pagetitle}\\\large{Corrigé}}
\date{Novembre 2023}
\author{DARVOUX Théo}

\hypersetup{
    colorlinks=true,
    citecolor=black,
    linktoc=all,
    linkcolor=blue
}

\pagestyle{fancy}
\cfoot{\thepage\ sur \pageref*{LastPage}}


\begin{document}
\renewcommand*\contentsname{Exercices.}
\renewcommand*{\cftsecleader}{\cftdotfill{\cftdotsep}}
\maketitle
\begin{center}
    \LARGE{Crédits : Ibrahim pour tout (j'aime pas les EDL)}\\
\end{center}
\hrule
\tableofcontents
\vspace{0.5cm}
\hrule


\thispagestyle{fancy}
\fancyhead[L]{MP2I Paul Valéry}
\fancyhead[C]{\pagetitle}
\fancyhead[R]{2023-2024}
\allowdisplaybreaks

\pagebreak

\section*{Exercice 11.1 [$\blacklozenge\lozenge\lozenge$]}
\begin{tcolorbox}[enhanced, width=7in, center, size=fbox, fontupper=\large, drop shadow southwest]
    Résoudre les équations différentielles ci-dessous
    \begin{center}
        1. $y' - 2y = 2$ sur $\mathbb{R}$ \hspace{0.5cm} 2. $(x^2+1)y'+xy=x$ \hspace{0.5cm} 3. $y' + \tan(x)y = \sin(2x)$ sur $]-\frac{\pi}{2}, \frac{\pi}{2}[$\\
        4. $y'-\ln(x)y = x^x$ sur $\mathbb{R}_+^*$ \hspace{1cm} 5. $(1-x)y' - y = \frac{1}{1-x}$ sur $]-\infty, 1[$
    \end{center}
    1. Solutions de l'équation homogène : $S_0=\{x\mapsto\lambda e^{2x} \, | \, \lambda\in\mathbb{R}\}$\\
    Solution particulière, avec $y$ constante : $S_p : x\mapsto -1$.\\
    Ensemble de solutions : $S = \{\lambda e^{2x} - 1 ~ | ~ \lambda\in\mathbb{R}\}$.\\[0.2cm]
    2. L'équation se réecrit comme $y' + \frac{x}{x^2+1}y=\frac{x}{x^2+1}$.\\
    Solutions de l'équation homogène : $S_0 = \{x\mapsto\frac{\lambda}{\sqrt{x^2+1}} ~ | ~ \lambda \in \mathbb{R}\}$\\
    Solution particulière : $S_p:x\mapsto1$ est solution évidente.\\
    Ensemble de solutions : $S = \{x\mapsto\frac{\lambda}{\sqrt{x^2+1}}+1 ~ | ~ \lambda\in\mathbb{R}\}$.\\[0.2cm]
    3.Soit $I = ~ ]-\frac{\pi}{2}, \frac{\pi}{2}[$.\\
    Solutions de l'équation homogène : $S_0 = \{x\mapsto \lambda \cos x ~ | ~ \lambda \in \mathbb{R}\}$.\\
    Solution particulière : Soit $u\in S_0$ et $\lambda:I\rightarrow\mathbb{K}$ dérivable sur $I$. On cherche $z=\lambda'u$.
    \begin{align*}
        z \text{ est solution } &\iff \forall{x\in I}, ~ \lambda'(x)\cos(x) = \sin(2x)\\
        &\iff \forall{x}\in I, ~ \lambda'(x) = \frac{\sin(2x)}{\cos(x)}=2\sin(x)\\
        &\iff \lambda =-2\cos
    \end{align*}
    Ainsi, $z=-2\cos^2$.\\
    Ensemble de solutions : $S = \{x\mapsto \lambda\cos x - 2\cos^2x\ ~ | ~ \lambda \in \mathbb{R}\}$.
\end{tcolorbox}

\addcontentsline{toc}{section}{\protect\numberline{}Exercice 11.1}

\end{document}
 