\documentclass[10pt]{article}

\usepackage[T1]{fontenc}
\usepackage[left=2cm, right=2cm, top=2cm, bottom=2cm, paperheight=29cm]{geometry}
\usepackage[skins]{tcolorbox}
\usepackage{hyperref, fancyhdr, lastpage, tocloft, ragged2e, multicol}
\usepackage{amsmath, amssymb, amsthm, stmaryrd}
\usepackage{tkz-tab}
\usepackage{systeme}

\def\pagetitle{Équations Différentielles Linéaires d'ordre 1}
\setlength{\headheight}{13pt}

\title{\bf{\pagetitle}\\\large{Corrigé}}
\date{Novembre 2023}
\author{DARVOUX Théo}

\hypersetup{
    colorlinks=true,
    citecolor=black,
    linktoc=all,
    linkcolor=blue
}

\pagestyle{fancy}
\cfoot{\thepage\ sur \pageref*{LastPage}}


\begin{document}
\renewcommand*\contentsname{Exercices.}
\renewcommand*{\cftsecleader}{\cftdotfill{\cftdotsep}}
\maketitle
\begin{center}
    \LARGE{Crédits : Ibrahim pour tout (j'aime pas les EDL)}\\
\end{center}
\hrule
\tableofcontents
\vspace{0.5cm}
\hrule

\begin{center}
    \LARGE{Je rédige pas la variation de la constante parce que trop long $\blacklozenge\blacklozenge\blacklozenge$}
\end{center}


\thispagestyle{fancy}
\fancyhead[L]{MP2I Paul Valéry}
\fancyhead[C]{\pagetitle}
\fancyhead[R]{2023-2024}
\allowdisplaybreaks

\pagebreak

\section*{Exercice 11.1 [$\blacklozenge\lozenge\lozenge$]}
\begin{tcolorbox}[enhanced, width=7in, center, size=fbox, fontupper=\large, drop shadow southwest]
    Résoudre les équations différentielles ci-dessous
    \begin{center}
        1. $y' - 2y = 2$ sur $\mathbb{R}$ \hspace{0.5cm} 2. $(x^2+1)y'+xy=x$ \hspace{0.5cm} 3. $y' + \tan(x)y = \sin(2x)$ sur $]-\frac{\pi}{2}, \frac{\pi}{2}[$\\
        4. $y'-\ln(x)y = x^x$ sur $\mathbb{R}_+^*$ \hspace{1cm} 5. $(1-x)y' - y = \frac{1}{1-x}$ sur $]-\infty, 1[$
    \end{center}
    1. Solutions de l'équation homogène : $S_0=\{x\mapsto\lambda e^{2x} ~ | ~ \lambda\in\mathbb{R}\}$\\
    Solution particulière, avec $y$ constante : $S_p : x\mapsto -1$.\\
    Ensemble de solutions : $S = \{\lambda e^{2x} - 1 ~ | ~ \lambda\in\mathbb{R}\}$.\\[0.2cm]
    2. L'équation se réecrit comme $y' + \frac{x}{x^2+1}y=\frac{x}{x^2+1}$.\\
    Solutions de l'équation homogène : $S_0 = \{x\mapsto\frac{\lambda}{\sqrt{x^2+1}} ~ | ~ \lambda \in \mathbb{R}\}$\\
    Solution particulière : $S_p:x\mapsto1$ est solution évidente.\\
    Ensemble de solutions : $S = \{x\mapsto\frac{\lambda}{\sqrt{x^2+1}}+1 ~ | ~ \lambda\in\mathbb{R}\}$.\\[0.2cm]
    3.Soit $I = ~ ]-\frac{\pi}{2}, \frac{\pi}{2}[$.\\
    Solutions de l'équation homogène : $S_0 = \{x\mapsto \lambda \cos x ~ | ~ \lambda \in \mathbb{R}\}$.\\
    Solution particulière : Soit $u\in S_0$ et $\lambda:I\rightarrow\mathbb{K}$ dérivable sur $I$. On cherche $z=\lambda'u$.
    \begin{align*}
        z \text{ est solution } &\iff \forall{x\in I}, ~ \lambda'(x)\cos(x) = \sin(2x)\\
        &\iff \forall{x}\in I, ~ \lambda'(x) = \frac{\sin(2x)}{\cos(x)}=2\sin(x)\\
        &\iff \lambda =-2\cos
    \end{align*}
    Ainsi, $z=-2\cos^2$.\\
    Ensemble de solutions : $S = \{x\mapsto \lambda\cos x - 2\cos^2x\ ~ | ~ \lambda \in \mathbb{R}\}$.\\
    4. Soit $I=\mathbb{R}^*_+$.\\
    Solutions de l'équation homogène : $S_0=\{x\mapsto\lambda \frac{x^x}{e^x} ~ | ~ \lambda \in\mathbb{R}\}$\\
    Solution particulière : Soit $u\in S_0$ et $\lambda : I \rightarrow \mathbb{K}$ dérivable sur $I$. On cherche $z=\lambda'u$.
    \begin{align*}
        z \text{ est solution} &\iff \forall{x\in I}, ~ \lambda'(x)\frac{x^x}{e^x} = x^x\\
        &\iff \forall{x \in I}, ~ \lambda'(x) = e^x\\
        &\iff \lambda = e^\cdot
    \end{align*}
    Ainsi, $z:x \mapsto x^x$\\
    Ensemble de solutions : $S = \{x\mapsto \lambda\frac{x^x}{e^x} + x^x ~ | ~ \lambda \in \mathbb{R}\}$\\
    5. Soit $I=]-\infty,1[$. L'équation se réecrit comme $y' - \frac{1}{1-x}y = \frac{1}{(1-x)^2}$.\\
    Solutions de l'équation homogène : $S_0 = \{x\mapsto\frac{\lambda}{1-x} ~ | ~ \lambda\in\mathbb{R}\}$.\\
    Solution particulière : Soit $u\in S_0$ et $\lambda:I\rightarrow\mathbb{K}$ dérivable sur $I$. On cherche $z=\lambda'u$.
    \begin{align*}
        z \text{ est solution} &\iff \forall{x\in I}, ~ \frac{\lambda'(x)}{1-x} = \frac{1}{(1-x)^2}\\
        &\iff \forall{x\in I}, ~ \lambda'(x) = \frac{1}{1-x}\\
        &\iff \forall{x\in I}, ~ \lambda(x) = -\ln(1-x) 
    \end{align*} 
    Ainsi, $z:x\mapsto -\frac{\ln(1-x)}{1-x}$.\\
    Ensemble de solutions : $S = \{x \mapsto \frac{\lambda}{1-x} - \frac{\ln(1-x)}{1-x} ~ | ~ \lambda \in \mathbb{R}\}$\\
    \qed
\end{tcolorbox}

\addcontentsline{toc}{section}{\protect\numberline{}Exercice 11.1}

\section*{Exercice 11.2 [$\blacklozenge\lozenge\lozenge$]}
\begin{tcolorbox}[enhanced, width=7in, center, size=fbox, fontupper=\large, drop shadow southwest]
    Résoudre sur $R_+^*$ le problème de Cauchy $\begin{cases} y' - \frac{2}{x}y = x^2\cos x\\ y(\pi)=0 \end{cases}$.\\
    Solution homogène : $S_0 = \{x \mapsto \lambda x^2 ~ | ~ \lambda \in\mathbb{R}\}$.\\
    Solution particulière : Soit $u\in S_0$ et $\lambda : I \rightarrow \mathbb{K}$ dérivable sur $I$. On cherche $z = \lambda'u$.
    \begin{align*}
        z \text{ est solution} &\iff \forall{x\in I} ~ \lambda'(x)x^2 = x^2\cos x\\
        &\iff \forall{x\in I} ~ \lambda'(x) = \cos x\\
        &\iff \lambda = \sin
    \end{align*}
    Ainsi, $z:x\mapsto x^2\sin x$.\\
    Ensemble de solutions : $S=\{x\mapsto\lambda x^2 + x^2\sin x ~ | ~ \lambda \in \mathbb{R}\}$\\
    Conditions initiales : Soit $y \in S$. On a :
    \begin{align*}
        y(\pi) = 0 &\iff \exists{\lambda\in\mathbb{R}} ~ | ~ \lambda\pi^2 + \pi^2\sin(\pi) = 0\\
        &\iff \lambda\pi^2 = 0\\
        &\iff \lambda = 0
    \end{align*}
    L'unique solution de ce problème de Cauchy est donc : $y:x\mapsto x^2\sin x$.\\
    \qed
\end{tcolorbox}

\addcontentsline{toc}{section}{\protect\numberline{}Exercice 11.2}

\section*{Exercice 11.3 [$\blacklozenge\blacklozenge\lozenge$]}
\begin{tcolorbox}[enhanced, width=7in, center, size=fbox, fontupper=\large, drop shadow southwest]
    Trouver toutes les fonctions $f$ dérivables sur $\mathbb{R}$ telles que
    \begin{equation*}
        \forall{x \in \mathbb{R}} ~ f'(x) + f(x) = \int_0^1{f(t)dt}
    \end{equation*}
    \emph{Analyse.}\\[0.1cm]
    On suppose qu'il existe $y$ dérivable sur $\mathbb{R}$ solution de cette équation.\\[0.1cm]
    Soit $x\in\mathbb{R}$.\\
    En dérivant l'égalité, on obtient : $y''(x) + y'(x) = 0$. On pose $g(x)=y'(x)$.\\
    On a : $g'(x) + g(x) = 0$.\\
    Solution générale : $S = \{x\mapsto \lambda e^{-x} ~ | ~ x\in\mathbb{R}\}$.\\
    Ainsi, $g \in S$ et $\exists (\lambda, \mu) \in \mathbb{R}^2 ~ | ~ y(x) = -\lambda e^{-x} + \mu$.\\
    On a :
    \begin{align*}
        y'(x) + y(x) = \int_0^1y(t)dt &\iff \lambda e^{-x} - \lambda e^{-x} + \mu = \left[\lambda e^{-t} + \mu t\right]_0^1\\
        &\iff \mu = \lambda e^{-1} + \mu - \lambda\\
        &\iff \lambda(e^{-1} -1) = 0\\
        &\iff \lambda = 0
    \end{align*}
    Ainsi, l'ensemble des solutions est : $\{x\mapsto\mu ~ | ~ \mu \in \mathbb{R}\}$.\\
    \emph{Synthèse}.\\
    Soit $x\in\mathbb{R}$ et $\mu\in\mathbb{R} ~ | ~ y(x)=\mu$. On a $y'(x) + y(x) = \mu$ et $\int_0^1y(t)dt=\int_0^1\mu dt=\mu$\\
    \qed
\end{tcolorbox}

\addcontentsline{toc}{section}{\protect\numberline{}Exercice 11.3}


\section*{Exercice 11.4 [$\blacklozenge\blacklozenge\blacklozenge$] <<Recollement>>}
\begin{tcolorbox}[enhanced, width=7in, center, size=fbox, fontupper=\large, drop shadow southwest]
    Soit l'équation différentielle $x^2y' - y = 0$.\\
    1. Résoudre sur $\mathbb{R}^*_+$ et sur $\mathbb{R}^*_-$.\\
    2. Trouver toutes les solutions définies sur $\mathbb{R}$
\end{tcolorbox}

\addcontentsline{toc}{section}{\protect\numberline{}Exercice 11.4}
\end{document}
 